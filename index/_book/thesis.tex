% This is the Reed College LaTeX thesis template. Most of the work
% for the document class was done by Sam Noble (SN), as well as this
% template. Later comments etc. by Ben Salzberg (BTS). Additional
% restructuring and APA support by Jess Youngberg (JY).
% Your comments and suggestions are more than welcome; please email
% them to cus@reed.edu
%
% See http://web.reed.edu/cis/help/latex.html for help. There are a
% great bunch of help pages there, with notes on
% getting started, bibtex, etc. Go there and read it if you're not
% already familiar with LaTeX.
%
% Any line that starts with a percent symbol is a comment.
% They won't show up in the document, and are useful for notes
% to yourself and explaining commands.
% Commenting also removes a line from the document;
% very handy for troubleshooting problems. -BTS

% As far as I know, this follows the requirements laid out in
% the 2002-2003 Senior Handbook. Ask a librarian to check the
% document before binding. -SN

%%
%% Preamble
%%
% \documentclass{<something>} must begin each LaTeX document
\documentclass[12pt,twoside]{reedthesis}
% Packages are extensions to the basic LaTeX functions. Whatever you
% want to typeset, there is probably a package out there for it.
% Chemistry (chemtex), screenplays, you name it.
% Check out CTAN to see: http://www.ctan.org/
%%
\usepackage{graphicx,latexsym}
\usepackage[french]{babel} 
\usepackage{amsmath}
\usepackage{amssymb,amsthm}
\usepackage[dvipsnames]{xcolor} % tk: for more color
\usepackage{xcolor}
\usepackage{eso-pic}
\usepackage{longtable,booktabs,setspace}
\usepackage{chemarr} %% Useful for one reaction arrow, useless if you're not a chem major
\usepackage[hyphens]{url}
\usepackage{tikz}
\usetikzlibrary{calc}
\newcommand\HRule{\rule{\textwidth}{1pt}}
% Added by CII
\usepackage{hyperref}
\usepackage{lmodern}
\usepackage{float}
\floatplacement{figure}{H}
% End of CII addition
\usepackage{rotating}
\usepackage{upgreek} % tk : pour pouvoir utiliser le symbole µ droit (pas en itallic)
\usepackage{pdfpages}
\usepackage{lscape}
\newcommand{\blandscape}{\begin{landscape}}
\newcommand{\elandscape}{\end{landscape}}
\usepackage[utf8]{inputenc}




% Next line commented out by CII
%%% \usepackage{natbib}
% Comment out the natbib line above and uncomment the following two lines to use the new
% biblatex-chicago style, for Chicago A. Also make some changes at the end where the
% bibliography is included.
%\usepackage{biblatex-chicago}
%\bibliography{thesis}


% Added by CII (Thanks, Hadley!)
% Use ref for internal links
\renewcommand{\hyperref}[2][???]{\autoref{#1}}
\def\chapterautorefname{Chapter}
\def\sectionautorefname{Section}
\def\subsectionautorefname{Subsection}
% End of CII addition

% Added by CII
\usepackage{caption}
\captionsetup{width=5in}
% End of CII addition

% \usepackage{times} % other fonts are available like times, bookman, charter, palatino


% To pass between YAML and LaTeX the dollar signs are added by CII
\title{THÈSE}
\author{Thomas Karaouzene}
\labo{}
% The month and year that you submit your FINAL draft TO THE LIBRARY (May or December)
\date{31 octobre 2017}
\division{}
\advisor{Pierre Ray}
%If you have two advisors for some reason, you can use the following
% Uncommented out by CII
\altadvisor{Nicolas Thierry-Mieg}
% End of CII addition

%%% Remember to use the correct department!
\department{Ingénierie de la Santé, de la Cognition et Environnement (EDISCE)}
% if you're writing a thesis in an interdisciplinary major,
% uncomment the line below and change the text as appropriate.
% check the Senior Handbook if unsure.
%\thedivisionof{The Established Interdisciplinary Committee for}
% if you want the approval page to say "Approved for the Committee",
% uncomment the next line
%\approvedforthe{Committee}

% Added by CII
%%% Copied from knitr
%% maxwidth is the original width if it's less than linewidth
%% otherwise use linewidth (to make sure the graphics do not exceed the margin)
\makeatletter
\def\maxwidth{ %
  \ifdim\Gin@nat@width>\linewidth
    \linewidth
  \else
    \Gin@nat@width
  \fi
}
\makeatother

\renewcommand{\contentsname}{Table of Contents}
% End of CII addition

\setlength{\parskip}{0pt}

% Added by CII
  %\setlength{\parskip}{\baselineskip}
  \usepackage[parfill]{parskip}

\providecommand{\tightlist}{%
  \setlength{\itemsep}{0pt}\setlength{\parskip}{0pt}}

\Acknowledgements{

}

\Dedication{

}

\Preface{
This is an example of a thesis setup to use the reed thesis document
class (for LaTeX) and the R bookdown package, in general.
}

\Abstract{

}

	\usepackage{tikz}
% End of CII addition
%%
%% End Preamble
%%
%

\usepackage{amsthm}
\newtheorem{theorem}{Theorem}[section]
\newtheorem{lemma}{Lemma}[section]
\theoremstyle{definition}
\newtheorem{definition}{Definition}[section]
\newtheorem{corollary}{Corollary}[section]
\newtheorem{proposition}{Proposition}[section]
\theoremstyle{definition}
\newtheorem{example}{Example}[section]
\theoremstyle{remark}
\newtheorem*{remark}{Remark}
\begin{document}

% Everything below added by CII
      \maketitle
  
  \frontmatter % this stuff will be roman-numbered
  \pagestyle{empty} % this removes page numbers from the frontmatter

  
      \begin{preface}
      This is an example of a thesis setup to use the reed thesis document
      class (for LaTeX) and the R bookdown package, in general.
    \end{preface}
  
      \hypersetup{linkcolor=black}
    \setcounter{tocdepth}{3}
    \tableofcontents
  
      \listoftables
  
      \listoffigures
  
  
  
  \mainmatter % here the regular arabic numbering starts
  \pagestyle{fancyplain} % turns page numbering back on

  \chapter{Delete line 6 if you only have one
  advisor}\label{delete-line-6-if-you-only-have-one-advisor}
  
  \chapter{Mise en place d'une stratégie pour l'analyse des données
  exomiques -- application en recherche
  clinique}\label{mise-en-place-dune-strategie-pour-lanalyse-des-donnees-exomiques-application-en-recherche-clinique}
  
  \newpage
  
  En 2011, les bases moléculaires d'environ 3700 pathologies à
  transmission Mendélienne avaient été élucidées. Cependant, pour une
  quantité équivalente de pathologies Mendéliennes (ou suspectées de
  l'être) cette cause reste un mystère (Amberger, Bocchini, \& Hamosh,
  \protect\hyperlink{ref-Amberger2011}{2011}). Avec plusieurs centaines de
  pathologies caractérisées depuis 2010 (S. B. Ng et al., n.d.), les
  séquençages WGS et WES ont, depuis leur émergence, révolutionnés les
  méthodes de recherche dans le cadre d'étude phénotype-génotype en
  permettant de manière rapide et à moindre coup le séquençage de la
  quasi-totalité des gènes humains. Dès lors, le défis de ces analyses
  n'est plus le séquençage de l'ADN mais l'interprétation des données
  massives produites. En effet, l'un des plus grands challenges des
  analyses phénotype-génotype réalisées par WES réside dans l'analyse de
  l'importante quantité de variant portés par chaque individu s'élevant à
  plusieurs dizaines de milliers lorsque l'on compare avec le génome de
  référence. Même après avoir retiré les variants retrouvés fréquemment
  dans la population générale, des méthodes additionnelles sont
  nécessaires pour prédire, parmi les variants restant, lesquels induisent
  des conséquences fonctionnelles sérieuses afin de les prioriser (Pelak
  et al., \protect\hyperlink{ref-Pelak2010}{2010}). De nombreux logiciels
  tel que Variant Effect Predictor (W. McLaren et al.,
  \protect\hyperlink{ref-McLaren2016}{2016}), SnpEff (Cingolani et al.,
  \protect\hyperlink{ref-Cingolani2012}{2012}) ou encore ANNOVAR (K. Wang,
  Li, \& Hakonarson, \protect\hyperlink{ref-Wang2010}{2010}) permettent
  d'identifier quels sont les variants qui ont un effet tronquant sur la
  protéine. Cependant, avec en moyenne 165 variants homozygotes ayant un
  effet tronquant retrouvés dans chaque exomes (Pelak et al.,
  \protect\hyperlink{ref-Pelak2010}{2010}) ces méthodes, bien qu'efficaces
  sont souvent insuffisantes.
  
  D'autres logiciels tel que Exomiser (P. N. Robinson et al.,
  \protect\hyperlink{ref-Robinson2014}{2014}) vont, à partir d'une liste
  de variants \textbf{déjà} appelés effectuer les étapes d'annotation, de
  filtrage et de priorisation. Malgré l'efficacité de ces logiciels, aucun
  d'entre eux ne couvrent l'ensemble des étapes allant de l'alignement des
  \emph{reads} à la priorisation des variants. La plupart ayant pour point
  de départ une liste de variants appelés en amont. Ils ne contrôlent donc
  en aucune manière les étapes d'alignement et d'appel des variants. Or,
  comme il a été dit plus tôt, ces deux étapes constituent la base de
  l'analyse {[}{]}.
  
  Ce chapitre décrit à la fois la constitution d'un pipeline d'analyse des
  données de séquençage exomique recouvrant l'ensemble des étapes allant
  de l'allignement des séquences à la priorisation des variants ainsi que
  son utilisation dans le cadre de la recherche de mutations entrainant
  différents phénotypes d'infertilité d'une part de cas familiaux composés
  de duos ou trio et, pour finir, d'une large cohorte d'individus non
  apparentés présentant tous le même phénotype.
  
  \newpage
  
  \section{Méthode : Description du
  pipeline}\label{methode-description-du-pipeline}
  
  \subsection{\texorpdfstring{L'alignement des
  \emph{reads}}{L'alignement des reads}}\label{lalignement-des-reads}
  
  Comme expliqué plus tôt, l'étape d'alignement à pour objectif de
  repositioner l'ensemble des \emph{reads} d'un individu le long d'un
  génome de référence. Cette étape peut ainsi être comparée à la
  reconstruction d'un puzzle dans lequel chaque \emph{reads} peut-être
  assimilé à l'une des pièces tandis que le génome de référence serait ici
  le modèle (\textbf{Figure : }\ref{fig:picdnamapping}).
  
  L'ensemble de nos exomes ayant été réalisés en \emph{paired-end}, les
  deux extrémités de chaque fragment sont séquencées. Chaque \emph{end}
  d'un même \emph{read} peut donc être considérée comme un \emph{read} à
  part entière qui sont alignées \textbf{indépendamment} le long du génome
  de référence. L'information fournit par le \emph{paired-end} n'étant
  utilisé qu'à \emph{posteriori} en tant que critère qualité. Au sein de
  notre pipeline, cette étape est effectué par le logiciel MAGIC (Su et
  al., \protect\hyperlink{ref-Su2014}{2014}) qui dans le cadre de nos
  études, s'est basé sur la version hg19 / GHRC37 du génome de référence.
  Suite à cet alignement, plusieurs critères sont observés afin de filtrer
  les \emph{reads} présentant une faible qualité d'alignement.
  
  Ainsi, le premier de ces filtres consiste à tout d'abord filtrer
  l'ensemble des \emph{reads} dupliqués, c'est à dire les \emph{reads}
  ayant des séquences parfaitement identiques, ceux-ci étant souvent le
  résultats d'un excès d'amplification au moment des PCRs effectuées en
  amont. De la même manière, afin d'éviter toute ambiguité au momen de
  l'interprétation des résultats, l'ensemble des \emph{reads} s'étant
  alignés sur plusieurs région du génome sont aussi filtrés. Une fois cela
  fait, nous vérifions la ``compatibilité'' des deux \emph{ends} composant
  chacun des \emph{reads} restant. Un \emph{reads} est dit compatible
  lorsque les deux \emph{ends} qui le composent s'alignent face à face
  (une sur le brin sens du génome de référence et l'autre sur le brin
  anti-sens) et couvrent une zone ne faisant pas plus de 3 fois la taille
  médiane de l'insert. Les \emph{reads} dont les deux \emph{ends} se sont
  alignées mais ne remplissant pas ces conditions seront dit ``Non
  compatible'', ceux dont une seule des deux \emph{ends} s'est alignés
  seront appelés ``orphelins'' et enfin ceux pour lesquels aucune des deux
  \emph{ends} ne se sont allignées sont appellés ``non-aligné''.
  L'ensemble des \emph{reads} ``non-compatible'', ``orphelins'' et
  ``non-alignés'' sont, en raison de leur faible qualité, filtré et donc
  non considérés pour les analyses en aval. Les \emph{reads} ayant passé
  l'ensemble des critères qualité mentionnés précédemment seront, eux,
  utilisés pour effectuer l'appel des variants.
  
  \newpage
  
  \begin{figure}
  
  {\centering \includegraphics[scale=0.35]{figure/dna_mapping} 
  
  }
  
  \caption{Représentation schématique de l'alignement de *reads paired-end* : **A** : Représentation du génome de référence ainsi que de *reads paired end* avant l'étape d'alignement. Les *reads paired-end* sont composés d'une extrémité *forward* (en vert) complémentaire du brin sens du génome de référence et d'une extrémité réverse (en jaune), complémentaire du brin anti-sens du génome de référence. Chacune de ces extrémités est séparées par un insert de taille connue mais de séquence inconnu. **B** : Après l'étape d'alignement, chaque *read* est repositioné sur la région du génome avec laquelle il présente la plus grande homologie de séquence. Le nombre de *reads* différents recouvrant une même position du génome de référence est appellée couverture}\label{fig:picdnamapping}
  \end{figure}
  
  \newpage
  
  \subsection{L'appel des variants}\label{lappel-des-variants}
  
  Si l'allignement des séquences peut être comparé à la reconstruction
  d'un puzzle, l'appel des variant pourrait lui être vu comme un jeu des 7
  erreurs, au cours duquel, pour chaque position couverte, les différence
  entre la séquence de l'individu séquencé et le génome de référence
  seront listés et appelé variants. Comme nous l'avons vu plus
  \protect\hyperlink{varcall}{tôt}, il est fortement conseillé d'effectuer
  l'appel des variants en tenant compte de l'aligneur choisi (Nielsen,
  Paul, Albrechtsen, \& Song, \protect\hyperlink{ref-Nielsen2011}{2011},
  M. A. DePristo et al. (\protect\hyperlink{ref-DePristo2011}{2011}),
  Lunter \& Goodson (\protect\hyperlink{ref-Lunter2011}{2011})). C'est
  pourquoi, nous avons développé notre propre algorithme d'appel des
  variants spécialement conçu pour l'analyse des données de MAGIC. Ainsi,
  l'appel des variants sera directement basé sur quatre comptages
  (R\(_+\), R\(_-\), V\(_+\) et V\(_-\)) fourni directement par MAGIC pour
  chaque position suffisement couvertes :
  
  \begin{enumerate}
  \def\labelenumi{\arabic{enumi}.}
  \tightlist
  \item
    \textbf{R}\(_+\) \textbf{et R}\(_-\) : Ces deux comptages
    correspondent au nombre de \emph{reads} \emph{forward} (+) et
    \emph{reverse} (-) sur lesquels est observé l'allèle de
    \textbf{référence} (R) à une position donnée.\\
  \item
    \textbf{V}\(_+\) \textbf{et V}\(_-\) : À l'inverse de R\(_+\) et
    R\(_-\), ces comptages correspondent au nombre de \emph{reads}
    \emph{forward} et \emph{reverse} sur lesquels est observé un allèle de
    \textbf{variant} (V) à une position donnée.
  \end{enumerate}
  
  Ainsi, les sommes : \(R_+ + V_+\) et \(R_- + V_-\) indiqueront
  respectivement la couverture d'une position en ne tenant que des
  \emph{reads forward} et \emph{reverse}. En fonction de ces couverturs
  nos appels seront classés en trois catégories :
  
  \begin{enumerate}
  \def\labelenumi{\arabic{enumi}.}
  \tightlist
  \item
    \textbf{Les appels \emph{double strand} (DS) :} Qualifie les positions
    ayant une couverture \(\ge\) 10 sur \textbf{les deux} strands. Ces
    appels sont ceux sont ceux ayant la meilleure qualité.\\
  \item
    \textbf{Les appels \emph{single strand} (SS) :} Ces appels définissent
    les positions pour lesquels \textbf{un des deux} \emph{strands}
    présentent une couverture \(\le\) 10. Dans ce cas, ce \emph{strand}
    est ignoré et l'appel est effectué uniquement en utilisant le second
    \emph{strand}.\\
  \item
    \textbf{Les appels \emph{non strand} (NS) :} Les positions NS sont
    celles pour lesquelles la couverture est \(\le\) 10 sur \textbf{les
    deux} strands. Aucun appel n'est effectué à ces positions qui
    \textbf{ne sont pas conservés dans la suite des analyses}.
  \end{enumerate}
  
  Ensuite, chaque position couverte, des appels indépendants seront
  effectués pour chaque \emph{strand} de telle sorte que, pour chaqune de
  ces position si :
  
  \begin{enumerate}
  \def\labelenumi{\arabic{enumi}.}
  \tightlist
  \item
    0 à 20\% des \emph{reads} portent un variant, la position est appelée
    \textbf{homozygote référence}.\\
  \item
    40 à 75\% des \emph{reads} portent un variant, la position est appelée
    \textbf{hétérozygote}.\\
  \item
    85 à 100\% des \emph{reads} portent un variant, la position est
    appelée \textbf{homozygote variant}.\\
  \item
    20 à 40\% des \emph{reads} portent un variant, l'appel sera considéré
    comme \textbf{ambigu bas}.\\
  \item
    75 à 85\% des \emph{reads} portent un variant, l'appel sera considéré
    comme \textbf{ambigu haut}.
  \end{enumerate}
  
  Pour les positions SS, l'appel final sera directement \ldots{}. Pour les
  positions DS, la concordance des appels fournis par chaque \emph{end}
  est verifié. Ainsi, un variant sera considéré :
  
  \begin{enumerate}
  \def\labelenumi{\arabic{enumi}.}
  \tightlist
  \item
    \textbf{Homozygote référence} si les deux appels sont homozygote
    référence, ou, un des appels est homozygote référence et l'autre
    ambigu bas.
  \item
    \textbf{Hétérozygote} si les deux appels sont hétérozygotes, ou, si
    l'un des appels est hétérozygote et l'autre ambigu bas ou haut.\\
  \item
    \textbf{Homozygote variant} si les deux appels sont homozygote
    variant, ou, un des appels est homozygote variant et l'autre ambigu
    haut
  \item
    \textbf{Ambigu} si les deux appels sont ambigu bas ou si ils sont tous
    les deux ambigu haut.\\
  \item
    \textbf{Discordant} pour toutes les combinaisons restantes.
  \end{enumerate}
  
  Dans le cadre de nos analyses, les appels ambigu et discordant sont
  filtrés.
  
  \newpage
  
  \subsection{L'annotation}\label{lannotation}
  
  Chaque variant retenu sera ensuite annoté tout d'abord par le logiciel
  \emph{variant effect predictor} (VEP) (W. McLaren et al.,
  \protect\hyperlink{ref-McLaren2016}{2016}) qui nous indiquera pour
  chaque variant la conséquence que celui-ci aura sur la séquence codante
  de l'ensemble des transcrits Ensembl qu'il chevauche (\textbf{Figure :
  }\ref{fig:figvepcsq}) (\textbf{Table : }\ref{tab:tabvepcsq}). Ensuite,
  nous ajoutons, pour chaque gène, son expression tissulaire en nous
  basant sur les données Ensembl (Aken et al.,
  \protect\hyperlink{ref-Aken2017}{2017}) générées par le projet Illumina
  BodyMap qui recense les données RNAseq des gènes humains pour 16 tissus
  différents. Suite à cela nous ajoutons, lorsque celle-ci est disponible,
  la fréquence du variant dans les bases de données ExAC (Lek et al.,
  \protect\hyperlink{ref-Lek2016}{2016}), ESP600
  (\href{http://evs.gs.washington.edu/EVS/}{Exome Variant Server, NHLBI GO
  Exome Sequencing Project (ESP), Seattle, WA}) et 1000Genomes (1000
  Genomes Project Consortium et al.,
  \protect\hyperlink{ref-1000GenomesProjectConsortium2015}{2015}) donnant
  ainsi une estimation de sa fréquence dans la population générale. De
  même, la particularité de ce pipeline est qu'elle conserve l'ensemble
  des variants identifiés dans les études effectuées précédemment
  permettant d'ajouter aux annotations la fréquence d'un variant chez les
  individus déjà séquencé et donc la fréquence d'un variant dans chaque
  phénotype étudié créant ainsi une base de données interne qui pourra
  servir de contrôle dans les études ultérieur.
  
  \begin{figure}
  
  {\centering \includegraphics[scale=.9]{figure/vep_csq} 
  
  }
  
  \caption[Listes des différentes conséquences prédites par VEP et leur positionnement sur le transcrit]{Listes des différentes conséquences prédites par VEP et leur positionnement sur le transcrit d'après [VEP site](http://www.ensembl.org/info/genome/variation/consequences.jpg)}\label{fig:figvepcsq}
  \end{figure}
  
  \newpage
  
  \blandscape
  
  \begin{longtable}[]{@{}lll@{}}
  \caption{\label{tab:tabvepcsq} Liste simplifiée des conséquences prédites
  par VEP avec leur description et impact associée}\tabularnewline
  \toprule
  \begin{minipage}[b]{0.18\columnwidth}\raggedright\strut
  VEP consequence\strut
  \end{minipage} & \begin{minipage}[b]{0.11\columnwidth}\raggedright\strut
  VEP impact\strut
  \end{minipage} & \begin{minipage}[b]{0.63\columnwidth}\raggedright\strut
  Description\strut
  \end{minipage}\tabularnewline
  \midrule
  \endfirsthead
  \toprule
  \begin{minipage}[b]{0.18\columnwidth}\raggedright\strut
  VEP consequence\strut
  \end{minipage} & \begin{minipage}[b]{0.11\columnwidth}\raggedright\strut
  VEP impact\strut
  \end{minipage} & \begin{minipage}[b]{0.63\columnwidth}\raggedright\strut
  Description\strut
  \end{minipage}\tabularnewline
  \midrule
  \endhead
  \begin{minipage}[t]{0.18\columnwidth}\raggedright\strut
  Splice acceptor / donor\strut
  \end{minipage} & \begin{minipage}[t]{0.11\columnwidth}\raggedright\strut
  HIGH\strut
  \end{minipage} & \begin{minipage}[t]{0.63\columnwidth}\raggedright\strut
  A splice variant that changes the 2 base region at the 3' / 5' end of an
  intron\strut
  \end{minipage}\tabularnewline
  \begin{minipage}[t]{0.18\columnwidth}\raggedright\strut
  Stop gained\strut
  \end{minipage} & \begin{minipage}[t]{0.11\columnwidth}\raggedright\strut
  HIGH\strut
  \end{minipage} & \begin{minipage}[t]{0.63\columnwidth}\raggedright\strut
  A sequence variant whereby at least one base of a codon is changed,
  resulting in a premature stop codon, leading to a shortened
  transcript\strut
  \end{minipage}\tabularnewline
  \begin{minipage}[t]{0.18\columnwidth}\raggedright\strut
  Frameshift\strut
  \end{minipage} & \begin{minipage}[t]{0.11\columnwidth}\raggedright\strut
  HIGH\strut
  \end{minipage} & \begin{minipage}[t]{0.63\columnwidth}\raggedright\strut
  A sequence variant which causes a disruption of the translational
  reading frame, because the number of nucleotides inserted or deleted is
  not a multiple of three\strut
  \end{minipage}\tabularnewline
  \begin{minipage}[t]{0.18\columnwidth}\raggedright\strut
  Stop lost\strut
  \end{minipage} & \begin{minipage}[t]{0.11\columnwidth}\raggedright\strut
  HIGH\strut
  \end{minipage} & \begin{minipage}[t]{0.63\columnwidth}\raggedright\strut
  A sequence variant where at least one base of the terminator codon
  (stop) is changed, resulting in an elongated transcript\strut
  \end{minipage}\tabularnewline
  \begin{minipage}[t]{0.18\columnwidth}\raggedright\strut
  Start lost\strut
  \end{minipage} & \begin{minipage}[t]{0.11\columnwidth}\raggedright\strut
  HIGH\strut
  \end{minipage} & \begin{minipage}[t]{0.63\columnwidth}\raggedright\strut
  A codon variant that changes at least one base of the canonical start
  codo\strut
  \end{minipage}\tabularnewline
  \begin{minipage}[t]{0.18\columnwidth}\raggedright\strut
  Inframe insertion / deletion\strut
  \end{minipage} & \begin{minipage}[t]{0.11\columnwidth}\raggedright\strut
  MODERATE\strut
  \end{minipage} & \begin{minipage}[t]{0.63\columnwidth}\raggedright\strut
  An inframe non synonymous variant that inserts / deletes bases into in
  the coding sequenc\strut
  \end{minipage}\tabularnewline
  \begin{minipage}[t]{0.18\columnwidth}\raggedright\strut
  Missense\strut
  \end{minipage} & \begin{minipage}[t]{0.11\columnwidth}\raggedright\strut
  MODERATE\strut
  \end{minipage} & \begin{minipage}[t]{0.63\columnwidth}\raggedright\strut
  A sequence variant, that changes one or more bases, resulting in a
  different amino acid sequence but where the length is preserved\strut
  \end{minipage}\tabularnewline
  \begin{minipage}[t]{0.18\columnwidth}\raggedright\strut
  Splice region\strut
  \end{minipage} & \begin{minipage}[t]{0.11\columnwidth}\raggedright\strut
  LOW\strut
  \end{minipage} & \begin{minipage}[t]{0.63\columnwidth}\raggedright\strut
  A sequence variant in which a change has occurred within the region of
  the splice site, either within 1-3 bases of the exon or 3-8 bases of the
  intron\strut
  \end{minipage}\tabularnewline
  \begin{minipage}[t]{0.18\columnwidth}\raggedright\strut
  Stop retained\strut
  \end{minipage} & \begin{minipage}[t]{0.11\columnwidth}\raggedright\strut
  LOW\strut
  \end{minipage} & \begin{minipage}[t]{0.63\columnwidth}\raggedright\strut
  A sequence variant where at least one base in the terminator codon is
  changed, but the terminator remains\strut
  \end{minipage}\tabularnewline
  \begin{minipage}[t]{0.18\columnwidth}\raggedright\strut
  Synonymous\strut
  \end{minipage} & \begin{minipage}[t]{0.11\columnwidth}\raggedright\strut
  LOW\strut
  \end{minipage} & \begin{minipage}[t]{0.63\columnwidth}\raggedright\strut
  A sequence variant where there is no resulting change to the encoded
  amino acid\strut
  \end{minipage}\tabularnewline
  \begin{minipage}[t]{0.18\columnwidth}\raggedright\strut
  5 / 3 prime UTR\strut
  \end{minipage} & \begin{minipage}[t]{0.11\columnwidth}\raggedright\strut
  MODIFIER\strut
  \end{minipage} & \begin{minipage}[t]{0.63\columnwidth}\raggedright\strut
  A UTR variant of the 5' / 3' UTR\strut
  \end{minipage}\tabularnewline
  \begin{minipage}[t]{0.18\columnwidth}\raggedright\strut
  Intron\strut
  \end{minipage} & \begin{minipage}[t]{0.11\columnwidth}\raggedright\strut
  MODIFIER\strut
  \end{minipage} & \begin{minipage}[t]{0.63\columnwidth}\raggedright\strut
  A transcript variant occurring within an intron\strut
  \end{minipage}\tabularnewline
  \begin{minipage}[t]{0.18\columnwidth}\raggedright\strut
  NMD transcript\strut
  \end{minipage} & \begin{minipage}[t]{0.11\columnwidth}\raggedright\strut
  MODIFIER\strut
  \end{minipage} & \begin{minipage}[t]{0.63\columnwidth}\raggedright\strut
  A variant in a transcript that is the target of NMD\strut
  \end{minipage}\tabularnewline
  \begin{minipage}[t]{0.18\columnwidth}\raggedright\strut
  Non coding transcript\strut
  \end{minipage} & \begin{minipage}[t]{0.11\columnwidth}\raggedright\strut
  MODIFIER\strut
  \end{minipage} & \begin{minipage}[t]{0.63\columnwidth}\raggedright\strut
  A transcript variant of a non coding RNA gene\strut
  \end{minipage}\tabularnewline
  \bottomrule
  \end{longtable}
  
  \elandscape
  \newpage
  
  \subsection{Le filtrage des variants}\label{le-filtrage-des-variants}
  
  L'étape de filtrage est extrêmement importante si l'on souhaite analyser
  de manière efficace les données provenant de WES. C'est pourquoi elle
  occupe une place importante dans notre pipeline. L'intégralité des
  paramètres de cette étape peuvent être modifiés par l'utilisateur de
  sorte à faire correspondre les critères de filtre aux besoins de
  l'étude. Afin de rendre son utilisation le plus efficace possible, nous
  avons souhaité définir des paramètres par défauts pertinent dans la
  plupart des études de séquençage exomique de sorte que à moins que le
  contraire ne soit spécifié les filtres suivant seront appliqués :
  
  \begin{enumerate}
  \def\labelenumi{\arabic{enumi}.}
  \item
    \textbf{Filtre 1 : L'union des variants :} Dans le cas ou des
    individus présentant un lien de parenté et présentant le même
    phénotype sont analysé, seuls les variants observés chez l'ensemble
    des individus sont conservés.
  \item
    \textbf{Filtre 2 : Génotype des variants :} Ce pipeline d'analyse a
    avant tout été développé pour la recherche de variant impliqué dans
    des pathologies à transmission récessives. Ainsi, à moins que le
    contraire ne soit spécifié, l'ensemble des variants hétérozygotes sont
    systématiquement filtrés. .
  \item
    \textbf{Filtre 4 : Les transcrits ``non pertinents'' :} Au cours de
    nos analyses nous nous sommes concentré uniquement sur les transcrits
    codant pour une protéine. Ainsi, l'ensemble des transcrits annotés
    comme étant non codant furent filtrés. De même pour les transcrits
    annotés comme étant NMD (\emph{nonsense-mediated decay}). En effet, ce
    mécanisme a pour but de contrôler la qualité des ARNm cellulaires chez
    les eucaryotes (Y.-F. Chang, Imam, \& Wilkinson,
    \protect\hyperlink{ref-Chang2007}{2007}) en éliminant les ARNm qui
    comportent un codon stop prématuré (K. E. Baker \& Parker,
    \protect\hyperlink{ref-Baker2004}{2004}), pouvant être le résultat
    d'une erreur de transcription, d'une mutation ou encore d'une erreur
    d'épissage. Il est donc peu probable que les variants présents sur des
    transcrits annotés NMD soient responsables du phénotype. Dès lors, ces
    transcrits ont été également filtrés. Ainsi, l'ensemble des variants
    impactant \textbf{uniquement} des transcrits non codant et / ou annoté
    NMD sont filtrés.
  \item
    \textbf{Filtre 3 : Impact du variant :} Afin de ne conserver que les
    variants ayant le plus probablement délétère sur la protéine, seuls
    sont conservés ceux impactant la séquence codante d'un transcrit. De
    plus les variants synonymes ne sont pas conservés (exceptés ceux se
    trouvant proches des régions d'épissage) car ceux-ci n'ont aucun effet
    sur la séquence protéique. Pour les variants faux sens (changement
    d'un seul acide-aminé de la séquence protéique) il est plus difficile
    de se trancher, dès lors, seuls ceux étant prédit comme
    \emph{tolerated} par SIFT (Kumar, Henikoff, \& Ng,
    \protect\hyperlink{ref-Kumar2009}{2009}) \textbf{et} comme
    \emph{benign} par Polyphen (Adzhubei et al.,
    \protect\hyperlink{ref-Adzhubei2010}{2010}) sont filtrés.
  \item
    \textbf{Fréquence des variants :} La fréquence d'un variant dans la
    population générale est un moyen rapide d'avoir une prédiction fiable
    de l'effet délétère ou non de celui-ci. En effet, il est peu probable
    qu'un variant retrouvé fréquemment dans la population générale soit
    causal d'une pathologie sévère. C'est pourquoi, l'ensemble des
    variants ayant une fréquence \(\ge\) 1\% dans l'une des trois bases de
    données que sont ExAC, ESP et 1KG sont filtrés.
  \item
    \textbf{Présence des variants dans la cohorte contrôle :} Au sien de
    notre pipeline, les données de l'ensemble des patients analysés dans
    les études anterieures sont conservés créant ainsi une base de donnée
    interne de variants. Dès lors, il devient possible d'utiliser chacun
    de ces patient comme contrôle lorsqu'ils ne sont pas porteur du même
    phénotype que celui des patients des études ulterieures. Ce filtre se
    révèle particulièrement intéréssant lorsque plusieurs patients
    porteurs de phénotypes différents ont subi le même protocole de
    séquençage ainsi l'ensemble des variants faux-positifs résultant
    d'artéfact liés au différentes étapes en amont de l'analyse
    bioinformatique pourront alors être filtré. De même ce filtre permet
    de mettre en évidence les variants propores à une population lorsque
    des patients provenant de la même région géographique et ne présentant
    toujours pas le même phénotype sont comparés.
  \end{enumerate}
  
  \newpage
  
  \subsection{La priorisation des gènes}\label{la-priorisation-des-genes}
  
  Malgré l'ensemble des filtres appliqué, il est possible que d\ldots{}..
  
  \newpage
  
  \section{Résultats 1 : Analyse de 3 cas
  familiaux}\label{resultats-1-analyse-de-3-cas-familiaux}
  
  \subsection{Résultats des différents étapes de
  l'analyse}\label{resultats-des-differents-etapes-de-lanalyse}
  
  \newpage
  
  \subsection{Article n° 1}\label{article-n-1}
  
  \textbf{SPINK2 deficiency causes infertility by inducing sperm defects
  in heterozygotes and azoospermia in homozygotes}
  
  Kherraf ZE\textsuperscript{*}, Christou-Kent M\textsuperscript{*},
  \textbf{Karaouzène T}, Amiri-Yekta A, Martinez G, Vargas AS, Lambert E,
  Borel C, Dorphin B, Aknin-Seifer I, Mitchell MJ, Metzler-Guillemain C,
  Escoffier J, Nef S, Grepillat M, Thierry-Mieg N, Satre V, Bailly M,
  Boitrelle F, Pernet-Gallay K, Hennebicq S, Fauré J, Bottari SP, Coutton
  C, Ray PF, Arnoult C
  
  \textsuperscript{*} Co-premiers auteurs
  
  EMBO Molecular Medicine, Mai 2017
  
  \newpage
  
  \subsubsection{Contexte et objectifs}\label{contexte-et-objectifs}
  
  \newpage
  
  \includepdf[pages=-]{bib/SPINK2_2017.pdf}
  
  \newpage
  
  \subsubsection{Principaux résultats}\label{principaux-resultats}
  
  \subsubsection{Discussion et
  Perspectives}\label{discussion-et-perspectives}
  
  \subsection{Article n° 2}\label{article-n-2}
  
  \subsubsection{Homozygous mutation of PLCZ1 leads to defective human
  oocyte activation and infertility that is not rescued by the WW-binding
  protein
  PAWP}\label{homozygous-mutation-of-plcz1-leads-to-defective-human-oocyte-activation-and-infertility-that-is-not-rescued-by-the-ww-binding-protein-pawp}
  
  Jessica Escoffier J\textsuperscript{*}, Lee HC\textsuperscript{*},
  Yassine S\textsuperscript{*}, Zouari R, Martinez G, \textbf{Karaouzène
  T}, Coutton C, Kherraf ZE, Halouani L, Triki C, Nef S, Thierry-Mieg N,
  Savinov SN, Fissore R, Ray PF, Arnoult C
  
  Human Molecular Genetics, Décembre 2015
  
  \newpage
  
  \subsubsection{Contexte et objectifs}\label{contexte-et-objectifs-1}
  
  \newpage
  
  \includepdf[pages=-]{bib/PLCZ1_2016}
  
  \newpage
  
  \subsubsection{Principaux résultats}\label{principaux-resultats-1}
  
  \subsubsection{Discussion et
  Perspectives}\label{discussion-et-perspectives-1}
  
  \subsection{Article n° 3}\label{article-n-3}
  
  \subsubsection{Whole-exome sequencing of familial cases of multiple
  morphological abnormalities of the sperm flagella (MMAF) reveals new
  DNAH1
  mutations}\label{whole-exome-sequencing-of-familial-cases-of-multiple-morphological-abnormalities-of-the-sperm-flagella-mmaf-reveals-new-dnah1-mutations}
  
  Amiri-Yekta A\textsuperscript{*}, Coutton C\textsuperscript{*}, Kherraf
  ZE, \textbf{Karaouzène T}, Le Tanno P, Sanati MH, Sabbaghian M, Almadani
  N, Sadighi Gilani MA, Seyedeh Hanieh Hosseini, Bahrami S, Daneshipour A,
  Bini M, Arnoult C, Colombo R, Gourabi H, Ray PF
  
  \textsuperscript{*} Co-premiers auteurs
  
  Human Reproduction, Octobre 2016
  
  \newpage
  
  \subsubsection{Contexte et objectifs}\label{contexte-et-objectifs-2}
  
  \newpage
  
  \includepdf[pages=-]{bib/Fam_DNAH1_2016.pdf}
  
  \newpage
  
  \subsubsection{Principaux résultats}\label{principaux-resultats-2}
  
  \subsubsection{Discussion et
  Perspectives}\label{discussion-et-perspectives-2}
  
  \newpage
  
  Dans cette partie, nous allons, après avoir décrit notre pipeline,
  détailler les résultats de l'analyse des données de WES de 75 patients
  tous atteints d'un phénotype d'infertilité. Ces études seront séparées
  en deux parties distinctes, la première se concentrera sur l'études de 6
  familles incluant 13 de ces patients. Le seconde portera sur l'analyse
  des 62 patients non-apparentés restant présentant tous un phénotype
  MMAF.
  
  Après avoir été séquencés, les données recueillies pour ces patients
  sont procéssées au sein du même pipeline d'analyse qui comprend quatre
  étapes allant de l'alignement des \emph{reads} au filtrage des variants
  :
  
  \subsection{Utilisation du pipeline dans des cas familiaux
  :}\label{utilisation-du-pipeline-dans-des-cas-familiaux}
  
  Dans cette partie, je me concentre sur l'analyse bioinformatique des
  résultats des séquençages exomiques effectués entre 2012 et 2014 de 13
  individus infertiles provenant de 6 familles différentes. Parmi
  celles-ci, 3 phénotypes différents ont été observés :
  
  \begin{enumerate}
  \def\labelenumi{\arabic{enumi}.}
  \tightlist
  \item
    \textbf{\protect\hyperlink{infquant}{L'Azoospermie} :} Comme nous
    avons pu le voir, l'azoospermie est un phénotype d'infertilité
    masculine caractérisé par l'absence de spermatozoïde dans l'éjaculât\\
  \item
    \textbf{Échec de fécondation :} Ce phénotype d'infertilité se
    caractérise par l'incapacité des spermatozoïdes à féconder
    l'ovocyte.\\
  \item
    \textbf{MMAF} : Le syndrome MMAF (\emph{multiple morphological
    abnormalities of the sperm flagella}) caractérise comme son nom
    l'indique les patients présentant une majorité de spermatozoïdes
    atteins par une mosaïque d'anomalie morphologique du flagelle.
  \end{enumerate}
  
  Parmi ces 6 chacune composée de 2 à 3 frères, les familles AZ, FF et
  MMAF2 présentent un historique de consanguinité, les parents étant soit
  cousins germains, soit cousins au second degré. La consanguinité
  favorisant la transmission de variants à l'état homozygote, nous avons
  décidé, dans un premier temps de concentrer nos analyses uniquement sur
  les variants (SNVs et indels) homozygotes pour l'ensemble des familles.
  Pour les 3 familles n'ayant pas d'historique de consanguinité, ce choix
  nous permet de réduire la liste des variants candidats de sorte à
  faciliter les analyses. L'études des variants hétérozygotes sera
  effectuée \emph{a posteriori} pour les familles dont la cause génétique
  du phénotype n'a pas pu être identifiée en se limitant aux variants
  homozygotes. Un récapitulatif des familles et de leur phénotype est
  disponible dans la table \ref{tab:tabrecapfam}.
  
  \newpage
  
  \begin{landscape}
  \begin{longtable}[t]{llllrl}
  \caption{\label{tab:tabrecapfam}Tableau récapitulatif des familles séquencées et de leur phénotype}\\
  \toprule
  Family & Consanguinity & Individuals & Phenotype & Year & Place\\
  \midrule
  AZ & Yes & AZ1, AZ2 & Azoospermia & 2012 & Mount Sinai Institut\\
  FF & Yes & FF1, FF2 & Fertilization failure & 2014 & Genoscope (Evry)\\
  MMAF1 & No & MMAF1.1, MMAF1.2 & MMAF & 2014 & Genoscope (Evry)\\
  MMAF2 & Yes & MMAF2.1, MMAF2.2 & MMAF & 2014 & Genoscope (Evry)\\
  MMAF3 & No & MMAF3.1, MMAF3.2 & MMAF & 2014 & Genoscope (Evry)\\
  MMAF4 & No & MMAF4.1, MMAF4.2, MMAF4.3 & MMAF & 2014 & Genoscope (Evry)\\
  \bottomrule
  \end{longtable}
  \end{landscape}
  
  \newpage  
  
  \subsubsection{Résultats des exomes}\label{resultats-des-exomes}
  
  \paragraph{Résultat de l'alignement}\label{resultat-de-lalignement}
  
  Pour rappel, l'\href{\%7B\#lalignement\%7D}{alignement} consiste à
  repositionner l'ensemble des \emph{reads} générés au cours de l'étape de
  séquençage le long d'un génome de référence.
  
  La quantité de \emph{reads} composant les exomes de chaque individu peut
  varier en fonction de plusieurs paramètres et n'est donc pas égale pour
  chaque patient bien que l'ordre de grandeur reste le même avec une
  médiane de 91438630 \emph{reads}. Seuls les deux frères AZ1 et AZ2 se
  distinguent avec près de 3 fois plus de \emph{reads} que les autres
  patients. Cette différence peut être expliquée car ces deux patients
  sont les deux seuls à voir été séquencés au Mount Sinaï Institut or leur
  protocole d'amplification précédent le séquençage contient un nombre de
  cycles de PCR supérieur à ceux appliqués au Génopole d'Évry où ont été
  séquencés les autres patients. Il faut noter que ce nombre plus
  important de \emph{reads} n'est en rien le reflet d'une meilleure
  qualité. En effet, celui-ci est causé par une grande quantité de
  \emph{reads} dupliqués qui seront pour la plupart filtrés au cours des
  analyses ultérieures (\textbf{Table :} \ref{tab:tabrecapfam},
  \textbf{Figure : }\ref{fig:readsselection} - \textbf{A}).
  
  La première étape du contrôle qualité des \emph{reads} consiste à
  filtrer les \emph{reads} ne s'étant pas alignés sur le génome. Ces
  \emph{reads} sont extrêmement minoritaires puisqu'ils ne représentent
  qu'entre 1.2 et 5.5 \% des \emph{reads} de nos individus (\textbf{Figure
  : }\ref{fig:readsselection} - \textbf{B}).
  
  Dans nos analyses, seuls les \emph{reads} compatibles sont conservés,
  c'est à dire environs 89.5 \% des \emph{reads} s'étant correctement
  alignés. (\textbf{Figure : }\ref{fig:readsselection} - \textbf{C}).
  
  La dernière étape de ce contrôle-qualité consiste à analyser le nombre
  de site auxquels se sont alignés les \emph{reads}. En effet, certaine
  zone du génome étant dupliqué, l'une des problématiques des
  \emph{short-reads} est qu'il est possible que ceux-ci s'alignent à
  plusieurs régions différentes du génome. Afin d'éviter toute ambiguïté,
  seul ceux s'étant aligné sur un site unique sont conservés pour la suite
  des analyses. Ces \emph{reads} représente entre 92.3 et 96.9 \% des
  \emph{reads} ayant passé les précédents filtres (\textbf{Figure :
  }\ref{fig:readsselection} - \textbf{C}).
  
  \newpage
  
  \begin{figure}
  
  {\centering \includegraphics{thesis_files/figure-latex/readsselection-1} 
  
  }
  
  \caption[Processus simplifié du contrôle qualité des *reads*]{Processus simplifié du contrôle qualité des *reads* : Pour chacun des graphiques, les *reads* représentés en vert sont conservés tandis que ceux en rouge sont filtrés. **A** : Quantité de *reads* bruts générés pour chaque patient au cours de l'étape de séquençage. La médiane des *reads* est représentée en bleue. **B** : Pourcentage pour chaque individu de *reads* s'étant aligné correctement et ne s'étant pas alignés sur le génome de référence. **C** : Distribution pour chaque patient des *reads* compatibles (Comp), non compatibles (Non comp) et orphelins (Orphans). **D** : Présentation pour chaque *reads* du nombre de site auxquels ils s'alignent}\label{fig:readsselection}
  \end{figure}
  
  \newpage
  
  \paragraph{Résultat de l'appel des
  variants}\label{resultat-de-lappel-des-variants}
  
  Dans nos données, les appels SS sont majoritaires et représentent
  environ 48.1 \% de nos appels (contre 35.6 \% d'appels DS). Au vus de
  l'importance de ces appels, nous avons fait le choix de les conserver
  afin de ne pas filtrer une quantité trop importante de données. Ces
  appels seront cependant considérés comme étant de faible qualité, de
  fait, leurs analyses et interprétation seront plus précautionneuses. En
  revanche, au vus de la trop grande incertitude de l'appel des variants
  NS, ceux-ci sont systématiquement filtrés éliminant ainsi entre 10.3 et
  18.7 \% des positions appelées pour chaque patient (\textbf{Figure :
  }\ref{fig:plotvarcall} - \textbf{A}).
  
  Les appels discordant et ambigus sont filtrés, soit environ 86.3 \% des
  variants DS.
  
  Il est intéressant de noter que bien que les variants \emph{single
  strand} (SS) soient conservés, on peut s'attendre à ce qu'environ 13.7
  \% de ceux-ci soient aberrants, ceux-ci n'ayant pu subir le même
  contrôle que les SS (\textbf{Figure : }\ref{fig:plotvarcall} -
  \textbf{B}).
  
  Pour l'ensemble des variants ayant passé les filtres énoncés ci-dessus,
  c'est à dire les variants SS et les variants DS avec appels concordants,
  le génotype est déterminé en fonction du pourcentage de \emph{reads}
  portant le variant à cette position. Par exemple, si à une position
  donnée, 0\% des \emph{reads} portent un variant, l'individu sera appelé
  ``Homozygote référence'', si 50\% des \emph{reads} sont porteurs d'un
  variant, l'appel sera ``hétérozygote'' et si 100\% des \emph{reads}
  portent un variant, l'appel sera ``Homozygote variant''. Ainsi, pour
  chaque individu nous avons pu établir une liste de SNVs et d'indels avec
  leur génotype associé. Pour chacun de nos 13 patients les ordres de
  grandeur du nombre de variants appelés sont identique. Ainsi pour chaque
  patient nous avons appelés environ 43670 variants hétérozygotes (41044
  SNVs et 2626 indels) et 65040 variants homozygotes (32520 SNVs et 1809
  indels) (\textbf{Figure : }\ref{fig:plotvarcall} - \textbf{C}).
  
  \newpage
  
  \begin{figure}
  
  {\centering \includegraphics{thesis_files/figure-latex/plotvarcall-1} 
  
  }
  
  \caption[Contrôle qualité des variants appelés]{Contrôle qualité des variants appelés : Pour chacun des graphiques, les variants représentés en vert et en orange sont conservés tandis que ceux en rouge sont filtrés. **A** : Distribution du *stranding* des appels pour chaque patient. **B** : Comparaison des appels entre les deux *ends* des variants appelés DS. **C** : Distribution des SNVs et indels en fonction de leur génotype pour chaque patients (représentés par une barre}\label{fig:plotvarcall}
  \end{figure}
  
  \newpage
  
  \paragraph{Résultats de l'annotation}\label{resultats-de-lannotation}
  
  L'annotation des variants appelés consiste à ajouter un maximum
  d'informations sur les variants. Ces informations seront ensuite
  utilisées afin de filtrer et / ou prioriser les variants. Dans ces
  analyses nous avons utilisé le logiciel \emph{Variant Effect Predictor}
  (VEP) (W. McLaren et al., \protect\hyperlink{ref-McLaren2016}{2016}) qui
  va prédire l'effet qu'auront ces variants sur l'ensemble des transcrits
  (et gènes) qu'ils chevauchent. Dans le cas de substitution faux-sens,
  c'est à dire entrainant le changement d'un seul acide-aminé de la
  séquence protéique, nous utiliserons les prédictions fournies par SIFT
  et PolyPhen afin d'estimer leur pathogénicité. Pour finir nous ajoutons,
  lorsqu'elle est disponible, la fréquence de chacun de ces variants dans
  les bases de données ExAC, 1KG et ESP6500.
  
  Après avoir annoté nos variants, nous avons pu constater que pour chaque
  patient 24975 gènes sont en moyenne affecté par au moins un variant
  homozygote pour en moyenne 122735 transcrits (soit environ 5 transcrits
  par gènes). Il faut noter que parmi ces gènes se trouvent à la fois des
  gènes codant pour des protéine \textbf{et} d'autres non codant
  (\textbf{Figure : }\ref{fig:plotvarannotation} - \textbf{A}).
  
  Chaque variant affectera l'ensemble des transcrits qu'il chevauche,
  ainsi un même variant pourra impacter plusieurs transcrits. Ces impacts
  sont ensuite classés par VEP en quatre catégories qui sont, de la plus
  délétère à la moins délétère : \emph{HIGH}, \emph{MODERATE}, \emph{LOW},
  \emph{MODIFIER} (\textbf{Table :}\ref{tab:tabvepcsq}). Comme attendu,
  les variants ayant un impact tronquant se retrouvent être les moins
  fréquent chez chacun de nos patients. Ceci est d'autant plus flagrant
  pour l'impact \emph{HIGH} qui regroupe, entre autres, les variants
  créant un codon stop ou encore ceux causant un décalage du cadre de
  lecture (\textbf{Table :}\ref{tab:tabvepcsq}), se retrouvent en quantité
  extrêmement faible puisqu'ils ne représentent en moyenne que 0.15 \% des
  variants, soit une moyenne de 466 hétérozygotes et 370 homozygotes par
  patient) (\textbf{Figure : }\ref{fig:plotvarannotation} - \textbf{B}).
  
  \newpage
  
  \begin{figure}
  
  {\centering \includegraphics{thesis_files/figure-latex/plotvarannotation-1} 
  
  }
  
  \caption[Annotation des variants]{Annotation des variants : **A** : Quantification du nombre de gènes (en bleu) / transcrits (en rose) impactés par au moins un variant pour chaque patient chacun représentés par une barre. **B** : Distribution des impacts HIGH MODERATE LOW et MODIFIER en fonction des patients et du génotype du variant}\label{fig:plotvarannotation}
  \end{figure}
  
  \newpage
  
  \hypertarget{filterdescription}{\paragraph{Résultats du
  filtrage}\label{filterdescription}}
  
  Les étapes précédentes nous ont permis de mettre en évidence pour chaque
  patient une liste de variants passant l'ensemble de nos critères
  qualités. Ces variants ont dès lors pu être annotés nous permettant
  notamment d'avoir connaissance de leurs impacts sur les différents
  transcrits qu'ils chevauchent ou encore leur fréquence dans la
  population générale. Désormais, afin de ne conserver que les variants
  ayant la plus forte probabilité d'être responsable du phénotype de ces
  patients, nous avons appliqué successivement six filtres basés à la fois
  sur les différentes annotations que nous avons ajoutées mais aussi sur
  nos connaissances du mode de transmission du phénotype :
  
  \begin{enumerate}
  \def\labelenumi{\arabic{enumi}.}
  \tightlist
  \item
    \textbf{Filtre 1 : L'union des variants :} Dans ces différentes
    études, nous avons à chaque fois séquencé des frères (deux ou trois)
    présentant phénotype. Ainsi nous avons pu formuler l'hypothèse d'une
    cause génétique commune entre les différents patients d'une même
    famille et donc filtrer l'ensemble des variants qui ne sont pas
    partagés les deux ou trois frères atteints testés.\\
  \item
    \textbf{Filtre 2 : Génotype des variants :} Dans ces études, nous
    avons émis l'hypothèse d'une transmission récessive du phénotype.
    Ainsi, seuls les variants homozygotes ont été conservés.
    (\textbf{Figure : }\ref{fig:plotvarcall},
    \ref{fig:plotcomparefilter}).\\
  \item
    \textbf{Filtre 3 : Impact du variant :} Afin de ne conserver que les
    variants ayant un effet potentiellement délétère sur la protéine, nous
    avons filtré les variants intronique et ceux tombant dans les
    séquences UTRs. De même les variants synonymes ne sont pas conservés
    (exceptés ceux se trouvant proches des régions d'épissage) car ceux-ci
    n'ont aucun effet sur la séquence protéique. Pour les variants faux
    sens (changement d'un seul aa de la séquence protéique) il est plus
    difficile de se trancher, nous avons donc utilisé les logiciels SIFT
    (Kumar et al., \protect\hyperlink{ref-Kumar2009}{2009}) et Polyphen
    (Adzhubei et al., \protect\hyperlink{ref-Adzhubei2010}{2010}) et
    filtré l'ensemble des faux-sens prédits comme \emph{tolerated} par
    SIFT et \emph{benign} par Polyphen.\\
  \item
    \textbf{Filtre 4 : Les transcrits ``non pertinents'' :} Au cours de
    nos analyses nous nous sommes concentré uniquement sur les transcrits
    codant pour une protéine. Ainsi, l'ensemble des transcrits annotés
    comme étant non codant furent filtrés. De même le mécanisme NMD
    (\emph{nonsense-mediated decay}) a pour but de contrôler la qualité
    des ARNm cellulaires chez les eucaryotes (Y.-F. Chang et al.,
    \protect\hyperlink{ref-Chang2007}{2007}) en éliminant les ARNm qui
    comportent un codon stop prématuré (K. E. Baker \& Parker,
    \protect\hyperlink{ref-Baker2004}{2004}), pouvant être le résultat
    d'une erreur de transcription, d'une mutation ou encore d'une erreur
    d'épissage. Il est donc peu probable que les variants présents sur des
    transcrits annotés NMD soient responsables du phénotype. Dès lors, ces
    transcrits ont été également filtrés. Ainsi, nous avons pu retirer de
    nos listes de variants l'ensemble des mutations impactant
    \textbf{uniquement} des transcrits non codant et / ou annoté NMD.
    Cette étape de filtre permet à elle seule de filtrer systématiquement
    entre 13712 et 17992 transcrits différents par patients, soit une
    moyenne de 1834 variants par individus (\textbf{Figure :
    }\ref{fig:plotfilternonpertinanttr}).
  \end{enumerate}
  
  \begin{figure}
  
  {\centering \includegraphics{thesis_files/figure-latex/plotfilternonpertinanttr-1} 
  
  }
  
  \caption[Filtrage des transcrits jugés "non pertinents" et des variants les chevauchant]{Filtrage des transcrits jugés "non pertinents" et des variants les chevauchant : Pour chaque patient nous avons filtrer les transcrits jugés "non pertinents" pour l'analyse, c'est à dire ceux ne codant pas pour une protéine et ceux annoté NMD. Dès lors, l'intégralité des variants chevauchant uniquement des transcrits non pertinents ont pu systématiquement être filtrés (boites rouges). Les autres furent conservés (boites vertes)}\label{fig:plotfilternonpertinanttr}
  \end{figure}
  
  \begin{enumerate}
  \def\labelenumi{\arabic{enumi}.}
  \setcounter{enumi}{4}
  \item
    \textbf{Fréquence des variants :} La fréquence d'un variant dans la
    population générale est un moyen rapide d'avoir une prédiction fiable
    de l'effet délétère ou non de celui-ci. En effet, il est peu probable
    qu'un variant retrouvé fréquemment dans la population générale soit
    causal d'une pathologie sévère. Ainsi nous avons filtré pour
    l'ensemble de nos patients l'ensemble des variants ayant une fréquence
    \(\ge\) 0.01 dans l'une des trois bases de données que sont ExAC, ESP
    et 1KG.
  \item
    \textbf{Présence des variants dans la cohorte contrôle :} Au cours de
    nos différentes études, nous avons été amenés à séquencé un total de
    134 individus présentant un des 6 phénotypes que nous avons étudiés
    (\textbf{Table : }\ref{tab:TODO}). Ces phénotypes étant très
    différent, on peut émettre l'hypothèse que leurs causes génétiques
    soient également différentes. De même, les variants recherchés étant
    rares, il est peu probable qu'un individu porte les variants de deux
    phénotypes différents. Ainsi, pour chacune des 6 familles, nous avons
    pu constituer une cohorte contrôle composée dans l'ensemble des
    patients précédemment analysés et ne présentant pas le même phénotype
    que celui étudié dans la famille (\textbf{Figure :}
    \ref{fig:plotsamplectrl}). Dès lors, nous avons pu filtrer l'ensemble
    des variants retrouvés à la fois chez nos patients et observés à
    l'état homozygote dans la cohorte contrôle. Cette cohorte contrôle
    présente ainsi le même rôle que les bases de données publiques. Sont
    intérêt principale par rapport à celles-ci est que les individus qui
    la composent ont pour la plupart la même origine ethnico-géographique
    que nos patients. De plus ceux-ci ont été séquencés en même temps dans
    les mêmes centres permettant ainsi d'identifier les artefacts dus aux
    protocoles de séquençage.
  \end{enumerate}
  
  \begin{figure}
  
  {\centering \includegraphics{thesis_files/figure-latex/plotsamplectrl-1} 
  
  }
  
  \caption[Nombre d'individus composant la cohorte contrôle de chaque famille]{Nombre d'individus composant la cohorte contrôle de chaque famille : Ici, chaque barre représente une famille et sa hauteur est déterminée par le nombre d'individus composant la cohorte contrôle à laquelle elle a été confronté. Chaque individu de la cohorte contrôle a été séquencés en WES par notre équipe. Afin d'être considéré comme "contrôle" et intégrer cette cohorte, un individu doit être sain ou présenter un phénotype d'infertilité différent de la famille étudiée. Par exemple, un individus MMAF pourra servir de contrôle aux familles AZ et FF mais pas aux familles MMAF1-4}\label{fig:plotsamplectrl}
  \end{figure}
  
  \newpage
  
  \newpage
  
  Comme on pouvait s'y attendre, ces six filtres ont un pouvoir
  discriminant extrêmement différent. En effet, tandis que le filtre
  ``Transcript relevance'' (filtre n°4) éliminer en moyenne 3.9 \% des
  variants de chaque individu, le filtre ``Variant impact'' (filtre n° 3)
  élimine jusqu'à 90.1 \% de ces mêmes variants. Cette différence n'est
  pas surprenante. En effet, comme nous l'avions vu plus tôt, les variants
  de la catégorie VEP \emph{MODIFIER} qui regroupe entre autres les
  variants chevauchant les séquences UTRs et introniques (\textbf{Table :}
  \ref{tab:tabvepcsq}) représentent en moyenne 88\% des variants de nos
  patients. Ceux-ci étant tous filtrés, on s'attendait donc à une valeur
  aussi élevée. On peut également constater l'importance de la cohorte
  contrôle qui, je le rappelle, permet de filtrer l'ensemble des variants
  homozygotes observés en son sein, puisque ce filtre permet retirer entre
  76.5 et 88.4\% des variants de chaque individus (\textbf{Figure :}
  \ref{fig:plotcomparefilter} - \textbf{A}).
  
  Cependant, regarder uniquement le pourcentage de variants filtrés par
  chaque filtre révèle une information partielle. En effet, dans ce cas de
  figure, on observe la quantité de variant éliminé par chaque filtre
  indépendamment les uns des autres. Ainsi, un même variant peut donc être
  filtré par plusieurs filtres. Dès lors, il faut également analyser la
  quantité de variants filtrés \textbf{spécifiquement} par chaque filtre.
  Ainsi, on peut constater que le classement des filtres en fonctions de
  leur stringence reste quasiment identique. Il est tout de même
  intéressant de noter que désormais le filtre ``Variant impact'' apparait
  moins efficace que les filtres ``Ctrl'' et ``Genotype'' en filtrant
  spécifiquement une moyenne de 253 variants par individu contre 423 pour
  le filtre génotype et 882 pour le filtre ``Ctrl''. Ainsi, ce dernier
  devient celui filtrant spécifiquement le plus de variants avec entre 364
  et 1060 variants spécifiquement filtrés par patients confirmant ainsi
  l'importance de ce filtre dans nos analyses. Aussi, les filtres
  ``Transcript relevance'', ``Union'' et ``Frequency'' apparaissent
  désormais comme étant anecdotiques en comparaison aux trois autres
  filtres puisqu'ils filtrent au maximum 43 variants spécifiques
  (\textbf{Figure :} \ref{fig:plotcomparefilter} - \textbf{B}).
  
  \newpage
  
  \begin{figure}
  
  {\centering \includegraphics{thesis_files/figure-latex/plotcomparefilter-1} 
  
  }
  
  \caption[Comparaison de l'efficacité de chacun des six filtres utilisés]{Comparaison de l'efficacité de chacun des six filtres utilisés : **A** : Comparaison du pourcentage de variants filtrés par chacun des six filtres indépendamment les uns des autres pour chaque patient (représenté par les points. Dès lors, un même variant peut-être filtré par plusieurs filtres. **B** : Comparaison du nombre de variant filtrés spécifiquement par chacun des filtres. Ici, un variant ne peut-être filtré que par un seul filtre}\label{fig:plotcomparefilter}
  \end{figure}
  
  \newpage
  
  Après avoir appliqué l'ensemble de ces filtres, seuls quelques variants
  subsistent nous permettant d'obtenir une liste de gènes restreinte pour
  chaque famille et ainsi de tirer des conclusions quant au variant
  responsable du phénotype de chacune d'entre elles (\textbf{Table :
  }\ref{tab:tablegene}).
  
  \newpage
  
  \begin{landscape}
  \begin{longtable}[t]{lllllllll}
  \caption{\label{tab:tablegene}Récapitulatif des variants ayant passé l'ensemble des filtres pour chaque famille}\\
  \toprule
  \multicolumn{1}{c}{ } & \multicolumn{1}{c}{ } & \multicolumn{4}{c}{Variant impact} & \multicolumn{3}{c}{Variant frequency} \\
  \cmidrule(l{2pt}r{2pt}){3-6} \cmidrule(l{2pt}r{2pt}){7-9}
  Family & Gene & HGVSc, HGVSp & Consequence & SIFT & PolyPhen & ExAC & ESP & 1KG\\
  \midrule
  AZ & GUF1 & c.443A>T ; p.Ser148Ile & missense & deleterious & proba damaging & 0.00207 & 0.0028 & 9e-04\\
  AZ & SPINK2 & c.56-3C>G ; . & splice region & . & . & . & . & .\\
  FF & PLCZ1 & c.1465G>T ; p.Ile489Phe & missense & deleterious & possib damaging & 8.24e-06 & . & .\\
  MMAF1 & PLA2G4B & c.1710-6delA ; . & splice region & . & . & . & . & .\\
  MMAF2 & DNAH1 & . ; . & splice acceptor & . & . & . & . & .\\
  \addlinespace
  MMAF2 & MYH11 & c.4625G>A ; p.Arg1542Gln & missense & . & proba damaging & 0.00234 & 0.0016 & 5e-04\\
  MMAF3 & WEE2 & . ; p.Pro92Leu & missense & deleterious & benign & 0.000372 & 2e-04 & .\\
  MMAF3 & ZFYVE28 & c.1729C>A ; p.Val577Met & missense & deleterious & benign & 0.000998 & 2e-04 & .\\
  MMAF3 & FCGR3A & c.133T>C ; p.Ala45Pro & missense & deleterious & proba damaging & . & . & .\\
  MMAF3 & GBP2 & c.412T>A ; p.Ala138Thr & missense & deleterious & proba damaging & 0.00176 & 0.0012 & 5e-04\\
  MMAF4 & TGIF2 & c.496C>A ; p.Leu166Met & missense & tolerated & proba damaging & . & . & .\\
  \bottomrule
  \end{longtable}
  \end{landscape}
  
  \newpage
  
  \begin{enumerate}
  \def\labelenumi{\arabic{enumi}.}
  \tightlist
  \item
    \textbf{Famille AZ} : Parmi les 2 gènes restant pour cette famille,
    \emph{SPINK2} est apparu comme étant un candidat évident. Notamment
    son expression étant spécifique au testicule tandis que celle de
    \emph{GUF1} est ubiquitaire (\textbf{Figure :
    }\ref{fig:plotexpspink2guf1}). De plus, des mutations du gène
    \emph{Spink2} chez la souris avait déjà été identifiée comme induisant
    des défauts de la spermatogenèse (B. Lee et al.,
    \protect\hyperlink{ref-Lee2011}{2011}).
  \end{enumerate}
  
  \begin{figure}
  
  {\centering \includegraphics{thesis_files/figure-latex/plotexpspink2guf1-1} 
  
  }
  
  \caption[Expression tissulaire des gènes *SPINK2* et *GUF1*]{Expression tissulaire des gènes *SPINK2* et *GUF1* : Données provenant du projet de transcriptome Illumina bodyMap}\label{fig:plotexpspink2guf1}
  \end{figure}
  
  \begin{enumerate}
  \def\labelenumi{\arabic{enumi}.}
  \setcounter{enumi}{1}
  \tightlist
  \item
    \textbf{Famille FF} : Pour cette famille, seul le gène
    \emph{PLC}\(\zeta 1\) a passé l'ensemble des filtres. Nos
    connaissances sur la fonction de se gène et notamment son rôle dans
    l'activation ovocytaire (Amdani, Jones, \& Coward,
    \protect\hyperlink{ref-Amdani2013}{2013}) ainsi que sa forte
    expression testiculaire ont fait de ce gène le candidat idéal pour
    expliquer le phénotype d'échec de fécondation de ces deux frères
    (\textbf{Figure : }\ref{fig:plotexpplcz1}).
  \end{enumerate}
  
  \begin{figure}
  
  {\centering \includegraphics{thesis_files/figure-latex/plotexpplcz1-1} 
  
  }
  
  \caption[Expression tissulaire du gène *PLCZ1*]{Expression tissulaire du gène *PLCZ1* : D'après les données du Illumina BodyMap}\label{fig:plotexpplcz1}
  \end{figure}
  
  \begin{enumerate}
  \def\labelenumi{\arabic{enumi}.}
  \setcounter{enumi}{2}
  \tightlist
  \item
    \textbf{Famille MMAF1} : L'analyse bibliographique des 1 gènes ayant
    passé l'ensemble des filtres n'a pas pu nous permettre de d'affirmer
    que l'un de ces gènes étaient responsable du phénotype MMAF de ces 2
    frères.\\
  \item
    \textbf{Famille MMAF2} : À l'issue des filtres, 2 gènes ressortaient
    chez ces deux frères : \emph{MYH11} et \emph{DNAH1}. Or, notre équipe
    ayant déjà établit le lien entre des mutations du gène \emph{DNAH1} et
    le syndrome MMAF (Ben Khelifa et al.,
    \protect\hyperlink{ref-BenKhelifa2014}{2014}) ce gène s'est révélé
    être un candidat idéal pour expliquer le phénotype de ces 2 frères. De
    plus, l'implication de \emph{MYH11} dans le phénotype de dissection
    aortique (Imai et al., \protect\hyperlink{ref-Imai2015}{2015}) l'ont
    écarté des candidats pour le phénotype MMAF.\\
  \item
    \textbf{Famille MMAF3} : Comme pour les gènes de la famille MMAF2,
    l'analyse bibliographique des 4 gènes ayant ici passé les filtres de
    même que l'étude de leurs expressions ne nous a pas permis de conclure
    que l'un d'entre eux étaient responsable du phénotype MMAF de ces 2
    frères. \newpage
  \end{enumerate}
  
  \begin{figure}
  
  {\centering \includegraphics{thesis_files/figure-latex/plotexpmmaf3-1} 
  
  }
  
  \caption[Expression tissulaire des gènes retenus pour la famille MMAF3]{Expression tissulaire des gènes retenus pour la famille MMAF3 : Données provenant du projet de transcriptome Illumina bodyMap}\label{fig:plotexpmmaf3}
  \end{figure}
  
  \begin{enumerate}
  \def\labelenumi{\arabic{enumi}.}
  \setcounter{enumi}{5}
  \tightlist
  \item
    \textbf{Famille MMAF4} : Seul le gène \emph{TGIF2} a passé l'ensemble
    des filtres pour la famille MMAF4. L'expression ubiquitaire de ce gène
    n'en font pas un candidat idéal. Cependant une étude de 2011 effectuée
    sur le wallaby décrit que la protéine TGIF2 est localisée
    spécifiquement dans le cytoplasme du spermatide, ainsi que dans le
    corps résiduel et la pièce intermédiaire du flagelle du spermatozoïde
    mature (Hu, Yu, Shaw, Renfree, \& Pask,
    \protect\hyperlink{ref-Hu2011}{2011}). Ces données pourraient corréler
    avec le phénotype MMAF de ces 3 frères bien que l'expression de ce
    gène soit ubiquitaire (\textbf{Figure : }\ref{fig:plotexptgif2}).
  \end{enumerate}
  
  \newpage 
  
  \begin{figure}
  
  {\centering \includegraphics{thesis_files/figure-latex/plotexptgif2-1} 
  
  }
  
  \caption[Expression tissulaire du gène *TGIF2*]{Expression tissulaire du gène *TGIF2* : D'après les données du Illumina BodyMap}\label{fig:plotexptgif2}
  \end{figure}
  
  \newpage
  
  \subsubsection{Discussion}\label{discussion}
  
  L'analyse de ces 6 familles nous a permis de mettre en évidence
  l'efficacité de notre pipeline d'analyse puisque pour 3 d'entre elles
  (soit 50\%) le variant causal a pu être identifié avec certitude
  (\textbf{Figure : }\ref{fig:plotremaininggenes}) et les résultats
  publiés dans trois revus dont je suis co-auteur :
  
  \begin{enumerate}
  \def\labelenumi{\arabic{enumi}.}
  \tightlist
  \item
    \textbf{Famille AZ} : \protect\hyperlink{spink2}{\textbf{SPINK2
    deficiency causes infertility by inducing sperm defects in
    heterozygotes and azoospermia in homozygotes}} : Dans cet article j'ai
    effectué non seulement l'intégralité des analyses bioinformatiques des
    données d'exomes de deux frères infertiles présentant un phénotype
    d'azoospermie mais j'ai aussi séquencé en Sanger les séquences
    codantes du gène \emph{SPINK2} pour une partie des 611 individus
    analysés ainsi que contribué à l'extraction de l'ARN testiculaire des
    souris pour l'analyse fonctionnelle du gène \emph{Spink2} sur le
    modèle murin.\\
  \item
    \textbf{Famille FF} : \protect\hyperlink{plcz}{\textbf{Homozygous
    mutation of PLCZ1 leads to defective human oocyte activation and
    infertility that is not rescued by the WW-binding protein PAWP}} :
    Dans cet article j'ai, effectué l'intégralité des analyses
    bioinformatiques des données d'exomes effectuées sur deux frères
    infertiles présentant des échecs de fécondation.\\
  \item
    \textbf{Famille MMAF2} :
    \protect\hyperlink{famdnah1}{\textbf{Whole-exome sequencing of
    familial cases of multiple morphological abnormalities of the sperm
    flagella (MMAF) reveals new DNAH1 mutations}} : Dans cet article j'ai,
    comme précédemment, effectué l'ensemble des analyses bioinformatiques
    des données d'exomes effectuées sur deux frères infertiles présentant
    des échecs de fécondation.
  \end{enumerate}
  
  Pour une d'entre elle, un candidat potentiel a pu être mis en évidence
  avec le gène \emph{TGIF2} et notre équipe travaille actuellement sur la
  caractérisation de ce gène afin de savoir s'il peut effectivement
  expliquer le phénotype MMAF de cette famille (\textbf{Figure :
  }\ref{fig:plotremaininggenes}).
  
  TODO : Il faut aller plus loin dans l'analyse et les arguments pour
  convaincre qu'il s'agit d'un bon candidat : quel type de mutation, ce
  gène est-il bien conservé, son expression n'est pas spécifique au
  testicule et ce gène serait impliqué dans un phénotype
  d'holoproencephaly\ldots{}
  
  Pour les 2 familles restantes, aucun variant n'a pu pour l'instant
  expliquer leur phénotype. L'explication la plus vraisemblable est que le
  variant ait été filtré par l'un de nos six filtres, probablement celui
  consistant à filtrer l'ensemble des variants hétérozygotes. En effet,
  l'hypothèse d'un variant causal homozygote était extrêmement crédible
  pour les familles AZ, FF et MMAF2 étant donné l'historique consanguin de
  ces 3 familles dont les parents sont à chaque fois apparentés. En
  revanche rien ne laisse supposer une telle chose pour les familles
  restantes. Cependant, le filtre des variants hétérozygotes pour
  l'ensemble des patients de ces 3 familles a été maintenu en première
  intention afin de faciliter les analyses en réduisant au maximum le
  nombre de variant. Au vus des résultats il apparait clair que les
  variants responsables de leur phénotype aient été filtrés pour au moins
  2 de ces familles. Dès lors, l'ensemble des analyses effectuées lors de
  l'étape de filtrage doivent être refaites en changeant les paramètres de
  filtrage. Cette fois-ci, les variants hétérozygotes seront conservés et
  les gènes sur lesquels au moins deux variants hétérozygotes seront
  recensés seront analysés en priorité. En effet, bien que les analyses
  exomiques nous fournissent en l'état pas d'informations suffisante pour
  savoir si ces deux variants sont présent sur le même allèle ou bien sur
  deux allèles différents, cela pourrait-être la signature de variants
  hétérozygotes composites. C'est donc sur ces analyses que se concentre
  actuellement notre équipe.
  
  \newpage
  
  \begin{figure}
  
  {\centering \includegraphics{thesis_files/figure-latex/plotremaininggenes-1} 
  
  }
  
  \caption[Nombre de gènes passant l'ensemble des filtres par famille]{Nombre de gènes passant l'ensemble des filtres par famille  :  Chaque barre représente une des familles analysées. La hauteur de cette barre correspond au nombre de gènes ayant passé l'ensemble des filtres pour chaque famille. Les barres vertes caractérisent les familles pour lesquelles le gène responsable de la pathologie a été identifié parmi la liste de gène (dans ce cas le symbole du gène est écrit au-dessus de la barre). La barre orange caractérise la famille pour laquelle un candidat potentiel a été identifié (le symbole du gène est écrit au-dessus suivit d'un "?"). Les barres rouges indiquent qu'aucun des gènes ayant passé les filtres pour ne semble expliquer le phénotype (dans ce cas il est écrit "???" au-dessus de la barre)}\label{fig:plotremaininggenes}
  \end{figure}
  
  \newpage
  
  \subsection{Etude d'une large cohorte de patients
  MMAF}\label{cohortemmah}
  
  \subsubsection{Description de la
  cohorte}\label{description-de-la-cohorte}
  
  Après avoir mis en évidence l'implication du gène \emph{DNAH1} dans le
  phénotype MMAF notre équipe s'est en partie spécialisé dans la
  caractérisation ce syndrome. Ainsi, entre 2012 et 2015, notre équipe a
  effectué le séquençage de 62 individus présentant tous ce phénotype afin
  d'en établir la cause génétique. Ces séquençages ont été effectué dans 3
  centres différents que sont Genoscope, MountSinai et Strasbourg et sur
  une seule plateforme de séquençage, le Illumina HiSeq2000.
  
  \begin{longtable}[t]{lrlr}
  \caption{\label{tab:tabcohort}Liste des différents projets de séquençages effectués}\\
  \toprule
  Place & Year & Platform & Nb of individuals\\
  \midrule
  MountSinai & 2012 & Illumina Hiseq2000 & 2\\
  Strasbourg & 2012 & Illumina Hiseq2000 & 13\\
  Genoscope & 2013 & Illumina Hiseq2000 & 13\\
  Genoscope & 2014 & Illumina Hiseq2000 & 31\\
  Genoscope & 2015 & Illumina Hiseq2000 & 6\\
  \bottomrule
  \end{longtable}
  
  \newpage
  
  \subsubsection{Application de la pipeline -
  Résultats}\label{application-de-la-pipeline---resultats}
  
  Après avoir appelé les variants de nos 62 patients, nous avons obtenu un
  total de 4484558 variants différents comprenant 4160274 SNVs et 324284
  indels. Ces variants étant répartit entre chaque patient qui portaient
  environs chacun 81618 SNV et 5148 indels dont 42.8 \% étaient
  homozygote. Comme on peut le voir, la proportion de chaque appel est
  relativement homogène lorsque l'on compare les patients ayant été
  séquencés dans le même centre la même année. Cependant, il est possible
  de noter de grandes disparités lorsque l'on compare les données
  provenant de différents centres ou bien du même centre avec plusieurs
  années de différences. Ces écarts peuvent-être causés par plusieurs
  facteur, tel que les différents kits de capture d'exons qui on put être
  utilisés puisque \ldots{} (\textbf{todo lister les différents kits de
  capture dans une table}) en revanche nous pouvons écarter un effet dus à
  la plateforme de séquençage ou encore le modèle de séquenceur puisque
  tous ces projets ont été réalisés sur des Illumina HiSeq2000
  (\textbf{Table : }\ref{tab:tabcohort}) (\textbf{Figure :
  }\ref{fig:plotbigmmafcall} - \textbf{A}).
  
  Le même constat peut être effectué lorsque l'on compare la qualité des
  appels puisque plus les projets de séquençage s'avèrent être récent,
  plus la proportion d'appel \emph{Single Strand} s'avère être faible
  tandis que la proportion d'appel \emph{Double Strand} (DS) est élevée.
  Ceci est une bonne chose, car, bien que ces deux appels soient conservés
  dans les analyses ultérieures, les appels DS sont de meilleure qualité
  que les appels SS. Cette augmentation des appels DS au cours du temps
  pourrait s'expliquer par une amélioration des protocoles de séquençage
  ainsi que des kits de capture. En revanche cela est à pondérer avec le
  taux croissant d'appels \emph{No-strand} (NS) au fur et à mesure des
  années pour atteindre environs 21.3 \% en 2015 avec un projet réalisé au
  Génoscope. Ces derniers appels étant systématiquement filtrés, ils
  n'altèreront en rien les résultats obtenus en aval hormis le fait qu'ils
  réduisent la quantité des données utilisées (\textbf{Figure :
  }\ref{fig:plotbigmmafcall} - \textbf{B} et \textbf{C}).
  
  \newpage
  
  \begin{figure}
  
  {\centering \includegraphics{thesis_files/figure-latex/plotbigmmafcall-1} 
  
  }
  
  \caption[Résultats de l'appel des variants par individus et par projet de séquençage]{Résultats de l'appel des variants par individus et par projet de séquençage : Chaque couleur définit un projet de séquençage caractérisé par un centre de séquençage et une année. **A** : Quantification pour chaque individus (représentés par les barres) du nombre de variants (SNVs et Indels) appelés homozygotes et hétérozygotes. **B** : Quantification des appels *Double Strand* (DS), *Single Strand* (SS) et *No strand* (NS) pour chaque projet de séquençage. **C** : Même chose en pourcentage}\label{fig:plotbigmmafcall}
  \end{figure}
  
  \newpage
  
  \subsubsection{Analyse des listes de
  gènes}\label{analyse-des-listes-de-genes}
  
  Après avoir appliqué les mêmes filtres que ceux décrit précédemment à
  l'exception du filtre n°1
  \protect\hyperlink{filterdescription}{``Union''} puisqu'ici nous avons
  uniquement des individus non apparentés, nous avons pu obtenir une liste
  de 1711 variants différents composés de 1470 SNVs et 241 indels et
  impactant un total de 1432 gènes distincts. Ces variants étant répartis
  sur l'ensemble de nos 62 patients ceux-ci portaient en moyenne 27 SNVS
  et 5 indels, de sorte que chacun d'entre eux avaient entre NA et NA
  gènes impactés par au moins un variants homozygote (\textbf{Figure :
  }\ref{fig:plotfilterbigmmaf} - \textbf{A} et \textbf{B}).
  
  \begin{verbatim}
  ## Warning: Removed 3 rows containing non-finite values (stat_boxplot).
  \end{verbatim}
  
  \begin{verbatim}
  ## Warning: Removed 3 rows containing missing values (geom_point).
  \end{verbatim}
  
  \begin{verbatim}
  ## Warning: Removed 3 rows containing missing values (position_stack).
  \end{verbatim}
  
  \begin{figure}
  
  {\centering \includegraphics{thesis_files/figure-latex/plotfilterbigmmaf-1} 
  
  }
  
  \caption[Résultats de l'étape de filtrage]{Résultats de l'étape de filtrage : **A** : Quantification du nombre de SNVs et indels ayant passé l'ensemble des filtres pour chaque patient. **B** : Nombre de gènes impactés par au moins un variant ayant passé les filtres pour chaque individu représenté par les barres. **C** : Présentation }\label{fig:plotfilterbigmmaf}
  \end{figure}
  
  \newpage
  
  Afin de nous orienter dans nos recherches, nous nous sommes basés sur
  une étude de 2012 qui établissait une liste des gènes humains pouvant
  être impliqués dans le cilliome humain (Ivliev, 't Hoen, Roon-Mom,
  Peters, \& Sergeeva, \protect\hyperlink{ref-Ivliev2012}{2012}) en se
  basnat entre autre sur les gènes fournis par la base de données CilDB
  (Arnaiz, Cohen, Tassin, \& Koll,
  \protect\hyperlink{ref-Arnaiz2015}{2015}) mais aussi des analyses
  \emph{in silico} permettant d'identifier des gènes jusqu'à présent
  jamais associés au ciliome. Ainsi, chaque gène était classé dans l'une
  des 3 catégories suivantes en fonction des preuves déjà existante (au
  moment de l'étude) permettant de lier un gène au cilliome humain :
  \textbf{Strong evidence from previous studies} (Strong), \textbf{Weak
  evidence from previous studies} (Weak) et \textbf{No evidence from
  previous studies} (Novel). L'utilisation de cette liste nous a permis
  d'ajouter une nouvelle annotation pertinente à nos gènes. En effet, le
  spermatozoïde humain est une cellule ciliée, et le flagelle en est le
  cil. Nous pouvons donc attendre à ce qu'une partie des gènes
  responsables du phénotype MMAF soit présents dans cette liste de 371
  gènes.
  
  Ainsi, 33 de nos 1432 gènes retenus faisaient partis de cette liste dont
  22 présentaient des preuves fortes de leur appartenance au cilliome. Il
  faut tout de même noter que bien que cette liste soit un bon outil pour
  orienter les recherches et prioriser certains gènes, elle ne peut
  constituer un critère suffisant pour filtrer les gènes n'en faisant pas
  partie. Par exemple le gène \emph{DNAH1}, de par son expression
  ubiquitaire n'a pas été intégré à cette liste (\textbf{Figure :
  }\ref{fig:plotdnah1} - \textbf{A}), or on connait désormais son
  implication dans le phénotype MMAF (\textbf{Table :} \ref{tab:tabcil},
  \textbf{Figure : }\ref{fig:plotcil} - \textbf{A}).
  
  Parmi l'ensemble des gènes retrouvés mutés, on peut constater que 1212
  d'entre eux, soit 84.6\%, ne sont observés muté, à l'état homozygote,
  uniquement chez un seul de nos patients tandis que 220 sont retrouvés
  muté à l'état homozygote chez au moins 2 patients. Ainsi, analyser en
  priorité ces 220 gènes permet à la fois de réduire considérablement
  notre liste, et donc de faciliter les analyses, mais aussi de pouvoir
  identifier en priorité les gènes affectant le plus grand nombre de
  patient et donc probablement les principaux acteurs du impliqués dans le
  phénotype MMAF (\textbf{Figure : }\ref{fig:plotcil} - \textbf{B}).
  
  \newpage 
  
  \begin{figure}
  
  {\centering \includegraphics{thesis_files/figure-latex/plotcil-1} 
  
  }
  
  \caption[Répartition des gènes retrouvés mutés chez nos patients dans les différentes classes de la liste du cilliome]{Répartition des gènes retrouvés mutés chez nos patients dans les différentes classes de la liste du cilliome : Chaque couleur définit une classe de la liste des gènes du cilliome décrit dans [@Ivliev2012]. Vert = *Strong evidence from previous studies* (*Strong*), Orange = *Weak evidence from previous studies* (*Weak*), rouge = *No evidence from previous studies* (*Novel*), bleu = Non présent dans la liste. **A** : Quantification du nombre de gène ayant passé les filtres au sein des 3 classes de la liste des gènes du cilliome. **B** : Quantification pour chacun des gènes du nombre d'individus présentant un variant homozygotes ayant passé l'ensemble des filtres}\label{fig:plotcil}
  \end{figure}
  
  \newpage  
  
  \begin{longtable}[t]{ll}
  \caption{\label{tab:tabcil}Liste des gènes prédits comme faisant partie du ciliome humain retrouvés mutés à l'état homozygote chez nos patients}\\
  \toprule
  Ciliome evidence & Gene\\
  \midrule
  Strong & AK7\\
  Strong & C21orf59\\
  Strong & C6orf118\\
  Strong & CCDC146\\
  Strong & CCDC147\\
  \addlinespace
  Strong & CCDC65\\
  Strong & CFAP44\\
  Strong & DLEC1\\
  Strong & EFCAB6\\
  Strong & FAM81B\\
  \addlinespace
  Strong & HYDIN\\
  Strong & KIF6\\
  Strong & KIF9\\
  Strong & NPHP1\\
  Strong & PROM1\\
  \addlinespace
  Strong & RSPH9\\
  Strong & SPEF2\\
  Strong & STK33\\
  Strong & TTC26\\
  Strong & TTC29\\
  \addlinespace
  Strong & WDR16\\
  Strong & ZMYND10\\
  Weak & C6\\
  Weak & FBXO15\\
  Weak & MIPEP\\
  \addlinespace
  Weak & SLFN13\\
  Weak & SPAG17\\
  Novel & ACYP1\\
  Novel & ARMC2\\
  Novel & C21orf58\\
  \addlinespace
  Novel & KIAA0319\\
  Novel & KIAA0556\\
  Novel & LRRC43\\
  \bottomrule
  \end{longtable}
  
  \newpage
  
  Nous avons ensuite \emph{designer} 5 analyses fonctionnant de manière
  pyramidale. C'est à dire que la première analyse permet d'identifier les
  candidats les plus évidents tandis que la cinquième ceux pour lesquels
  il y a moins d'indices. L'interêt de ce fonctionnement est qu'à chacune
  des analyses, l'ensemble des variants portés par les patients pour
  lesquels un candidat a pu être déterminer avec \textbf{certitudes}
  seront retirés dans analyses suivantes. Ce procédé nous permettra ainsi
  après chacune de ces 5 analyses d'alléger notre liste de variants, et
  donc de gènes, rendant l'analyse suivante plus simple :
  
  \begin{enumerate}
  \def\labelenumi{\arabic{enumi}.}
  \item
    \textbf{Analyse n°1} : Au moment de nos analyses, le gène \emph{DNAH1}
    était encore le seul décrit comme responsable du phénotype MMAF
    faisant de lui un candidat évident pour expliquer le phénotype MMAF de
    nos patients malgré son expression non spécifique au testicule
    (\textbf{Figure : }\ref{fig:plotdnah1} - \textbf{A}). C'est pourquoi
    nous avons recherché en priorité des patients portant des variants
    tronquants homozygotes sur ce gène.
  \item
    \textbf{Analyse n°2} : Dans second temps, nous avons sélectionné
    uniquement les gènes \textbf{présents dans la liste cilliome} sur
    lesquels \textbf{au moins deux} de nos patients présentaient au moins
    1 variant tronquant à l'état \textbf{homozygote}.
  \item
    \textbf{Analyse n°3} : Ensuite, nous avons étudié les gènes
    \textbf{absents dans la liste cilliome} mais sur lesquels on trouvait
    toujours \textbf{au moins deux} de nos patients présentant au moins 1
    variant tronquant à l'état \textbf{homozygote}.
  \item
    \textbf{Analyse n°4} : Dans un troisième temps nous sommes revenus à
    étudier les gènes \textbf{présents dans la liste cilliome} en
    considérant cette fois-ci les gènes sur lesquels \textbf{un seul} de
    nos patients présentaient au moins 1 variant tronquant à l'état
    \textbf{homozygote}.
  \item
    \textbf{Analyse n°5} : Pour finir nous avons étudié les gènes
    \textbf{absents dans la liste cilliome} sur lesquels \textbf{un seul}
    de nos patients présentaient au moins 1 variant tronquant à l'état
    \textbf{homozygote}.
  \end{enumerate}
  
  Chacune de ces analyses s'est ensuite déroulée en 5 étapes :
  
  \begin{enumerate}
  \def\labelenumi{\arabic{enumi}.}
  \item
    \textbf{Étape n°1} : Cette première étape consistait à récupérer
    l'ensemble des patients répondant aux critères de l'analyse en
    question, par exemple : l'ensemble des patients portant un variant
    homozygote sur le gène \emph{DNAH1} dans le cas de l'analyse n°1. Au
    vu de l'effet délétère des variants sélectionés au cours de cette
    étape, les preuves génétiques seront considérées comme de forte
    confiance (\emph{High trust}).
  \item
    \textbf{Étape n°2} : Pour l'ensemble des gènes retenus dans l'étape
    n°1, nous recherchons ensuite des patients portant, \textbf{toujours à
    l'état homozygote}, des variant aux effets \textbf{non tronquants} tel
    que des variants faux-sens ou encore des variants intronique situés
    proches des sites d'épissage. Dans le cas des variants faux-sens, les
    logiciels SIFT et PolyPhen sont ensuite utilisés afin de nous orienter
    quant à l'effet délétère du variant, bien que comme nous l'avons déjà
    vu, ces logiciels son contredisent régulièrement (Salgado, Bellgard,
    Desvignes, \& B??roud, \protect\hyperlink{ref-Salgado2016}{2016})
    (\textbf{Figure :} \ref{fig:vennpred}). Au vus de la difficulté à
    déterminer l'effet délétère de ces variants, les preuves génétiques
    seront considérées comme de confiance modérée (\emph{Moderate trust}).
  \item
    \textbf{Étape n°3} : Cette étape consiste à rechercher des patients
    éventuellement hétérozygotes composites, c'est à dire des patients
    portant deux variants hétérozygotes différents sur chacun des deux
    allèles d'un même gène. Malheureusement, dans le cadre des séquençages
    WES et WGS, il est impossible de connaitre le ``phasage'' des
    variants, c'est à dire que l'on ne peut déterminer si deux variants
    hétérozygotes sont situés sur le même allèles ou sur deux allèles
    différents (\textbf{Figure : }\ref{fig:compositehet}). Pour cela, des
    analyses de biologie moléculaire sont nécessaires. C'est pour cette
    raison que ces preuves génétiques seront considérées comme étant de
    faible confiance (\emph{Low trust}).
  \end{enumerate}
  
  \begin{figure}
  
  {\centering \includegraphics[scale=0.35]{figure/hetero_composites} 
  
  }
  
  \caption[Représentation schématique des ....TODO]{Représentation schématique des ....TODO : Un variant est dit homozygote lorsque le **même** variant est présents sur les deux allèles d'un gène et hétérozygote lorsqu'il est présent sur **un seul** des deux allèles. On parlera d'hétérozygotes *cis* lorsque deux variants hétérozygotes différents seront positionnés sur **le même** allèle et d'hétérozygote *trans* (ou composite) lorsque ces deux variants hétérozygotes seront positionnés sur **deux allèles différents**. En WES et en WGS il est impossible de différentié les hétérozygotes *cis* des hétérozygotes *trans*}\label{fig:compositehet}
  \end{figure}
  
  \begin{enumerate}
  \def\labelenumi{\arabic{enumi}.}
  \setcounter{enumi}{3}
  \item
    \textbf{Étape n°4} : Au cours de cette étape, nous allons étudier
    chacun des gènes sélectionnés et déterminer au vus des différentes
    preuves génétiques, des données d'expressions de ces gènes mais aussi
    des informations disponibles dans la litterature, si ceux-ci sont de
    bons candidats pour expliquer le phénotype MMAF de nos patients. Il
    est à noter que \textbf{cette étape est la seul à ne pas être
    automatisée}.
  \item
    \textbf{Étape n°5} : Pour finir, les données des patients pour
    lesquels un gène candidat a été identifié et dont les preuves
    génétiques sont considérées comme de forte confiance (c'est à dire les
    patients identifiés lors de l'étape n°1) sont retirés de notre liste
    de variants allégeant ainsi celle-ci pour les analyses suivante.
  \end{enumerate}
  
  \newpage  
  
  \paragraph{Analyse n°1 :}\label{analyse-n1}
  
  \begin{enumerate}
  \def\labelenumi{\arabic{enumi}.}
  \tightlist
  \item
    \textbf{\emph{DNAH1}} : Parmi l'ensemble de nos 62 patients MMAF 1
    portait un indel homozygote entrainant un décalage du cadre de lecture
    sur le gène \emph{DNAH1} et répertorié dans aucune des 3 bases de
    données qua nous avons utilisées. Le patient et \emph{Ghs90} était
    porteur de 3 variants homozygotes successifs entrainant tous un
    faux-sens tandis que le patient et \emph{Ghs95} portait lui 1 autre
    variant homozygote entrainant un faux-sens différent des 3 portés par
    et \emph{Ghs90}. On peut noter qu'aucun de ces 4 variants n'est
    répertorié dans les bases de données, cependant tous sont prédits
    comme \emph{benign} par PolyPhen tandis que SIFT ne donne aucune
    prédiction. La recherche de patients potentiellement hétérozygotes
    composites nous a permis de révéler 6 patients portant tous 2 variants
    hétérozygotes sur le gène \emph{DNAH1}. On peut alors noter le
    patients \emph{Ghs36} porteur d'un premier variant créant un codon
    stop de manière prématuré et d'un second variant entrainant un
    faux-sens absent des 3 bases de données et prédit comme \emph{probably
    damaging} par PolyPhen. Ainsi, s'ils étaient situés sur deux allèles
    différents, ces deux variants pourraient être des bons candidats pour
    expliquer le phénotype du patient \emph{Ghs36}. On peut noter
    également le patient \emph{Ghs129} portant deux faux-sens hétérozygote
    tout deux prédits comme \emph{probably damaging} par PolyPhen et dont
    un seul est retrouvé dans la base de données ExAC avec une très faible
    fréquence. Pour les 4 autres patients, il est plus difficile de se
    prononcer à ce stade compte tenus des fréquences parfois élevée des
    variants ou bien des prédictions fournies par PolyPhen.
  \end{enumerate}
  
  Cette première analyse nous a permis de révéler que 9 des 62 patients de
  notre cohorte portaient au moins 1 variant sur le gène \emph{DNAH1} et
  que pour 3 d'entre eux ce(s) variants étaient présents à l'état
  homozygote. Cependant, il faut noter que du fait de son effet tronquant
  sur la protéine, seul le variant homozygote porté par le patient et
  \emph{Ghs122} nous permets d'être certains de la causalité du phénotype
  MMAF. C'est pourquoi, seul les données de ce patient seront retirées de
  notre liste de variants réduisant celle-ci à 1661 variants chevauchant
  1395 gènes (\textbf{Table :} \ref{tab:annexetabafterdnah1}). Pour les 8
  autres patients, des analyses fonctionnelles complémentaires sont
  nécessaires.
  
  Ainsi, les mutations du gène \emph{DNAH1} seraient ainsi responsables de
  manière certaine de 2 \% des phénotype MMAF de notre cohorte. Ce
  pourcentage monte jusqu'à 15\% si l'on considère l'ensemble des patients
  identifiés dans cette analyse.
  
  Bien que ce pourcentage soit en deçà des 40\% observés dans notre étude
  précédente (Ben Khelifa et al.,
  \protect\hyperlink{ref-BenKhelifa2014}{2014}), ces résultats confirment
  néanmoins le rôle primordial de la protéine DNAH1 dans la structure du
  flagelle et l'implication majeure du gène \emph{DNAH1} dans le phénotype
  MMAF.
  
  \newpage
  
  \begin{figure}
  
  {\centering \includegraphics{thesis_files/figure-latex/plotdnah1-1} 
  
  }
  
  \caption[Analyse du gène *DNAH1*]{Analyse du gène *DNAH1* : Expression tissulaire du gène *DNAH1* d'après les données du projet Illumina BodyMap. Quantification du nombre de patients portant au moins un variant sur le gène *DNAH1* pour chacun des 3 niveau de confiance}\label{fig:plotdnah1}
  \end{figure}
  
  \newpage
  
  \begin{landscape}
  \begin{longtable}[t]{llllllll}
  \caption{\label{tab:tabdnah1high}Analyse n°1 : Liste des patients portant au moins un variant homozygote tronquant sur le gène *DNAH1*}\\
  \toprule
  \multicolumn{1}{c}{ } & \multicolumn{1}{c}{ } & \multicolumn{1}{c}{ } & \multicolumn{2}{c}{Variant impact} & \multicolumn{3}{c}{Variant frequency} \\
  \cmidrule(l{2pt}r{2pt}){4-5} \cmidrule(l{2pt}r{2pt}){6-8}
  Patient & Gene & Evidence & HGVSc, HGVSp & Consequence & ESP & 1KG & ExAC\\
  \midrule
  Ghs122 & DNAH1 & Not in the list & c.7533delC ; p.Gln2511HisfsTer27 & frameshift & . & . & .\\
  \bottomrule
  \end{longtable}
  \end{landscape}
  
  \begin{landscape}
  \begin{longtable}[t]{lllllllll}
  \caption{\label{tab:tabdnah1moderate}Analyse n°1 : Liste des patients portant au moins un variant homozygote non tronquant sur le gène *DNAH1*}\\
  \toprule
  \multicolumn{1}{c}{ } & \multicolumn{1}{c}{ } & \multicolumn{4}{c}{Variant impact} & \multicolumn{3}{c}{Variant frequency} \\
  \cmidrule(l{2pt}r{2pt}){3-6} \cmidrule(l{2pt}r{2pt}){7-9}
  Patient & Gene & HGVSc, HGVSp & Consequence & SIFT & PolyPhen & ESP & 1KG & ExAC\\
  \midrule
  Ghs90 & DNAH1 & c.2122G>C ; p.Ile708Leu & missense & . & benign & . & . & .\\
  Ghs90 & DNAH1 & c.2123A>C ; p.Ile708Thr & missense & . & benign & . & . & .\\
  Ghs90 & DNAH1 & c.2125A>C ; p.Phe709Leu & missense & . & benign & . & . & .\\
  Ghs95 & DNAH1 & c.9278C>G ; p.Ala3093Gly & missense & . & benign & . & . & .\\
  \bottomrule
  \end{longtable}
  \end{landscape}
  
  \newpage
  
  \begin{landscape}
  \begin{longtable}[t]{lllllllll}
  \caption{\label{tab:tabdnah1low}Analyse n°1 : Liste des patients portant au moins deux variant hétérozygotes sur le gène *DNAH1*}\\
  \toprule
  \multicolumn{1}{c}{ } & \multicolumn{1}{c}{ } & \multicolumn{4}{c}{Variant impact} & \multicolumn{3}{c}{Variant frequency} \\
  \cmidrule(l{2pt}r{2pt}){3-6} \cmidrule(l{2pt}r{2pt}){7-9}
  Patient & Gene & HGVSc, HGVSp & Consequence & SIFT & PolyPhen & ESP & 1KG & ExAC\\
  \midrule
  Ghs28 & DNAH1 & c.752C>G ; p.Glu251Gly & missense & . & benign & 1e-04 & . & 1.65e-05\\
  Ghs28 & DNAH1 & c.1172A>G ; p.Tyr391Cys & missense & . & benign & 0.0027 & 0.0019 & 0.00233\\
  Ghs36 & DNAH1 & c.845A>G ; p.Leu282Trp & missense & . & proba damaging & . & . & .\\
  Ghs36 & DNAH1 & c.3931G>T ; p.Gln1311Ter & stop gained & . & . & . & . & 8.29e-06\\
  Ghs42 & DNAH1 & c.6916G>A ; p.Ala2306Thr & missense & . & proba damaging & . & . & .\\
  \addlinespace
  Ghs42 & DNAH1 & . ; . & splice region & . & . & . & . & .\\
  Ghs87 & DNAH1 & c.3431C>A ; p.Ser1144Asn & missense & . & benign & . & . & 0.00024\\
  Ghs87 & DNAH1 & c.8885G>C ; p.Lys2962Thr & missense & . & proba damaging & 7e-04 & 5e-04 & 0.000457\\
  Ghs88 & DNAH1 & c.2209C>A ; p.Val737Met & missense & . & benign & 1e-04 & . & 0.000115\\
  Ghs88 & DNAH1 & c.3877T>A ; p.Asp1293Asn & missense & . & benign & 1e-04 & 0.0019 & 0.000149\\
  \addlinespace
  Ghs129 & DNAH1 & c.7153G>A ; p.Trp2385Arg & missense & . & proba damaging & . & . & .\\
  Ghs129 & DNAH1 & . ; p.Arg3169Gly & missense & . & proba damaging & . & . & 8.26e-06\\
  \bottomrule
  \end{longtable}
  \end{landscape}
  
  \newpage
  
  \paragraph{Analyse n°2}\label{analyse-n2}
  
  Comme dit précédemment, dans cette analyse nous nous sommes concentrés
  sur les gènes \textbf{présents dans la liste cilliome} sur lesquels
  \textbf{au moins deux} de nos patients présentaient au moins 1 variant
  \textbf{homozygote} ayant un effet tronquant sur la protéine. Nous avons
  ainsi pu identifier les 4 gènes suivants : \emph{ARMC2}, \emph{CCDC146},
  \emph{CFAP44} et \emph{TTC29} :
  
  \begin{enumerate}
  \def\labelenumi{\arabic{enumi}.}
  \item
    \textbf{\emph{ARMC2}} : Sur ce gène, 2 patients portent un variant
    homozygote tronquant : \emph{Ghs37} et \emph{Ghs93}. Le patient et
    \emph{Ghs37} portent un indel créant un décalage du cadre de lecture
    tandis que et \emph{Ghs93} porte un variant créant un codon stop
    prématuré (\textbf{Table : }\ref{tab:tabgrp1high}). Le patient et
    \emph{Ghs107} porte quant à lui un variant faux-sens prédit comme
    \emph{deleterious} par SIFT et \emph{probably damaging} par PolyPhen.
    Nous pouvons également noter qu'aucun des variants portés par ces 3
    sont absents des 3 bases de données. Les arguments génétiques
    mentionnés couplés à la forte expression testiculaire de ce gène
    (\textbf{Figure : }\ref{fig:plotgrp1high} - \textbf{B}) font de
    celui-ci un très bon candidat pour expliquer le phénotype de ces 3
    bien que des analyses fonctionnelles soient nécéssaires pour 1 d'entre
    eux.
  \item
    \textbf{\emph{CFAP44}} : Comme \emph{ARMC2}, 2 patients portent un
    variant homozygote tronquant. et \emph{Ghs22} porte un variant stop et
    et \emph{Ghs34} un variant affectant un site donneur d'épissage. Le
    fait qu'aucun de ces variant ne soir répertorié dans aucune des bases
    de données laisse supposer qu'ils sont tous deux très rares. Le
    patient et \emph{Ghs89} porte un variant faux-sens prédit comme
    \emph{tolerated} par SIFT et \emph{possibly damaging} par PolyPhen.
    Les preuves fortes impliquant ce gène dans le cilliome humain ainsi
    que les effets délétères retrouvé chez 2 des 3 portant des mutations
    sur ce gène font de \emph{CFAP44} un autre très bon candidat malgré
    son expression non spécifique au testicule (\textbf{Table :
    }\ref{tab:tabgrp1high} et \ref{tab:tabgrp1moderate}, \textbf{Figure :
    }\ref{fig:plotgrp1high} - \textbf{B}).
  \item
    \textbf{\emph{CCDC146}} : Sur ce gène, seuls les patients \emph{Ghs32}
    et \emph{Ghs35} sont retrouvés mutés. Cependant, ces deux patients
    portent tous deux des variants homozygotes induisant respectivement un
    variant un codon stop prématuré et un décalage du cadre de lecture. Le
    premier de ces variants est retrouvé à une fréquence très faible à la
    fois dans ESP et ExAC, tandis que le second est totalement absent de
    l'ensemble des bases de données. On peut également ajouter que ce gène
    décrit comme faisant partie du cilliome humain avec de fortes preuves
    présente également une expression testiculaire relativement élevée.
    Pour finir, la protéine CCDC146 codée par le gène \emph{CCDC146} avait
    déjà été décrite comme composant du centrosome spermatique, un
    organite ayant un rôle dans l'orientation des cellules et étant à
    l'origine des cils et des flagelles (E. N. Firat-Karalar, Sante,
    Elliott, \& Stearns, \protect\hyperlink{ref-Firat-Karalar2014}{2014})
    renforçant ainsi les arguments de l'implication de ce gène dans le
    phénotype MMAF nous permettent ainsi d'affirmer que ces variants sont
    responsables du phénotype MMAF de ces 2 patients (\textbf{Table :
    }\ref{tab:tabgrp1high} et \ref{tab:tabgrp1moderate}, \textbf{Figure :
    }\ref{fig:plotgrp1high} - \textbf{B}).
  \item
    \textbf{\emph{TTC29}} : Sur ce gène, les patients \emph{Ghs19} et
    \emph{Ghs26} portent la même variation retrouvée à très faible
    fréquence dans les trois bases de données et impactant un site donneur
    d'épissage du transcrit induisant la production d'une protéine
    aberrante. Ce gène à très forte expression testiculaire avait déjà été
    décrit en 2014 comme localisant au niveau de l'axonème du flagelle et
    qu'un \emph{knock-down} entrainait des défauts de la cilliogénèse
    (Chung et al., \protect\hyperlink{ref-Chung2014}{2014}) faisant de
    celui-ci un bon candidat pour expliquer le phénotype de ces 2 patients
    (\textbf{Table : }\ref{tab:tabgrp1high} et \ref{tab:tabgrp1moderate},
    \textbf{Figure : }\ref{fig:plotgrp1high} - \textbf{B}).
  \end{enumerate}
  
  Cette première analyse au cours de laquelle nous avons sélectionné les
  gènes retrouvés présents dans la liste cilliome et sur lesquels au moins
  deux de nos patients présentaient au moins 1 variant homozygote ayant un
  effet tronquant sur la protéine nous a permis de mettre en évidence 4
  nouveaux gènes candidats : \emph{ARMC2}, \emph{CCDC146}, \emph{CFAP44}
  et \emph{TTC29} retrouvés mutés à l'état homozygote chez 10 de nos
  patients dont 8 avec des variants tronquants soit 13.1\% des patients
  restant dans notre cohorte. Pour les 2 autres, des analyses
  fonctionnelles sont nécessaires afin de pouvoir être sûr ce leurs
  variants sont bien responsables de leur phénotype. La cause génétique
  responsable du phénotype des patients \emph{Ghs19}, \emph{Ghs22},
  \emph{Ghs26}, \emph{Ghs32}, \emph{Ghs34}, \emph{Ghs35}, \emph{Ghs37} et
  \emph{Ghs93} ayant été identifiées avec certitude, l'ensemble de leurs
  données de variants sont retirées de nos liste réduisant ainsi celle-ci
  à 1331 chevauchant 1164 et répartis sur 53. Les données des patients
  \emph{Ghs107} et \emph{Ghs89} sont, elles, conservées afin de voir si un
  meilleur candidat pourrait expliquer le phénotype de ces patients.
  
  \newpage
  
  \begin{figure}
  
  {\centering \includegraphics{thesis_files/figure-latex/plotgrp1high-1} 
  
  }
  
  \caption[Analyse des gènes sélectionnés dans l'Analyse n°1]{Analyse des gènes sélectionnés dans l'Analyse n°1 : Expression tissulaire des gènes retenus d'après les données du projet de transcriptome Illumina bodyMap. Résumé de l'Analyse 1, quantification du nombre de patients retrouvés mutés sur chacun des gènes retenus ainsi que du degré de confiance accordé à la cause génétique}\label{fig:plotgrp1high}
  \end{figure}
  
  \newpage
  
  \begin{landscape}
  \begin{longtable}[t]{llllllll}
  \caption{\label{tab:tabgrp1high}Analyse n°2 : List des gènes présents dans la liste ciliome sur lesquels au moins deux patients portent une mutation tronquante homozygote}\\
  \toprule
  \multicolumn{1}{c}{ } & \multicolumn{1}{c}{ } & \multicolumn{1}{c}{ } & \multicolumn{2}{c}{Variant impact} & \multicolumn{3}{c}{Variant frequency} \\
  \cmidrule(l{2pt}r{2pt}){4-5} \cmidrule(l{2pt}r{2pt}){6-8}
  Patient & Gene & Evidence & HGVSc, HGVSp & Consequence & ESP & 1KG & ExAC\\
  \midrule
  Ghs37 & ARMC2 & Novel & c.2353\_2354delTA ; p.Leu785MetfsTer5 & frameshift & . & . & .\\
  Ghs93 & ARMC2 & Novel & c.1023+1G>A ; . & splice donor & . & . & .\\
  Ghs32 & CCDC146 & Strong & c.1084G>T ; p.Arg362Ter & stop gained & 1e-04 & . & 2.47e-05\\
  Ghs35 & CCDC146 & Strong & c.2112delT ; p.Arg704SerfsTer7 & frameshift & . & . & .\\
  Ghs22 & CFAP44 & Strong & . ; p.Arg1059Ter & stop gained & . & . & .\\
  \addlinespace
  Ghs34 & CFAP44 & Strong & . ; . & splice donor & . & . & .\\
  Ghs19 & TTC29 & Strong & c.176+1T>A ; . & splice donor & 0.0012 & 5e-04 & 0.000158\\
  Ghs26 & TTC29 & Strong & c.176+1T>A ; . & splice donor & 0.0012 & 5e-04 & 0.000158\\
  \bottomrule
  \end{longtable}
  \end{landscape}
  
  \begin{landscape}
  \begin{longtable}[t]{lllllllll}
  \caption{\label{tab:tabgrp1moderate}Analyse n°2 : Liste des patients portant un variant non troquant homozygote sur un des gènes suivant : ARMC2, CCDC146, CFAP44  et  TTC29}\\
  \toprule
  \multicolumn{1}{c}{ } & \multicolumn{1}{c}{ } & \multicolumn{4}{c}{Variant impact} & \multicolumn{3}{c}{Variant frequency} \\
  \cmidrule(l{2pt}r{2pt}){3-6} \cmidrule(l{2pt}r{2pt}){7-9}
  Patient & Gene & HGVSc, HGVSp & Consequence & SIFT & PolyPhen & ESP & 1KG & ExAC\\
  \midrule
  Ghs89 & CFAP44 & c.1457A>T ; p.Ala486Val & missense & tolerated & possib damaging & 0.0012 & 0.0014 & 0.000692\\
  Ghs107 & ARMC2 & c.2279G>A ; p.Ile760Asn & missense & deleterious & proba damaging & . & . & .\\
  \bottomrule
  \end{longtable}
  \end{landscape}
  
  \newpage
  
  \paragraph{Analyse n°3}\label{analyse-n3}
  
  Pour rappel, au cours de cette analyse, nous avons étudié les gènes
  \textbf{absents dans la liste cilliome} mais sur lesquels on trouvait
  toujours \textbf{au moins deux} de nos patients présentant au moins 1
  variant tronquant à l'état \textbf{homozygote}. 17 patients différents
  portaient ainsi au moins un variant homozygote tronquant sur l'un des 8
  gènes suivants : \emph{BAZ1A}, \emph{CCDC129}, \emph{CFAP43},
  \emph{FSIP2}, \emph{ICA1}, \emph{NACA}, \emph{SART3} et \emph{TRAV26-1}.
  
  \begin{enumerate}
  \def\labelenumi{\arabic{enumi}.}
  \item
    \textbf{\emph{CFAP43}} : 7 patients portent au moins 1 variant sur le
    gène \emph{CFAP43}. Parmi ceux-ci, les patients \emph{Ghs102},
    \emph{Ghs105}, \emph{Ghs126}, \emph{Ghs17} et \emph{Ghs41} portent une
    mutation tronquante à l'état homozygote soit absente des bases de
    données soit présentent avec une très faible fréquence. Le patient et
    \emph{Ghs25} lui porte un variant homozygote intronique au situé au
    sein de la région d'épissage (et non sur un site donneur ou
    accepteur). Bien que ce type de variants puissent effectivement avoir
    un impact sur l'épissage, il pourrait également être sans effet, or,
    il est difficile de le prédire à ce stade. (TODO :: faire tourner un
    algo de prédiction). Le patient et \emph{Ghs132} en revanche semble
    plus intéressant, puisque celui-ci porte deux variant hétérozygotes
    sur le gène \emph{CFAP43} parmi lesquels un est un indel entrainant un
    décalage du cadre de lecture tandis que l'autre est un faux-sens
    prédit comme \emph{possibly damaging} par Polyphen, bien qu'il soit
    annoté \emph{tolerated} par SIFT. Malgré une expression ubiquitaire
    (\textbf{Figure : }\ref{fig:plotgrp2} - \textbf{B}), le nombre
    important de patients portant des variant sur celui-ci, notamment 5
    portant des variants tronquants homozygote font de ce gène un bon
    candidat.
  \item
    \textbf{\emph{FSIP2}} : Comme pour le gène \emph{CFAP43}, 7 de nos
    patients portaient 1 variant sur le gène \emph{FSIP2} cependant pour
    seulement 2 d'entre eux ce variant était tronquant à l'état
    homozygote. En effet, les patients \emph{Ghs20} et \emph{Ghs21}
    portent tous deux un indel entrainant un décalage du cadre de lecture
    dont aucun n'est répertorié dans les bases de données. Le patient et
    \emph{Ghs131} porte lui un faux-sens homozygote prédit comme
    \emph{benign} par PolyPhen. 4 autres patients portent au moins deux
    variants hétérozygote sur ce gène, cependant la plupart sont des
    faux-sens prédit également comme \emph{benign} par PolyPhen. Bien que
    l'effet sur la protéine des variants portés par 5 des 7 patients
    portant au moins un variant sur ce gène soit incertains, les variants
    tronquant portés par les patients \emph{Ghs20} et \emph{Ghs21} ainsi
    que sa forte expression testiculaire et le fait que son implication
    dans la structure de la gaine fibreuse du flagelle spermatique ait été
    montrée en 2003 (Brown, Miki, Harper, \& Eddy,
    \protect\hyperlink{ref-Brown2003}{2003}) font de ce gène un excellent
    candidat pour expliquer le phénotype d'au moins 2 patients. Pour les
    autres, des analyses fonctionnelles seront nécessaires.
  \item
    \textbf{\emph{SART3}} : Un total de 4 patients portait \textbf{le
    même} variants homozygote. Ces patients étant tous issus du même
    projet de séquençage (Strasbourg 2012) ce variant pourrait
    parfaitement être un artefact dû au protocole de séquençage. Cette
    hypothèse est d'autant plus probable que l'ensemble des patients
    séquencés dans ce projet présentaient un phénotype MMAF, ainsi, aucun
    des individus de notre cohorte contrôle n'a donc pu servir à filtrer
    les variants artefactuels issus de ce projet. De plus, on peut
    constater que l'expression tissulaire de ce gène est ubiquitaire. Pour
    ces différentes raisons, ce gène n'a pas été retenu comme candidat
    dans nos analyses.
  \item
    \textbf{\emph{ICA1} et \emph{TRAV26-1}} : Ces deux gènes sont chacun
    retrouvés présentant des variants homozygotes tronquant chez 6 de nos
    patients. Pour chaque gène on peut noter que c'est à chaque fois le
    même variant qui est partagé par les trois patients ce qui est asses
    étonnant étant donné le fait que ces patients ne sont pas apparentés
    et n'ont pas été séquencés lors du même projet de séquençage. De plus,
    on peut observer que le patient et \emph{Ghs40} porte un variant sur
    chacun de ces deux gènes, de même qu'il en portait également un sur le
    gène \emph{SART3}. L'expression testiculaire de ces deux gènes
    apparait elle aussi très faible. De plus, la bibliographie du gène
    \emph{ICA1} révèle que celui-ci est principalement exprimé dans le
    pancréas (Mally, Cirulli, Hayek, \& Otonkoski,
    \protect\hyperlink{ref-Mally1996}{1996}, Stassi, Schloot, \&
    Pietropaolo (\protect\hyperlink{ref-Stassi1997}{1997})) et que
    celui-ci serait à la fois lié au diabète de type 1 (S. Martin et al.,
    \protect\hyperlink{ref-Martin1995}{1995}, R. Gaedigk, Duncan,
    Miyazaki, Robinson, \& Dosch
    (\protect\hyperlink{ref-Gaedigk1994}{1994}), S. Martin, Lampasona,
    Dosch, \& Pietropaolo (\protect\hyperlink{ref-Martin1996}{1996})) et
    au syndrome de Gougerot-Sjögren (S. Winer et al.,
    \protect\hyperlink{ref-Winer2002}{2002}). Malgré le nombre important
    de patients présentant des variants homozygotes tronquant sur ces deux
    gènes, ceux-ci n'ont pas été retenus comme candidat pour expliquer le
    phénotype MMAF de ces 3 patients.
  \item
    \textbf{\emph{BAZ1A}} : Sur ce gène, les patient \emph{Ghs18} et
    \emph{Ghs94} portent tous deux le même indel à l'état homozygote.
    Celui-ci entrainant un décalage du cadre de lecture induit la
    formation d'une protéine aberrente. L'absence de cette indel dans les
    trois bases de données laisse supposer une faible fréquence de ce
    variant dans la population générale. Bien que ce gène présente une
    faible expression testiculaire, une étude de 2013 effectuée chez la
    souris a démontré son rôle majeur dans la spermatogénèse, les souris
    \emph{KO} \emph{Baz1a}\^{}\{-/-\} présentaient de nombreux défauts tel
    que des spermatozoïdes non motiles, présentant une morphologie de la
    tête et du flagelle aberrante (Dowdle et al.,
    \protect\hyperlink{ref-Dowdle2013}{2013}) faisant ainsi de ce gène un
    bon candidat pour expliquer le phénotype de ces 2 patients.
  \item
    \textbf{\emph{CCDC129}} : Ce gène est retrouvé muté à l'état
    homozygote chez 2 patients porteur du même indel, non répertorié dans
    les bases de données, causant un décalage du cadre de lecture. On peut
    également constate que ce gène possède une expression testiculaire
    forte et exclusive faisant ainsi de ce gène un bon candidat malgré une
    littérature pauvre à son sujet.
  \item
    \textbf{\emph{NACA}} : 2 patients partagent le même variant homozygote
    causant un décalage du cadre de lecture sur le gène \emph{NACA}. On
    peut noter que les patient \emph{Ghs37} et \emph{Ghs37} portaient
    porteurs respectifs de variant homozygote tronquant sur les gène
    \emph{CCDC146} et \emph{ARMC2} portaient également ce même variant.
    L'ensemble de ces patients provenant du même projet de séquençage
    (Strasbourg 2012), cela laisse supposer que comme pour \emph{SART3},
    ce variant est artefactuel. On peut d'ailleurs noter que le patient
    \emph{Ghs41} était lui aussi porteur du variant sur \emph{SART3}
    renforçant ainsi l'hypothèse de l'erreur de séquençage. Pour ces
    raisons, ce gène n'a pas été retenu en tant que candidat.
  \end{enumerate}
  
  Dans cette analyse, 8 gène ont dans un premier temps été identifiés.
  Cependant, une analyse plus approfondie a fait que seuls les gènes
  \emph{BAZ1A}, \emph{CCDC129}, \emph{CFAP43} et \emph{FSIP2} ont été
  retenus. Les variants présent sur les gènes \emph{SART3} et \emph{NACA}
  étant probablement des artefacts dus aux erreurs de séquençage. Dès
  lors, seuls 4 des 8 gènes identifiés présentaient des arguments
  suffisamment convainquant pour être considéré comme responsable du
  phénotype MMAF. Ainsi, 18 de nos patients sont porteurs de variants sur
  l'un de ces gènes dont 13 à l'état homozygote avec notamment les
  patients \emph{Ghs102}, \emph{Ghs105}, \emph{Ghs126}, \emph{Ghs132},
  \emph{Ghs17}, \emph{Ghs18}, \emph{Ghs20}, \emph{Ghs21}, \emph{Ghs41},
  \emph{Ghs91} et \emph{Ghs94} porteur de variants homozygotes tronquants.
  Les gènes \emph{BAZ1A}, \emph{CCDC129}, \emph{CFAP43} et \emph{FSIP2}
  étant de bons candidats et les données génétiques des 11 patients
  précédemment cités étant suffisamment fortes, l'ensemble de leurs
  variant furent ainsi retirés de notre liste contenant désormais 1093
  variants et 962 gènes différents.
  
  \newpage  
  
  \begin{figure}
  
  {\centering \includegraphics{thesis_files/figure-latex/plotgrp2-1} 
  
  }
  
  \caption[Analyse des gènes sélectionnés dans l'Analyse n°2]{Analyse des gènes sélectionnés dans l'Analyse n°2 : Expression tissulaire des gènes retenus d'après les données du projet de transcriptome Illumina bodyMap. Résumé de l'Analyse 2, quantification du nombre de patients retrouvés mutés sur chacun des gènes retenus ainsi que du degré de confiance accordé à la cause génétique}\label{fig:plotgrp2}
  \end{figure}
  
  \newpage  
  
  \begin{landscape}
  \begin{longtable}[t]{lllllll}
  \caption{\label{tab:tabgrp2high}Analyse n°3 : List des gènes sur lesquels au moins deux patients portent une mutation tronquante non présents dans la liste ciliome}\\
  \toprule
  \multicolumn{1}{c}{ } & \multicolumn{1}{c}{ } & \multicolumn{2}{c}{Variant impact} & \multicolumn{3}{c}{Variant frequency} \\
  \cmidrule(l{2pt}r{2pt}){3-4} \cmidrule(l{2pt}r{2pt}){5-7}
  Patient & Gene & HGVSc, HGVSp & Consequence & ESP & 1KG & ExAC\\
  \midrule
  Ghs91 & CCDC129 & . ; . & frameshift & . & . & .\\
  Ghs132 & CCDC129 & . ; . & frameshift & . & . & .\\
  Ghs17 & CFAP43 & c.2658G>A ; p.Trp886Ter & stop gained & 2e-04 & . & 9.88e-05\\
  Ghs41 & CFAP43 & c.2680C>T ; p.Arg894Ter & stop gained & . & . & 8.24e-06\\
  Ghs102 & CFAP43 & c.3882delA ; p.Glu1294AspfsTer47 & frameshift & . & . & .\\
  \addlinespace
  Ghs105 & CFAP43 & c.3541-2A>C ; . & splice acceptor & . & . & .\\
  Ghs126 & CFAP43 & c.3352C>T ; p.Arg1118Ter & stop gained & . & . & 3.29e-05\\
  Ghs20 & FSIP2 & c.2549\_2550insA ; p.Asn850LysfsTer4 & frameshift & . & . & .\\
  Ghs21 & FSIP2 & c.1177delG ; p.Gln393LysfsTer13 & frameshift & . & . & .\\
  Ghs40 & TRAV26-1 & c.75\_76insC ; p.Ser26LeufsTer6 & frameshift & . & . & .\\
  \addlinespace
  Ghs43 & TRAV26-1 & c.75\_76insC ; p.Ser26LeufsTer6 & frameshift & . & . & .\\
  Ghs55 & TRAV26-1 & c.75\_76insC ; p.Ser26LeufsTer6 & frameshift & . & . & .\\
  \bottomrule
  \end{longtable}
  \end{landscape}
  
  \newpage
  
  \begin{landscape}
  \begin{longtable}[t]{lllllllll}
  \caption{\label{tab:tabgrp2moderate}Analyse n°3 : Liste des patients portant un variant non troquant homozygote sur un des gènes suivant : *BAZ1A*, *CCDC129*, *CFAP43*, *FSIP2*, *ICA1*, *NACA*, *SART3*  et  *TRAV26-1*}\\
  \toprule
  \multicolumn{1}{c}{ } & \multicolumn{1}{c}{ } & \multicolumn{4}{c}{Variant impact} & \multicolumn{3}{c}{Variant frequency} \\
  \cmidrule(l{2pt}r{2pt}){3-6} \cmidrule(l{2pt}r{2pt}){7-9}
  Patient & Gene & HGVSc, HGVSp & Consequence & SIFT & PolyPhen & ESP & 1KG & ExAC\\
  \midrule
  Ghs25 & CFAP43 & c.2141+5T>A ; . & splice region & . & . & . & . & .\\
  Ghs131 & FSIP2 & . ; p.Ala86Val & missense & . & benign & . & . & 0.00121\\
  \bottomrule
  \end{longtable}
  \end{landscape}
  
  \begin{landscape}
  \begin{longtable}[t]{llp{11em}llllll}
  \caption{\label{tab:tabgrp2low}Analyse n°3 : Liste des patients portant un variant non troquant homozygote sur un des gènes suivant : *BAZ1A*, *CCDC129*, *CFAP43*, *FSIP2*, *ICA1*, *NACA*, *SART3*  et  *TRAV26-1*}\\
  \toprule
  \multicolumn{1}{c}{ } & \multicolumn{1}{c}{ } & \multicolumn{4}{c}{Variant impact} & \multicolumn{3}{c}{Variant frequency} \\
  \cmidrule(l{2pt}r{2pt}){3-6} \cmidrule(l{2pt}r{2pt}){7-9}
  Patient & Gene & HGVSc, HGVSp & Consequence & SIFT & PolyPhen & ESP & 1KG & ExAC\\
  \midrule
  Ghs40 & FSIP2 & . ; p.Asn495Ile & missense & . & benign & . & 0.0056 & 0.00157\\
  Ghs40 & FSIP2 & . ; p.Ile2960Met & missense & . & proba damaging & . & . & .\\
  Ghs40 & FSIP2 & . ; p.Lys5822Ile & missense & . & benign & 0.0034 & 0.0056 & 0.0019\\
  Ghs92 & FSIP2 & . ; p.Asn495Ile & missense & . & benign & . & 0.0056 & 0.00157\\
  Ghs92 & FSIP2 & . ; p.Lys5822Ile & missense & . & benign & 0.0034 & 0.0056 & 0.0019\\
  \addlinespace
  Ghs95 & FSIP2 & . ; p.Asn495Ile & missense & . & benign & . & 0.0056 & 0.00157\\
  Ghs95 & FSIP2 & . ; p.Lys5822Ile & missense & . & benign & 0.0034 & 0.0056 & 0.0019\\
  Ghs101 & FSIP2 & c.182T>C ; p.Leu61Pro & missense & . & unknown & . & . & .\\
  Ghs101 & FSIP2 & c.925A>T ; p.Arg309Cys & missense & . & benign & . & . & .\\
  Ghs132 & CFAP43 & c.1300\_1301insT ; p.Leu435SerfsTer26 & frameshift & . & . & . & . & .\\
  Ghs132 & CFAP43 & c.1040G>C ; p.Val347Ala & missense & tolerated & possib damaging & 2e-04 & . & 7.41e-05\\
  \bottomrule
  \end{longtable}
  \end{landscape}
  
  \newpage
  
  \paragraph{Analyse n°4}\label{analyse-n4}
  
  Dans cette troisième analyse, nous avons sélectionné à nouveau les gènes
  \textbf{présents dans la liste cilliome} en conservant cette fois-ci
  ceux sur lesquels \textbf{un seul} de nos patients présentaient au moins
  1 variant tronquant à l'état \textbf{homozygote}. Les 4 gènes suivant
  correspondaient à ces critères : \emph{C21orf59}, \emph{C6orf118},
  \emph{CCDC65} et \emph{SPEF2} tous présentant de fortes preuves
  indiquant leur appartenance au ciliome humain.
  
  \begin{enumerate}
  \def\labelenumi{\arabic{enumi}.}
  \item
    \textbf{\emph{C6orf118}} : Ce gène a été retrouvé muté à l'état
    homozygote chez le patient et \emph{Ghs40} qui portait une
    substitution entrainant la formation d'un codon stop prématuré. Il
    faut noter que ce même patient portait également des variants
    homozygotes tronquant sur les gènes \emph{SART3}, \emph{TRAV26-1}, et
    \emph{ICA1} cependant la forte expression testiculaire de ce gène en
    fait un meilleur candidat. Le patient et \emph{Ghs27} quant à lui
    porte deux variants faux-sens hétérozygote Le premier étant prédit
    probably \emph{damaging} par PolyPhen et \emph{tolerated low
    confidence} par SIFT tandis que le second est prédit \emph{possibly
    damaging} et \emph{tolerated}, il est difficile de se prononcer quant
    à l'effet délétère de ces deux variants sans effectuer d'analyses
    fonctionnelles. Il faut noter que \emph{C6orf118} présente une forte
    expression dans le poumon. De plus, ce gène a récemment été décrit
    comme étant associé au phénotype de tuberculose pulmonaire (E. P.
    Hong, Go, Kim, \& Park, \protect\hyperlink{ref-Hong2017}{2017}).
    Cependant cela n'est en rien contradictoire avec le phénotype MMAF de
    ces 2 patients, le poumon comprenant de nombreuses cellules ciliées,
    notamment au niveau de l'épithélium respiratoire, il n'est donc pas
    surprenant que des gènes du flagelle aient également une fonction au
    sein d'autres organes ciliés.
  \item
    \textbf{\emph{C21orf59} et \emph{CCDC65}} : Les patients \emph{Ghs127}
    et \emph{Ghs88} porte tout deux des indels induisant un décalage du
    cadre de lecture, le premier sur le gène \emph{C21orf59}, le second
    sur \emph{CCDC65} deux gènes déjà connu pour être impliqué dans la
    formations des cils. En effet, la protéine NYD-SP28 (ancien nom de
    CCDC65) avait déjà caractérisé comme faisant partie du flagelle
    spermatique (Y. Zheng et al., \protect\hyperlink{ref-Zheng2006}{2006})
    alors que \emph{C21orf59} contrôlerait à la fois la mobilité et la
    polarisation des cils chez \emph{zebrafish} (Jaffe et al.,
    \protect\hyperlink{ref-Jaffe2016}{2016}). On note aussi que ces deux
    gènes ont été associés à des pathologies ciliaires chez le
    \emph{zebrafish}, \emph{Chlamydomonas} ainsi que chez l'humain,
    pouvant entrainer un phénotype de dyskinésie ciliaire primaire, de
    même, la protéine FBB18 du gène orthologue de C21orf59 régulerait la
    motilité du flagelle chez \emph{Chlamydomonas} (Austin-Tse et al.,
    \protect\hyperlink{ref-Austin-Tse2013}{2013}). Ainsi, les arguments
    génétiques associés aux implications déjà avérées de ces 2 gènes dans
    la ciliogénèse font de \emph{CCDC65} et \emph{c21orf59} des excellents
    candidats pour expliquer les phénotype MMAF de nos 2 patients malgré
    l'expression non spécifique au testicule du gène \emph{C21orf59}.
  \item
    \textbf{\emph{SPEF2}} : Ce gène est retrouvé muté à l'état homozygote
    chez le patient et \emph{Ghs131} qui porte un indel créant un décalage
    du cadre de lecture. Malgré son expression non spécifique au
    testicule, plusieurs études ont démontré le rôle important de la
    protéine SPEF2 chez la souris dans la formation et la structure du
    flagelle spermatique de même que l'absence de cette protéine
    entrainait la formation d'un flagelle court et désorganisé (Sironen et
    al., \protect\hyperlink{ref-Sironen2010}{2010}, Sironen et al.
    (\protect\hyperlink{ref-Sironen2011}{2011})) faisant ainsi de ce gène
    un excellent candidat pour le phénotype MMAF du patient et
    \emph{Ghs131}.
  \end{enumerate}
  
  Analyser les gènes de la liste ciliome sur lesquels \textbf{un seul}
  patient portait un variant tronquant à l'état homozygote nous a permis
  d'identifier 4 nouveaux gènes : \emph{C21orf59}, \emph{C6orf118},
  \emph{CCDC65} et \emph{SPEF2}. Ainsi, 4 patients portaient un variant
  homozygote tronquant sur l'un d'eux tandis qu'un autre patient portait
  deux variant faux-sens hétérozygotes sur le gène \emph{C6orf118}. Parmi
  ces 4, seuls \emph{CCDC65} et \emph{C6orf118} présentaient une forte
  expression testiculaire, cependant tous avaient déjà été associés à la
  formation du cil chez l'humain ou chez une autre espèce animale. Bien
  que des analyses supplémentaires soient nécessaires afin de pouvoir
  affirmer que les deux variants hétérozygotes portés par le patients et
  \emph{Ghs27} sont bien responsables de son phénotype, les arguments
  génétiques ainsi que les informations fournies par la littérature nous
  ont permis d'affirmer que ces gènes sont responsables des phénotypes des
  patients \emph{Ghs127}, \emph{Ghs131}, \emph{Ghs40} et \emph{Ghs88}. Les
  données de variants de ces derniers ont ainsi été retirés réduisant à
  nouveau notre liste à 934 variants et 821 gènes différents.
  
  \newpage  
  
  \begin{figure}
  
  {\centering \includegraphics{thesis_files/figure-latex/plotgrp3-1} 
  
  }
  
  \caption[Analyse des gènes sélectionnés dans l'Analyse n°4]{Analyse des gènes sélectionnés dans l'Analyse n°4 : **A** : Expression tissulaire des gènes retenus dans cette analyse d'après les données du projet de transcriptome Illumina bodyMap. **B** : Résumé de l'Analyse 3, quantification du nombre de patients retrouvés mutés sur chacun des gènes retenus ainsi que du degré de confiance accordé à la cause génétique}\label{fig:plotgrp3}
  \end{figure}
  
  \newpage
  
  \begin{landscape}
  \begin{longtable}[t]{llllllll}
  \caption{\label{tab:tabgrp3high}Analyse n°4 : List des gènes présents dans la liste ciliome sur lesquels un seul patient portent une mutation homozygote tronquante}\\
  \toprule
  \multicolumn{1}{c}{ } & \multicolumn{1}{c}{ } & \multicolumn{1}{c}{ } & \multicolumn{2}{c}{Variant impact} & \multicolumn{3}{c}{Variant frequency} \\
  \cmidrule(l{2pt}r{2pt}){4-5} \cmidrule(l{2pt}r{2pt}){6-8}
  Patient & Gene & Evidence & HGVSc, HGVSp & Consequence & ESP & 1KG & ExAC\\
  \midrule
  Ghs40 & C6orf118 & Strong & c.1114C>T ; p.Arg372Ter & stop gained & . & . & 8.24e-06\\
  Ghs127 & CCDC65 & Strong & c.1208delG ; p.Asn403IlefsTer9 & frameshift & . & . & .\\
  Ghs131 & SPEF2 & Strong & c.3240delG ; p.Phe1080LeufsTer2 & frameshift & . & . & .\\
  \bottomrule
  \end{longtable}
  \end{landscape}
  
  \begin{landscape}
  \begin{longtable}[t]{lllllllll}
  \caption{\label{tab:tabgrp3low}Analyse n°4 : Liste des patients portant au moins deux variants hétérozygotes sur un des gènes suivant : *C21orf59*, *C6orf118*, *CCDC65*  et  *SPEF2*}\\
  \toprule
  \multicolumn{1}{c}{ } & \multicolumn{1}{c}{ } & \multicolumn{4}{c}{Variant impact} & \multicolumn{3}{c}{Variant frequency} \\
  \cmidrule(l{2pt}r{2pt}){3-6} \cmidrule(l{2pt}r{2pt}){7-9}
  Patient & Gene & HGVSc, HGVSp & Consequence & SIFT & PolyPhen & ESP & 1KG & ExAC\\
  \midrule
  Ghs27 & C6orf118 & c.1372G>C ; p.Tyr458His & missense & tolerated & proba damaging & 8e-04 & 0.0042 & 0.000297\\
  Ghs27 & C6orf118 & c.98C>T ; p.Pro33Leu & missense & tolerated & possib damaging & 0.0038 & 9e-04 & 0.000684\\
  \bottomrule
  \end{longtable}
  \end{landscape}
  
  \newpage
  
  \paragraph{Analyse n°5}\label{analyse-n5}
  
  Pour finir nous avons dans cette analyse sélectionné l'ensemble des
  variants chevauchant des gènes \textbf{absents dans la liste cilliome}
  sur lesquels \textbf{un seul} de nos patients présentaient au moins 1
  variant tronquant à l'état \textbf{homozygote}. Cela nous a permis
  d'obtenir une liste de 59 gènes différents retrouvés mutés chez 30 de
  nos patients.
  
  En raison de la grande quantité de gène, nous nous sommes tout d'abord
  concentré sur ceux retrouvés mutés à l'état homozygote chez au moins
  deux patients. Ainsi nous avons obtenus une liste de 6 gènes :
  \emph{CCDC66}, \emph{CCL3L3}, \emph{CFHR1}, \emph{DPY19L2P2},
  \emph{RNFT2} et \emph{RP6-206I17.1}
  
  \begin{enumerate}
  \def\labelenumi{\arabic{enumi}.}
  \tightlist
  \item
    \textbf{\emph{CCDC66}} : Le gène \emph{CCDC66} a été retrouvé muté à
    l'état homozygote chez \ldots{} patients parmi lesquels \ldots{}
    portait un variant tronquant entrainant un décalage du cadre de
    lecture et retrouvé dans aucune des bases de données. Malgré une
    expression testiculaire faible, l'implication récente de ce gène dans
    la ciliogénèse (Conkar et al.,
    \protect\hyperlink{ref-Conkar2017}{2017}) fait de ce gène un bon
    candidat bien que certaines études aient déjà démontré l'implication
    de celui-ci dans des pathologies rétiniennes à la fois chez le chien
    (Dekomien et al., \protect\hyperlink{ref-Dekomien2010}{2010}) mais
    aussi chez l'humain (Gerding et al.,
    \protect\hyperlink{ref-Gerding2011}{2011}, Khan et al.
    (\protect\hyperlink{ref-Khan2017}{2017})). En effet, ces anomalies
    rétiniennes.
  \end{enumerate}
  
  \newpage
  
  \begin{center}\includegraphics{thesis_files/figure-latex/unnamed-chunk-9-1} \end{center}
  
  \newpage
  
  \begin{center}\includegraphics{thesis_files/figure-latex/unnamed-chunk-10-1} \end{center}
  
  \newpage
  
  \begin{landscape}
  \begin{longtable}[t]{lllllll}
  \caption{\label{tab:tabgrp4high}Analyse n°5 : Liste des patients portant au moins un variant homozygote tronquant sur le gène sur l'un des 59 genes identifés dans cette analyse}\\
  \toprule
  \multicolumn{1}{c}{ } & \multicolumn{1}{c}{ } & \multicolumn{2}{c}{Variant impact} & \multicolumn{3}{c}{Variant frequency} \\
  \cmidrule(l{2pt}r{2pt}){3-4} \cmidrule(l{2pt}r{2pt}){5-7}
  Patient & Gene & HGVSc, HGVSp & Consequence & ESP & 1KG & ExAC\\
  \midrule
  Ghs95 & ABCA10 & c.1356\_1357delGG ; p.Ile453LeufsTer2 & frameshift & . & . & .\\
  Ghs52 & ACSBG2 & . ; . & splice acceptor & 0.0028 & 0.0019 & 0.00084\\
  Ghs92 & ALDOA & c.124\_125delAG ; p.Gln42AspfsTer30 & frameshift & . & . & .\\
  Ghs47 & CCDC9 & c.720+2C>G ; . & splice donor & . & . & 0.00343\\
  Ghs96 & CFHR1 & c.790+1T>A ; . & splice donor & 0.0093 & 0.0042 & 0.00231\\
  \addlinespace
  Ghs38 & CHST4 & c.247C>T ; p.Lys83Ter & stop gained & . & . & 1.65e-05\\
  Ghs87 & CLEC12A & c.121+1T>A ; . & splice donor & 0.0013 & 5e-04 & 0.00119\\
  Ghs42 & CNTLN & c.3115-2delT ; . & splice acceptor & . & . & .\\
  Ghs28 & DPY19L2P2 & . ; . & splice acceptor & . & . & .\\
  Ghs55 & EPB41L4A-AS2 & c.378delG ; p.Arg127GlyfsTer172 & frameshift & . & . & .\\
  \addlinespace
  Ghs42 & FCN3 & c.347delA ; p.Leu117SerfsTer65 & frameshift & . & . & .\\
  Ghs134 & GPR142 & c.215delG ; p.Pro72HisfsTer29 & frameshift & . & . & .\\
  Ghs133 & HIST1H4J & c.78dupA ; p.Ile27AsnfsTer85 & frameshift & . & . & .\\
  Ghs25 & HOGA1 & c.208A>T ; p.Arg70Ter & stop gained & . & . & 2.47e-05\\
  Ghs87 & KRT74 & c.748-2T>G ; . & splice acceptor & 0.0017 & 0.0014 & 0.000997\\
  \addlinespace
  Ghs55 & LRRC9 & c.3256A>T ; p.Arg1086Ter & stop gained & . & . & .\\
  Ghs134 & MCPH1 & c.2221G>T ; p.Arg741Ter & stop gained & 1e-04 & . & 2.48e-05\\
  Ghs38 & MTCH1 & c.28delT ; p.Trp11GlyfsTer82 & frameshift & . & . & .\\
  Ghs96 & NAT16 & c.808delT ; p.Ser270AlafsTer87 & frameshift & . & . & .\\
  Ghs125 & OR13A1 & c.804dupT ; p.Tyr269LeufsTer66 & frameshift & . & . & .\\
  \addlinespace
  Ghs90 & OR51H1P & c.89delG ; p.Leu33TrpfsTer16 & frameshift & . & . & .\\
  Ghs25 & PI4K2A & . ; p.Arg70Ter & stop gained & . & . & 2.47e-05\\
  Ghs24 & PLA2R1 & c.1953delA ; p.Ala653GlnfsTer35 & frameshift & . & . & .\\
  Ghs31 & PPFIBP2 & . ; . & frameshift & . & . & 2.47e-05\\
  Ghs33 & PSG4 & c.931C>T ; p.Glu311Ter & stop gained & 0.0024 & 0.0014 & 0.0054\\
  \addlinespace
  Ghs51 & PSPHP1 & . ; . & splice donor & . & 0.0051 & .\\
  Ghs87 & PTGR1 & . ; . & splice donor & . & . & 0.000231\\
  Ghs43 & RP6-206I17.1 & n.211-2N>G ; . & splice acceptor & . & . & .\\
  Ghs128 & RYK & c.56\_57insC ; p.Arg20GlnfsTer59 & frameshift & . & . & .\\
  Ghs87 & SDHAP2 & n.777+1C>A ; . & splice donor & . & . & 0.00896\\
  \addlinespace
  Ghs92 & SDSL & c.685delT ; p.Leu229TrpfsTer30 & frameshift & . & . & .\\
  Ghs97 & SERPINA10 & c.262G>T ; p.Arg88Ter & stop gained & 0.0053 & 0.0032 & 0.00749\\
  Ghs23 & SIGLECL1 & c.141delG ; p.Val48TrpfsTer10 & frameshift & . & . & .\\
  Ghs96 & SLC2A8 & c.802G>T ; p.Gln268Ter & stop gained & 7e-04 & 9e-04 & 0.000948\\
  Ghs90 & TRAP1 & c.139G>T ; p.Arg47Ter & stop gained & 1e-04 & . & 8.24e-05\\
  \addlinespace
  Ghs87 & TTLL2 & c.47+1T>C ; . & splice donor & . & . & .\\
  Ghs107 & UNC93A & c.676C>T ; p.Arg226Ter & stop gained & 0.0024 & 0.0028 & 0.00236\\
  Ghs87 & ZNF438 & c.244dupT ; p.Met83AspfsTer33 & frameshift & . & . & .\\
  Ghs23 & ZNF528 & c.381delG ; p.Ile129PhefsTer32 & frameshift & . & . & .\\
  Ghs38 & ZNF862 & . ; . & splice donor & . & . & .\\
  \bottomrule
  \end{longtable}
  \end{landscape}
  
  \begin{landscape}
  \begin{longtable}[t]{lllllllll}
  \caption{\label{tab:tabgrp4moderate}Analyse n°5 : Liste des patients portant au moins un variant homozygote non tronquant sur le gène sur l'un des 59 genes identifés dans cette analyse}\\
  \toprule
  \multicolumn{1}{c}{ } & \multicolumn{1}{c}{ } & \multicolumn{4}{c}{Variant impact} & \multicolumn{3}{c}{Variant frequency} \\
  \cmidrule(l{2pt}r{2pt}){3-6} \cmidrule(l{2pt}r{2pt}){7-9}
  Patient & Gene & HGVSc, HGVSp & Consequence & SIFT & PolyPhen & ESP & 1KG & ExAC\\
  \midrule
  Ghs28 & DPY19L2P2 & n.3419-4\_3419-3insGC ; . & splice region & . & . & . & . & .\\
  Ghs28 & DPY19L2P2 & n.3419-5\_3419-4insAT ; . & splice region & . & . & . & . & .\\
  Ghs36 & CCL3L3 & c.272G>C ; p.Leu91Pro & missense & deleterious & benign & . & . & .\\
  Ghs90 & CCDC66 & c.11+6T>G ; . & splice region & . & . & . & . & 4.03e-05\\
  \bottomrule
  \end{longtable}
  \end{landscape}
  
  \newpage
  
  \newpage
  
  \subsubsection{Discussion}\label{discussion-1}
  
  L'analyse de cette cohorte de 62 patients MMAF nous dans un premier
  temps permis de confirmer l'importance de l'implication du gène
  \emph{DNAH1} dans ce phénotype grâce à 9 patients présentant des
  variants sur ce gène dont 3 à l'état homozygote (dont 1 tronquant). Elle
  nous a également permis d'identifier 12 nouveaux gène candidats :
  \emph{ARMC2}, \emph{BAZ1A}, \emph{C21orf59}, \emph{C6orf118},
  \emph{CCDC129}, \emph{CCDC146}, \emph{CCDC65}, \emph{CFAP43},
  \emph{CFAP44}, \emph{FSIP2}, \emph{SPEF2} et \emph{TTC29}. Ainsi, 27 de
  nos patients soit 43.5\% de la cohorte présentaient au moins 1 variant
  homozygote sur l'un de ces gènes, et, chez 23 d'entre eux, ce variant
  induisaient un effet tronquant sur la séquence protéique. Les 4 autres
  portaint tous au moins deux variants hétérozygotes, tronquants ou non,
  sur un de ces gènes (\textbf{Figure : }\ref{fig:plotresumebigmmaf} -
  \textbf{A}).
  
  Parmi cet ensemble de patients, il faut noter que 5 d'entre eux porte
  des variants sur plusieurs des gènes candidats que nous avons identifiés
  (\textbf{Table :} \ref{tab:tabtwocandidats}, \textbf{Figure :
  }\ref{fig:plotresumebigmmaf} - \textbf{B}).
  
  Cependant, parmi ces différents variants certains semblent plus
  probables pour expliquer le phénotype :
  
  \begin{enumerate}
  \def\labelenumi{\arabic{enumi}.}
  \tightlist
  \item
    \textbf{Patient \emph{Ghs131} :} Ce patient porte à la fois un variant
    homozygote sur le gène \emph{FSIP2} et un autre sur le gène
    \emph{SPEF2}. Cependant, au vu de l'effet tronquant de celui présent
    sur le gène \emph{SPEF2}, il est plus probable que ce soit celui-ci
    qui soit responsable du phénotype de ce patient.\\
  \item
    \textbf{Patient \emph{Ghs132} :} Ce patient porte à la fois un variant
    homozygote causant un décalage du cadre de lecture sur le gène
    \emph{CCDC129} et deux variants hétérozygotes sur le gène
    \emph{CFAP43}. Au vu de l'incertitude lié au phasage des variants
    hétérozygote et de l'impact délétère de celui présent sur le gène
    \emph{CCDC129}, c'est ce dernier qui sera retenu pour ce patient dans
    un premier temps.\\
  \item
    \textbf{Patient \emph{Ghs40} :} Ce patient est porteur de 3 variants
    faux-sens hétérozygotes sur le gène \emph{FSIP2} et d'un variant
    créant un codon stop prématuré homozygote sur \emph{C6orf118}. En
    tenant compte du génotype et de l'efet délétère du variant, il est
    plus probable que celui chevauchant le gène \emph{C6orf118} soit
    responsable du phénotype.\\
  \item
    \textbf{Patient \emph{Ghs88} :} Comme le patient\emph{Ghs40}, celui-ci
    porte à la fois deux variants hétérozygotes dur le gène \emph{DNAH1}
    et un variant tronquant sur \emph{C21orf59}. Pour les même raison que
    celles précédement évoquées, ces ce dernier qui sera retenu.\\
  \item
    \textbf{Patient \emph{Ghs95} :} Pour les même raison, c'est pour ce
    patient le faux-sens homozygote présant sur le gène \emph{DNAH1} qui
    sera ici conservé.
  \end{enumerate}
  
  Ainsi, cette analyse révèle l'efficacité de notre pipeline puisqu'elle a
  permis d'identifier un gène candidat pour 38 des 62 patients de notre
  cohorte, soit 61.3\% de nos patients. Pour les 24 patients pour lesquels
  aucun candidat n'a été identifié, des analyses individuelles
  complémentaires sont nécessaires.
  
  Une partie de ces différents résultats ont déjà été publiés dans deux
  articles dont je suis co-auteur :
  
  \begin{enumerate}
  \def\labelenumi{\arabic{enumi}.}
  \item
    \textbf{Whole exome cohort study and analysis of mouse and Trypanosoma
    models demonstrate the importance of WDR proteins in flagellogenesis
    and male fertility}, \emph{Nat Genet} (soumis) : Cette article
    présente nos différents résultats dans la caractérisation des gènes
    \emph{WDR96} et \emph{WDR52} ainsi que les différentes preuves de leur
    implication dans le phénotype MMAF.
  \item
    \protect\hyperlink{famdnah1}{\textbf{Whole-exome sequencing of
    familial cases of multiple morphological abnormalities of the sperm
    flagella (MMAF) reveals new DNAH1 mutations}} : En plus des résultats
    évoqués précédemment pour la famille MMAF2, cet article inclus ceux de
    \ldots{} patients de cette cohorte présentant des variants sur le gène
    \emph{DNAH1}
  \end{enumerate}
  
  Pour les autres, notre équipe travaille actuellement à la
  caractérisation des différents gènes afin de comprendre les processus
  moléculaires
  
  \newpage
  
  \begin{figure}
  
  {\centering \includegraphics{thesis_files/figure-latex/plotresumebigmmaf-1} 
  
  }
  
  \caption[Conclusion des analyses WES de notre large cohorte MMAF, liste des gènes candidats]{Conclusion des analyses WES de notre large cohorte MMAF, liste des gènes candidats : **A** : Quantification du nombre de patient portant un ou plusieurs variants sur un des gène candidat. La couleur des barres dépend du type et du génotype du ou des variants portés par chaque patient. La barre bleue indique les patients pour lesquels aucun candidat n'a été identifié. **B** : Nombre de candidat potentiel par patient (parmi ceux pour lesquels au moins un gène candidat a été identifié)}\label{fig:plotresumebigmmaf}
  \end{figure}
  
  \newpage
  
  \begin{longtable}[t]{lllll}
  \caption{\label{tab:tabtwocandidats}Liste des patients portant au moins 1 variant sur deux gènes candidats}\\
  \toprule
  Patient & Gene & HGVSc, HGVSp & Consequence & Genotype\\
  \midrule
  Ghs131 & FSIP2 & . ; p.Ala86Val & missense & Homozygous\\
  Ghs131 & SPEF2 & c.3240delG ; p.Phe1080LeufsTer2 & frameshift & Homozygous\\
  Ghs132 & CCDC129 & . ; . & frameshift & Homozygous\\
  Ghs132 & CFAP43 & c.1300\_1301insT ; p.Leu435SerfsTer26 & frameshift & Heterozygous\\
  Ghs132 & CFAP43 & c.1040G>C ; p.Val347Ala & missense & Heterozygous\\
  \addlinespace
  Ghs40 & C6orf118 & c.1114C>T ; p.Arg372Ter & stop gained & Homozygous\\
  Ghs40 & FSIP2 & . ; p.Asn495Ile & missense & Heterozygous\\
  Ghs40 & FSIP2 & . ; p.Ile2960Met & missense & Heterozygous\\
  Ghs40 & FSIP2 & . ; p.Lys5822Ile & missense & Heterozygous\\
  Ghs88 & C21orf59 & c.79delA ; p.Phe28LeufsTer47 & frameshift & Homozygous\\
  \addlinespace
  Ghs88 & DNAH1 & c.2209C>A ; p.Val737Met & missense & Heterozygous\\
  Ghs88 & DNAH1 & c.3877T>A ; p.Asp1293Asn & missense & Heterozygous\\
  Ghs95 & DNAH1 & c.9278C>G ; p.Ala3093Gly & missense & Homozygous\\
  Ghs95 & FSIP2 & . ; p.Asn495Ile & missense & Heterozygous\\
  Ghs95 & FSIP2 & . ; p.Lys5822Ile & missense & Heterozygous\\
  \bottomrule
  \end{longtable}
  
  \newpage
  
  \newpage 
  
  \section{Conclusion}\label{conclusion}
  
  Au cours de ces différentes études nous avons pu identifier les variants
  pouvant expliquer les phénotypes de \ldots{} des différents patients que
  nous avons analysé que ce soit au sein d'études familiales ou bien au
  sein de plus large cohorte composés d'individus non apparentés. Bien que
  ces résultats soient satisfaisant, il faut noter que pour \ldots{}
  patients, soit \ldots{} \% d'entre eux aucun candidat n'a pu à ce jour
  être identifié. Pour ces patients, le WES permets désormais de nouvelles
  approches permettant d'identifier de larges variants structuraux
  (insertion ou délétions) pouvant eux aussi être responsable du phénotype
  qui ne sont pas détectés par les analyses classiques. Néanmoins, il
  semble clair que des avancés soient encore nécessaires afin d'améliorer
  l'efficacité de ce genre d'étude notamment en créant de nouveaux filtres
  permettant ainsi d'épurer les listes de variants facilitant ainsi
  l'identification des gènes candidats.
  
  \chapter*{References}\label{references}
  \addcontentsline{toc}{chapter}{References}
  
  \hypertarget{refs}{}
  \hypertarget{ref-1000GenomesProjectConsortium2015}{}
  1000 Genomes Project Consortium, T. 1. G. P., Auton, A., Brooks, L. D.,
  Durbin, R. M., Garrison, E. P., Kang, H. M., \ldots{} Abecasis, G. R.
  (2015). A global reference for human genetic variation. \emph{Nature},
  \emph{526}(7571), 68--74. \url{http://doi.org/10.1038/nature15393}
  
  \hypertarget{ref-Adzhubei2010}{}
  Adzhubei, I. A., Schmidt, S., Peshkin, L., Ramensky, V. E., Gerasimova,
  A., Bork, P., \ldots{} Sunyaev, S. R. (2010). A method and server for
  predicting damaging missense mutations. \emph{Nature Methods},
  \emph{7}(4), 248--9. \url{http://doi.org/10.1038/nmeth0410-248}
  
  \hypertarget{ref-Aken2017}{}
  Aken, B. L., Achuthan, P., Akanni, W., Amode, M. R., Bernsdorff, F.,
  Bhai, J., \ldots{} Flicek, P. (2017). Ensembl 2017. \emph{Nucleic Acids
  Research}, \emph{45}(D1), D635--D642.
  \url{http://doi.org/10.1093/nar/gkw1104}
  
  \hypertarget{ref-Amberger2011}{}
  Amberger, J., Bocchini, C., \& Hamosh, A. (2011). A new face and new
  challenges for Online Mendelian Inheritance in Man (OMIM). \emph{Human
  Mutation}, \emph{32}(5), 564--567.
  \url{http://doi.org/10.1002/humu.21466}
  
  \hypertarget{ref-Amdani2013}{}
  Amdani, S. N., Jones, C., \& Coward, K. (2013). Phospholipase C zeta
  (PLC\(\zeta\)): Oocyte activation and clinical links to male factor
  infertility. \emph{Advances in Biological Regulation}, \emph{53}(3),
  292--308. \url{http://doi.org/10.1016/j.jbior.2013.07.005}
  
  \hypertarget{ref-Arnaiz2015}{}
  Arnaiz, O., Cohen, J., Tassin, A., \& Koll, F. (2015). Remodeling Cildb,
  a popular database for cilia and links for ciliopathies. \emph{Cilia},
  \emph{4}, P21. \url{http://doi.org/10.1186/2046-2530-4-S1-P21}
  
  \hypertarget{ref-Austin-Tse2013}{}
  Austin-Tse, C., Halbritter, J., Zariwala, M. A., Gilberti, R. M., Gee,
  H. Y., Hellman, N., \ldots{} Hildebrandt, F. (2013). Zebrafish
  Ciliopathy Screen Plus Human Mutational Analysis Identifies C21orf59 and
  CCDC65 Defects as Causing Primary Ciliary Dyskinesia. \emph{American
  Journal of Human Genetics}, \emph{93}(4), 672--86.
  \url{http://doi.org/10.1016/j.ajhg.2013.08.015}
  
  \hypertarget{ref-Baker2004}{}
  Baker, K. E., \& Parker, R. (2004). Nonsense-mediated mRNA decay:
  terminating erroneous gene expression. \emph{Current Opinion in Cell
  Biology}, \emph{16}(3), 293--9.
  \url{http://doi.org/10.1016/j.ceb.2004.03.003}
  
  \hypertarget{ref-BenKhelifa2014}{}
  Ben Khelifa, M., Coutton, C., Zouari, R., Karaouzène, T., Rendu, J.,
  Bidart, M., \ldots{} Ray, P. F. (2014). Mutations in DNAH1, which
  encodes an inner arm heavy chain dynein, lead to male infertility from
  multiple morphological abnormalities of the sperm flagella.
  \emph{American Journal of Human Genetics}, \emph{94}(1), 95--104.
  \url{http://doi.org/10.1016/j.ajhg.2013.11.017}
  
  \hypertarget{ref-Brown2003}{}
  Brown, P. R., Miki, K., Harper, D. B., \& Eddy, E. M. (2003). A-Kinase
  Anchoring Protein 4 Binding Proteins in the Fibrous Sheath of the Sperm
  Flagellum. \emph{Biology of Reproduction}, \emph{68}(6), 2241--2248.
  \url{http://doi.org/10.1095/biolreprod.102.013466}
  
  \hypertarget{ref-Chang2007}{}
  Chang, Y.-F., Imam, J. S., \& Wilkinson, M. F. (2007). The
  Nonsense-Mediated Decay RNA Surveillance Pathway. \emph{Annual Review of
  Biochemistry}, \emph{76}(1), 51--74.
  \url{http://doi.org/10.1146/annurev.biochem.76.050106.093909}
  
  \hypertarget{ref-Chung2014}{}
  Chung, M.-I., Kwon, T., Tu, F., Brooks, E. R., Gupta, R., Meyer, M.,
  \ldots{} Wallingford, J. B. (2014). Coordinated genomic control of
  ciliogenesis and cell movement by RFX2. \emph{ELife}, \emph{3}, e01439.
  \url{http://doi.org/10.7554/eLife.01439}
  
  \hypertarget{ref-Cingolani2012}{}
  Cingolani, P., Platts, A., Wang, L. L., Coon, M., Nguyen, T., Wang, L.,
  \ldots{} Ruden, D. M. (2012). A program for annotating and predicting
  the effects of single nucleotide polymorphisms, SnpEff. \emph{Fly},
  \emph{6}(2), 80--92. \url{http://doi.org/10.4161/fly.19695}
  
  \hypertarget{ref-Conkar2017}{}
  Conkar, D., Culfa, E., Odabasi, E., Rauniyar, N., Yates, J. R., \&
  Firat-Karalar, E. N. (2017). The centriolar satellite protein CCDC66
  interacts with CEP290 and functions in cilium formation and trafficking.
  \emph{Journal of Cell Science}, \emph{130}(8). Retrieved from
  \url{http://jcs.biologists.org/content/130/8/1450.long}
  
  \hypertarget{ref-Dekomien2010}{}
  Dekomien, G., Vollrath, C., Petrasch-Parwez, E., Boev?, M. H., Akkad, D.
  A., Gerding, W. M., \& Epplen, J. T. (2010). Progressive retinal atrophy
  in Schapendoes dogs: mutation of the newly identified CCDC66 gene.
  \emph{Neurogenetics}, \emph{11}(2), 163--174.
  \url{http://doi.org/10.1007/s10048-009-0223-z}
  
  \hypertarget{ref-DePristo2011}{}
  DePristo, M. A., Banks, E., Poplin, R., Garimella, K. V., Maguire, J.
  R., Hartl, C., \ldots{} Pritchard, E. (2011). A framework for variation
  discovery and genotyping using next-generation DNA sequencing data.
  \emph{Nature Genetics}, \emph{43}(5), 491--498.
  \url{http://doi.org/10.1038/ng.806}
  
  \hypertarget{ref-Dowdle2013}{}
  Dowdle, J. A., Mehta, M., Kass, E. M., Vuong, B. Q., Inagaki, A., Egli,
  D., \ldots{} Keeney, S. (2013). Mouse BAZ1A (ACF1) is dispensable for
  double-strand break repair but is essential for averting improper gene
  expression during spermatogenesis. \emph{PLoS Genetics}, \emph{9}(11),
  e1003945. \url{http://doi.org/10.1371/journal.pgen.1003945}
  
  \hypertarget{ref-Firat-Karalar2014}{}
  Firat-Karalar, E. N., Sante, J., Elliott, S., \& Stearns, T. (2014).
  Proteomic analysis of mammalian sperm cells identifies new components of
  the centrosome. \emph{Journal of Cell Science}, \emph{127}(Pt 19),
  4128--33. \url{http://doi.org/10.1242/jcs.157008}
  
  \hypertarget{ref-Gaedigk1994}{}
  Gaedigk, R., Duncan, A. M., Miyazaki, I., Robinson, B. H., \& Dosch, H.
  M. (1994). ICA1 encoding p69, a protein linked to the development of
  type 1 diabetes, maps to human chromosome 7p22. \emph{Cytogenetics and
  Cell Genetics}, \emph{66}(4), 274--6. Retrieved from
  \url{http://www.ncbi.nlm.nih.gov/pubmed/8162706}
  
  \hypertarget{ref-Gerding2011}{}
  Gerding, W. M., Schreiber, S., Schulte-Middelmann, T., de Castro
  Marques, A., Atorf, J., Akkad, D. A., \ldots{} Petrasch-Parwez, E.
  (2011). Ccdc66 null mutation causes retinal degeneration and
  dysfunction. \emph{Human Molecular Genetics}, \emph{20}(18), 3620--3631.
  \url{http://doi.org/10.1093/hmg/ddr282}
  
  \hypertarget{ref-Hong2017}{}
  Hong, E. P., Go, M. J., Kim, H.-L., \& Park, J. W. (2017). Risk
  prediction of pulmonary tuberculosis using genetic and conventional risk
  factors in adult Korean population. \emph{PloS One}, \emph{12}(3),
  e0174642. \url{http://doi.org/10.1371/journal.pone.0174642}
  
  \hypertarget{ref-Hu2011}{}
  Hu, Y., Yu, H., Shaw, G., Renfree, M. B., \& Pask, A. J. (2011).
  Differential roles of TGIF family genes in mammalian reproduction.
  \emph{BMC Developmental Biology}, \emph{11}, 58.
  \url{http://doi.org/10.1186/1471-213X-11-58}
  
  \hypertarget{ref-Imai2015}{}
  Imai, Y., Morita, H., Takeda, N., Miya, F., Hyodo, H., Fujita, D.,
  \ldots{} Komuro, I. (2015). A deletion mutation in myosin heavy chain 11
  causing familial thoracic aortic dissection in two Japanese pedigrees.
  \emph{International Journal of Cardiology}, \emph{195}, 290--292.
  \url{http://doi.org/10.1016/j.ijcard.2015.05.178}
  
  \hypertarget{ref-Ivliev2012}{}
  Ivliev, A. E., 't Hoen, P. A. C., Roon-Mom, W. M. C. van, Peters, D. J.
  M., \& Sergeeva, M. G. (2012). Exploring the Transcriptome of Ciliated
  Cells Using In Silico Dissection of Human Tissues. \emph{PLoS ONE},
  \emph{7}(4), e35618. \url{http://doi.org/10.1371/journal.pone.0035618}
  
  \hypertarget{ref-Jaffe2016}{}
  Jaffe, K. M., Grimes, D. T., Schottenfeld-Roames, J., Werner, M. E., Ku,
  T.-S. J., Kim, S. K., \ldots{} Burdine, R. D. (2016). c21orf59/kurly
  Controls Both Cilia Motility and Polarization. \emph{Cell Reports},
  \emph{14}(8), 1841--9. \url{http://doi.org/10.1016/j.celrep.2016.01.069}
  
  \hypertarget{ref-Khan2017}{}
  Khan, A. O., Budde, B. S., Nürnberg, P., Kawalia, A., Lenzner, S., \&
  Bolz, H. J. (2017). Genome-wide linkage and sequence analysis challenge
  CCDC66 as a human retinal dystrophy candidate gene and support a
  distinct NMNAT1-related fundus phenotype. \emph{Clinical Genetics}.
  \url{http://doi.org/10.1111/cge.13022}
  
  \hypertarget{ref-Kumar2009}{}
  Kumar, P., Henikoff, S., \& Ng, P. C. (2009). Predicting the effects of
  coding non-synonymous variants on protein function using the SIFT
  algorithm. \emph{Nature Protocols}, \emph{4}(7), 1073--1081.
  \url{http://doi.org/10.1038/nprot.2009.86}
  
  \hypertarget{ref-Lee2011}{}
  Lee, B., Park, I., Jin, S., Choi, H., Kwon, J. T., Kim, J., \ldots{}
  Cho, C. (2011). Impaired spermatogenesis and fertility in mice carrying
  a mutation in the Spink2 gene expressed predominantly in testes.
  \emph{The Journal of Biological Chemistry}, \emph{286}(33), 29108--17.
  \url{http://doi.org/10.1074/jbc.M111.244905}
  
  \hypertarget{ref-Lek2016}{}
  Lek, M., Karczewski, K. J., Minikel, E. V., Samocha, K. E., Banks, E.,
  Fennell, T., \ldots{} Exome Aggregation Consortium, D. G. (2016).
  Analysis of protein-coding genetic variation in 60,706 humans.
  \emph{Nature}, \emph{536}(7616), 285--91.
  \url{http://doi.org/10.1038/nature19057}
  
  \hypertarget{ref-Lunter2011}{}
  Lunter, G., \& Goodson, M. (2011). Stampy: A statistical algorithm for
  sensitive and fast mapping of Illumina sequence reads. \emph{Genome
  Research}, \emph{21}(6), 936--939.
  \url{http://doi.org/10.1101/gr.111120.110}
  
  \hypertarget{ref-Mally1996}{}
  Mally, M. I., Cirulli, V., Hayek, A., \& Otonkoski, T. (1996). ICA69 is
  expressed equally in the human endocrine and exocrine pancreas.
  \emph{Diabetologia}, \emph{39}(4), 474--80. Retrieved from
  \url{http://www.ncbi.nlm.nih.gov/pubmed/8777998}
  
  \hypertarget{ref-Martin1995}{}
  Martin, S., Kardorf, J., Schulte, B., Lampeter, E. F., Gries, F. A.,
  Melchers, I., \ldots{} Pfützner, A. (1995). Autoantibodies to the islet
  antigen ICA69 occur in IDDM and in rheumatoid arthritis.
  \emph{Diabetologia}, \emph{38}(3), 351--5. Retrieved from
  \url{http://www.ncbi.nlm.nih.gov/pubmed/7758883}
  
  \hypertarget{ref-Martin1996}{}
  Martin, S., Lampasona, V., Dosch, M., \& Pietropaolo, M. (1996). Islet
  cell autoantigen 69 antibodies in IDDM. \emph{Diabetologia},
  \emph{39}(6), 747. Retrieved from
  \url{http://www.ncbi.nlm.nih.gov/pubmed/8781774}
  
  \hypertarget{ref-McLaren2016}{}
  McLaren, W., Gil, L., Hunt, S. E., Riat, H. S., Ritchie, G. R. S.,
  Thormann, A., \ldots{} Cunningham, F. (2016). The Ensembl Variant Effect
  Predictor. \emph{Genome Biology}, \emph{17}(1), 122.
  \url{http://doi.org/10.1186/s13059-016-0974-4}
  
  \hypertarget{ref-Ng}{}
  Ng, S. B., Buckingham, K. J., Lee, C., Bigham, A. W., Tabor, H. K.,
  Dent, K. M., \ldots{} Bamshad, M. J. (n.d.). Exome sequencing identifies
  the cause of a Mendelian disorder. \url{http://doi.org/10.1038/ng.499}
  
  \hypertarget{ref-Nielsen2011}{}
  Nielsen, R., Paul, J. S., Albrechtsen, A., \& Song, Y. S. (2011).
  Genotype and SNP calling from next-generation sequencing data.
  \emph{Nature Reviews. Genetics}, \emph{12}(6), 443--51.
  \url{http://doi.org/10.1038/nrg2986}
  
  \hypertarget{ref-Pelak2010}{}
  Pelak, K., Shianna, K. V., Ge, D., Maia, J. M., Zhu, M., Smith, J. P.,
  \ldots{} Goldstein, D. B. (2010). The characterization of twenty
  sequenced human genomes. \emph{PLoS Genetics}, \emph{6}(9), e1001111.
  \url{http://doi.org/10.1371/journal.pgen.1001111}
  
  \hypertarget{ref-Robinson2014}{}
  Robinson, P. N., Köhler, S., Oellrich, A., Sanger Mouse Genetics
  Project, S. M. G., Wang, K., Mungall, C. J., \ldots{} Smedley, D.
  (2014). Improved exome prioritization of disease genes through
  cross-species phenotype comparison. \emph{Genome Research},
  \emph{24}(2), 340--8. \url{http://doi.org/10.1101/gr.160325.113}
  
  \hypertarget{ref-Salgado2016}{}
  Salgado, D., Bellgard, M. I., Desvignes, J. P., \& B??roud, C. (2016).
  How to Identify Pathogenic Mutations among All Those Variations: Variant
  Annotation and Filtration in the Genome Sequencing Era. \emph{Human
  Mutation}, \emph{37}(12), 1272--1282.
  \url{http://doi.org/10.1002/humu.23110}
  
  \hypertarget{ref-Sironen2010}{}
  Sironen, A., Hansen, J., Thomsen, B., Andersson, M., Vilkki, J.,
  Toppari, J., \& Kotaja, N. (2010). Expression of SPEF2 During Mouse
  Spermatogenesis and Identification of IFT20 as an Interacting Protein1.
  \emph{Biology of Reproduction}, \emph{82}(3), 580--590.
  \url{http://doi.org/10.1095/biolreprod.108.074971}
  
  \hypertarget{ref-Sironen2011}{}
  Sironen, A., Kotaja, N., Mulhern, H., Wyatt, T. A., Sisson, J. H.,
  Pavlik, J. A., \ldots{} Lee, L. (2011). Loss of SPEF2 function in mice
  results in spermatogenesis defects and primary ciliary dyskinesia.
  \emph{Biology of Reproduction}, \emph{85}(4), 690--701.
  \url{http://doi.org/10.1095/biolreprod.111.091132}
  
  \hypertarget{ref-Stassi1997}{}
  Stassi, G., Schloot, N., \& Pietropaolo, M. (1997). Islet cell
  autoantigen 69 kDa (ICA69) is preferentially expressed in the human
  islets of Langerhans than exocrine pancreas. \emph{Diabetologia},
  \emph{40}(1), 120--2. Retrieved from
  \url{http://www.ncbi.nlm.nih.gov/pubmed/9028728}
  
  \hypertarget{ref-Su2014}{}
  Su, Z., Łabaj, P. P., Li, S. S., Thierry-Mieg, J., Thierry-Mieg, D.,
  Shi, W., \ldots{} Shi, L. (2014). A comprehensive assessment of RNA-seq
  accuracy, reproducibility and information content by the Sequencing
  Quality Control Consortium. \emph{Nature Biotechnology}, \emph{32}(9),
  903--14. \url{http://doi.org/10.1038/nbt.2957}
  
  \hypertarget{ref-Wang2010}{}
  Wang, K., Li, M., \& Hakonarson, H. (2010). ANNOVAR: functional
  annotation of genetic variants from high-throughput sequencing data.
  \emph{Nucleic Acids Research}, \emph{38}(16), e164--e164.
  \url{http://doi.org/10.1093/nar/gkq603}
  
  \hypertarget{ref-Winer2002}{}
  Winer, S., Astsaturov, I., Cheung, R., Tsui, H., Song, A., Gaedigk, R.,
  \ldots{} Dosch, H.-M. (2002). Primary Sj?gren's syndrome and deficiency
  of ICA69. \emph{The Lancet}, \emph{360}(9339), 1063--1069.
  \url{http://doi.org/10.1016/S0140-6736(02)11144-5}
  
  \hypertarget{ref-Zheng2006}{}
  Zheng, Y., Zhang, J., Wang, L., Zhou, Z., Xu, M., Li, J., \& Sha, J.-H.
  (2006). Cloning and characterization of a novel sperm tail protein,
  NYD-SP28. \emph{International Journal of Molecular Medicine},
  \emph{18}(6), 1119--25. Retrieved from
  \url{http://www.ncbi.nlm.nih.gov/pubmed/17089017}


  % Index?

\end{document}

