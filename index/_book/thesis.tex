% This is the Reed College LaTeX thesis template. Most of the work
% for the document class was done by Sam Noble (SN), as well as this
% template. Later comments etc. by Ben Salzberg (BTS). Additional
% restructuring and APA support by Jess Youngberg (JY).
% Your comments and suggestions are more than welcome; please email
% them to cus@reed.edu
%
% See http://web.reed.edu/cis/help/latex.html for help. There are a
% great bunch of help pages there, with notes on
% getting started, bibtex, etc. Go there and read it if you're not
% already familiar with LaTeX.
%
% Any line that starts with a percent symbol is a comment.
% They won't show up in the document, and are useful for notes
% to yourself and explaining commands.
% Commenting also removes a line from the document;
% very handy for troubleshooting problems. -BTS

% As far as I know, this follows the requirements laid out in
% the 2002-2003 Senior Handbook. Ask a librarian to check the
% document before binding. -SN

%%
%% Preamble
%%
% \documentclass{<something>} must begin each LaTeX document
\documentclass[12pt,twoside]{reedthesis}
% Packages are extensions to the basic LaTeX functions. Whatever you
% want to typeset, there is probably a package out there for it.
% Chemistry (chemtex), screenplays, you name it.
% Check out CTAN to see: http://www.ctan.org/
%%
\usepackage{graphicx,latexsym}
\usepackage[french]{babel} 
\usepackage{amsmath}
\usepackage{amssymb,amsthm}
\usepackage[dvipsnames]{xcolor} % tk: for more color
\usepackage{xcolor}
\usepackage{eso-pic}
\usepackage{longtable,booktabs,setspace}
\usepackage{chemarr} %% Useful for one reaction arrow, useless if you're not a chem major
\usepackage[hyphens]{url}
\usepackage{tikz}
\usetikzlibrary{calc}
\newcommand\HRule{\rule{\textwidth}{1pt}}
% Added by CII
\usepackage{hyperref}
\usepackage{lmodern}
\usepackage{float}
\floatplacement{figure}{H}
% End of CII addition
\usepackage{rotating}
\usepackage{upgreek} % tk : pour pouvoir utiliser le symbole µ droit (pas en itallic)
\usepackage{lscape}
\newcommand{\blandscape}{\begin{landscape}}
\newcommand{\elandscape}{\end{landscape}}

% Next line commented out by CII
%%% \usepackage{natbib}
% Comment out the natbib line above and uncomment the following two lines to use the new
% biblatex-chicago style, for Chicago A. Also make some changes at the end where the
% bibliography is included.
%\usepackage{biblatex-chicago}
%\bibliography{thesis}


% Added by CII (Thanks, Hadley!)
% Use ref for internal links
\renewcommand{\hyperref}[2][???]{\autoref{#1}}
\def\chapterautorefname{Chapter}
\def\sectionautorefname{Section}
\def\subsectionautorefname{Subsection}
% End of CII addition

% Added by CII
\usepackage{caption}
\captionsetup{width=5in}
% End of CII addition

% \usepackage{times} % other fonts are available like times, bookman, charter, palatino


% To pass between YAML and LaTeX the dollar signs are added by CII
\title{THÈSE}
\author{Thomas Karaouzene}
\labo{}
% The month and year that you submit your FINAL draft TO THE LIBRARY (May or December)
\date{31 octobre 2017}
\division{}
\advisor{Pierre Ray}
%If you have two advisors for some reason, you can use the following
% Uncommented out by CII
\altadvisor{Nicolas Thierry-Mieg}
% End of CII addition

%%% Remember to use the correct department!
\department{Ingénierie de la Santé, de la Cognition et Environnement (EDISCE)}
% if you're writing a thesis in an interdisciplinary major,
% uncomment the line below and change the text as appropriate.
% check the Senior Handbook if unsure.
%\thedivisionof{The Established Interdisciplinary Committee for}
% if you want the approval page to say "Approved for the Committee",
% uncomment the next line
%\approvedforthe{Committee}

% Added by CII
%%% Copied from knitr
%% maxwidth is the original width if it's less than linewidth
%% otherwise use linewidth (to make sure the graphics do not exceed the margin)
\makeatletter
\def\maxwidth{ %
  \ifdim\Gin@nat@width>\linewidth
    \linewidth
  \else
    \Gin@nat@width
  \fi
}
\makeatother

\renewcommand{\contentsname}{Table of Contents}
% End of CII addition

\setlength{\parskip}{0pt}

% Added by CII
  %\setlength{\parskip}{\baselineskip}
  \usepackage[parfill]{parskip}

\providecommand{\tightlist}{%
  \setlength{\itemsep}{0pt}\setlength{\parskip}{0pt}}

\Acknowledgements{

}

\Dedication{

}

\Preface{
This is an example of a thesis setup to use the reed thesis document
class (for LaTeX) and the R bookdown package, in general.
}

\Abstract{

}

	\usepackage{tikz}
% End of CII addition
%%
%% End Preamble
%%
%

\usepackage{amsthm}
\newtheorem{theorem}{Theorem}[section]
\newtheorem{lemma}{Lemma}[section]
\theoremstyle{definition}
\newtheorem{definition}{Definition}[section]
\newtheorem{corollary}{Corollary}[section]
\newtheorem{proposition}{Proposition}[section]
\theoremstyle{definition}
\newtheorem{example}{Example}[section]
\theoremstyle{remark}
\newtheorem*{remark}{Remark}
\begin{document}

% Everything below added by CII
      \maketitle
  
  \frontmatter % this stuff will be roman-numbered
  \pagestyle{empty} % this removes page numbers from the frontmatter

  
      \begin{preface}
      This is an example of a thesis setup to use the reed thesis document
      class (for LaTeX) and the R bookdown package, in general.
    \end{preface}
  
      \hypersetup{linkcolor=black}
    \setcounter{tocdepth}{3}
    \tableofcontents
  
      \listoftables
  
      \listoffigures
  
  
  
  \mainmatter % here the regular arabic numbering starts
  \pagestyle{fancyplain} % turns page numbering back on

  \chapter{Delete line 6 if you only have one
  advisor}\label{delete-line-6-if-you-only-have-one-advisor}
  
  \chapter*{Remerciements}\label{remerciements}
  \addcontentsline{toc}{chapter}{Remerciements}
  
  \chapter*{Résumé}\label{resume}
  \addcontentsline{toc}{chapter}{Résumé}
  
  \chapter{Introduction}\label{introInf}
  
  \chapter{Investigation génétique et physiologique de la
  globozoospermie}\label{globo}
  
  \chapter{Mise en place d'une stratégie pour l'analyse des données
  exomiques -- application en recherche
  clinique}\label{mise-en-place-dune-strategie-pour-lanalyse-des-donnees-exomiques-application-en-recherche-clinique}
  
  \section{Intro}\label{intro}
  
  \newpage
  
  \section{Résultats}\label{resultats}
  
  \subsection{Description de la
  pipeline}\label{description-de-la-pipeline}
  
  Notre pipeline d'analyse effectue l'ensemble des étapes allant de
  l'alignement des données jusqu'au filtrage des variants
  
  \begin{enumerate}
  \def\labelenumi{\arabic{enumi}.}
  \tightlist
  \item
    \textbf{L'alignement} : L'alignement des \emph{reads} le long du
    génome de référence est effectué par le logiciel MAGIC (Su et al.,
    \protect\hyperlink{ref-Su2014}{2014}). Celui-ci l'intégralité pour
    l'ensemble des analyses en aval l'ensemble des \emph{reads} dupliqués
    et / ou s'alignant à plusieurs zones du génome. Au cours de cette
    étape, MAGIC va produire également quatre comptages pour chaque
    position couverte du génome : R+, V+, R- et V- :
  
    \begin{enumerate}
    \def\labelenumii{\alph{enumii}.}
    \tightlist
    \item
      \textbf{R+ et R-} : Ces deux comptages correspondent au nombre de
      \emph{reads} \emph{forward} (+) et \emph{reverse} (-) sur lesquels
      est observé l'allèle de \textbf{référence} (R) à une position
      donnée.\\
    \item
      \textbf{V+ et V-} : À l'inverse de R+ et R-, ces comptages
      correspondent au nombre de \emph{reads} \emph{forward} et
      \emph{reverse} sur lesquels est observé un allèle de
      \textbf{variant} (V) à une position donnée.\\
    \end{enumerate}
  \item
    \textbf{L'appel des variants} : Comme nous l'avons vu plus
    \protect\hyperlink{varcall}{tôt}, il est fortement conseillé
    d'effectuer l'appel des variants en tenant compte de l'aligneur choisi
    (Nielsen, Paul, Albrechtsen, \& Song,
    \protect\hyperlink{ref-Nielsen2011}{2011}, M. A. DePristo et al.
    (\protect\hyperlink{ref-DePristo2011}{2011}), Lunter \& Goodson
    (\protect\hyperlink{ref-Lunter2011}{2011})). C'est pourquoi, nous
    avons conçu notre propre algorithme d'appel des variants spécialement
    conçu pour l'analyse des données de MAGIC. Ainsi, l'appel des variants
    sera directement basé sur les quatre comptages vus précédemment. Tout
    d'abord, les positions ayant une couverture \textless{} 10 sur l'un
    des deux \emph{strands} sera considérée comme de faible qualité,
    celles ayant une couverture \textless{} 10 sur les deux \emph{strands}
    seront exclus. Ensuite pour chaque variant, des appels indépendants
    seront effectués pour chaque \emph{strand}. L'appel final sera une
    synthèse de ces deux appels où seul les cas où ces deux appels sont
    concordants seront considérés comme de bonne qualité.\\
  \item
    \textbf{L'annotation} : Chaque variant retenu sera ensuite annoté tout
    d'abord par le logiciel \emph{variant effect predictor} (VEP) (W.
    McLaren et al., \protect\hyperlink{ref-McLaren2016}{2016}) qui nous
    indiquera pour chaque variant la conséquence que celui-ci aura sur la
    séquence codante de l'ensemble des transcrits Ensembl qu'il chevauche
    (\textbf{Figure : }\ref{fig:figvepcsq}) (\textbf{Table :
    }\ref{tab:tabvepcsq}). Suite à cela nous ajoutons, lorsque celle-ci
    est disponible, la fréquence du variant dans les bases de données ExAC
    (Lek et al., \protect\hyperlink{ref-Lek2016}{2016}), ESP600 {[}TODO{]}
    et 1000Genomes {[}TODO{]} donnant ainsi une estimation de sa fréquence
    dans la population générale. De même, la particularité de ce pipeline
    est qu'elle conserve l'ensemble des variants identifiés dans les
    études effectuées précédemment permettant d'ajouter aux annotations la
    fréquence d'un variant chez les individus déjà séquencé et donc la
    fréquence d'un variant dans chaque phénotype étudié créant ainsi une
    base de données interne qui pourra servir de contrôle dans les études
    ultérieur.
  \end{enumerate}
  
  \begin{figure}
  
  {\centering \includegraphics[scale=.9]{figure/vep_csq} 
  
  }
  
  \caption[Listes des différentes conséquences prédites par VEP et leur positionnement sur le transcrit]{Listes des différentes conséquences prédites par VEP et leur positionnement sur le transcrit d'après [VEP site](http://www.ensembl.org/info/genome/variation/consequences.jpg)}\label{fig:figvepcsq}
  \end{figure}
  
  \begin{enumerate}
  \def\labelenumi{\arabic{enumi}.}
  \setcounter{enumi}{3}
  \tightlist
  \item
    \textbf{Le filtrage des variants} : L'étape de filtrage est
    extrêmement importante si l'on souhaite analyser de manière efficace
    les données provenant de WES. C'est pourquoi elle occupe une place
    importante dans notre pipeline. L'intégralité des paramètres de cette
    étape peuvent être modifier par l'utilisateur de sorte à faire
    correspondre les critères de filtre aux besoins de l'étude. Afin de
    rendre son utilisation le plus efficace possible, nous avons souhaité
    définir des paramètres par défauts pertinent dans la plupart des
    études de séquençage exomique de sorte que à moins que le contraire ne
    soit spécifié, seul les variants impactant les transcrits codant pour
    une protéine sont conservés. De même les variants synonymes ou
    affectant les séquences UTRs sont filtrés ainsi que les variants ayant
    une fréquence \(\ge\) 1\% dans les bases dans l'une des bases données
    (ExAC, ESP6500 ou 1KH). Aussi, pour un phénotype donné, l'ensemble des
    variants observés chez les individus étudiés présentant un phénotype
    différent sont de même enlevés de la liste finale.
  \end{enumerate}
  
  \newpage
  
  \blandscape
  
  \begin{longtable}[]{@{}lll@{}}
  \caption{\label{tab:tabvepcsq} Liste simplifiée des conséquences prédites
  par VEP avec leur description et impact associée}\tabularnewline
  \toprule
  \begin{minipage}[b]{0.18\columnwidth}\raggedright\strut
  VEP consequence\strut
  \end{minipage} & \begin{minipage}[b]{0.11\columnwidth}\raggedright\strut
  VEP impact\strut
  \end{minipage} & \begin{minipage}[b]{0.63\columnwidth}\raggedright\strut
  Description\strut
  \end{minipage}\tabularnewline
  \midrule
  \endfirsthead
  \toprule
  \begin{minipage}[b]{0.18\columnwidth}\raggedright\strut
  VEP consequence\strut
  \end{minipage} & \begin{minipage}[b]{0.11\columnwidth}\raggedright\strut
  VEP impact\strut
  \end{minipage} & \begin{minipage}[b]{0.63\columnwidth}\raggedright\strut
  Description\strut
  \end{minipage}\tabularnewline
  \midrule
  \endhead
  \begin{minipage}[t]{0.18\columnwidth}\raggedright\strut
  Splice acceptor / donor\strut
  \end{minipage} & \begin{minipage}[t]{0.11\columnwidth}\raggedright\strut
  HIGH\strut
  \end{minipage} & \begin{minipage}[t]{0.63\columnwidth}\raggedright\strut
  A splice variant that changes the 2 base region at the 3' / 5' end of an
  intron\strut
  \end{minipage}\tabularnewline
  \begin{minipage}[t]{0.18\columnwidth}\raggedright\strut
  Stop gained\strut
  \end{minipage} & \begin{minipage}[t]{0.11\columnwidth}\raggedright\strut
  HIGH\strut
  \end{minipage} & \begin{minipage}[t]{0.63\columnwidth}\raggedright\strut
  A sequence variant whereby at least one base of a codon is changed,
  resulting in a premature stop codon, leading to a shortened
  transcript\strut
  \end{minipage}\tabularnewline
  \begin{minipage}[t]{0.18\columnwidth}\raggedright\strut
  Frameshift\strut
  \end{minipage} & \begin{minipage}[t]{0.11\columnwidth}\raggedright\strut
  HIGH\strut
  \end{minipage} & \begin{minipage}[t]{0.63\columnwidth}\raggedright\strut
  A sequence variant which causes a disruption of the translational
  reading frame, because the number of nucleotides inserted or deleted is
  not a multiple of three\strut
  \end{minipage}\tabularnewline
  \begin{minipage}[t]{0.18\columnwidth}\raggedright\strut
  Stop lost\strut
  \end{minipage} & \begin{minipage}[t]{0.11\columnwidth}\raggedright\strut
  HIGH\strut
  \end{minipage} & \begin{minipage}[t]{0.63\columnwidth}\raggedright\strut
  A sequence variant where at least one base of the terminator codon
  (stop) is changed, resulting in an elongated transcript\strut
  \end{minipage}\tabularnewline
  \begin{minipage}[t]{0.18\columnwidth}\raggedright\strut
  Start lost\strut
  \end{minipage} & \begin{minipage}[t]{0.11\columnwidth}\raggedright\strut
  HIGH\strut
  \end{minipage} & \begin{minipage}[t]{0.63\columnwidth}\raggedright\strut
  A codon variant that changes at least one base of the canonical start
  codo\strut
  \end{minipage}\tabularnewline
  \begin{minipage}[t]{0.18\columnwidth}\raggedright\strut
  Inframe insertion / deletion\strut
  \end{minipage} & \begin{minipage}[t]{0.11\columnwidth}\raggedright\strut
  MODERATE\strut
  \end{minipage} & \begin{minipage}[t]{0.63\columnwidth}\raggedright\strut
  An inframe non synonymous variant that inserts / deletes bases into in
  the coding sequenc\strut
  \end{minipage}\tabularnewline
  \begin{minipage}[t]{0.18\columnwidth}\raggedright\strut
  Missense\strut
  \end{minipage} & \begin{minipage}[t]{0.11\columnwidth}\raggedright\strut
  MODERATE\strut
  \end{minipage} & \begin{minipage}[t]{0.63\columnwidth}\raggedright\strut
  A sequence variant, that changes one or more bases, resulting in a
  different amino acid sequence but where the length is preserved\strut
  \end{minipage}\tabularnewline
  \begin{minipage}[t]{0.18\columnwidth}\raggedright\strut
  Splice region\strut
  \end{minipage} & \begin{minipage}[t]{0.11\columnwidth}\raggedright\strut
  LOW\strut
  \end{minipage} & \begin{minipage}[t]{0.63\columnwidth}\raggedright\strut
  A sequence variant in which a change has occurred within the region of
  the splice site, either within 1-3 bases of the exon or 3-8 bases of the
  intron\strut
  \end{minipage}\tabularnewline
  \begin{minipage}[t]{0.18\columnwidth}\raggedright\strut
  Stop retained\strut
  \end{minipage} & \begin{minipage}[t]{0.11\columnwidth}\raggedright\strut
  LOW\strut
  \end{minipage} & \begin{minipage}[t]{0.63\columnwidth}\raggedright\strut
  A sequence variant where at least one base in the terminator codon is
  changed, but the terminator remains\strut
  \end{minipage}\tabularnewline
  \begin{minipage}[t]{0.18\columnwidth}\raggedright\strut
  Synonymous\strut
  \end{minipage} & \begin{minipage}[t]{0.11\columnwidth}\raggedright\strut
  LOW\strut
  \end{minipage} & \begin{minipage}[t]{0.63\columnwidth}\raggedright\strut
  A sequence variant where there is no resulting change to the encoded
  amino acid\strut
  \end{minipage}\tabularnewline
  \begin{minipage}[t]{0.18\columnwidth}\raggedright\strut
  5 / 3 prime UTR\strut
  \end{minipage} & \begin{minipage}[t]{0.11\columnwidth}\raggedright\strut
  MODIFIER\strut
  \end{minipage} & \begin{minipage}[t]{0.63\columnwidth}\raggedright\strut
  A UTR variant of the 5' / 3' UTR\strut
  \end{minipage}\tabularnewline
  \begin{minipage}[t]{0.18\columnwidth}\raggedright\strut
  Intron\strut
  \end{minipage} & \begin{minipage}[t]{0.11\columnwidth}\raggedright\strut
  MODIFIER\strut
  \end{minipage} & \begin{minipage}[t]{0.63\columnwidth}\raggedright\strut
  A transcript variant occurring within an intron\strut
  \end{minipage}\tabularnewline
  \begin{minipage}[t]{0.18\columnwidth}\raggedright\strut
  NMD transcript\strut
  \end{minipage} & \begin{minipage}[t]{0.11\columnwidth}\raggedright\strut
  MODIFIER\strut
  \end{minipage} & \begin{minipage}[t]{0.63\columnwidth}\raggedright\strut
  A variant in a transcript that is the target of NMD\strut
  \end{minipage}\tabularnewline
  \begin{minipage}[t]{0.18\columnwidth}\raggedright\strut
  Non coding transcript\strut
  \end{minipage} & \begin{minipage}[t]{0.11\columnwidth}\raggedright\strut
  MODIFIER\strut
  \end{minipage} & \begin{minipage}[t]{0.63\columnwidth}\raggedright\strut
  A transcript variant of a non coding RNA gene\strut
  \end{minipage}\tabularnewline
  \bottomrule
  \end{longtable}
  
  \elandscape
  
  \newpage
  
  \subsection{Utilisation du pipeline dans des cas familiaux
  :}\label{utilisation-du-pipeline-dans-des-cas-familiaux}
  
  \subsubsection{Description des familles}\label{description-des-familles}
  
  Dans cette partie, je me concentre sur l'analyse bioinformatique des
  résultats des séquençages exomiques effectués entre 2012 et 2014 de 13
  individus infertiles provenant de 6 familles différentes. Parmi
  celles-ci, 3 phénotypes différents ont été observés :
  
  \begin{enumerate}
  \def\labelenumi{\arabic{enumi}.}
  \tightlist
  \item
    \textbf{\protect\hyperlink{infquant}{L'Azoospermie} :} Comme nous
    avons pu le voir, l'azoospermie est un phénotype d'infertilité
    masculine caractérisé par l'absence de spermatozoïde dans l'éjaculât\\
  \item
    \textbf{Échec de fécondation :} Ce phénotype d'infertilité se
    caractérise par l'incapacité des spermatozoïdes à féconder
    l'ovocyte.\\
  \item
    \textbf{MMAF} : Le syndrome MMAF (\emph{multiple morphological
    abnormalities of the sperm flagella}) caractérise comme son nom
    l'indique les patients présentant une majorité de spermatozoïdes
    atteins par une mosaïque d'anomalie morphologique du flagelle.
  \end{enumerate}
  
  Un récapitulatif des familles et de leur phénotype est disponible dans
  la table \ref{tab:tabrecapfam}.
  
  \begin{longtable}[t]{lrlrl}
  \caption{\label{tab:tabrecapfam}Tableau récapitulatif des familles séquencées et de leur phénotype}\\
  \toprule
  Familly & Individuals & Phenotype & Year & Place\\
  \midrule
  Az & 2 & Azoospermia & 2012 & Mount Sinai Institut\\
  FF & 2 & Fertilization failure & 2014 & Genoscope (Evry)\\
  MMAF1 & 2 & MMAF & 2014 & Genoscope (Evry)\\
  MMAF2 & 2 & MMAF & 2014 & Genoscope (Evry)\\
  MMAF3 & 2 & MMAF & 2014 & Genoscope (Evry)\\
  MMAF4 & 3 & MMAF & 2014 & Genoscope (Evry)\\
  \bottomrule
  \end{longtable}
  
  \newpage  
  
  \subsubsection{Resultats des exomes}\label{resultats-des-exomes}
  
  \paragraph{Résultat de l'alignement}\label{resultat-de-lalignement}
  
  Pour rappel, l'\href{\%7B\#lalignement\%7D}{alignement} consiste à
  repositionner l'ensemble des \emph{reads} générés au cours de l'étape de
  séquençage le long d'un génome de référence.
  
  La quantité de \emph{reads} composant les exomes de chaque individu peut
  varier en fonction de plusieurs paramètres et n'est donc pas égale pour
  chaque patient bien que l'ordre de grandeur reste le même exceptés,
  c'est à dire une médiane de 91438630 \emph{reads}. Seul les deux frères
  AZ1 et AZ2 se distinguent près de 3 fois plus de \emph{reads} que pour
  les autres patients. Cette différence peut être expliqué car ces deux
  patients sont les deux seuls à voir été séquencé au Mount Sinaï Institut
  or leur protocole d'amplification précédent le séquençage contient un
  nombre de cycles de PCR supérieur à ceux appliqué au Génopole d'Évry où
  ont été séquencé les autres patients (\textbf{Table :}
  \ref{tab:tabrecapfam}, \textbf{Figure : }\ref{fig:readsselection} -
  \textbf{A}).
  
  L'ensemble de nos exomes ayant été réalisés en \emph{paired-end}, les
  deux extrémités de chaque fragment sont séquencés chaque \emph{end} d'un
  même \emph{read} peut donc être considéré comme un \emph{read} à part
  entière qui sont alignées \textbf{indépendamment} le long du génome de
  référence. L'information fournit par le \emph{paired-end} n'étant
  utilisé qu'à \emph{posteriori} en tant que critère qualité. La première
  étape du contrôle qualité des \emph{reads} consiste filtrer les
  \emph{reads} ne s'étant pas aligné sur le génome. Ces \emph{reads} sont
  extrêmement minoritaires puisqu'ils représentent entre 1.2 et 5.5 \% des
  \emph{reads} de nos individus (\textbf{Figure :
  }\ref{fig:readsselection} - \textbf{B}).
  
  Une fois cela fait, nous vérifions la ``compatibilité'' des deux
  \emph{ends} composant chacun des \emph{reads} s'étant correctement
  alignés. Un \emph{reads} est dit compatible lorsque les deux \emph{ends}
  qui le composent s'alignent face à face (une sur le \emph{strand} + et
  l'autre sur le \emph{strand} -) et couvrent une zone ne faisant pas plus
  de 3 fois la taille médiane de l'insert. Les \emph{reads} dont les deux
  \emph{ends} se sont alignées mais ne remplissant pas ces conditions
  seront dit ``Non compatible'', ceux dont une seule des deux \emph{ends}
  s'est alignés seront appelés ``orphelins''. Dans nos analyses, seuls les
  \emph{reads} compatibles sont conservés, c'est à dire environs \ldots{}
  \% (médiane) des \emph{reads} s'étant correctement alignés.
  (\textbf{Figure : }\ref{fig:readsselection} - \textbf{C}).
  
  La dernière étape de ce contrôle-qualité consiste à analyser le nombre
  de site auxquels se sont alignés les \emph{reads}. En effet, certaine
  zone du génome étant dupliqué, l'une des problématiques des
  \emph{short-reads} est qu'il est possible que ceux-ci s'alignent à
  plusieurs régions différentes du génome. Afin d'éviter toute ambiguïté,
  seul ceux s'étant aligné sur un site unique sont conservés pour la suite
  des analyses. Ces \emph{reads} représente entre 92.3 et 96.9 \% des
  \emph{reads} ayant passé les précédents filtres (\textbf{Figure :
  }\ref{fig:readsselection} - \textbf{C}).
  
  Les \emph{reads} ayant passé l'ensemble des critères qualité mentionnés
  précédemment seront ensuite utilisés pour effectuer l'appel des
  variants.
  
  \newpage
  
  \begin{figure}
  
  {\centering \includegraphics{thesis_files/figure-latex/readsselection-1} 
  
  }
  
  \caption[Processus simplifié du contrôle qualité des *reads*]{Processus simplifié du contrôle qualité des *reads* : Pour chacun des graphiques, les *reads* représentés en vert sont conservés tandis que ceux en rouge sont filtrés. **A** : Quantité de *reads* bruts générés pour chaque patient au cours de l'étape de séquençage. La médiane des *reads* est représentée en bleue. **B** : Pourcentage pour chaque individu de *reads* s'étant aligné correctement et ne s'étant pas alignés sur le génome de référence. **C** : Distribution pour chaque patient des *reads* compatibles (Comp), non compatibles (Non comp) et orphelins (Orphans). **D** : Présentation pour chaque *reads* du nombre de site auxquels ils s'alignent}\label{fig:readsselection}
  \end{figure}
  
  \newpage
  
  \paragraph{Résultat de l'appel des
  variants}\label{resultat-de-lappel-des-variants}
  
  Comme dit précédemment, l'appel des variants fait suite à l'alignement
  et consiste à comparer la séquence d'un individu avec celle d'un génome
  de référence afin d'en relever les différences. La particularité de
  notre algorithme d'appel est d'effectuer pour chaque position deux
  appels indépendants Le premier sera effectué en utilisant uniquement les
  \emph{reads forward} et le second le \emph{reads reverse}. Encore une
  fois, plusieurs filtres sont appliqués de sorte à conserver uniquement
  les variants les plus qualitatifs.
  
  Tout d'abord, nos appels sont classés en trois catégories :
  
  \begin{enumerate}
  \def\labelenumi{\arabic{enumi}.}
  \tightlist
  \item
    \textbf{Les appels \emph{double strand} (DS) :} Qualifie les positions
    ayant une couverture \(\ge\) 10 sur \textbf{les deux} strands. Ces
    appels sont ceux sont ceux ayant la meilleure qualité
  \item
    \textbf{Les appels \emph{single strand} (SS) :} Ces appels définissent
    les positions pour lesquels \textbf{un des deux} \emph{strands}
    présentent une couverture \(\le\) 10. Dans ce cas, ce \emph{strand}
    est ignoré et l'appel est effectué uniquement en utilisant le second
    \emph{strand}.\\
  \item
    \textbf{Les appels \emph{non strand} (NS) :} Les positions NS sont
    celles pour lesquelles la couverture est \(\le\) 10 sur \textbf{les
    deux} strands. Aucun appel n'est effectué à ces positions.
  \end{enumerate}
  
  Dans nos données, les appels SS sont majoritaires et représentent
  environ 48.1 \% de nos appels (contre 35.6 \% d'appels DS). Au vus de
  l'importance de ces appels, nous avons fait le choix de les conserver
  afin de ne pas filtrer une quantité trop importante de données. Ces
  appels seront cependant considérés comme étant de faible qualité, de
  fait, leurs analyses et interprétation seront plus précautionneuses En
  revanche, au vus de la trop grande incertitude de l'appel des variants
  NS, ceux-ci sont systématiquement filtrés éliminant ainsi entre 10.3 et
  18.7 \% des positions appelées (\textbf{Figure : }\ref{fig:plotvarcall}
  - \textbf{A}).
  
  Un second filtre est appliqué aux variants ayant été précédemment
  appelés DS. Celui-ci consiste à comparer les appels effectués
  indépendamment sur chacune des deux \emph{ends} et à vérifier leur
  concordance, c'est à dire que les deux appels soit identique. Les appels
  discordant et ambigus sont filtrer, ainsi environs 86.3 \% des variants
  DS passent ce filtre. Il est intéressant de noter que bien que les
  variants \emph{single strand} (SS) soient conservés, on peut s'attendre
  à ce qu'environ 13.7 \% de ceux-ci soient aberrants, ceux-ci n'ayant pu
  subir le même contrôle que les SS (\textbf{Figure :
  }\ref{fig:plotvarcall} - \textbf{B}).
  
  Pour l'ensemble des variants ayant passé les filtres énoncés ci-dessus,
  c'est à dire les variants SS et les variants DS avec appels concordants,
  le génotype est déterminé en fonction du pourcentage de \emph{reads}
  portant le variant à cette position. Par exemple, si à une position
  donnée, 0\% des \emph{reads} portent un variant, l'individu sera appelé
  ``Homozygote référence'', si 50\% des \emph{reads} sont portent un
  variant, l'appel sera ``hétérozygote'' et si 100\% des \emph{reads}
  portent un variant, l'appel sera ``Homozygote variant''. Ainsi, pour
  chaque individu nous avons pu établir une liste de SNVs et d'indels avec
  leur génotype associé. Pour chacun de nos 13 patients les ordres de
  grandeur du nombre de variants appelés sont identique. Ainsi pour chaque
  patient nous avons appelés environ 43670 variants hétérozygotes (41044
  SNVs et 2626 indels) et 65040 variants homozygotes (32520 SNVs et 1809
  indels) (\textbf{Figure : }\ref{fig:plotvarcall} - \textbf{C}).
  
  \newpage
  
  \begin{figure}
  
  {\centering \includegraphics{thesis_files/figure-latex/plotvarcall-1} 
  
  }
  
  \caption[Contrôle qualité des variants appelés]{Contrôle qualité des variants appelés : Pour chacun des graphiques, les variants représentés en vert et en orange sont conservés tandis que ceux en rouge sont filtrés. **A** : Distribution du *stranding* des appels pour chaque patient. **B** : Comparaison des appels entre les deux *ends* des variants appelés DS. **C** : Distribution des SNVs et indels en fonction de leur génotype pour chaque patients (représentés par une barre}\label{fig:plotvarcall1}
  \end{figure}\begin{figure}
  
  {\centering \includegraphics{thesis_files/figure-latex/plotvarcall-2} 
  
  }
  
  \caption[Contrôle qualité des variants appelés]{Contrôle qualité des variants appelés : Pour chacun des graphiques, les variants représentés en vert et en orange sont conservés tandis que ceux en rouge sont filtrés. **A** : Distribution du *stranding* des appels pour chaque patient. **B** : Comparaison des appels entre les deux *ends* des variants appelés DS. **C** : Distribution des SNVs et indels en fonction de leur génotype pour chaque patients (représentés par une barre}\label{fig:plotvarcall2}
  \end{figure}
  
  \newpage
  
  \paragraph{Résultats de l'annotation}\label{resultats-de-lannotation}
  
  L'annotation des variants appelés consiste à ajouter un maximum
  d'informations sur les variants. Ces informations seront ensuite
  utilisées afin de filtrer et / ou prioriser les variants. Dans ces
  analyses nous avons utilisé le logiciel \emph{Variant Effect Predictor}
  (VEP) (W. McLaren et al., \protect\hyperlink{ref-McLaren2016}{2016}) qui
  va à la fois prédire l'effet qu'auront ces variants sur l'ensemble des
  transcrits (et gènes) qu'ils chevauchent, ajouter, lorsqu'elle est
  disponible, la fréquence de chacun de ces variants dans les bases de
  données ExAC, 1000Genomes (1KG) et ESP6500. Pour finir VEP nous
  permettra de connaitre les prédictions de pathogénicités fournies par
  SIFT et PolyPhen pour les variants faux-sens.
  
  Après avoir annoter nos variants par VEP, nous avons pu constater que
  pour chaque patient 24975 gènes sont en moyenne affecté par au moins un
  variant pour en moyenne 122735 transcrits (soit environs 5 transcrits
  par gènes) (\textbf{Figure : }\ref{fig:plotvarannotation} - \textbf{A}).
  
  Chaque variant affectera l'ensemble des transcrits qu'il chevauche,
  ainsi un même variant pourra impacter plusieurs transcrits. Ces impacts
  sont ensuite classés par VEP en quatre catégories qui sont, de la plus
  délétère à la moins délétère : HIGH, MODERATE, LOW, MODIFIER
  (\textbf{Table :}\ref{tab:tabvepcsq}). Comme attendu, les variants ayant
  un impact tronquant se retrouvent être les moins fréquent chez chacun de
  nos patients. Ceci est d'autant plus flagrant pour l'impact HIGH qui
  regroupe, entre autres, les variants créant un codon stop ou encore ceux
  causant un décalage du cadre de lecture (\textbf{Table
  :}\ref{tab:tabvepcsq}), se retrouvent en quantité extrêmement faible
  puisqu'ils ne représentent en moyenne que 0.15 \% des variants, soit une
  moyenne de 466 hétérozygotes et 370 homozygotes par patient)
  (\textbf{Figure : }\ref{fig:plotvarannotation} - \textbf{B}).
  
  Parmi ces variants, certains étaient déjà recensés dans une des trois
  base donnée (ExAC, ESP et 1KG). Ainsi, on peut observer qu'entre 38.6 et
  55.5 \% de nos variant étaient listés dans ExAC et entre 33.1 et 43.8 \%
  dans ESP. En revanche environ 87.1 \% d'entre eux sont recensés dans 1KG
  (\textbf{Figure : }\ref{fig:plotvarannotation} - \textbf{C}) (À discuter
  !!!!!).
  
  (À discuter !!!!!) (\textbf{Figure : }\ref{fig:plotvarannotation} -
  \textbf{D})
  
  LES FIGURES SUR LA FRÉQUENCE SONT À DISCUTER CAR LEUR INTERPRÉTATION ME
  LAISSE PERPLEX (SURTOUT LA PROPORTION DE NOS VARIANTS PRÉSENTS DANS 1KG)
  
  \newpage
  
  \begin{figure}
  
  {\centering \includegraphics{thesis_files/figure-latex/plotvarannotation-1} 
  
  }
  
  \caption[Annotation des variants par VEP]{Annotation des variants par VEP : **A** : Quantification du nombre de gènes (en bleu) / transcrits (en rose) impactés par au moins un variant pour chaque patient chacun représentés par une barre. **B** : Distribution des impact HIGH MODERATE LOW et MODIFIER en fonction des patients et du génotype du variant. **C** : Pourcentage de nos variants retrouvés au sein des trois bases de données : ExAC, ESP et 1KG. **D** : Distribution des fréquences de nos variants au sein des trois bases de données : ExAC, ESP et 1KG}\label{fig:plotvarannotation}
  \end{figure}
  
  \newpage
  
  \paragraph{Résultats du filtrage}\label{resultats-du-filtrage}
  
  Les étapes précédentes nous ont permis de mettre en évidence pour chaque
  patient une liste de variants passant l'ensemble de nos critères
  qualités. Ces variants ont dès lors put être annotés nous permettant
  entre autres d'avoir connaissance de leurs l'impacts sur les différents
  transcrits qu'ils chevauchent ou encore leur fréquence dans la
  population générale. Désormais, afin de ne conserver que les variants
  ayant la plus forte probabilité d'être responsable du phénotype de ces
  patients, nous avons appliqué successivement six filtres basés à la fois
  sur les différentes annotations que nous avons ajoutées mais aussi sur
  nos connaissances du mode de transmission du phénotype :
  
  \begin{enumerate}
  \def\labelenumi{\arabic{enumi}.}
  \tightlist
  \item
    \textbf{Filtre 1 : L'union des variants :} Dans ces différentes
    études, nous avons à chaque fois séquencé des duos ou des trios
    d'individus provenant de même fratries et étant caractérisés par le
    même phénotype. Ainsi nous avons pu formuler l'hypothèse d'une cause
    génétique commune entre les différents patients d'une même famille et
    donc filtrer l'ensemble des variants qui ne sont pas partagés par
    l'ensemble des membres de la fratrie.\\
  \item
    \textbf{Filtre 2 : Génotype des variants :} Dans ces études, nous
    avons émis l'hypothèse d'une transmission récessive du phénotype.
    Ainsi, seul les variants homozygotes ont été conservés.
    (\textbf{Figure : }\ref{fig:resvarcall}, \ref{fig:comparefilter}).\\
  \item
    \textbf{Filtre 3 : Impact du variant :} Afin de ne conserver que les
    variants ayant un effet potentiellement tronquant sur la protéine,
    nous avons filtré les variants intronique et ceux tombant dans les
    séquences UTRs. De même les variants synonymes ne sont pas conservés
    (exceptés ceux se trouvant proches des régions d'épissage) car ceux-ci
    n'ont aucun effet sur séquences protéique. Pour les variants faux sens
    (changement d'un seul aa de la séquence protéique) il est plus
    difficile de se décider {[}TODO insert citation{]} nous avons donc
    utilisé les logiciels SIFT (Kumar, Henikoff, \& Ng,
    \protect\hyperlink{ref-Kumar2009}{2009}) et Polyphen (Adzhubei et al.,
    \protect\hyperlink{ref-Adzhubei2010}{2010}) et filtré l'ensemble des
    faux-sens prédit comme \emph{tolerated} par SIFT et \emph{benign} par
    Polyphen.\\
  \item
    \textbf{Filtre 4 : Les transcrits ``non pertinents'' :} Au cours de
    nos analyses nous nous sommes concentré uniquement sur les transcrits
    codant pour une protéine. Ainsi, l'ensemble des transcrits annotés
    comme étant non codant furent filtrés. De même Le mécanisme NMD
    (\emph{nonsense-mediated decay}) a pour but de contrôler la qualité
    des ARNm cellulaires chez les eucaryotes (Y.-F. Chang, Imam, \&
    Wilkinson, \protect\hyperlink{ref-Chang2007}{2007}) en éliminant les
    ARNm qui comportent un codon stop prématuré (Baker \& Parker,
    \protect\hyperlink{ref-Baker2004}{2004}), pouvant être le résultat
    d'une erreur de transcription, d'une mutation ou encore d'une erreur
    d'épissage. Il est donc peu probable que les variants présents sur
    transcrits annotés NMD soient responsables du phénotype. Dès lors, ces
    transcrits furent eux aussi filtrés. Ainsi, nous avons pu retirer de
    nos listes de variants l'ensemble des mutations impactant
    \textbf{uniquement} des transcrits non codant et / ou annoté NMD.
    Cette étape de filtre permet à elle seule de systématiquement filtrer
    entre 36576 et 44581 transcrits différents par patients, soit une
    moyenne de NaN variants par individus (\textbf{Figure :
    }\ref{fig:plotfilternonpertinanttr}).
  \end{enumerate}
  
  \begin{figure}
  
  {\centering \includegraphics{thesis_files/figure-latex/plotfilternonpertinanttr-1} 
  
  }
  
  \caption[Filtrage des transcrits jugés "non pertinents" et des variants les chevauchant]{Filtrage des transcrits jugés "non pertinents" et des variants les chevauchant : Pour chaque patients nous avons filtrer les transcrits jugés "non pertinents" pour l'analyse, c'est à dire ceux ne codant pas pour une protéine et ceux annoté NMD. Dès lors, l'intégralité des variants chevauchant uniquement des transcrits non pertinents ont put systématiquement être filtrés (boites rouges). les autres furent conservés (boites vertes)}\label{fig:plotfilternonpertinanttr}
  \end{figure}
  
  \begin{enumerate}
  \def\labelenumi{\arabic{enumi}.}
  \setcounter{enumi}{4}
  \tightlist
  \item
    \textbf{Fréquence des variants :} La fréquence d'un variant dans la
    population générale est un moyen rapide d'avoir un avis sur l'effet
    délétère de celui-ci. En effet, il est peu probable qu'un retrouvé
    fréquemment dans la population générale soit causal d'une pathologie
    sévère. Ainsi nous avons filtré pour l'ensemble de nos patients
    l'ensemble des variants ayant une fréquence \(\ge\) 0.01 dans l'une
    des trois bases de données que sont ExAC, ESP et 1KG.\\
  \item
    \textbf{Présence des variants dans la cohorte contrôle :} Au cours de
    nos différentes études, nous avons été amenés à séquencé 134.
    L'ensemble de ces individus peuvent être soit sains soit présenter
    l'un des 6 phénotypes étudié au cours de nos différentes études
    (\textbf{Table : }\ref{tab:TODO}). Ces phénotypes étant très
    différent, il n'est pas aberrant d'émettre l'hypothèse qu'ils que
    leurs causes génétiques le soient également. De même, les variants
    recherchés étant rares, il est peu probable qu'un individu porte les
    variants de deux phénotypes différents. Ainsi, pour chacune des 6
    familles, nous avons pu constituer une cohorte contrôle composée dans
    l'ensemble des patients précédemment analysés et ne présentant pas le
    même phénotype que celui étudié dans la famille (\textbf{Figure :}
    \ref{fig:plotsamplectrl}). Dès lors, nous avons pu filtrer l'ensemble
    des variants retrouvés à la fois chez nos patients et observés à
    l'état homozygote dans la cohorte contrôle.
  \end{enumerate}
  
  \begin{figure}
  
  {\centering \includegraphics{thesis_files/figure-latex/plotsamplectrl-1} 
  
  }
  
  \caption[Nombre d'individus composant la cohorte contrôle de chaque famille]{Nombre d'individus composant la cohorte contrôle de chaque famille : Ici, chaque barre représente une famille et sa hauteur est déterminée par le nombre d'individus composant la cohorte contrôle à laquelle elle a été confronté. Chaque individu de la cohorte contrôle a été séquencés en WES par notre équipe. Afin d'être considéré comme "contrôle" et intégrer cette cohorte, un individu doit être sain ou présenter un phénotype d'infertilité différent de la famille étudiée. Par exemple, un individus MMAF pourra servir de contrôle aux familles AZ et FF mais pas aux familles MMAF1-4}\label{fig:plotsamplectrl}
  \end{figure}
  
  \newpage
  
  \newpage
  
  Comme on pouvait s'y attendre, ces six filtres ont un pouvoir
  discriminant extrêmement différent (\textbf{Figure :}
  \ref{fig:plotcomparefilter}). En effet, tandis que le filtre
  ``Transcript relevance'' (filtre n°4) éliminer en moyenne 3.9 \% des
  variants de chaque individu tandis que le filtre ``Variant impact''
  (filtre n° 3) élimine jusqu'à 90.1 \% de ces mêmes variants
  (\textbf{Figure :} \ref{fig:plotcomparefilter} - \textbf{A}). Cette
  différence n'est pas surprenante. En effet, comme nous l'avions vu plus
  tôt, les variants de la catégorie VEP MODIFIER qui regroupe entre autres
  les variants chevauchant les séquences UTRs et introniques
  (\textbf{Table :} ) représentent en moyenne \ldots{} \% des variants de
  nos patients (\textbf{Figure :} \ref{fig:plotvarannotation} -
  \textbf{A}). Ceux-ci étant tous filtrés, on s'attendait donc à une
  valeur aussi élevée. On peut également constater l'importance de la
  cohorte contrôle qui, je le rappelle, permet de filtrer l'ensemble des
  variants homozygotes observés en son sein, puisque ce filtre permet
  retirer entre 76.5 et 88.4\% des variants de chaque individus
  (\textbf{Figure :} \ref{fig:plotvarannotation} - \textbf{A}).
  
  Cependant, regarder uniquement le pourcentage de variants filtrés par
  chaque filtre révèle une information partielle. En effet, dans ce cas de
  figure, on observe la quantité de variant éliminé par chaque filtre
  indépendamment les uns des autres. Ainsi, un même variant peut donc être
  filtrer par plusieurs filtres. Dès lors, il faut également analyser la
  quantité de variants filtrés \textbf{spécifiquement} par chaque filtre.
  Ainsi, on peut constater que le classement des filtres en fonctions de
  leur stringeance reste quasi identique (\textbf{Figure :}
  \ref{fig:plotcomparefilter} - \textbf{B}) il est tout de même
  intéressant de noter que désormais le filtre ``Variant impact'' apparait
  moins efficace que les filtres ``Ctrl'' et ``Genotype'' en filtrant
  spécifiquement une moyenne de 253 variants par individu contre 423 pour
  le filtre génotype et 882 pour le filtre ``Ctrl''. Ainsi, ce dernier
  devient celui filtrant spécifiquement le plus de variants avec entre 364
  et 1060 variants spécifiquement filtrés par patients confirmant ainsi
  l'importance de ce filtre dans nos analyses. Aussi, les filtres
  ``Transcript relevance'', ``Union'' et ``Frequency'' apparaissent
  désormais comme étant anecdotiques en comparaison aux trois autres
  filtres puisqu'ils filtrent au maximum 43 variants spécifiques
  (\textbf{Figure :} \ref{fig:plotcomparefilter} - \textbf{B}).
  
  \newpage
  
  \begin{figure}
  
  {\centering \includegraphics{thesis_files/figure-latex/plotcomparefilter-1} 
  
  }
  
  \caption[Comparaison de l'efficacité de chacun des six filtres utilisés]{Comparaison de l'efficacité de chacun des six filtres utilisés : **A** : Comparaison du pourcentage de variants filtrés par chacun des six filtres indépendamment les uns des autres pour chaque patient (représenté par les points. Dès lors, un même variants peut être filtré par plusieurs filtres. **B** : Comparaison du nombre de variant filtrés spécifiquement par chacun des filtres. Ici, un variant ne peut-être filtré que par un seul filtre}\label{fig:plotcomparefilter}
  \end{figure}
  
  \newpage
  
  Après avoir appliqué l'ensemble de ces filtres, seuls quelques variants
  subsistent nous permettant d'obtenir une liste de gènes restreinte pour
  chaque famille (\textbf{Table : }\ref{tab:tablegene}) et ainsi de tirer
  des conclusions quant au variant responsable du phénotype.
  
  \begin{enumerate}
  \def\labelenumi{\arabic{enumi}.}
  \item
    \textbf{Famille AZ} : Parmi les 2 gènes restant pour cette famille,
    \emph{SPINK2} est apparu comme étant un candidat évident. Notamment
    son expression étant spécifique au testicule tandis que celle de
    \emph{GUF1} est ubiquitaire (TODO fig). De plus, des mutations du gène
    \emph{Spink2} chez la souris avait déjà été identifiée comme induisant
    des défauts de la spermatogenèse (Lee et al.,
    \protect\hyperlink{ref-Lee2011}{2011}).
  \item
    \textbf{Famille FF} : Pour cette famille, le gène
    \emph{PLC}\(\zeta 1\) a passé l'ensemble des filtres. Nos
    connaissances sur la fonction de se gène et notamment son rôle dans
    l'activation ovocytaire (TODO: REF) on fait de ce gène le candidat
    idéal pour expliquer le phénotype de ces deux frères.\\
  \item
    \textbf{Famille MMAF1} : L'analyse bibliographique des 2 gènes ayant
    passé l'ensemble des filtres n'a ici pu nous permettre de d'affirmer
    que l'un de ces gènes étaient responsable du phénotype MMAF de ces 2
    frères.\\
  \item
    \textbf{Famille MMAF2} : À l'issue des filtres, 2 gènes ressortaient
    chez ces deux frères : \emph{MYH11} et \emph{DNAH1}. Or, notre équipe
    ayant déjà, il y a quelques années établit le lien entre des mutations
    du gène \emph{DNAH1} et le syndrome MMAF (Ben Khelifa et al.,
    \protect\hyperlink{ref-BenKhelifa2014}{2014}) ce gène s'est révélé
    être un candidat idéal pour expliquer le phénotype de ces 2 frères. De
    plus, l'implication de \emph{MYH11} dans le phénotype de dissection
    aortique (Imai et al., \protect\hyperlink{ref-Imai2015}{2015}) l'ont
    écarté des candidats pour le phénotype MMAF.\\
  \item
    \textbf{Famille MMAF3} : Comme pour les gènes de la famille MMAF2,
    l'analyse bibliographique des 5 gènes ayant ici passé les filtres de
    même que l'étude de leurs expressions ne nous a pas permis de conclure
    que l'un d'entre eux étaient responsable du phénotype MMAF de ces 2
    frères.\\
  \item
    \textbf{Famille MMAF4} : Seul le gène \emph{TGIF2} a passé l'ensemble
    des filtres pour la famille MMAF4. L'expression ubiquitaire de ce gène
    n'en font pas un candidat idéal. Cependant une étude de 2011 effectuée
    sur le wallaby décrit que la protéine TGIF2 localise spécifiquement
    dans le cytoplasme du spermatide, ainsi que dans le corps résiduel et
    la pièce intermédiaire du flagelle du spermatozoïde mature (Hu, Yu,
    Shaw, Renfree, \& Pask, \protect\hyperlink{ref-Hu2011}{2011}). Ces
    données pourraient corréler avec le phénotype MMAF de ces 3 frères.
  \end{enumerate}
  
  \newpage
  
  \begin{longtable}[t]{llllll}
  \caption{\label{tab:tablegene}Liste des gènes ayant passé l'ensemble des filtres pour chaque famille}\\
  \toprule
  AZ & FF & MMAF1 & MMAF2 & MMAF3 & MMAF4\\
  \midrule
  SPINK2 & PLCZ1 & PLA2G4B & MYH11 & PCSK5 & TGIF2\\
  GUF1 &  & JMJD7-PLA2G4B & DNAH1 & WEE2 & \\
   &  &  &  & GBP2 & \\
   &  &  &  & FCGR3A & \\
   &  &  &  & ZFYVE28 & \\
  \bottomrule
  \end{longtable}
  
  \subsubsection{Discussion}\label{discussion}
  
  L'analyse de ces 6 familles nous a permis de mettre en évidence
  l'efficacité de notre pipeline d'analyse puisque pour 3 d'entre elles
  (soit 50\%) le variant causal a pu être identifié avec certitude
  (\textbf{Figure : }\ref{fig:plotremaininggenes}) et les résultats
  publiés dans trois revus dont je suis co-auteur :
  
  \begin{enumerate}
  \def\labelenumi{\arabic{enumi}.}
  \tightlist
  \item
    \textbf{Famille AZ} : \protect\hyperlink{spink2}{\textbf{SPINK2
    deficiency causes infertility by inducing sperm defects in
    heterozygotes and azoospermia in homozygotes}} : Dans cet article j'ai
    effectué non seulement l'intégralité des analyses bioinformatiques des
    données d'exomes de deux frères infertiles présentant un phénotype
    d'azoospermie mais aussi séquencer en Sanger les séquences codantes du
    gène \emph{SPINK2} pour une partie des 611 individus analyser ainsi
    que contribué à l'extraction de l'ARN testiculaire des souris pour
    l'analyse fonctionnelle du gène \emph{Spink2} sur le modèle murin.\\
  \item
    \textbf{Famille FF} : \protect\hyperlink{plcz}{\textbf{Homozygous
    mutation of PLCZ1 leads to defective human oocyte activation and
    infertility that is not rescued by the WW-binding protein PAWP}} :
    Dans cet article j'ai, effectué l'intégralité des analyses
    bioinformatiques des données d'exomes effectués sur deux frères
    infertiles présentant des échecs de fécondation.\\
  \item
    \textbf{Famille MMAF2} :
    \protect\hyperlink{famdnah1}{\textbf{Whole-exome sequencing of
    familial cases of multiple morphological abnormalities of the sperm
    flagella (MMAF) reveals new DNAH1 mutations}} : Dans cet article j'ai,
    comme précédemment, effectué l'ensemble des analyses bioinformatiques
    des données d'exomes effectués sur deux frères infertiles présentant
    des échecs de fécondation.
  \end{enumerate}
  
  Pour une d'entre elle, un candidat potentiel a pu être mis en évidence
  avec le gène \emph{TGIF2} et notre équipe travaille actuellement sur la
  caractérisation de ce gène afin de savoir s'il peut effectivement
  expliquer le phénotype MMAF de cette famille (\textbf{Figure :
  }\ref{fig:plotremaininggenes}).
  
  Pour les 2 familles restantes, aucun variant n'a pu pour l'instant
  expliquer leur phénotype. L'explication la plus vraisemblable est que le
  variant ait été filtré par l'un de nos six filtres, probablement celui
  consistant à filtrer l'ensemble des variants hétérozygotes. En effet,
  l'hypothèse d'un variant causal homozygote était extrêmement crédible
  pour les familles AZ, FF et MMAF2 étant donné l'historique consanguin de
  ces 3 familles dont les parents sont à chaque fois apparentés. En
  revanche rien ne laisse supposé une telle chose pour les familles
  restantes. Cependant, le filtre des variants hétérozygotes pour
  l'ensemble des patients de ces 3 familles a été maintenu en première
  intention afin de faciliter les analyses en réduisant au maximum le
  nombre de variant. Au vus des résultats il apparait clair que les
  variants responsablent de leur phénotype aient été filtrés pour au moins
  2 de ces familles. Dès lors, l'ensemble des analyses effectuées lors de
  l'étape de filtrage doivent être refaites en changeant les paramètres de
  filtrage. Cette fois-ci, les variants hétérozygotes seront conservés et
  les gènes sur lesquels au moins deux variants hétérozygotes seront
  recensés seront analysés en priorité. En effet, bien que les analyses
  exomiques nous fournissent en l'état pas d'informations suffisante pour
  savoir si ces deux variants sont présent sur le même allèle ou bien sur
  deux allèles différents, cela pourrait-être la signature de variants
  hétérozygotes composites. C'est donc sur ces analyses que se concentre
  actuellement notre équipe.
  
  \begin{figure}
  
  {\centering \includegraphics{thesis_files/figure-latex/plotremaininggenes-1} 
  
  }
  
  \caption[Nombre de gènes passant l'ensemble des filtres par famille]{Nombre de gènes passant l'ensemble des filtres par famille  :  Chaque barre représente une des familles analysées. La hauteur de cette barre correspond au nombre de gènes ayant passé l'ensemble des filtres pour chaque famille. Les barres vertes caractérisent les familles pour lesquelles le gène responsable de la pathologie a été identifié parmi la liste de gène (dans ce cas le symbole du gène est écrit au-dessus de la barre). La barre orange caractérise la famille pour laquelle un candidat potentiel a été identifié (le symbole du gène est écrit au-dessus suivit d'un "?"). Les barres rouges indiquent qu'aucun des gènes ayant passé les filtres pour ne semble expliquer le phénotype (dans ce cas il est écrit "???" au-dessus de la barre)}\label{fig:plotremaininggenes}
  \end{figure}
  
  \newpage
  
  \subsection{Etude d'une large cohorte de patients
  MMAF}\label{cohortemmah}
  
  \subsubsection{Description de la
  cohorte}\label{description-de-la-cohorte}
  
  Historique : après avoir mis en évidence DNAH1 -\textgreater{} MMAF
  notre équipe s'est en partie spécialisé dans ce syndrome.
  
  ainsi, entre (année) et année, notre équipe a effectué le séquençage de
  \ldots{} individus présentant ce phénotype afin d'en établir la cause
  génétique. parmi ces patients, la majorité provenait d'Afrique du Nord,
  cependant \ldots{} vfenaient de et de \ldots{} ces séquençage ont été
  effectué dans \ldots{} centres diférents que sont (listes des centre de
  séquençage) et sur \ldots{} plateforme : liste des plateformes
  
  \begin{longtable}[t]{lrr}
  \caption{\label{tab:tabcohort}Liste des différents projets de séquençages effectués}\\
  \toprule
  Place & Year & Nb of sequenced individuals\\
  \midrule
  MountSinai & 2012 & 2\\
  Strasbourg & 2012 & 13\\
  Genoscope & 2013 & 13\\
  Genoscope & 2014 & 28\\
  Genoscope & 2015 & 6\\
  \bottomrule
  \end{longtable}
  
  \newpage
  
  \subsubsection{Application de la pipeline -
  Résultats}\label{application-de-la-pipeline---resultats}
  
  Après avoir appelé les variants de nos \ldots{} patients, nous avons
  obtenu une total de 4484558 variants différents comprenant 4160274 SNVs
  et 324284. Ces variants étant répartit entre chaque patients qui
  portaient environs chacun 81618 SNV et 5148 indels dont 0 \% étaient
  homozygote. Comme on peut le voir, la proportion de chaque appel est
  relativement homogène lorsque l'on compare les patients ayant été
  séquencés dans le même centre la même année. Cependant, il est possible
  de noter de grandes disparités lorsque l'on compare les données
  provenant de différents centres ou bien du même centre avec plusieurs
  années de différences. Ces écarts peuvent-être causés par plusieurs
  facteur, tel que les différents kits de capture d'exons qui on put être
  utilisés puisque \ldots{} (todo lister les différents kit de capture
  dans une table) en revanche nous pouvons écarté un effet dus à la
  platforme de séquençage ou encore le modèle de séquenceur puisque tout
  ces projets ont été réalisés sur des Illumina HiSeq2000 (\textbf{Table :
  }\ref{tab:tabcohort}) (\textbf{Figure : }\ref{fig:plotbigmmafcall} -
  \textbf{A}).
  
  Le même constat peut être effectué lorsque l'on compare la qualité des
  appels puisque plus les projets de séquençage s'avèrent être récent,
  plus la proportion d'appel \emph{Single Strand} s'avère être faible
  tandis que la proportion d'appel \emph{Double Strand} (DS) est élevée.
  Ceci est une bonne chose, car, bien que ces deux appels soient conservés
  dans les analyses ulterieures, les appels DS sont de meilleur qualité
  que les appels SS. Cette augmentation des appels DS au cours du temps
  pourrait s'expliquer par une amélioration des protocole de séquençage
  qinsi que des kit de capture. En revanche cela est à pondérer avec le
  taux croissant d'appels \emph{No-strand} (NS) au fur et à mesure des
  années pour atteindre environs \ldots{} \% en (\ldots{}Année) avec un
  projet réalisé au Génoscope. Ces derniers appels étant systématiquement
  filtrés, ils n'altèreront en rien les résultats obtenus en aval ormis le
  fait qu'ils réduisent la quantité des données utilisées (\textbf{Figure
  : }\ref{fig:plotbigmmafcall} - \textbf{B} et \textbf{C}).
  
  \newpage
  
  \begin{figure}
  
  {\centering \includegraphics{thesis_files/figure-latex/plotbigmmafcall-1} 
  
  }
  
  \caption[Résultats de l'appel des variants par individus et par projet de séquençage]{Résultats de l'appel des variants par individus et par projet de séquençage : Chaque couleur définit un projet de séquençage caractérisé par un centre de séquençage et une année. **A** : Quantification pour chaque individus (représentés par les barres) du nombre de variants (SNVs et Indels) appelés homozygotes et hétérozygotes. **B** : Quantification des appels *Double Strand* (DS), *Single Strand* (SS) et *No strand* (NS) pour chaque projet de séquençage. **C** : Même chose en pourcentage}\label{fig:plotbigmmafcall}
  \end{figure}
  
  \newpage
  
  \subsubsection{Analyse des listes de
  gènes}\label{analyse-des-listes-de-genes}
  
  Après avoir appliqué les mêmes filtres que ceux décrit précédement à
  l'exception du filtre n°\ldots{} Union puisqu'ici nous avons uniquement
  des individus non apparentés, nous avons put obtenir une liste de 1568
  variants différents composés de 1359 SNVs et 209 indels et impactant un
  total de 1306 gènes distincts. Ces variants étant répartis sur
  l'ensemble de nos \ldots{} patients ceux-ci portaient en moyenne 25 SNVS
  et 4 indels, de sorte que chacun d'entre eux avaient entre 1 et 73 gènes
  impactés par au moins un variants (\textbf{Figure :
  }\ref{fig:plotfilterbigmmaf} - \textbf{A} et \textbf{B}).
  
  Parmis l'ensemble de nos patients, 3 révélèrent porter au moins un
  variant passant l'ensemble des filtres sur le gène \emph{DNAH1},
  candidat évident pour ce phénotype. Ainsi, le patient Ghs90 porte 3
  variants successifs induisant 3 variation faux-sens, le patient Ghs95
  porte lui un seul variant entrainant lui aussi un faux-sens et le
  patient Ghs122 porte un indel entrainant un décalage du cadre de lecture
  (\textbf{Table : }\ref{tab:tabdnah1}).
  
  \begin{longtable}[t]{lllll}
  \caption{\label{tab:tabdnah1}liste des variants passant les filtres et chevauchant le gène *DNAH1*}\\
  \toprule
  Run ID & Variant coordinates & Genotype & Consequence & Gene symbol\\
  \midrule
  Ghs122 & 3-52414073-CC-C & Homozygous & frameshift & DNAH1\\
  Ghs90 & 3-52382919-A-C & Homozygous & missense & DNAH1\\
  Ghs90 & 3-52382920-T-C & Homozygous & missense & DNAH1\\
  Ghs90 & 3-52382922-T-C & Homozygous & missense & DNAH1\\
  Ghs95 & 3-52422540-C-G & Homozygous & missense & DNAH1\\
  \addlinespace
  Ghs129 & 3-52409423-T-A & Heterozygous & missense & DNAH1\\
  Ghs129 & 3-52423486-C-G & Heterozygous & missense & DNAH1\\
  Ghs28 & 3-52361911-A-G & Heterozygous & missense & DNAH1\\
  Ghs28 & 3-52366296-A-G & Heterozygous & missense & DNAH1\\
  Ghs36 & 3-52362004-T-G & Heterozygous & missense & DNAH1\\
  \addlinespace
  Ghs36 & 3-52391702-C-T & Heterozygous & stop & DNAH1\\
  Ghs42 & 3-52407000-G-A & Heterozygous & missense & DNAH1\\
  Ghs42 & 3-52417872-C-G & Heterozygous & splice & DNAH1\\
  Ghs87 & 3-52387600-G-A & Heterozygous & missense & DNAH1\\
  Ghs87 & 3-52420751-A-C & Heterozygous & missense & DNAH1\\
  \addlinespace
  Ghs88 & 3-52383006-G-A & Heterozygous & missense & DNAH1\\
  Ghs88 & 3-52391648-G-A & Heterozygous & missense & DNAH1\\
  \bottomrule
  \end{longtable}
  
  Ensuite, afin de nous concentrer sur les gènes ayant le plus de risque
  d'être impliquer dans le phénotype MMAF, nous avons étudié en priorité
  ceux sur lesquels plusieurs patients portaient au moins un variant ayant
  passé les filtres. Ainsi, nous avons obtenu une liste de 194 gènes dont
  135 (soit 70 \%) étaient retrouvés variants chez uniquement 2 patients
  (\textbf{Figure : }\ref{fig:plotfilterbigmmaf} - \textbf{C}).
  
  \newpage
  
  \begin{figure}
  
  {\centering \includegraphics{thesis_files/figure-latex/plotfilterbigmmaf-1} 
  
  }
  
  \caption[TODOOOOOOOOOOOOOOOOOO]{TODOOOOOOOOOOOOOOOOOO : **A** : Quantification du nombre de SNVs et indels ayant passé l'ensemble des filtres pour chaque patients. **B** : Nombre de gènes impactés par au moins un variant ayant passé les filtres pour chaque individus représentés par les barres. **C** : Présentation }\label{fig:plotfilterbigmmaf}
  \end{figure}
  
  \newpage
  
  L'étude de ces gènes nous a tout d'abord permis d'identifier deux gènes
  : le gène \emph{WDR96} (récemment renommé \emph{CFAP43}) et le gène
  \emph{WDR52} récemment renommé \emph{CFAP44}). Ces gènes ont
  respectivement été retrouvés mutés chez 6 et 3 patients. En plus du
  nombre important de patients portant une tronquante mutation sur ces
  gènes (\textbf{Table : }\ref{tab:tabwdr96} et \ref{tab:tabwdr52}), tout
  deux sont retrouvés comme étant sur-exprimés dans le testicule dans les
  bases de donnée publiques. De même, ces deux gènes avaient déjà été
  décrits dans la litterature comme étant impliqués dans la structure et /
  ou le fonctionnement du flagelle spermatique (Ivliev, 't Hoen, Roon-Mom,
  Peters, \& Sergeeva, \protect\hyperlink{ref-Ivliev2012}{2012}). Ces
  informations ont ainsi fait de ces deux gènes des candidats idéaux pour
  expliquer le phénotype de ces 9 individus.
  
  De même, cette étude nous a permis égallement de relever 3 patients
  présentants des mutations dans le gènes \emph{FSIP2} (\textbf{Table :
  }\ref{tab:tabfsip2}. Comme pour les gènes \emph{WDR96} et \emph{WDR52},
  ce gène présente une forte et quasi exclusive expression testiculaire.
  De plus, une équipe a démontré en 2003 l'implication de ce gène dans la
  structure de la gaine fibreuse su flagelle spermatique (Brown, Miki,
  Harper, \& Eddy, \protect\hyperlink{ref-Brown2003}{2003}) faisant
  égallement de ce gène un excellent candidat dans l'explication du
  phénotype MMAF de ces patients.
  
  Après avoir identifié la cause génétique du phénotype MMAF de ces 15
  patients (3 patients \emph{DNAH1}, 6 \emph{WDR96}, 3 \emph{WDR52} et 3
  \emph{FSIP2}) grâce à des variants homozygotes, nous avons chercher
  l'ensemble des patients pouvant être hétérozygotes composites, c'est à
  dire des patients portant deux allèles différents mutés sur le même
  gène, l'un venant de la mère, l'autre venant du père. Pour cela, nous
  avons recansser l'ensemble des patients portant au moins deux variants
  hétérozygotes différents sur un de ces 4 gènes. Il faut tout de même
  noter la limite de cette approche dûe au fait de la non connaissance du
  phasage des variants, c'est à dire qu'il nous est impossible de
  déterminer si les deux variants sont situés sur le même allèle ou bien
  sur deux allèles différents.
  
  Néanmoins, cette stratégie nous a permis d'identifier 6 nouveaux
  patients présentant des mutation sur le gène \emph{DNAH1}, 1 sur WDR96
  et 5 sur \emph{FSIP2}.
  
  \newpage
  
  \begin{longtable}[t]{lllll}
  \caption{\label{tab:tabwdr96}liste des variants passant les filtres et chevauchant le gène *WDR96*}\\
  \toprule
  Run ID & Variant coordinates & Genotype & Consequence & Gene symbol\\
  \midrule
  Ghs102 & 10-105905296-TT-T & Homozygous & frameshift & WDR96\\
  Ghs105 & 10-105912486-T-G & Homozygous & splice & WDR96\\
  Ghs126 & 10-105921781-G-A & Homozygous & stop & WDR96\\
  Ghs17 & 10-105928535-C-T & Homozygous & stop & WDR96\\
  Ghs25 & 10-105944769-C-T & Homozygous & splice & WDR96\\
  \addlinespace
  Ghs41 & 10-105928513-G-A & Homozygous & stop & WDR96\\
  Ghs132 & 10-105953765-A-AA & Heterozygous & frameshift & WDR96\\
  Ghs132 & 10-105963485-A-G & Heterozygous & missense & WDR96\\
  \bottomrule
  \end{longtable}
  
  \begin{longtable}[t]{lllll}
  \caption{\label{tab:tabwdr52}liste des variants passant les filtres et chevauchant le gène *WDR52*}\\
  \toprule
  Run ID & Variant coordinates & Genotype & Consequence & Gene symbol\\
  \midrule
  Ghs22 & 3-113063450-G-A & Homozygous & stop & WDR52\\
  Ghs34 & 3-113114596-C-T & Homozygous & splice & WDR52\\
  Ghs89 & 3-113119409-G-A & Homozygous & missense & WDR52\\
  \bottomrule
  \end{longtable}
  
  \begin{longtable}[t]{lllll}
  \caption{\label{tab:tabfsip2}liste des variants passant les filtres et chevauchant le gène *FSIP2*}\\
  \toprule
  Run ID & Variant coordinates & Genotype & Consequence & Gene symbol\\
  \midrule
  Ghs131 & 2-186603611-C-T & Homozygous & missense & FSIP2\\
  Ghs20 & 2-186654145-A-AA & Homozygous & frameshift & FSIP2\\
  Ghs21 & 2-186618487-AC-A & Homozygous & frameshift & FSIP2\\
  Ghs101 & 2-186603536-T-C & Heterozygous & missense & FSIP2\\
  Ghs101 & 2-186611436-C-T & Heterozygous & missense & FSIP2\\
  \addlinespace
  Ghs122 & 2-186658089-C-A & Heterozygous & missense & FSIP2\\
  Ghs122 & 2-186661399-A-G & Heterozygous & missense & FSIP2\\
  Ghs40 & 2-186626721-A-T & Heterozygous & missense & FSIP2\\
  Ghs40 & 2-186660476-A-G & Heterozygous & missense & FSIP2\\
  Ghs40 & 2-186671231-A-T & Heterozygous & missense & FSIP2\\
  \addlinespace
  Ghs92 & 2-186626721-A-T & Heterozygous & missense & FSIP2\\
  Ghs92 & 2-186671231-A-T & Heterozygous & missense & FSIP2\\
  Ghs95 & 2-186626721-A-T & Heterozygous & missense & FSIP2\\
  Ghs95 & 2-186671231-A-T & Heterozygous & missense & FSIP2\\
  \bottomrule
  \end{longtable}
  
  \newpage
  
  Comme précédemment, la cause génétique expliquant le phénotype de ces 12
  patients ayant été identifiée, leurs données ont été retirées des
  analyses ulterieure réduisant ainsi nos listes à 947 variants et 801
  gènes distincts.
  
  \subsubsection{Disscution}\label{disscution}
  
  L'analyse de cette cohorte de \ldots{} patients MMAF nous à la fois
  permis de confirmer l'importance de l'implication du gène \emph{DNAH1}
  dans ce phénotype grâce à 9 patients présentant des variants sur ce
  gène, mais aussi de mettre en évidence de nouveaux gènes impliqués dans
  ce phénotype. Ainsi, notre pipeline d'analyse a permis d'identifier la
  cause génétique induisant le phénotype MMAF chez \ldots{} individus de
  notre cohorte, soit chez \ldots{} \% des individus la composant.
  
  Ces résultats ont permis l'écriture de \ldots{} articles dont je suis
  co-auteur :
  
  \begin{enumerate}
  \def\labelenumi{\arabic{enumi}.}
  \item
    \textbf{Whole exome cohort study and analysis of mouse and Trypanosoma
    models demonstrate the importance of WDR proteins in flagellogenesis
    and male fertility}, \emph{Nat Genet} (soummis) : Cette article
    présente nos différents résultats dans la caractérisation des gènes
    \emph{WDR96} et \emph{WDR52} ainsi que les différentes preuves de leur
    implication dans le phénotype MMAF.
  \item
    (papier DNAH1) :
  \end{enumerate}
  
  Les résultats sur le gène \emph{FSIP2} ne sont pour l'instant pas
  publiés, notre équipe travaillant à l'heure actuelle à la
  caractérisation de ce gène.
  
  \section{Conclusion}\label{conclusion}
  
  \chapter{MutaScript}\label{mutascript}
  
  \chapter*{Conclusion}\label{conclusion-1}
  \addcontentsline{toc}{chapter}{Conclusion}
  
  \chapter{The First Appendix}\label{the-first-appendix}
  
  \chapter*{References}\label{references}
  \addcontentsline{toc}{chapter}{References}
  
  \hypertarget{refs}{}
  \hypertarget{ref-Adzhubei2010}{}
  Adzhubei, I. A., Schmidt, S., Peshkin, L., Ramensky, V. E., Gerasimova,
  A., Bork, P., \ldots{} Sunyaev, S. R. (2010). A method and server for
  predicting damaging missense mutations. \emph{Nature Methods},
  \emph{7}(4), 248--9. \url{http://doi.org/10.1038/nmeth0410-248}
  
  \hypertarget{ref-Baker2004}{}
  Baker, K. E., \& Parker, R. (2004). Nonsense-mediated mRNA decay:
  terminating erroneous gene expression. \emph{Current Opinion in Cell
  Biology}, \emph{16}(3), 293--9.
  \url{http://doi.org/10.1016/j.ceb.2004.03.003}
  
  \hypertarget{ref-BenKhelifa2014}{}
  Ben Khelifa, M., Coutton, C., Zouari, R., Karaouzène, T., Rendu, J.,
  Bidart, M., \ldots{} Ray, P. F. (2014). Mutations in DNAH1, which
  encodes an inner arm heavy chain dynein, lead to male infertility from
  multiple morphological abnormalities of the sperm flagella.
  \emph{American Journal of Human Genetics}, \emph{94}(1), 95--104.
  \url{http://doi.org/10.1016/j.ajhg.2013.11.017}
  
  \hypertarget{ref-Brown2003}{}
  Brown, P. R., Miki, K., Harper, D. B., \& Eddy, E. M. (2003). A-Kinase
  Anchoring Protein 4 Binding Proteins in the Fibrous Sheath of the Sperm
  Flagellum. \emph{Biology of Reproduction}, \emph{68}(6), 2241--2248.
  \url{http://doi.org/10.1095/biolreprod.102.013466}
  
  \hypertarget{ref-Chang2007}{}
  Chang, Y.-F., Imam, J. S., \& Wilkinson, M. F. (2007). The
  Nonsense-Mediated Decay RNA Surveillance Pathway. \emph{Annual Review of
  Biochemistry}, \emph{76}(1), 51--74.
  \url{http://doi.org/10.1146/annurev.biochem.76.050106.093909}
  
  \hypertarget{ref-DePristo2011}{}
  DePristo, M. A., Banks, E., Poplin, R., Garimella, K. V., Maguire, J.
  R., Hartl, C., \ldots{} Pritchard, E. (2011). A framework for variation
  discovery and genotyping using next-generation DNA sequencing data.
  \emph{Nature Genetics}, \emph{43}(5), 491--498.
  \url{http://doi.org/10.1038/ng.806}
  
  \hypertarget{ref-Hu2011}{}
  Hu, Y., Yu, H., Shaw, G., Renfree, M. B., \& Pask, A. J. (2011).
  Differential roles of TGIF family genes in mammalian reproduction.
  \emph{BMC Developmental Biology}, \emph{11}, 58.
  \url{http://doi.org/10.1186/1471-213X-11-58}
  
  \hypertarget{ref-Imai2015}{}
  Imai, Y., Morita, H., Takeda, N., Miya, F., Hyodo, H., Fujita, D.,
  \ldots{} Komuro, I. (2015). A deletion mutation in myosin heavy chain 11
  causing familial thoracic aortic dissection in two Japanese pedigrees.
  \emph{International Journal of Cardiology}, \emph{195}, 290--292.
  \url{http://doi.org/10.1016/j.ijcard.2015.05.178}
  
  \hypertarget{ref-Ivliev2012}{}
  Ivliev, A. E., 't Hoen, P. A. C., Roon-Mom, W. M. C. van, Peters, D. J.
  M., \& Sergeeva, M. G. (2012). Exploring the Transcriptome of Ciliated
  Cells Using In Silico Dissection of Human Tissues. \emph{PLoS ONE},
  \emph{7}(4), e35618. \url{http://doi.org/10.1371/journal.pone.0035618}
  
  \hypertarget{ref-Kumar2009}{}
  Kumar, P., Henikoff, S., \& Ng, P. C. (2009). Predicting the effects of
  coding non-synonymous variants on protein function using the SIFT
  algorithm. \emph{Nature Protocols}, \emph{4}(7), 1073--1081.
  \url{http://doi.org/10.1038/nprot.2009.86}
  
  \hypertarget{ref-Lee2011}{}
  Lee, B., Park, I., Jin, S., Choi, H., Kwon, J. T., Kim, J., \ldots{}
  Cho, C. (2011). Impaired spermatogenesis and fertility in mice carrying
  a mutation in the Spink2 gene expressed predominantly in testes.
  \emph{The Journal of Biological Chemistry}, \emph{286}(33), 29108--17.
  \url{http://doi.org/10.1074/jbc.M111.244905}
  
  \hypertarget{ref-Lek2016}{}
  Lek, M., Karczewski, K. J., Minikel, E. V., Samocha, K. E., Banks, E.,
  Fennell, T., \ldots{} Exome Aggregation Consortium, D. G. (2016).
  Analysis of protein-coding genetic variation in 60,706 humans.
  \emph{Nature}, \emph{536}(7616), 285--91.
  \url{http://doi.org/10.1038/nature19057}
  
  \hypertarget{ref-Lunter2011}{}
  Lunter, G., \& Goodson, M. (2011). Stampy: A statistical algorithm for
  sensitive and fast mapping of Illumina sequence reads. \emph{Genome
  Research}, \emph{21}(6), 936--939.
  \url{http://doi.org/10.1101/gr.111120.110}
  
  \hypertarget{ref-McLaren2016}{}
  McLaren, W., Gil, L., Hunt, S. E., Riat, H. S., Ritchie, G. R. S.,
  Thormann, A., \ldots{} Cunningham, F. (2016). The Ensembl Variant Effect
  Predictor. \emph{Genome Biology}, \emph{17}(1), 122.
  \url{http://doi.org/10.1186/s13059-016-0974-4}
  
  \hypertarget{ref-Nielsen2011}{}
  Nielsen, R., Paul, J. S., Albrechtsen, A., \& Song, Y. S. (2011).
  Genotype and SNP calling from next-generation sequencing data.
  \emph{Nature Reviews. Genetics}, \emph{12}(6), 443--51.
  \url{http://doi.org/10.1038/nrg2986}
  
  \hypertarget{ref-Su2014}{}
  Su, Z., Łabaj, P. P., Li, S. S., Thierry-Mieg, J., Thierry-Mieg, D.,
  Shi, W., \ldots{} Shi, L. (2014). A comprehensive assessment of RNA-seq
  accuracy, reproducibility and information content by the Sequencing
  Quality Control Consortium. \emph{Nature Biotechnology}, \emph{32}(9),
  903--14. \url{http://doi.org/10.1038/nbt.2957}


  % Index?

\end{document}

