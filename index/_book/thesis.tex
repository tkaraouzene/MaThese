% This is the Reed College LaTeX thesis template. Most of the work
% for the document class was done by Sam Noble (SN), as well as this
% template. Later comments etc. by Ben Salzberg (BTS). Additional
% restructuring and APA support by Jess Youngberg (JY).
% Your comments and suggestions are more than welcome; please email
% them to cus@reed.edu
%
% See http://web.reed.edu/cis/help/latex.html for help. There are a
% great bunch of help pages there, with notes on
% getting started, bibtex, etc. Go there and read it if you're not
% already familiar with LaTeX.
%
% Any line that starts with a percent symbol is a comment.
% They won't show up in the document, and are useful for notes
% to yourself and explaining commands.
% Commenting also removes a line from the document;
% very handy for troubleshooting problems. -BTS

% As far as I know, this follows the requirements laid out in
% the 2002-2003 Senior Handbook. Ask a librarian to check the
% document before binding. -SN

%%
%% Preamble
%%
% \documentclass{<something>} must begin each LaTeX document
\documentclass[12pt,twoside]{reedthesis}
% Packages are extensions to the basic LaTeX functions. Whatever you
% want to typeset, there is probably a package out there for it.
% Chemistry (chemtex), screenplays, you name it.
% Check out CTAN to see: http://www.ctan.org/
%%
\usepackage{graphicx,latexsym}
\usepackage[french]{babel} 
\usepackage{amsmath}
\usepackage{amssymb,amsthm}
\usepackage[dvipsnames]{xcolor} % tk: for more color
\usepackage{xcolor}
\usepackage{eso-pic}
\usepackage{longtable,booktabs,setspace}
\usepackage{chemarr} %% Useful for one reaction arrow, useless if you're not a chem major
\usepackage[hyphens]{url}
\usepackage{tikz}
\usetikzlibrary{calc}
\newcommand\HRule{\rule{\textwidth}{1pt}}
% Added by CII
\usepackage{hyperref}
\usepackage{lmodern}
\usepackage{float}
\floatplacement{figure}{H}
% End of CII addition
\usepackage{rotating}
\usepackage{upgreek} % tk : pour pouvoir utiliser le symbole µ droit (pas en itallic)
\usepackage{pdfpages}
\usepackage{lscape}
\newcommand{\blandscape}{\begin{landscape}}
\newcommand{\elandscape}{\end{landscape}}
\usepackage[utf8]{inputenc}




% Next line commented out by CII
%%% \usepackage{natbib}
% Comment out the natbib line above and uncomment the following two lines to use the new
% biblatex-chicago style, for Chicago A. Also make some changes at the end where the
% bibliography is included.
%\usepackage{biblatex-chicago}
%\bibliography{thesis}


% Added by CII (Thanks, Hadley!)
% Use ref for internal links
\renewcommand{\hyperref}[2][???]{\autoref{#1}}
\def\chapterautorefname{Chapter}
\def\sectionautorefname{Section}
\def\subsectionautorefname{Subsection}
% End of CII addition

% Added by CII
\usepackage{caption}
\captionsetup{width=5in}
% End of CII addition

% \usepackage{times} % other fonts are available like times, bookman, charter, palatino


% To pass between YAML and LaTeX the dollar signs are added by CII
\title{THÈSE}
\author{Thomas Karaouzene}
\labo{}
% The month and year that you submit your FINAL draft TO THE LIBRARY (May or December)
\date{31 octobre 2017}
\division{}
\advisor{Pierre Ray}
%If you have two advisors for some reason, you can use the following
% Uncommented out by CII
\altadvisor{Nicolas Thierry-Mieg}
% End of CII addition

%%% Remember to use the correct department!
\department{Ingénierie de la Santé, de la Cognition et Environnement (EDISCE)}
% if you're writing a thesis in an interdisciplinary major,
% uncomment the line below and change the text as appropriate.
% check the Senior Handbook if unsure.
%\thedivisionof{The Established Interdisciplinary Committee for}
% if you want the approval page to say "Approved for the Committee",
% uncomment the next line
%\approvedforthe{Committee}

% Added by CII
%%% Copied from knitr
%% maxwidth is the original width if it's less than linewidth
%% otherwise use linewidth (to make sure the graphics do not exceed the margin)
\makeatletter
\def\maxwidth{ %
  \ifdim\Gin@nat@width>\linewidth
    \linewidth
  \else
    \Gin@nat@width
  \fi
}
\makeatother

\renewcommand{\contentsname}{Table of Contents}
% End of CII addition

\setlength{\parskip}{0pt}

% Added by CII
  %\setlength{\parskip}{\baselineskip}
  \usepackage[parfill]{parskip}

\providecommand{\tightlist}{%
  \setlength{\itemsep}{0pt}\setlength{\parskip}{0pt}}

\Acknowledgements{

}

\Dedication{

}

\Preface{
This is an example of a thesis setup to use the reed thesis document
class (for LaTeX) and the R bookdown package, in general.
}

\Abstract{

}

	\usepackage{tikz}
% End of CII addition
%%
%% End Preamble
%%
%

\usepackage{amsthm}
\newtheorem{theorem}{Theorem}[section]
\newtheorem{lemma}{Lemma}[section]
\theoremstyle{definition}
\newtheorem{definition}{Definition}[section]
\newtheorem{corollary}{Corollary}[section]
\newtheorem{proposition}{Proposition}[section]
\theoremstyle{definition}
\newtheorem{example}{Example}[section]
\theoremstyle{remark}
\newtheorem*{remark}{Remark}
\begin{document}

% Everything below added by CII
      \maketitle
  
  \frontmatter % this stuff will be roman-numbered
  \pagestyle{empty} % this removes page numbers from the frontmatter

  
      \begin{preface}
      This is an example of a thesis setup to use the reed thesis document
      class (for LaTeX) and the R bookdown package, in general.
    \end{preface}
  
      \hypersetup{linkcolor=black}
    \setcounter{tocdepth}{3}
    \tableofcontents
  
      \listoftables
  
      \listoffigures
  
  
  
  \mainmatter % here the regular arabic numbering starts
  \pagestyle{fancyplain} % turns page numbering back on

  \chapter{Delete line 6 if you only have one
  advisor}\label{delete-line-6-if-you-only-have-one-advisor}
  
  \chapter*{Résumé}\label{resume}
  \addcontentsline{toc}{chapter}{Résumé}
  
  \newpage
  
  L'infertilité masculine est définie comme étant l'incapacité d'aboutir à
  une grossesse après 12 mois ou plus de rapports sexuels réguliers non
  protégés. Dix à quinze pourcents des couples font face à des problèmes
  d'infertilité soit plus de 70 millions de personnes dans le monde
  faisant de cette pathologie un véritable enjeu de santé publique. Bien
  que multifactorielle, l'infertilité masculine a une composante génétique
  importante et encore peu explorée. L'objectif de mon travail a été à la
  fois de poursuivre les investigations génétiques et moléculaires mené au
  sein de laboratoire sur le phénotype de globozoospermie, mais aussi de
  mettre en place un pipeline d'analyse pour les données de séquençage
  haut-débit obtenues dans le cadre d'étude portant sur des patients
  présentant divers phénotypes d'infertilité.
  
  Ainsi, j'ai d'abord contribué à la caractérisation du gène
  \emph{DPY19L2} dont la délétion homozygote est responsable de la majeure
  partie des cas de globozoospermie, une teratozoospermie caractérisée
  entre autres par la présence de spermatozoïdes à têtes rondes et
  dépourvu d'acrosome dans l'éjaculât. Après avoir contribué à
  caractériser les mécaniques responsables de cette délétion récurrente,
  j'ai pu réaliser une étude comparative des transcriptomes testiculaires
  de souris sauvage \emph{Dpy19l2}\textsuperscript{+/+} et de souris KO *
  Dpy19l2\emph{\textsuperscript{-/-}, présentant le même phénotype que
  l'humain, sur puce à ADN. L'objectif de cette étude ayant pour but
  principal de mettre en évidence des dérégulations transcriptomiques chez
  les souris KO pouvant notamment expliquer le faible taux de réussite des
  fécondations }in vitro* effectuées avec des spermatozoïdes
  globozoocéphales, même dans les cas où celles-ci sont effectuées par
  \emph{intra cytoplasmic sperm injection} (ICSI). Cette étude a ainsi
  permis de mettre en évidence la dérégulation de 76 gènes chez la souris
  KO parmi lesquels 70 été retrouvé sur-exprimés et 6 sous exprimés. Parmi
  ceux-ci, 23 étaient impliqués dans la liaison d'acide nucléique et de
  protéine pouvant ainsi expliquer les défauts d'ancrage de l'acrosome au
  noyau chez les spermatozoïdes globozoocéphales.
  
  Dans un second temps, l'émergence du séquençage nouvelle génération
  (NGS) ayant, de par la masse des données générées, totalement bouleversé
  les enjeux ainsi que les méthodes d'analyse des données en génétique,
  j'ai pu au cours de ma thèse mettre en place un pipeline d'analyse des
  données de séquençage exomique. Ce pipeline a été conçu afin de prendre
  en compte l'intégralité des étapes d'analyses effectuée en aval du
  processus de séquençage en utilisant à la fois des logiciels et
  algorithmes existant ainsi que d'autre développés au sein de notre
  équipe. Ainsi, l'alignement des séquences sur un génome de référence est
  effectué par le logiciel MAGIC, l'appel des variants a été développé au
  sein de notre équipe afin de , l'annotation des variant est
  principalement effectué par le logiciel \emph{Variant Effect Predictor}
  tandis que les étape filtrage des variants et la priorisation des gènes
  a été développé de sorte à être personnalisable en fonction de chaque
  étude bien qu'elles aient été optimisées une application dans un
  contexte de recherche de variants impliqués dans des pathologies à
  transmission Mendélienne. Ce manuscrit détaillera donc ce pipeline ainsi
  que son utilisation dans trois études différentes. La première est
  l'analyse de trois cas familiaux présentant des phénotypes différents
  d'infertilité masculine, la deuxième décrira une étude effectuée sur 15
  femmes infertiles tandis que la dernière se concentrera sur une cohorte
  de 62 patients présentant tous un phénotype d'anomalies des flagelles
  spermatiques, le syndrome MMAF.
  
  \newpage
  
  \chapter*{Abstract}\label{abstract}
  \addcontentsline{toc}{chapter}{Abstract}
  
  \newpage
  
  Même chose en anglais
  
  \chapter*{References}\label{references}
  \addcontentsline{toc}{chapter}{References}


  % Index?

\end{document}

