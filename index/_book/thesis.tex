% This is the Reed College LaTeX thesis template. Most of the work
% for the document class was done by Sam Noble (SN), as well as this
% template. Later comments etc. by Ben Salzberg (BTS). Additional
% restructuring and APA support by Jess Youngberg (JY).
% Your comments and suggestions are more than welcome; please email
% them to cus@reed.edu
%
% See http://web.reed.edu/cis/help/latex.html for help. There are a
% great bunch of help pages there, with notes on
% getting started, bibtex, etc. Go there and read it if you're not
% already familiar with LaTeX.
%
% Any line that starts with a percent symbol is a comment.
% They won't show up in the document, and are useful for notes
% to yourself and explaining commands.
% Commenting also removes a line from the document;
% very handy for troubleshooting problems. -BTS

% As far as I know, this follows the requirements laid out in
% the 2002-2003 Senior Handbook. Ask a librarian to check the
% document before binding. -SN

%%
%% Preamble
%%
% \documentclass{<something>} must begin each LaTeX document
\documentclass[12pt,twoside]{reedthesis}
% Packages are extensions to the basic LaTeX functions. Whatever you
% want to typeset, there is probably a package out there for it.
% Chemistry (chemtex), screenplays, you name it.
% Check out CTAN to see: http://www.ctan.org/
%%
\usepackage{graphicx,latexsym}
\usepackage[french]{babel} 
\usepackage{amsmath}
\usepackage{amssymb,amsthm}
\usepackage[dvipsnames]{xcolor} % tk: for more color
\usepackage{xcolor}
\usepackage{eso-pic}
\usepackage{longtable,booktabs,setspace}
\usepackage{chemarr} %% Useful for one reaction arrow, useless if you're not a chem major
\usepackage[hyphens]{url}
\usepackage{tikz}
\usetikzlibrary{calc}
\newcommand\HRule{\rule{\textwidth}{1pt}}
% Added by CII
\usepackage{hyperref}
\usepackage{lmodern}
\usepackage{float}
\floatplacement{figure}{H}
% End of CII addition
\usepackage{rotating}
\usepackage{upgreek} % tk : pour pouvoir utiliser le symbole µ droit (pas en itallic)
\usepackage{lscape}
\newcommand{\blandscape}{\begin{landscape}}
\newcommand{\elandscape}{\end{landscape}}

% Next line commented out by CII
%%% \usepackage{natbib}
% Comment out the natbib line above and uncomment the following two lines to use the new
% biblatex-chicago style, for Chicago A. Also make some changes at the end where the
% bibliography is included.
%\usepackage{biblatex-chicago}
%\bibliography{thesis}


% Added by CII (Thanks, Hadley!)
% Use ref for internal links
\renewcommand{\hyperref}[2][???]{\autoref{#1}}
\def\chapterautorefname{Chapter}
\def\sectionautorefname{Section}
\def\subsectionautorefname{Subsection}
% End of CII addition

% Added by CII
\usepackage{caption}
\captionsetup{width=5in}
% End of CII addition

% \usepackage{times} % other fonts are available like times, bookman, charter, palatino


% To pass between YAML and LaTeX the dollar signs are added by CII
\title{THÈSE}
\author{Thomas Karaouzene}
\labo{}
% The month and year that you submit your FINAL draft TO THE LIBRARY (May or December)
\date{31 octobre 2017}
\division{}
\advisor{Pierre Ray}
%If you have two advisors for some reason, you can use the following
% Uncommented out by CII
\altadvisor{Nicolas Thierry-Mieg}
% End of CII addition

%%% Remember to use the correct department!
\department{Ingénierie de la Santé, de la Cognition et Environnement (EDISCE)}
% if you're writing a thesis in an interdisciplinary major,
% uncomment the line below and change the text as appropriate.
% check the Senior Handbook if unsure.
%\thedivisionof{The Established Interdisciplinary Committee for}
% if you want the approval page to say "Approved for the Committee",
% uncomment the next line
%\approvedforthe{Committee}

% Added by CII
%%% Copied from knitr
%% maxwidth is the original width if it's less than linewidth
%% otherwise use linewidth (to make sure the graphics do not exceed the margin)
\makeatletter
\def\maxwidth{ %
  \ifdim\Gin@nat@width>\linewidth
    \linewidth
  \else
    \Gin@nat@width
  \fi
}
\makeatother

\renewcommand{\contentsname}{Table of Contents}
% End of CII addition

\setlength{\parskip}{0pt}

% Added by CII
  %\setlength{\parskip}{\baselineskip}
  \usepackage[parfill]{parskip}

\providecommand{\tightlist}{%
  \setlength{\itemsep}{0pt}\setlength{\parskip}{0pt}}

\Acknowledgements{

}

\Dedication{

}

\Preface{
This is an example of a thesis setup to use the reed thesis document
class (for LaTeX) and the R bookdown package, in general.
}

\Abstract{

}

	\usepackage{tikz}
% End of CII addition
%%
%% End Preamble
%%
%

\usepackage{amsthm}
\newtheorem{theorem}{Theorem}[section]
\newtheorem{lemma}{Lemma}[section]
\theoremstyle{definition}
\newtheorem{definition}{Definition}[section]
\newtheorem{corollary}{Corollary}[section]
\newtheorem{proposition}{Proposition}[section]
\theoremstyle{definition}
\newtheorem{example}{Example}[section]
\theoremstyle{remark}
\newtheorem*{remark}{Remark}
\begin{document}

% Everything below added by CII
      \maketitle
  
  \frontmatter % this stuff will be roman-numbered
  \pagestyle{empty} % this removes page numbers from the frontmatter

  
      \begin{preface}
      This is an example of a thesis setup to use the reed thesis document
      class (for LaTeX) and the R bookdown package, in general.
    \end{preface}
  
      \hypersetup{linkcolor=black}
    \setcounter{tocdepth}{3}
    \tableofcontents
  
      \listoftables
  
      \listoffigures
  
  
  
  \mainmatter % here the regular arabic numbering starts
  \pagestyle{fancyplain} % turns page numbering back on

  \chapter{Delete line 6 if you only have one
  advisor}\label{delete-line-6-if-you-only-have-one-advisor}
  
  \chapter*{Remerciements}\label{remerciements}
  \addcontentsline{toc}{chapter}{Remerciements}
  
  \chapter*{Résumé}\label{resume}
  \addcontentsline{toc}{chapter}{Résumé}
  
  \chapter{Introduction}\label{introInf}
  
  \chapter{Investigation génétique et physiologique de la
  globozoospermie}\label{globo}
  
  \chapter{Mise en place d'une stratégie pour l'analyse des données
  exomiques -- application en recherche
  clinique}\label{mise-en-place-dune-strategie-pour-lanalyse-des-donnees-exomiques-application-en-recherche-clinique}
  
  \section{Intro}\label{intro}
  
  \newpage
  
  \section{Résultats}\label{resultats}
  
  Dans cette partie, nous allons détailler les résultats de l'analyse des
  données de WES de 75 patients tous atteins d'un phénotype d'infertilité.
  Ces études seront séparées en deux parties distinctes, la première se
  concentrera sur l'études de \ldots{} familles incluant 13 de ces
  patients. Le seconde portera sur l'analyse des 62 patients restant étant
  tous non-apparentés et présentant un phénotype MMAF.
  
  \subsection{Description de la
  pipeline}\label{description-de-la-pipeline}
  
  Après avoir été séquencés les données receuillies pour ces patients sont
  procéssés au sein de la même pipeline d'analyse qui comprend quatre
  étapes allant de l'alignement de \emph{reads} au filtrage des variants :
  
  \begin{enumerate}
  \def\labelenumi{\arabic{enumi}.}
  \tightlist
  \item
    \textbf{L'alignement} : L'alignement des \emph{reads} le long du
    génome de référence (hg19 / GHRC37) est effectué par le logiciel MAGIC
    (Su et al., \protect\hyperlink{ref-Su2014}{2014}). Afin d'écarté toute
    ambiguité au moment de l'interprétation de l'alignement, l'intégralité
    des \emph{reads} dupliqués et / ou s'alignant à plusieurs zones du
    génome seront filtrés et ne seront donc pas pris en compte pour
    l'ensemble des analyses en aval. Suite à celà,, MAGIC va produire
    quatre comptages pour chaque position couverte du génome : R+, V+, R-
    et V- :
  
    \begin{enumerate}
    \def\labelenumii{\alph{enumii}.}
    \tightlist
    \item
      \textbf{R+ et R-} : Ces deux comptages correspondent au nombre de
      \emph{reads} \emph{forward} (+) et \emph{reverse} (-) sur lesquels
      est observé l'allèle de \textbf{référence} (R) à une position
      donnée.\\
    \item
      \textbf{V+ et V-} : À l'inverse de R+ et R-, ces comptages
      correspondent au nombre de \emph{reads} \emph{forward} et
      \emph{reverse} sur lesquels est observé un allèle de
      \textbf{variant} (V) à une position donnée.\\
    \end{enumerate}
  \item
    \textbf{L'appel des variants} : Comme nous l'avons vu plus
    \protect\hyperlink{varcall}{tôt}, il est fortement conseillé
    d'effectuer l'appel des variants en tenant compte de l'aligneur choisi
    (Nielsen, Paul, Albrechtsen, \& Song,
    \protect\hyperlink{ref-Nielsen2011}{2011}, M. A. DePristo et al.
    (\protect\hyperlink{ref-DePristo2011}{2011}), Lunter \& Goodson
    (\protect\hyperlink{ref-Lunter2011}{2011})). C'est pourquoi, nous
    avons conçu notre propre algorithme d'appel des variants spécialement
    conçu pour l'analyse des données de MAGIC. Ainsi, l'appel des variants
    sera directement basé sur les quatre comptages vus précédemment. Tout
    d'abord, les positions ayant une couverture \textless{} 10 sur l'un
    des deux \emph{strands} sera considérée comme de faible qualité,
    celles ayant une couverture \textless{} 10 sur les deux \emph{strands}
    seront exclus. Ensuite pour chaque variant, des appels indépendants
    seront effectués pour chaque \emph{strand}. L'appel final sera une
    synthèse de ces deux appels où seul les cas où ces deux appels sont
    concordants seront considérés comme de bonne qualité.\\
  \item
    \textbf{L'annotation} : Chaque variant retenu sera ensuite annoté tout
    d'abord par le logiciel \emph{variant effect predictor} (VEP) (W.
    McLaren et al., \protect\hyperlink{ref-McLaren2016}{2016}) qui nous
    indiquera pour chaque variant la conséquence que celui-ci aura sur la
    séquence codante de l'ensemble des transcrits Ensembl qu'il chevauche
    (\textbf{Figure : }\ref{fig:figvepcsq}) (\textbf{Table :
    }\ref{tab:tabvepcsq}). Ensite, nous ajoutons récuperons pour chaque
    gène son expression tissulaire en nous basant sur le projet Illumina
    BodyMap {[}TODO ref{]} qui recanse les données RNAseq des gènes
    humains pour 16 tissus différents. Suite à cela nous ajoutons, lorsque
    celle-ci est disponible, la fréquence du variant dans les bases de
    données ExAC (Lek et al., \protect\hyperlink{ref-Lek2016}{2016}),
    ESP600 {[}TODO{]} et 1000Genomes {[}TODO{]} donnant ainsi une
    estimation de sa fréquence dans la population générale. De même, la
    particularité de ce pipeline est qu'elle conserve l'ensemble des
    variants identifiés dans les études effectuées précédemment permettant
    d'ajouter aux annotations la fréquence d'un variant chez les individus
    déjà séquencé et donc la fréquence d'un variant dans chaque phénotype
    étudié créant ainsi une base de données interne qui pourra servir de
    contrôle dans les études ultérieur.
  \end{enumerate}
  
  \begin{figure}
  
  {\centering \includegraphics[scale=.9]{figure/vep_csq} 
  
  }
  
  \caption[Listes des différentes conséquences prédites par VEP et leur positionnement sur le transcrit]{Listes des différentes conséquences prédites par VEP et leur positionnement sur le transcrit d'après [VEP site](http://www.ensembl.org/info/genome/variation/consequences.jpg)}\label{fig:figvepcsq}
  \end{figure}
  
  \begin{enumerate}
  \def\labelenumi{\arabic{enumi}.}
  \setcounter{enumi}{3}
  \tightlist
  \item
    \textbf{Le filtrage des variants} : L'étape de filtrage est
    extrêmement importante si l'on souhaite analyser de manière efficace
    les données provenant de WES. C'est pourquoi elle occupe une place
    importante dans notre pipeline. L'intégralité des paramètres de cette
    étape peuvent être modifier par l'utilisateur de sorte à faire
    correspondre les critères de filtre aux besoins de l'étude. Afin de
    rendre son utilisation le plus efficace possible, nous avons souhaité
    définir des paramètres par défauts pertinent dans la plupart des
    études de séquençage exomique de sorte que à moins que le contraire ne
    soit spécifié, seul les variants impactant les transcrits codant pour
    une protéine sont conservés. De même les variants synonymes ou
    affectant les séquences UTRs sont filtrés ainsi que les variants ayant
    une fréquence \(\ge\) 1\% dans les bases dans l'une des bases données
    (ExAC, ESP6500 ou 1KH). Aussi, pour un phénotype donné, l'ensemble des
    variants observés chez les individus étudiés présentant un phénotype
    différent sont de même enlevés de la liste finale.
  \end{enumerate}
  
  \newpage
  
  \blandscape
  
  \begin{longtable}[]{@{}lll@{}}
  \caption{\label{tab:tabvepcsq} Liste simplifiée des conséquences prédites
  par VEP avec leur description et impact associée}\tabularnewline
  \toprule
  \begin{minipage}[b]{0.18\columnwidth}\raggedright\strut
  VEP consequence\strut
  \end{minipage} & \begin{minipage}[b]{0.11\columnwidth}\raggedright\strut
  VEP impact\strut
  \end{minipage} & \begin{minipage}[b]{0.63\columnwidth}\raggedright\strut
  Description\strut
  \end{minipage}\tabularnewline
  \midrule
  \endfirsthead
  \toprule
  \begin{minipage}[b]{0.18\columnwidth}\raggedright\strut
  VEP consequence\strut
  \end{minipage} & \begin{minipage}[b]{0.11\columnwidth}\raggedright\strut
  VEP impact\strut
  \end{minipage} & \begin{minipage}[b]{0.63\columnwidth}\raggedright\strut
  Description\strut
  \end{minipage}\tabularnewline
  \midrule
  \endhead
  \begin{minipage}[t]{0.18\columnwidth}\raggedright\strut
  Splice acceptor / donor\strut
  \end{minipage} & \begin{minipage}[t]{0.11\columnwidth}\raggedright\strut
  HIGH\strut
  \end{minipage} & \begin{minipage}[t]{0.63\columnwidth}\raggedright\strut
  A splice variant that changes the 2 base region at the 3' / 5' end of an
  intron\strut
  \end{minipage}\tabularnewline
  \begin{minipage}[t]{0.18\columnwidth}\raggedright\strut
  Stop gained\strut
  \end{minipage} & \begin{minipage}[t]{0.11\columnwidth}\raggedright\strut
  HIGH\strut
  \end{minipage} & \begin{minipage}[t]{0.63\columnwidth}\raggedright\strut
  A sequence variant whereby at least one base of a codon is changed,
  resulting in a premature stop codon, leading to a shortened
  transcript\strut
  \end{minipage}\tabularnewline
  \begin{minipage}[t]{0.18\columnwidth}\raggedright\strut
  Frameshift\strut
  \end{minipage} & \begin{minipage}[t]{0.11\columnwidth}\raggedright\strut
  HIGH\strut
  \end{minipage} & \begin{minipage}[t]{0.63\columnwidth}\raggedright\strut
  A sequence variant which causes a disruption of the translational
  reading frame, because the number of nucleotides inserted or deleted is
  not a multiple of three\strut
  \end{minipage}\tabularnewline
  \begin{minipage}[t]{0.18\columnwidth}\raggedright\strut
  Stop lost\strut
  \end{minipage} & \begin{minipage}[t]{0.11\columnwidth}\raggedright\strut
  HIGH\strut
  \end{minipage} & \begin{minipage}[t]{0.63\columnwidth}\raggedright\strut
  A sequence variant where at least one base of the terminator codon
  (stop) is changed, resulting in an elongated transcript\strut
  \end{minipage}\tabularnewline
  \begin{minipage}[t]{0.18\columnwidth}\raggedright\strut
  Start lost\strut
  \end{minipage} & \begin{minipage}[t]{0.11\columnwidth}\raggedright\strut
  HIGH\strut
  \end{minipage} & \begin{minipage}[t]{0.63\columnwidth}\raggedright\strut
  A codon variant that changes at least one base of the canonical start
  codo\strut
  \end{minipage}\tabularnewline
  \begin{minipage}[t]{0.18\columnwidth}\raggedright\strut
  Inframe insertion / deletion\strut
  \end{minipage} & \begin{minipage}[t]{0.11\columnwidth}\raggedright\strut
  MODERATE\strut
  \end{minipage} & \begin{minipage}[t]{0.63\columnwidth}\raggedright\strut
  An inframe non synonymous variant that inserts / deletes bases into in
  the coding sequenc\strut
  \end{minipage}\tabularnewline
  \begin{minipage}[t]{0.18\columnwidth}\raggedright\strut
  Missense\strut
  \end{minipage} & \begin{minipage}[t]{0.11\columnwidth}\raggedright\strut
  MODERATE\strut
  \end{minipage} & \begin{minipage}[t]{0.63\columnwidth}\raggedright\strut
  A sequence variant, that changes one or more bases, resulting in a
  different amino acid sequence but where the length is preserved\strut
  \end{minipage}\tabularnewline
  \begin{minipage}[t]{0.18\columnwidth}\raggedright\strut
  Splice region\strut
  \end{minipage} & \begin{minipage}[t]{0.11\columnwidth}\raggedright\strut
  LOW\strut
  \end{minipage} & \begin{minipage}[t]{0.63\columnwidth}\raggedright\strut
  A sequence variant in which a change has occurred within the region of
  the splice site, either within 1-3 bases of the exon or 3-8 bases of the
  intron\strut
  \end{minipage}\tabularnewline
  \begin{minipage}[t]{0.18\columnwidth}\raggedright\strut
  Stop retained\strut
  \end{minipage} & \begin{minipage}[t]{0.11\columnwidth}\raggedright\strut
  LOW\strut
  \end{minipage} & \begin{minipage}[t]{0.63\columnwidth}\raggedright\strut
  A sequence variant where at least one base in the terminator codon is
  changed, but the terminator remains\strut
  \end{minipage}\tabularnewline
  \begin{minipage}[t]{0.18\columnwidth}\raggedright\strut
  Synonymous\strut
  \end{minipage} & \begin{minipage}[t]{0.11\columnwidth}\raggedright\strut
  LOW\strut
  \end{minipage} & \begin{minipage}[t]{0.63\columnwidth}\raggedright\strut
  A sequence variant where there is no resulting change to the encoded
  amino acid\strut
  \end{minipage}\tabularnewline
  \begin{minipage}[t]{0.18\columnwidth}\raggedright\strut
  5 / 3 prime UTR\strut
  \end{minipage} & \begin{minipage}[t]{0.11\columnwidth}\raggedright\strut
  MODIFIER\strut
  \end{minipage} & \begin{minipage}[t]{0.63\columnwidth}\raggedright\strut
  A UTR variant of the 5' / 3' UTR\strut
  \end{minipage}\tabularnewline
  \begin{minipage}[t]{0.18\columnwidth}\raggedright\strut
  Intron\strut
  \end{minipage} & \begin{minipage}[t]{0.11\columnwidth}\raggedright\strut
  MODIFIER\strut
  \end{minipage} & \begin{minipage}[t]{0.63\columnwidth}\raggedright\strut
  A transcript variant occurring within an intron\strut
  \end{minipage}\tabularnewline
  \begin{minipage}[t]{0.18\columnwidth}\raggedright\strut
  NMD transcript\strut
  \end{minipage} & \begin{minipage}[t]{0.11\columnwidth}\raggedright\strut
  MODIFIER\strut
  \end{minipage} & \begin{minipage}[t]{0.63\columnwidth}\raggedright\strut
  A variant in a transcript that is the target of NMD\strut
  \end{minipage}\tabularnewline
  \begin{minipage}[t]{0.18\columnwidth}\raggedright\strut
  Non coding transcript\strut
  \end{minipage} & \begin{minipage}[t]{0.11\columnwidth}\raggedright\strut
  MODIFIER\strut
  \end{minipage} & \begin{minipage}[t]{0.63\columnwidth}\raggedright\strut
  A transcript variant of a non coding RNA gene\strut
  \end{minipage}\tabularnewline
  \bottomrule
  \end{longtable}
  
  \elandscape
  
  \newpage
  
  \subsection{Utilisation du pipeline dans des cas familiaux
  :}\label{utilisation-du-pipeline-dans-des-cas-familiaux}
  
  \subsubsection{Description des familles}\label{description-des-familles}
  
  Dans cette partie, je me concentre sur l'analyse bioinformatique des
  résultats des séquençages exomiques effectués entre 2012 et 2014 de 13
  individus infertiles provenant de 6 familles différentes. Parmi
  celles-ci, 3 phénotypes différents ont été observés :
  
  \begin{enumerate}
  \def\labelenumi{\arabic{enumi}.}
  \tightlist
  \item
    \textbf{\protect\hyperlink{infquant}{L'Azoospermie} :} Comme nous
    avons pu le voir, l'azoospermie est un phénotype d'infertilité
    masculine caractérisé par l'absence de spermatozoïde dans l'éjaculât\\
  \item
    \textbf{Échec de fécondation :} Ce phénotype d'infertilité se
    caractérise par l'incapacité des spermatozoïdes à féconder
    l'ovocyte.\\
  \item
    \textbf{MMAF} : Le syndrome MMAF (\emph{multiple morphological
    abnormalities of the sperm flagella}) caractérise comme son nom
    l'indique les patients présentant une majorité de spermatozoïdes
    atteins par une mosaïque d'anomalie morphologique du flagelle.
  \end{enumerate}
  
  Parmi ces 6 chacune composée de 2 à 3 frères, 3 d'entre elles présentent
  un historique de consanguinité, les parents étant soit cousins germains,
  pour les familles \ldots{} et \ldots{}, soit cousins au second degré,
  pour la famille \ldots{} . La consanguinité favorisant la transmission
  de variants à l'état homozygote, nous avons décidé, dans un premiers
  temps de concentrer nos analyses uniquement sur les variants (SNVs et
  indels) homozygotes pour l'ensemble des familles. Pour les 3 familles
  n'ayant pas d'historique de consanguinité, ce choix nous permet de
  réduire la liste des variants candidats de sorte à faciliter les
  analyses. L'études des variants hétérozygotes sera effectuée \emph{a
  posteriori} pour les familles dont la cause génétique du phénotype n'a
  pas pu être identifiée en se limitant aux variants homozygotes. Un
  récapitulatif des familles et de leur phénotype est disponible dans la
  table \ref{tab:tabrecapfam}.
  
  \newpage
  
  \blandscape
  
  \begin{longtable}[t]{llllrl}
  \caption{\label{tab:tabrecapfam}Tableau récapitulatif des familles séquencées et de leur phénotype}\\
  \toprule
  Family & Consanguinity & Individuals & Phenotype & Year & Place\\
  \midrule
  AZ & Yes & AZ1, AZ2 & Azoospermia & 2012 & Mount Sinai Institut\\
  FF & Yes & FF1, FF2 & Fertilization failure & 2014 & Genoscope (Evry)\\
  MMAF1 & No & MMAF1.1, MMAF1.2 & MMAF & 2014 & Genoscope (Evry)\\
  MMAF2 & Yes & MMAF2.1, MMAF2.2 & MMAF & 2014 & Genoscope (Evry)\\
  MMAF3 & No & MMAF3.1, MMAF3.2 & MMAF & 2014 & Genoscope (Evry)\\
  MMAF4 & No & MMAF4.1, MMAF4.2, MMAF4.3 & MMAF & 2014 & Genoscope (Evry)\\
  \bottomrule
  \end{longtable}
  
  \elandscape
  \newpage  
  
  \subsubsection{Résultats des exomes}\label{resultats-des-exomes}
  
  \paragraph{Résultat de l'alignement}\label{resultat-de-lalignement}
  
  Pour rappel, l'\href{\%7B\#lalignement\%7D}{alignement} consiste à
  repositionner l'ensemble des \emph{reads} générés au cours de l'étape de
  séquençage le long d'un génome de référence.
  
  La quantité de \emph{reads} composant les exomes de chaque individu peut
  varier en fonction de plusieurs paramètres et n'est donc pas égale pour
  chaque patient bien que l'ordre de grandeur reste le même avec une
  médiane de 91438630 \emph{reads}. Seuls les deux frères AZ1 et AZ2 se
  distinguent avec près de 3 fois plus de \emph{reads} que les autres
  patients. Cette différence peut être expliqué car ces deux patients sont
  les deux seuls à voir été séquencé au Mount Sinaï Institut or leur
  protocole d'amplification précédent le séquençage contient un nombre de
  cycles de PCR supérieur à ceux appliqué au Génopole d'Évry où ont été
  séquencé les autres patients. Il faut noter que ce nombre plus important
  de \emph{reads} n'est en rien le reflet d'une meilleure qualité. En
  effet, ce nombre important de \emph{reads} est causé par une grande
  quantité de \emph{reads} dupliqués qui seront pour la plupart filtrés au
  cours des analyses ulterieure (\textbf{Table :} \ref{tab:tabrecapfam},
  \textbf{Figure : }\ref{fig:readsselection} - \textbf{A}).
  
  L'ensemble de nos exomes ayant été réalisés en \emph{paired-end}, les
  deux extrémités de chaque fragment sont séquencées. Chaque \emph{end}
  d'un même \emph{read} peut donc être considéré comme un \emph{read} à
  part entière qui sont alignées \textbf{indépendamment} le long du génome
  de référence. L'information fournit par le \emph{paired-end} n'étant
  utilisé qu'à \emph{posteriori} en tant que critère qualité. La première
  étape du contrôle qualité des \emph{reads} consiste filtrer les
  \emph{reads} ne s'étant pas aligné sur le génome. Ces \emph{reads} sont
  extrêmement minoritaires puisqu'ils ne représentent qu'entre 1.2 et 5.5
  \% des \emph{reads} de nos individus (\textbf{Figure :
  }\ref{fig:readsselection} - \textbf{B}).
  
  Une fois cela fait, nous vérifions la ``compatibilité'' des deux
  \emph{ends} composant chacun des \emph{reads} s'étant correctement
  alignés. Un \emph{reads} est dit compatible lorsque les deux \emph{ends}
  qui le composent s'alignent face à face (une sur le \emph{strand} + et
  l'autre sur le \emph{strand} -) et couvrent une zone ne faisant pas plus
  de 3 fois la taille médiane de l'insert. Les \emph{reads} dont les deux
  \emph{ends} se sont alignées mais ne remplissant pas ces conditions
  seront dit ``Non compatible'', ceux dont une seule des deux \emph{ends}
  s'est alignés seront appelés ``orphelins''. Dans nos analyses, seuls les
  \emph{reads} compatibles sont conservés, c'est à dire environs 89.5 \%
  des \emph{reads} s'étant correctement alignés. (\textbf{Figure :
  }\ref{fig:readsselection} - \textbf{C}).
  
  La dernière étape de ce contrôle-qualité consiste à analyser le nombre
  de site auxquels se sont alignés les \emph{reads}. En effet, certaine
  zone du génome étant dupliqué, l'une des problématiques des
  \emph{short-reads} est qu'il est possible que ceux-ci s'alignent à
  plusieurs régions différentes du génome. Afin d'éviter toute ambiguïté,
  seul ceux s'étant aligné sur un site unique sont conservés pour la suite
  des analyses. Ces \emph{reads} représente entre 92.3 et 96.9 \% des
  \emph{reads} ayant passé les précédents filtres (\textbf{Figure :
  }\ref{fig:readsselection} - \textbf{C}).
  
  Les \emph{reads} ayant passé l'ensemble des critères qualité mentionnés
  précédemment seront ensuite utilisés pour effectuer l'appel des
  variants.
  
  \newpage
  
  \begin{figure}
  
  {\centering \includegraphics{thesis_files/figure-latex/readsselection-1} 
  
  }
  
  \caption[Processus simplifié du contrôle qualité des *reads*]{Processus simplifié du contrôle qualité des *reads* : Pour chacun des graphiques, les *reads* représentés en vert sont conservés tandis que ceux en rouge sont filtrés. **A** : Quantité de *reads* bruts générés pour chaque patient au cours de l'étape de séquençage. La médiane des *reads* est représentée en bleue. **B** : Pourcentage pour chaque individu de *reads* s'étant aligné correctement et ne s'étant pas alignés sur le génome de référence. **C** : Distribution pour chaque patient des *reads* compatibles (Comp), non compatibles (Non comp) et orphelins (Orphans). **D** : Présentation pour chaque *reads* du nombre de site auxquels ils s'alignent}\label{fig:readsselection}
  \end{figure}
  
  \newpage
  
  \paragraph{Résultat de l'appel des
  variants}\label{resultat-de-lappel-des-variants}
  
  Comme dit précédemment, l'appel des variants fait suite à l'alignement
  et consiste à comparer la séquence d'un individu avec celle d'un génome
  de référence afin d'en relever les différences. La particularité de
  notre algorithme d'appel est d'effectuer pour chaque position deux
  appels indépendants Le premier sera effectué en utilisant uniquement les
  \emph{reads forward} et le second le \emph{reads reverse}. Encore une
  fois, plusieurs filtres sont appliqués de sorte à conserver uniquement
  les variants les plus qualitatifs.
  
  Tout d'abord, nos appels sont classés en trois catégories :
  
  \begin{enumerate}
  \def\labelenumi{\arabic{enumi}.}
  \tightlist
  \item
    \textbf{Les appels \emph{double strand} (DS) :} Qualifie les positions
    ayant une couverture \(\ge\) 10 sur \textbf{les deux} strands. Ces
    appels sont ceux sont ceux ayant la meilleure qualité
  \item
    \textbf{Les appels \emph{single strand} (SS) :} Ces appels définissent
    les positions pour lesquels \textbf{un des deux} \emph{strands}
    présentent une couverture \(\le\) 10. Dans ce cas, ce \emph{strand}
    est ignoré et l'appel est effectué uniquement en utilisant le second
    \emph{strand}.\\
  \item
    \textbf{Les appels \emph{non strand} (NS) :} Les positions NS sont
    celles pour lesquelles la couverture est \(\le\) 10 sur \textbf{les
    deux} strands. Aucun appel n'est effectué à ces positions.
  \end{enumerate}
  
  Dans nos données, les appels SS sont majoritaires et représentent
  environ 48.1 \% de nos appels (contre 35.6 \% d'appels DS). Au vus de
  l'importance de ces appels, nous avons fait le choix de les conserver
  afin de ne pas filtrer une quantité trop importante de données. Ces
  appels seront cependant considérés comme étant de faible qualité, de
  fait, leurs analyses et interprétation seront plus précautionneuses En
  revanche, au vus de la trop grande incertitude de l'appel des variants
  NS, ceux-ci sont systématiquement filtrés éliminant ainsi entre 10.3 et
  18.7 \% des positions appelées (\textbf{Figure : }\ref{fig:plotvarcall}
  - \textbf{A}).
  
  Un second filtre est appliqué aux variants ayant été précédemment
  appelés DS. Celui-ci consiste à comparer les appels effectués
  indépendamment sur chacune des deux \emph{ends} et à vérifier leur
  concordance, c'est à dire que les deux appels soit identique. Les appels
  discordant et ambigus sont filtrés, soit environ 86.3 \% des variants
  DS. Il est intéressant de noter que bien que les variants \emph{single
  strand} (SS) soient conservés, on peut s'attendre à ce qu'environ 13.7
  \% de ceux-ci soient aberrants, ceux-ci n'ayant pu subir le même
  contrôle que les SS (\textbf{Figure : }\ref{fig:plotvarcall} -
  \textbf{B}).
  
  Pour l'ensemble des variants ayant passé les filtres énoncés ci-dessus,
  c'est à dire les variants SS et les variants DS avec appels concordants,
  le génotype est déterminé en fonction du pourcentage de \emph{reads}
  portant le variant à cette position. Par exemple, si à une position
  donnée, 0\% des \emph{reads} portent un variant, l'individu sera appelé
  ``Homozygote référence'', si 50\% des \emph{reads} sont porteurs d'un
  variant, l'appel sera ``hétérozygote'' et si 100\% des \emph{reads}
  portent un variant, l'appel sera ``Homozygote variant''. Ainsi, pour
  chaque individu nous avons pu établir une liste de SNVs et d'indels avec
  leur génotype associé. Pour chacun de nos 13 patients les ordres de
  grandeur du nombre de variants appelés sont identique. Ainsi pour chaque
  patient nous avons appelés environ 43670 variants hétérozygotes (41044
  SNVs et 2626 indels) et 65040 variants homozygotes (32520 SNVs et 1809
  indels) (\textbf{Figure : }\ref{fig:plotvarcall} - \textbf{C}).
  
  \newpage
  
  \begin{figure}
  
  {\centering \includegraphics{thesis_files/figure-latex/plotvarcall-1} 
  
  }
  
  \caption[Contrôle qualité des variants appelés]{Contrôle qualité des variants appelés : Pour chacun des graphiques, les variants représentés en vert et en orange sont conservés tandis que ceux en rouge sont filtrés. **A** : Distribution du *stranding* des appels pour chaque patient. **B** : Comparaison des appels entre les deux *ends* des variants appelés DS. **C** : Distribution des SNVs et indels en fonction de leur génotype pour chaque patients (représentés par une barre}\label{fig:plotvarcall}
  \end{figure}
  
  \newpage
  
  \paragraph{Résultats de l'annotation}\label{resultats-de-lannotation}
  
  L'annotation des variants appelés consiste à ajouter un maximum
  d'informations sur les variants. Ces informations seront ensuite
  utilisées afin de filtrer et / ou prioriser les variants. Dans ces
  analyses nous avons utilisé le logiciel \emph{Variant Effect Predictor}
  (VEP) (W. McLaren et al., \protect\hyperlink{ref-McLaren2016}{2016}) qui
  va à la fois prédire l'effet qu'auront ces variants sur l'ensemble des
  transcrits (et gènes) qu'ils chevauchent, ajouter, lorsqu'elle est
  disponible, la fréquence de chacun de ces variants dans les bases de
  données ExAC, 1000Genomes (1KG) et ESP6500. Pour finir VEP nous
  permettra de connaitre les prédictions de pathogénicités fournies par
  SIFT et PolyPhen pour les variants faux-sens.
  
  Après avoir annoté nos variants par VEP, nous avons pu constater que
  pour chaque patient 24975 gènes sont en moyenne affecté par au moins un
  variant pour en moyenne 122735 transcrits (soit environ 5 transcrits par
  gènes). Il faut noter que parmi ces gènes se trouvent à la fois des
  gènes codant pour des protéine \textbf{et} d'autres non codant
  (\textbf{Figure : }\ref{fig:plotvarannotation} - \textbf{A}).
  
  Chaque variant affectera l'ensemble des transcrits qu'il chevauche,
  ainsi un même variant pourra impacter plusieurs transcrits. Ces impacts
  sont ensuite classés par VEP en quatre catégories qui sont, de la plus
  délétère à la moins délétère : HIGH, MODERATE, LOW, MODIFIER
  (\textbf{Table :}\ref{tab:tabvepcsq}). Comme attendu, les variants ayant
  un impact tronquant se retrouvent être les moins fréquent chez chacun de
  nos patients. Ceci est d'autant plus flagrant pour l'impact HIGH qui
  regroupe, entre autres, les variants créant un codon stop ou encore ceux
  causant un décalage du cadre de lecture (\textbf{Table
  :}\ref{tab:tabvepcsq}), se retrouvent en quantité extrêmement faible
  puisqu'ils ne représentent en moyenne que 0.15 \% des variants, soit une
  moyenne de 466 hétérozygotes et 370 homozygotes par patient)
  (\textbf{Figure : }\ref{fig:plotvarannotation} - \textbf{B}).
  
  Parmi ces variants, certains étaient déjà recensés dans une des trois
  base donnée (ExAC, ESP et 1KG). Ainsi, on peut observer qu'entre 38.6 et
  55.5 \% de nos variant étaient listés dans ExAC et entre 33.1 et 43.8 \%
  dans ESP. En revanche environ 87.1 \% d'entre eux sont recensés dans 1KG
  (\textbf{Figure : }\ref{fig:plotvarannotation} - \textbf{C}) (À discuter
  !!!!!).
  
  (À discuter !!!!!) (\textbf{Figure : }\ref{fig:plotvarannotation} -
  \textbf{D})
  
  LES FIGURES SUR LA FRÉQUENCE SONT À DISCUTER CAR LEUR INTERPRÉTATION ME
  LAISSE PERPLEX (SURTOUT LA PROPORTION DE NOS VARIANTS PRÉSENTS DANS 1KG)
  
  \newpage
  
  \begin{figure}
  
  {\centering \includegraphics{thesis_files/figure-latex/plotvarannotation-1} 
  
  }
  
  \caption[Annotation des variants par VEP]{Annotation des variants par VEP : **A** : Quantification du nombre de gènes (en bleu) / transcrits (en rose) impactés par au moins un variant pour chaque patient chacun représentés par une barre. **B** : Distribution des impacts HIGH MODERATE LOW et MODIFIER en fonction des patients et du génotype du variant. **C** : Pourcentage des variants retrouvés au sein des trois bases de données : ExAC, ESP et 1KG. **D** : Distribution des fréquences de nos variants au sein des trois bases de données : ExAC, ESP et 1KG}\label{fig:plotvarannotation}
  \end{figure}
  
  \newpage
  
  \hypertarget{filterdescription}{\paragraph{Résultats du
  filtrage}\label{filterdescription}}
  
  Les étapes précédentes nous ont permis de mettre en évidence pour chaque
  patient une liste de variants passant l'ensemble de nos critères
  qualités. Ces variants ont dès lors pu être annotés nous permettant
  notamment d'avoir connaissance de leurs impacts sur les différents
  transcrits qu'ils chevauchent ou encore leur fréquence dans la
  population générale. Désormais, afin de ne conserver que les variants
  ayant la plus forte probabilité d'être responsable du phénotype de ces
  patients, nous avons appliqué successivement six filtres basés à la fois
  sur les différentes annotations que nous avons ajoutées mais aussi sur
  nos connaissances du mode de transmission du phénotype :
  
  \begin{enumerate}
  \def\labelenumi{\arabic{enumi}.}
  \tightlist
  \item
    \textbf{Filtre 1 : L'union des variants :} Dans ces différentes
    études, nous avons à chaque fois séquencé des frères (deux ou trois)
    présentant phénotype. Ainsi nous avons pu formuler l'hypothèse d'une
    cause génétique commune entre les différents patients d'une même
    famille et donc filtrer l'ensemble des variants qui ne sont pas
    partagés les deux ou trois frères atteints testés.\\
  \item
    \textbf{Filtre 2 : Génotype des variants :} Dans ces études, nous
    avons émis l'hypothèse d'une transmission récessive du phénotype.
    Ainsi, seuls les variants homozygotes ont été conservés.
    (\textbf{Figure : }\ref{fig:resvarcall}, \ref{fig:comparefilter}).\\
  \item
    \textbf{Filtre 3 : Impact du variant :} Afin de ne conserver que les
    variants ayant un effet potentiellement délétère sur la protéine, nous
    avons filtré les variants intronique et ceux tombant dans les
    séquences UTRs. De même les variants synonymes ne sont pas conservés
    (exceptés ceux se trouvant proches des régions d'épissage) car ceux-ci
    n'ont aucun effet sur la séquence protéique. Pour les variants faux
    sens (changement d'un seul aa de la séquence protéique) il est plus
    difficile de se trancher, nous avons donc utilisé les logiciels SIFT
    (Kumar, Henikoff, \& Ng, \protect\hyperlink{ref-Kumar2009}{2009}) et
    Polyphen (Adzhubei et al., \protect\hyperlink{ref-Adzhubei2010}{2010})
    et filtré l'ensemble des faux-sens prédits comme \emph{tolerated} par
    SIFT et \emph{benign} par Polyphen.\\
  \item
    \textbf{Filtre 4 : Les transcrits ``non pertinents'' :} Au cours de
    nos analyses nous nous sommes concentré uniquement sur les transcrits
    codant pour une protéine. Ainsi, l'ensemble des transcrits annotés
    comme étant non codant furent filtrés. De même le mécanisme NMD
    (\emph{nonsense-mediated decay}) a pour but de contrôler la qualité
    des ARNm cellulaires chez les eucaryotes (Y.-F. Chang, Imam, \&
    Wilkinson, \protect\hyperlink{ref-Chang2007}{2007}) en éliminant les
    ARNm qui comportent un codon stop prématuré (Baker \& Parker,
    \protect\hyperlink{ref-Baker2004}{2004}), pouvant être le résultat
    d'une erreur de transcription, d'une mutation ou encore d'une erreur
    d'épissage. Il est donc peu probable que les variants présents sur des
    transcrits annotés NMD soient responsables du phénotype. Dès lors, ces
    transcrits ont été également filtrés. Ainsi, nous avons pu retirer de
    nos listes de variants l'ensemble des mutations impactant
    \textbf{uniquement} des transcrits non codant et / ou annoté NMD.
    Cette étape de filtre permet à elle seule de filtrer systématiquement
    entre 36576 et 44581 transcrits différents par patients, soit une
    moyenne de 3107 variants par individus (\textbf{Figure :
    }\ref{fig:plotfilternonpertinanttr}).
  \end{enumerate}
  
  \begin{figure}
  
  {\centering \includegraphics{thesis_files/figure-latex/plotfilternonpertinanttr-1} 
  
  }
  
  \caption[Filtrage des transcrits jugés "non pertinents" et des variants les chevauchant]{Filtrage des transcrits jugés "non pertinents" et des variants les chevauchant : Pour chaque patients nous avons filtrer les transcrits jugés "non pertinents" pour l'analyse, c'est à dire ceux ne codant pas pour une protéine et ceux annoté NMD. Dès lors, l'intégralité des variants chevauchant uniquement des transcrits non pertinents ont put systématiquement être filtrés (boites rouges). les autres furent conservés (boites vertes)}\label{fig:plotfilternonpertinanttr}
  \end{figure}
  
  \begin{enumerate}
  \def\labelenumi{\arabic{enumi}.}
  \setcounter{enumi}{4}
  \tightlist
  \item
    \textbf{Fréquence des variants :} La fréquence d'un variant dans la
    population générale est un moyen rapide d'avoir une prédiction fiable
    de l'effet délétère ou non de celui-ci. En effet, il est peu probable
    qu'un variant retrouvé fréquemment dans la population générale soit
    causal d'une pathologie sévère. Ainsi nous avons filtré pour
    l'ensemble de nos patients l'ensemble des variants ayant une fréquence
    \(\ge\) 0.01 dans l'une des trois bases de données que sont ExAC, ESP
    et 1KG.\\
  \item
    \textbf{Présence des variants dans la cohorte contrôle :} Au cours de
    nos différentes études, nous avons été amenés à séquencé un total de
    134 un des 6 phénotypes étudiés au cours de nos différentes études
    (\textbf{Table : }\ref{tab:TODO}). Ces phénotypes étant très
    différent, on peut émettre l'hypothèse que leurs causes génétiques
    soient également différentes. De même, les variants recherchés étant
    rares, il est peu probable qu'un individu porte les variants de deux
    phénotypes différents. Ainsi, pour chacune des 6 familles, nous avons
    pu constituer une cohorte contrôle composée dans l'ensemble des
    patients précédemment analysés et ne présentant pas le même phénotype
    que celui étudié dans la famille (\textbf{Figure :}
    \ref{fig:plotsamplectrl}). Dès lors, nous avons pu filtrer l'ensemble
    des variants retrouvés à la fois chez nos patients et observés à
    l'état homozygote dans la cohorte contrôle. Cette cohorte contrôle
    présente ainsi le même rôle que les bases de données publiques. Sont
    intéret principale par rapport à celles-ci est que les individus qui
    la composent ont pour la plupart la même origine ethnico-géographique
    que nos patients. De plus ceux-ci ont été séquencés en même temps dans
    les mêmes centres permettant ainsi d'identifier les artefacts dûs aux
    protocols de séquençage.
  \end{enumerate}
  
  \begin{figure}
  
  {\centering \includegraphics{thesis_files/figure-latex/plotsamplectrl-1} 
  
  }
  
  \caption[Nombre d'individus composant la cohorte contrôle de chaque famille]{Nombre d'individus composant la cohorte contrôle de chaque famille : Ici, chaque barre représente une famille et sa hauteur est déterminée par le nombre d'individus composant la cohorte contrôle à laquelle elle a été confronté. Chaque individu de la cohorte contrôle a été séquencés en WES par notre équipe. Afin d'être considéré comme "contrôle" et intégrer cette cohorte, un individu doit être sain ou présenter un phénotype d'infertilité différent de la famille étudiée. Par exemple, un individus MMAF pourra servir de contrôle aux familles AZ et FF mais pas aux familles MMAF1-4}\label{fig:plotsamplectrl}
  \end{figure}
  
  \newpage
  
  \newpage
  
  Comme on pouvait s'y attendre, ces six filtres ont un pouvoir
  discriminant extrêmement différent (\textbf{Figure :}
  \ref{fig:plotcomparefilter}). En effet, tandis que le filtre
  ``Transcript relevance'' (filtre n°4) éliminer en moyenne 3.9 \% des
  variants de chaque individu tandis que le filtre ``Variant impact''
  (filtre n° 3) élimine jusqu'à 90.1 \% de ces mêmes variants
  (\textbf{Figure :} \ref{fig:plotcomparefilter} - \textbf{A}). Cette
  différence n'est pas surprenante. En effet, comme nous l'avions vu plus
  tôt, les variants de la catégorie VEP MODIFIER qui regroupe entre autres
  les variants chevauchant les séquences UTRs et introniques
  (\textbf{Table :} \ref{fig:tabvepcsq}) représentent en moyenne \ldots{}
  \% des variants de nos patients (\textbf{Figure :}
  \ref{fig:plotvarannotation} - \textbf{A}). Ceux-ci étant tous filtrés,
  on s'attendait donc à une valeur aussi élevée. On peut également
  constater l'importance de la cohorte contrôle qui, je le rappelle,
  permet de filtrer l'ensemble des variants homozygotes observés en son
  sein, puisque ce filtre permet retirer entre 76.5 et 88.4\% des variants
  de chaque individus (\textbf{Figure :} \ref{fig:plotvarannotation} -
  \textbf{A}).
  
  Cependant, regarder uniquement le pourcentage de variants filtrés par
  chaque filtre révèle une information partielle. En effet, dans ce cas de
  figure, on observe la quantité de variant éliminé par chaque filtre
  indépendamment les uns des autres. Ainsi, un même variant peut donc être
  filtré par plusieurs filtres. Dès lors, il faut également analyser la
  quantité de variants filtrés \textbf{spécifiquement} par chaque filtre.
  Ainsi, on peut constater que le classement des filtres en fonctions de
  leur stringance reste quasiment identique (\textbf{Figure :}
  \ref{fig:plotcomparefilter} - \textbf{B}) il est tout de même
  intéressant de noter que désormais le filtre ``Variant impact'' apparait
  moins efficace que les filtres ``Ctrl'' et ``Genotype'' en filtrant
  spécifiquement une moyenne de 253 variants par individu contre 423 pour
  le filtre génotype et 882 pour le filtre ``Ctrl''. Ainsi, ce dernier
  devient celui filtrant spécifiquement le plus de variants avec entre 364
  et 1060 variants spécifiquement filtrés par patients confirmant ainsi
  l'importance de ce filtre dans nos analyses. Aussi, les filtres
  ``Transcript relevance'', ``Union'' et ``Frequency'' apparaissent
  désormais comme étant anecdotiques en comparaison aux trois autres
  filtres puisqu'ils filtrent au maximum 43 variants spécifiques
  (\textbf{Figure :} \ref{fig:plotcomparefilter} - \textbf{B}).
  
  \newpage
  
  \begin{figure}
  
  {\centering \includegraphics{thesis_files/figure-latex/plotcomparefilter-1} 
  
  }
  
  \caption[Comparaison de l'efficacité de chacun des six filtres utilisés]{Comparaison de l'efficacité de chacun des six filtres utilisés : **A** : Comparaison du pourcentage de variants filtrés par chacun des six filtres indépendamment les uns des autres pour chaque patient (représenté par les points. Dès lors, un même variant peut être filtré par plusieurs filtres. **B** : Comparaison du nombre de variant filtrés spécifiquement par chacun des filtres. Ici, un variant ne peut-être filtré que par un seul filtre}\label{fig:plotcomparefilter}
  \end{figure}
  
  \newpage
  
  Après avoir appliqué l'ensemble de ces filtres, seuls quelques variants
  subsistent nous permettant d'obtenir une liste de gènes restreinte pour
  chaque famille (\textbf{Table : }\ref{tab:tablegene}) et ainsi de tirer
  des conclusions quant au variant responsable du phénotype.
  
  \begin{enumerate}
  \def\labelenumi{\arabic{enumi}.}
  \tightlist
  \item
    \textbf{Famille AZ} : Parmi les 2 gènes restant pour cette famille,
    \emph{SPINK2} est apparu comme étant un candidat évident. Notamment
    son expression étant spécifique au testicule tandis que celle de
    \emph{GUF1} est ubiquitaire (\textbf{Figure :
    }\ref{fig:plotexpspink2guf1}). De plus, des mutations du gène
    \emph{Spink2} chez la souris avait déjà été identifiée comme induisant
    des défauts de la spermatogenèse (Lee et al.,
    \protect\hyperlink{ref-Lee2011}{2011}).
  \end{enumerate}
  
  \begin{figure}
  
  {\centering \includegraphics{thesis_files/figure-latex/plotexpspink2guf1-1} 
  
  }
  
  \caption[Expression tissulaire des gènes *SPINK2* et *GUF1*]{Expression tissulaire des gènes *SPINK2* et *GUF1* : Données provenant du projet de transcriptome Illumina bodyMap}\label{fig:plotexpspink2guf1}
  \end{figure}
  
  \begin{enumerate}
  \def\labelenumi{\arabic{enumi}.}
  \setcounter{enumi}{1}
  \tightlist
  \item
    \textbf{Famille FF} : Pour cette famille, le gène
    \emph{PLC}\(\zeta 1\) a passé l'ensemble des filtres. Nos
    connaissances sur la fonction de se gène et notamment son rôle dans
    l'activation ovocytaire (TODO: REF) ainsi que sa forte expression
    testiculaire ont fait de ce gène le candidat idéal pour expliquer le
    phénotype de ces deux frères (\textbf{Figure :
    }\ref{fig:plotexpplcz1}).
  \end{enumerate}
  
  \begin{figure}
  
  {\centering \includegraphics{thesis_files/figure-latex/plotexpplcz1-1} 
  
  }
  
  \caption[Expression tissulaire du gène *PLCZ1*]{Expression tissulaire du gène *PLCZ1* : D'après les données du Illumina BodyMap}\label{fig:plotexpplcz1}
  \end{figure}
  
  \begin{enumerate}
  \def\labelenumi{\arabic{enumi}.}
  \setcounter{enumi}{2}
  \tightlist
  \item
    \textbf{Famille MMAF1} : L'analyse bibliographique des 2 gènes ayant
    passé l'ensemble des filtres n'a pas pu nous permettre de d'affirmer
    que l'un de ces gènes étaient responsable du phénotype MMAF de ces 2
    frères.\\
  \item
    \textbf{Famille MMAF2} : À l'issue des filtres, 2 gènes ressortaient
    chez ces deux frères : \emph{MYH11} et \emph{DNAH1}. Or, notre équipe
    ayant déjà établit le lien entre des mutations du gène \emph{DNAH1} et
    le syndrome MMAF (Ben Khelifa et al.,
    \protect\hyperlink{ref-BenKhelifa2014}{2014}) ce gène s'est révélé
    être un candidat idéal pour expliquer le phénotype de ces 2 frères. De
    plus, l'implication de \emph{MYH11} dans le phénotype de dissection
    aortique (Imai et al., \protect\hyperlink{ref-Imai2015}{2015}) l'ont
    écarté des candidats pour le phénotype MMAF.\\
  \item
    \textbf{Famille MMAF3} : Comme pour les gènes de la famille MMAF2,
    l'analyse bibliographique des 7 gènes ayant ici passé les filtres de
    même que l'étude de leurs expressions ne nous a pas permis de conclure
    que l'un d'entre eux étaient responsable du phénotype MMAF de ces 2
    frères.
  \end{enumerate}
  
  \begin{figure}
  
  {\centering \includegraphics{thesis_files/figure-latex/plotexpmmaf3-1} 
  
  }
  
  \caption[Expression tissulaire des gènes retenus pour la famille MMAF3]{Expression tissulaire des gènes retenus pour la famille MMAF3 : Données provenant du projet de transcriptome Illumina bodyMap}\label{fig:plotexpmmaf3}
  \end{figure}
  
  \begin{enumerate}
  \def\labelenumi{\arabic{enumi}.}
  \setcounter{enumi}{5}
  \tightlist
  \item
    \textbf{Famille MMAF4} : Seul le gène \emph{TGIF2} a passé l'ensemble
    des filtres pour la famille MMAF4. L'expression ubiquitaire de ce gène
    n'en font pas un candidat idéal. Cependant une étude de 2011 effectuée
    sur le wallaby décrit que la protéine TGIF2 est localisée
    spécifiquement dans le cytoplasme du spermatide, ainsi que dans le
    corps résiduel et la pièce intermédiaire du flagelle du spermatozoïde
    mature (Hu, Yu, Shaw, Renfree, \& Pask,
    \protect\hyperlink{ref-Hu2011}{2011}). Ces données pourraient corréler
    avec le phénotype MMAF de ces 3 frères bien que l'expression de ce
    gène soit ubiquitaire (\textbf{Figure : }\ref{fig:plotexptgif2}).
  \end{enumerate}
  
  \newpage 
  
  \begin{figure}
  
  {\centering \includegraphics{thesis_files/figure-latex/plotexptgif2-1} 
  
  }
  
  \caption[Expression tissulaire du gène *TGIF2*]{Expression tissulaire du gène *TGIF2* : D'après les données du Illumina BodyMap}\label{fig:plotexptgif2}
  \end{figure}
  
  \newpage
  
  \subsubsection{Discussion}\label{discussion}
  
  L'analyse de ces 6 familles nous a permis de mettre en évidence
  l'efficacité de notre pipeline d'analyse puisque pour 3 d'entre elles
  (soit 50\%) le variant causal a pu être identifié avec certitude
  (\textbf{Figure : }\ref{fig:plotremaininggenes}) et les résultats
  publiés dans trois revus dont je suis co-auteur :
  
  \begin{enumerate}
  \def\labelenumi{\arabic{enumi}.}
  \tightlist
  \item
    \textbf{Famille AZ} : \protect\hyperlink{spink2}{\textbf{SPINK2
    deficiency causes infertility by inducing sperm defects in
    heterozygotes and azoospermia in homozygotes}} : Dans cet article j'ai
    effectué non seulement l'intégralité des analyses bioinformatiques des
    données d'exomes de deux frères infertiles présentant un phénotype
    d'azoospermie mais j'ai aussi séquencé en Sanger les séquences
    codantes du gène \emph{SPINK2} pour une partie des 611 individus
    analysés ainsi que contribué à l'extraction de l'ARN testiculaire des
    souris pour l'analyse fonctionnelle du gène \emph{Spink2} sur le
    modèle murin.\\
  \item
    \textbf{Famille FF} : \protect\hyperlink{plcz}{\textbf{Homozygous
    mutation of PLCZ1 leads to defective human oocyte activation and
    infertility that is not rescued by the WW-binding protein PAWP}} :
    Dans cet article j'ai, effectué l'intégralité des analyses
    bioinformatiques des données d'exomes effectuées sur deux frères
    infertiles présentant des échecs de fécondation.\\
  \item
    \textbf{Famille MMAF2} :
    \protect\hyperlink{famdnah1}{\textbf{Whole-exome sequencing of
    familial cases of multiple morphological abnormalities of the sperm
    flagella (MMAF) reveals new DNAH1 mutations}} : Dans cet article j'ai,
    comme précédemment, effectué l'ensemble des analyses bioinformatiques
    des données d'exomes effectuées sur deux frères infertiles présentant
    des échecs de fécondation.
  \end{enumerate}
  
  Pour une d'entre elle, un candidat potentiel a pu être mis en évidence
  avec le gène \emph{TGIF2} et notre équipe travaille actuellement sur la
  caractérisation de ce gène afin de savoir s'il peut effectivement
  expliquer le phénotype MMAF de cette famille (\textbf{Figure :
  }\ref{fig:plotremaininggenes}).
  
  TODO : Il faut aller plus loin dans l'analyse et les arguments pour
  convaincre qu'il s'agit d'un bon candidat : quel type de mutation, ce
  gène est-il bien conservé, son expression n'est pas spécifique au
  testicule et ce gène serait impliqué dans un phénotype
  d'holoproencephaly\ldots{}
  
  Pour les 2 familles restantes, aucun variant n'a pu pour l'instant
  expliquer leur phénotype. L'explication la plus vraisemblable est que le
  variant ait été filtré par l'un de nos six filtres, probablement celui
  consistant à filtrer l'ensemble des variants hétérozygotes. En effet,
  l'hypothèse d'un variant causal homozygote était extrêmement crédible
  pour les familles AZ, FF et MMAF2 étant donné l'historique consanguin de
  ces 3 familles dont les parents sont à chaque fois apparentés. En
  revanche rien ne laisse supposer une telle chose pour les familles
  restantes. Cependant, le filtre des variants hétérozygotes pour
  l'ensemble des patients de ces 3 familles a été maintenu en première
  intention afin de faciliter les analyses en réduisant au maximum le
  nombre de variant. Au vus des résultats il apparait clair que les
  variants responsables de leur phénotype aient été filtrés pour au moins
  2 de ces familles. Dès lors, l'ensemble des analyses effectuées lors de
  l'étape de filtrage doivent être refaites en changeant les paramètres de
  filtrage. Cette fois-ci, les variants hétérozygotes seront conservés et
  les gènes sur lesquels au moins deux variants hétérozygotes seront
  recensés seront analysés en priorité. En effet, bien que les analyses
  exomiques nous fournissent en l'état pas d'informations suffisante pour
  savoir si ces deux variants sont présent sur le même allèle ou bien sur
  deux allèles différents, cela pourrait-être la signature de variants
  hétérozygotes composites. C'est donc sur ces analyses que se concentre
  actuellement notre équipe.
  
  \begin{longtable}[t]{lll}
  \caption{\label{tab:tablegene}Récapitulatif des variants ayant passé l'ensemble des filtres pour chaque famille}\\
  \toprule
  Family & Gene & Impact\\
  \midrule
  MMAF3 & WEE2 & missense\\
  MMAF3 & FCGR3A & missense\\
  MMAF3 & FCGR3A & missense\\
  MMAF3 & FCGR3A & missense\\
  MMAF3 & GBP2 & missense\\
  \addlinespace
  MMAF2 & MYH11 & missense\\
  MMAF2 & DNAH1 & splice acceptor\\
  MMAF3 & PCSK5 & missense\\
  AZ & GUF1 & missense\\
  MMAF1 & PLA2G4B & splice region\\
  \addlinespace
  MMAF1 & JMJD7-PLA2G4B & splice region\\
  MMAF3 & ZFYVE28 & missense\\
  MMAF4 & TGIF2 & missense\\
  AZ & SPINK2 & splice region\\
  FF & PLCZ1 & missense\\
  \addlinespace
  FF & PLCZ1 & missense\\
  FF & PLCZ1 & missense\\
  \bottomrule
  \end{longtable}
  
  \begin{figure}
  
  {\centering \includegraphics{thesis_files/figure-latex/plotremaininggenes-1} 
  
  }
  
  \caption[Nombre de gènes passant l'ensemble des filtres par famille]{Nombre de gènes passant l'ensemble des filtres par famille  :  Chaque barre représente une des familles analysées. La hauteur de cette barre correspond au nombre de gènes ayant passé l'ensemble des filtres pour chaque famille. Les barres vertes caractérisent les familles pour lesquelles le gène responsable de la pathologie a été identifié parmi la liste de gène (dans ce cas le symbole du gène est écrit au-dessus de la barre). La barre orange caractérise la famille pour laquelle un candidat potentiel a été identifié (le symbole du gène est écrit au-dessus suivit d'un "?"). Les barres rouges indiquent qu'aucun des gènes ayant passé les filtres pour ne semble expliquer le phénotype (dans ce cas il est écrit "???" au-dessus de la barre)}\label{fig:plotremaininggenes}
  \end{figure}
  
  \newpage
  
  \subsection{Etude d'une large cohorte de patients
  MMAF}\label{cohortemmah}
  
  \subsubsection{Description de la
  cohorte}\label{description-de-la-cohorte}
  
  Historique : après avoir mis en évidence DNAH1 -\textgreater{} MMAF
  notre équipe s'est en partie spécialisé dans ce syndrome.
  
  ainsi, entre (année) et année, notre équipe a effectué le séquençage de
  \ldots{} individus présentant ce phénotype afin d'en établir la cause
  génétique. parmi ces patients, la majorité provenait d'Afrique du Nord,
  cependant \ldots{} vfenaient de et de \ldots{} ces séquençage ont été
  effectué dans \ldots{} centres diférents que sont (listes des centre de
  séquençage) et sur \ldots{} plateforme : liste des plateformes
  
  \begin{longtable}[t]{lrr}
  \caption{\label{tab:tabcohort}Liste des différents projets de séquençages effectués}\\
  \toprule
  Place & Year & Nb of sequenced individuals\\
  \midrule
  MountSinai & 2012 & 2\\
  Strasbourg & 2012 & 13\\
  Genoscope & 2013 & 13\\
  Genoscope & 2014 & 28\\
  Genoscope & 2015 & 6\\
  \bottomrule
  \end{longtable}
  
  \newpage
  
  \subsubsection{Application de la pipeline -
  Résultats}\label{application-de-la-pipeline---resultats}
  
  Après avoir appelé les variants de nos 62 patients, nous avons obtenu un
  total de 4484558 variants différents comprenant 4160274 SNVs et 324284
  indels. Ces variants étant répartit entre chaque patient qui portaient
  environs chacun 81618 SNV et 5148 indels dont 42.8 \% étaient
  homozygote. Comme on peut le voir, la proportion de chaque appel est
  relativement homogène lorsque l'on compare les patients ayant été
  séquencés dans le même centre la même année. Cependant, il est possible
  de noter de grandes disparités lorsque l'on compare les données
  provenant de différents centres ou bien du même centre avec plusieurs
  années de différences. Ces écarts peuvent-être causés par plusieurs
  facteur, tel que les différents kits de capture d'exons qui on put être
  utilisés puisque \ldots{} (\textbf{todo lister les différents kits de
  capture dans une table}) en revanche nous pouvons écarter un effet dus à
  la plateforme de séquençage ou encore le modèle de séquenceur puisque
  tous ces projets ont été réalisés sur des Illumina HiSeq2000
  (\textbf{Table : }\ref{tab:tabcohort}) (\textbf{Figure :
  }\ref{fig:plotbigmmafcall} - \textbf{A}).
  
  Le même constat peut être effectué lorsque l'on compare la qualité des
  appels puisque plus les projets de séquençage s'avèrent être récent,
  plus la proportion d'appel \emph{Single Strand} s'avère être faible
  tandis que la proportion d'appel \emph{Double Strand} (DS) est élevée.
  Ceci est une bonne chose, car, bien que ces deux appels soient conservés
  dans les analyses ultérieures, les appels DS sont de meilleure qualité
  que les appels SS. Cette augmentation des appels DS au cours du temps
  pourrait s'expliquer par une amélioration des protocoles de séquençage
  ainsi que des kits de capture. En revanche cela est à pondérer avec le
  taux croissant d'appels \emph{No-strand} (NS) au fur et à mesure des
  années pour atteindre environs 21.3 \% en 2015 avec un projet réalisé au
  Génoscope. Ces derniers appels étant systématiquement filtrés, ils
  n'altèreront en rien les résultats obtenus en aval hormis le fait qu'ils
  réduisent la quantité des données utilisées (\textbf{Figure :
  }\ref{fig:plotbigmmafcall} - \textbf{B} et \textbf{C}).
  
  \newpage
  
  \begin{figure}
  
  {\centering \includegraphics{thesis_files/figure-latex/plotbigmmafcall-1} 
  
  }
  
  \caption[Résultats de l'appel des variants par individus et par projet de séquençage]{Résultats de l'appel des variants par individus et par projet de séquençage : Chaque couleur définit un projet de séquençage caractérisé par un centre de séquençage et une année. **A** : Quantification pour chaque individus (représentés par les barres) du nombre de variants (SNVs et Indels) appelés homozygotes et hétérozygotes. **B** : Quantification des appels *Double Strand* (DS), *Single Strand* (SS) et *No strand* (NS) pour chaque projet de séquençage. **C** : Même chose en pourcentage}\label{fig:plotbigmmafcall}
  \end{figure}
  
  \newpage
  
  \subsubsection{Analyse des listes de
  gènes}\label{analyse-des-listes-de-genes}
  
  Après avoir appliqué les mêmes filtres que ceux décrit précédemment à
  l'exception du filtre n°1
  \protect\hyperlink{filterdescription}{``Union''} puisqu'ici nous avons
  uniquement des individus non apparentés, nous avons pu obtenir une liste
  de 1711 variants différents composés de 1470 SNVs et 241 indels et
  impactant un total de 1432 gènes distincts. Ces variants étant répartis
  sur l'ensemble de nos 62 patients ceux-ci portaient en moyenne 27 SNVS
  et 5 indels, de sorte que chacun d'entre eux avaient entre 1 et 86 gènes
  impactés par au moins un variants homozygote (\textbf{Figure :
  }\ref{fig:plotfilterbigmmaf} - \textbf{A} et \textbf{B}).
  
  \begin{figure}
  
  {\centering \includegraphics{thesis_files/figure-latex/plotfilterbigmmaf-1} 
  
  }
  
  \caption[Résultats de l'étape de filtrage]{Résultats de l'étape de filtrage : **A** : Quantification du nombre de SNVs et indels ayant passé l'ensemble des filtres pour chaque patient. **B** : Nombre de gènes impactés par au moins un variant ayant passé les filtres pour chaque individu représenté par les barres. **C** : Présentation }\label{fig:plotfilterbigmmaf}
  \end{figure}
  
  \newpage
  
  Afin de déterminer parmi cette ensemble de gènes ceux responsables du
  phénotype de nos patients, nous avons procédés en trois étapes :
  
  \begin{enumerate}
  \def\labelenumi{\arabic{enumi}.}
  \tightlist
  \item
    \textbf{Étape n°1} : Cette étape consiste à séléctionner uniquement
    les variants ayant un effet tronquant sur la protéine comme un
    décalage du cadre de lecture, l'apparition d'un codon stop prématuré
    ou encore une perturbation des sites accepteurs / donneur d'épissage.
    Une analyse de l'expression testiculaire des gènes retenus ainsi
    qu'une étude bibliographique nous permet ensuite de séléctionner tout
    ou partie de ceux-ci. Du fait des effets extremement délétères de ces
    variants, les patients ressortant de cette première étape sont
    considérés comme de confience élevée (\emph{High trust}).\\
  \item
    \textbf{Étape n°2} : Pour l'ensemble des gènes retenus dans l'étape
    n°1, nous recherchons ensuite des patients portant, toujours à l'état
    homozygote, des variant aux effet non tronquant tel que des variants
    faux-sens ou encore des variants intronique situés proches des sites
    d'épissage. Dans le cas des variants faux-sens, les logiciels SIFT et
    PolyPhen sont ensuite utilisés afin de nous orienter quant à l'effet
    délétère du variant, bien que comme nous l'avons déjà vu, ces
    logiciels son contredisent regulièrement {[}TODO : ref!!!{]}. Au vus
    de la difficulté à déterminer l'effet délétère de ces variants, les
    patients identifiés au cours de cette étape sont marqués comme de
    confience modérée (\emph{Moderate trust}).\\
  \item
    \textbf{Étape n°3} : Cette étape consiste à recherchere des patients
    éventuellement hétérozygotes composites, c'est à dire des patients
    portant deux variants hétérozygotes différents sur chacun des deux
    allèles d'un même gène. Malheuresement, dans le cadre de séquençage
    WES WGS, il est impossible de connaitre le ``phasage'' des variants,
    c'est à dire que l'on ne peut déterminer si deux variants
    hétérozygotes sont situés sur le même allèles ou sur deux allèles
    différents. Pour cela, des analyses de biologie moléculaire sont
    nécéssaire. C'est pour cette raison que les patients identifier au
    cours de cette étape sont labellisés comme étant de faible confience
    (\emph{Low trust}).
  \item
    \textbf{Étape n°4} : Cette étape a pour but d'alleger le nombre de
    variant et donc de gène restant à analyser. Pour cela, nous retirons
    les données des patients pour lesquels la cause génétique a pu être
    déterminée \textbf{avec certitude}, c'est à dire ceux portant des
    variants homozygotes tronquants (c'est à dire ceux identifiés lors de
    l'étape n°1). En effet, la plupart des variants ressortant de l'étape
    n°2 sont des variants faux-sens, or, malgré les prédiction de SIFT et
    PolyPhen, il est impossible d'affirmer avec certitude que ces variants
    ont un réel impact sur la protéine sans effectuer d'analyses
    fonctionelles. De même pour les patients identifiés lors de l'étape
    n°3 pour lesquels il est impossible sans analyses de moléculaire de
    déterminer si les deux variants hétérozygotes affectent le même allèle
    ou bien deux allèles différents.\\
    \newpage  
  \end{enumerate}
  
  \paragraph{\texorpdfstring{\emph{DNAH1}, un acteur primordial dans le
  phénotype
  MMAF}{DNAH1, un acteur primordial dans le phénotype MMAF}}\label{dnah1-un-acteur-primordial-dans-le-phenotype-mmaf}
  
  Au moment de nos analyses, le gène \emph{DNAH1} était encore le seul
  décrit comme responsable du phénotype MMAF faisant de lui un candidat
  évident pour expliquer le phénotype MMAF de nos patients malgrés on
  expression non spécifique au testicule (\textbf{Figure :
  }\ref{fig:plotdnah1} - \textbf{A}). C'est pourquoi nous avons appliquer
  les 3 étapes précédement décrites en ciblant spécifiquement les patients
  ayant des variants chevauchant les gène \emph{DNAH1}.
  
  \begin{enumerate}
  \def\labelenumi{\arabic{enumi}.}
  \tightlist
  \item
    \textbf{Étape n°1} : Parmis l'ensemble de nos 62 patients MMAF 1
    portait un indel homozygote entrainant un décalage du cadre de lecture
    sur le gène \emph{DNAH1} (\textbf{Table : }\ref{tab:tabgrp1high}.\\
  \item
    \textbf{Étape n°2} : La recherche de variants homozygote non tronquant
    sur le gène \emph{DNAH1} nous a permis d'identifier 2 nouveaux
    patients : \ldots{} . On note ainsi que le patient \emph{Ghs90} est
    porteur de \ldots{} variants homozygotes successif, tous entrainant
    des faux-sens prédits comm \emph{benign} par PolyPhen et pour lesquels
    SIFT ne proposen aucune prédiction. Le patient \emph{Ghs95}, lui porte
    égallement un variant faux-sens différent de \ldots{} observés chez
    \emph{Ghs90} mais également prédit comme \emph{benign} par PolyPhen.
    Il est cependant à ce stade impossible de conclure avec certitudes que
    ces variants soient effectivement responsables du phénotype MMAF de
    ces 2 patients compte tenus des prédictions fournis par PolyPhen et du
    fait qu'il est difficile de conclure quant à l'effet d'un
    variant-faux-sens sans effectuer d'analyses fonctionelles. Cependant,
    l'abscence de ces \ldots{} dans chacune des 3 bases de donées que nous
    avons utilisés, c'est à dire ExAC, 1KG et ESP6500, laisse supposer que
    ceux-ci soient êtremement rare
  \item
    \textbf{Étape n°3} : La recherche d'hétérozygote composites nous a
    permis de révéler 6 patients portant tous 2 variants hétérozygotessur
    le gène \emph{DNAH1}. On oeut alors noter le patients \emph{Ghs36}
    portant 2 variant hétérozygotes. Le premier d'entre eux créé un codon
    stop de manière prématuré et n'est retrouvé ni dans ESP6500 ni dan 1KG
    tandis que sa fréquence dans la base de donnée ExAC est de \ldots{}.
    Le second est un variant faux-sens abscent des 3 bases de données et
    prédit comme \emph{probably damaging} par PolyPhen. Ainsi, s'ils
    étaient situés sur deux allèles différents, ces deux variants
    pourraient être des bons candidat pour expliquer le phénotype du
    patient \emph{Ghs36}. On peut noter égallement le patient
    \emph{Ghs129} portant deux faux-sens hétérozygote tout deux prédits
    comme \emph{probably damaging} par PolyPhen et dont un seul est
    retrouvé dans la base de donnée exac avec une fréquence de \ldots{} .
    Pour les \ldots{} autres patients, il est plus dificile de se
    prononcer à ce stade compte tenus des fréquence parfois élevée des
    variants ou bien des prédictions fournies par PolyPhen.\\
  \item
    \textbf{Étape n°4} : Au vus des résultats, seuls les données du
    patient \emph{Ghs122} ont été retirées de notre liste de variants,
    celui-ci étant le seul à porter un variant homozygote tronquant. Cela
    a ainsi permis de réduire notre liste à 1661 variants distincts
    cheveauchant 1395 gènes différents.
  \end{enumerate}
  
  Ainsi, cette première analyse nous a permis de révéler que 9 des 62
  patients de notre cohorte portaient au moins 1 variant sur le gène
  \emph{DNAH1} et que pour 3 d'entre eux ce(s) variants étaient présents à
  l'état homozygote. Cependant, il faut noter que du fait de son effet
  tronquant sur la proétine, seul le variant homozygote porté par le
  patient \emph{Ghs122} nous permets d'être certains de la causalité du
  phénotype MMAF. Pour les 8 autres patient, des analyses fonctionelles
  complémentaires sont nécéssaires (\textbf{Figure : }\ref{fig:plotdnah1}
  - \textbf{B}).
  
  Ainsi, les mutations du gène \emph{DNAH1} seraient ainsi responsables de
  manière certaine de 2 \% des phénotype MMAF de notre cohorte. Ce
  pourcentage monte jusqu'à 15 \% si l'on considère l'ensemble des
  patients identifiés dans cette analyse.
  
  Bien que ce pourcentage soit en deçà des 40\% (TODO: à confirmer!)
  observés dans notre étude précédente (Ben Khelifa et al.,
  \protect\hyperlink{ref-BenKhelifa2014}{2014}), ces résultats confirment
  néanmoins le rôle primordial de la protéine DNAH1 dans la structure du
  flagelle et l'implication majeure du gène \emph{DNAH1} dans le phénotype
  MMAF.
  
  \newpage
  
  \begin{figure}
  
  {\centering \includegraphics{thesis_files/figure-latex/plotdnah1-1} 
  
  }
  
  \caption[Analyse du gène *DNAH1*]{Analyse du gène *DNAH1* : Expression tissulaire du gène *DNAH1* d'après les données du projet Illumina BodyMap. Quantification du nombre de patients portant au moins un variant sur le gène *DNAH1* pour chacun des 3 niveau de confiance}\label{fig:plotdnah1}
  \end{figure}
  
  \newpage
  
  \begin{longtable}[t]{llllll}
  \caption{\label{tab:tabdnah1high}Liste des patients portant au moins un variant homozygote tronquant sur le gène *DNAH1*}\\
  \toprule
  Gene & Patient & Consequence & ESP & 1KG & ExAC\\
  \midrule
  DNAH1 & Ghs122 & frameshift & none & none & none\\
  \bottomrule
  \end{longtable}
  
  \begin{longtable}[t]{llllllll}
  \caption{\label{tab:unnamed-chunk-8}Liste des patients portant au moins un variant homozygote non tronquant sur le gène *DNAH1*}\\
  \toprule
  Gene & Patient & Consequence & SIFT & PolyPhen & ESP & 1KG & ExAC\\
  \midrule
  DNAH1 & Ghs90 & missense & none & benign & none & none & none\\
  DNAH1 & Ghs90 & missense & none & benign & none & none & none\\
  DNAH1 & Ghs90 & missense & none & benign & none & none & none\\
  DNAH1 & Ghs95 & missense & none & benign & none & none & none\\
  \bottomrule
  \end{longtable}
  
  \begin{longtable}[t]{llllllll}
  \caption{\label{tab:unnamed-chunk-9}Liste des patients portant au moins deux variant hétérozygotes sur le gène *DNAH1*}\\
  \toprule
  Gene & Patient & Consequence & SIFT & PolyPhen & ESP & 1KG & ExAC\\
  \midrule
  DNAH1 & Ghs28 & missense & none & benign & 1e-04 & none & 1.65e-05\\
  DNAH1 & Ghs28 & missense & none & benign & 0.0027 & 0.0019 & 0.00233\\
  DNAH1 & Ghs36 & missense & none & probably damaging & none & none & none\\
  DNAH1 & Ghs36 & stop gained & none & none & none & none & 8.29e-06\\
  DNAH1 & Ghs42 & missense & none & probably damaging & none & none & none\\
  \addlinespace
  DNAH1 & Ghs42 & splice region & none & none & none & none & none\\
  DNAH1 & Ghs87 & missense & none & benign & none & none & 0.00024\\
  DNAH1 & Ghs87 & missense & none & probably damaging & 7e-04 & 5e-04 & 0.000457\\
  DNAH1 & Ghs88 & missense & none & benign & 1e-04 & none & 0.000115\\
  DNAH1 & Ghs88 & missense & none & benign & 1e-04 & 0.0019 & 0.000149\\
  \addlinespace
  DNAH1 & Ghs129 & missense & none & probably damaging & none & none & none\\
  DNAH1 & Ghs129 & missense & none & probably damaging & none & none & 8.26e-06\\
  \bottomrule
  \end{longtable}
  
  \newpage
  
  \paragraph{Les nouveaux acteurs}\label{les-nouveaux-acteurs}
  
  Comme dit à l'instant, la recherche de variants sur le gène \emph{DNAH1}
  nous permettrait d'expliquer jusqu'à 15 \% des phénotypes des patients
  de notre cohorte. Cela signifie donc que pour au moins 85 \% d'entre
  eux, la cause génétique du phénotype MMAF est donc expliquée par
  d'autres facteurs.
  
  Afin de nous orienter dans nos recherches, nous nous sommes basés sur
  une étude de 2012 qui établissait une liste des gènes humains pouvant
  être impliqués dans cilliome, c'est à dire (todo def cilliome) (Ivliev,
  't Hoen, Roon-Mom, Peters, \& Sergeeva,
  \protect\hyperlink{ref-Ivliev2012}{2012}). La constitution de cette
  liste se basait à la fois sur les données de CilDB {[}ref ? {]} et de
  MEDLINE {[} ref ? {]} mais aussi des analyse \emph{in silico} permettant
  d'effectuer des prédiction. Ainsi, chaque gène était classé dans l'une
  des 3 catégories suivantes en fonction des preuves déjà existante (au
  moment de l'étude) permettant de lier un gène au cilliome humain :
  \textbf{Strong evidence from previous studies} (Strong), \textbf{Weak
  evidence from previous studies} (Weak) et \textbf{No evidence from
  previous studies} (Novel). L'utilisation de cette liste nous a permis
  d'ajouter une nouvelle annotation pertinente à nos gènes. En effet, le
  spermatozoïde humain est une cellule ciliée, et le flagelle en est le
  cil. Nous pouvons donc attendre à ce qu'une partie des gènes
  responsables du phénotype MMAF soit présents dans cette liste de 371
  gènes.
  
  Ainsi, 32 de nos 1395 gènes retenus faisaient partis de cette liste dont
  21 présentaient des preuves fortes de leur appartenance au cilliome. Il
  faut tout de même noter que bien que cette liste soit un bon outil pour
  orienter les recherches et prioriser certains gènes, elle ne peut
  constituer un critère suffisant pour filtrer les gènes n'en faisant pas
  partie. Par exemple le gène \emph{DNAH1}, de par son expression
  ubiquitaire n'a pas été intégré à cette liste (\textbf{Figure :
  }\ref{fig:expdnah1}, or on connait désormais son implication dans le
  phénotype MMAF (\textbf{Figure : }\ref{fig:plotfilterbigmmaf2} -
  \textbf{A}).
  
  Suite à cela, afin de nous concentrer en priorité sur les gènes
  entrainant un phénotype MMAF chez le plus grand nombre d'individus, nous
  avons sélectionnés ceux sur lesquels plusieurs patients portaient au
  moins un variant ayant passé l'ensemble des filtres nous permettant
  alors d'obtenir une liste de 212 gènes dont 145 (soit 68 \%) étaient
  retrouvés variants chez uniquement 2 patients (\textbf{Figure :
  }\ref{fig:plotfilterbigmmaf2} - \textbf{B}).
  
  Ces deux informations nous ont ensuite permis de \emph{designer} 4
  analyses que nous avons effectué de manière successives nous permettant
  ainsi de séléctionner dans un premier temps les gènes candidats les plus
  évidents nous permettant ainsi d'épurer progressivement notre liste de
  variants rendant ainsi plus facile l'identification des candidats les
  moins évidents.
  
  \begin{enumerate}
  \def\labelenumi{\arabic{enumi}.}
  \item
    \textbf{Analyse n°1} : Dans un permier temps, nous avons étudié
    uniquement les gènes \textbf{présents dans la liste cilliome} sur
    lesquels \textbf{au moins deux} de nos patients présentaient au moins
    1 variant tronquant à l'état \textbf{homozygote}.
  \item
    \textbf{Analyse n°2} : Ensuite, nous avons étudié les gènes
    \textbf{absents dans la liste cilliome} mais sur lesquls on trouvait
    toujours \textbf{au moins deux} de nos patients présentant au moins 1
    variant tronquant à l'état \textbf{homozygote}.
  \item
    \textbf{Analyse n°3} : Dans un troisième temps nous sommes revenus à
    étudier les gènes \textbf{présents dans la liste cilliome} en
    considérant cette fois-ci les gènes sur lesquels \textbf{un seul} de
    nos patients présentaient au moins 1 variant tronquant à l'état
    \textbf{homozygote}.
  \item
    \textbf{Analyse n°4} : Pour finir nous avons étudié les gènes
    \textbf{absents dans la liste cilliome} sur lesquels \textbf{un seul}
    de nos patients présentaient au moins 1 variant tronquant à l'état
    \textbf{homozygote}.
  \end{enumerate}
  
  Il faut noter qu'au sein des chacun de ces 4 analyses nous avons
  appliquer les étapes n°1-4 décrites précédemment.
  
  \newpage
  
  \begin{figure}
  
  {\centering \includegraphics{thesis_files/figure-latex/plotcil-1} 
  
  }
  
  \caption[TODOOOOOOOOOOOOOOOOOO]{TODOOOOOOOOOOOOOOOOOO : Chaque couleur définit une classe de la liste des gènes du cilliome décrit dans [@Ivliev2012]. Vert = *Strong evidence from previous studies* (*Strong*), Orange = *Weak evidence from previous studies* (*Weak*), rouge = *No evidence from previous studies* (*Novel*), bleu = Non présent dans la liste. **A** : Quantification du nombre de gène ayant passé les filtres au sein des 3 classes de la liste des gènes du cilliome. **B** : TODOOOOOO : Besoin d'aide pour le nom de l'axe des Y !!!!!}\label{fig:plotcil}
  \end{figure}
  
  \newpage
  
  \subparagraph{Analyse n°1}\label{analyse-n1}
  
  Comme dit précédemment, dans cette analyse nous nous sommes concentrés
  sur les gènes \textbf{présents dans la liste cilliome} sur lesquels
  \textbf{au moins deux} de nos patients présentaient au moins 1 variant
  \textbf{homozygote} ayant un effet tronquant sur la protéine.
  
  \begin{enumerate}
  \def\labelenumi{\arabic{enumi}.}
  \item
    \textbf{Étape n°1} : Nous sommes ainsi arrivé à une liste de 4 gènes
    différents : TTC29, CFAP44, CCDC146, ARMC2 retrouvés mutés à l'état
    homozygote chez 8 de nos patients. Parmis eux, les patients
    \emph{Ghs37} et \emph{Ghs93} portent respectivement un indel créant un
    décalage du cadre de lecture et un variant stop affectant le gène
    \emph{ARMC2} dont aucun n'est retrouvé dans les bases de données.
  \item
    \textbf{Étape n°2} : Nous avons ensuite étendu notre champs de
    recherche aux patients portant des variants homozygotes sur l'un d'eux
    en ne se limitant pas cette fois-ci aux variants tronquant. Nous avons
    ainsi pu identifier 2 nouveaux patients présentant tout deux des
    variants faux-sens homozygote, l'un sur le gène \emph{ARMC2}, l'autre
    sur \emph{CFAP44}. On peut noter que ces deux variants faux-sens sont
    prédit comme délétères par au moins 1 des deux logiciels de prédiction
    utilisés (SIFT et PolyPhen).
  \item
    \textbf{Étape n°3} : Pour finir avec ce premier lot de gènes, nous
    avons comme précédemment cherché des patients potentiellement
    hétérozygotes composites, c'est à dire portant au moins deux variants
    hétérozygotes, tronquant ou non. Cependant, aucun de nos patients
    restant ne remplissaient ces critères
  \item
    \textbf{Étape n°4} : Cette première analyse explique potentiellement
    la cause génétique responsable du phénotype MMAF de 10 patients.
    Cependant pour 2 d'entre eux, Les données de ces patients sont alors
    retirées de notre liste de variants , nous avons comme précédemment
    avec les patients \emph{DNAH1} retirer l'intégralité de leurs variants
    des analyses ulterieures afin de réduire à nouveau la liste des
    variants et des gènes à analyser facilitant ainsi les analyses. Ainsi,
    \ldots{} variants sur \ldots{} gènes.
  \end{enumerate}
  
  Pour résumer, cette première analyse nous a permis de mettre en évidence
  2 gènes dont différents plusieurs preuves laissent supposer leur
  implication dans le phénotype MMAF de 10 de nos patients. Pamis ceux-ci,
  8 portent un variant tronquant à l'état homozygote nous donnant ainsi
  une grande confiance dans le fait que le variant soit effectivement à
  l'origine du phénotype. Pour les 2 autres patients, les variants
  faux-sens portés par ces patients laisse planer un doute, cependant, le
  fait que \ldots{}\ldots{}\ldots{}.. (\textbf{Figure :
  }\ref{fig:plotgrp1high} - \textbf{B})
  
  \begin{enumerate}
  \def\labelenumi{\arabic{enumi}.}
  \tightlist
  \item
    \textbf{\emph{ARMC2}} : Le gène \emph{ARMC2} est classé dans la
    catégorie \emph{No evidence from previous studies} des gènes du
    cilliome de notre liste. Cependant son expression testiculaire forte
    et exclusive font de lui un bon candidat pour expliquer le phénotype
    MMAF de 3 portant respectivement à l'état homozygote un variant
    faux-sens, un impactant le site d'épissage et l'autre entrainant un
    décalage du cadre de lecture
  \end{enumerate}
  
  présentant tous une forte expression testiculaire (\textbf{Table :
  }\ref{tab:tabgrp1high}, \textbf{Figure : }\ref{fig:plotgrp1high} -
  \textbf{A}). Parmi ces gènes, on peut noter que \emph{CCDC146} codant
  pour la protéine CCDC146 avait déjà été décrit comme composant du
  centrosome spermatique, un organite ayant un rôle dans l'orientation des
  cellules et étant à l'origine des cils et des flagelles (Firat-Karalar,
  Sante, Elliott, \& Stearns,
  \protect\hyperlink{ref-Firat-Karalar2014}{2014}) reforçant ainsi les
  arguments de l'implication de ce gène dans le phénotype MMAF. L'analyse
  bibliographique des 3 autres gènes n'a cependant rien révélés ceux-ci
  étant jusqu'à présent peu étudiés.
  
  \newpage
  
  \begin{figure}
  
  {\centering \includegraphics{thesis_files/figure-latex/plotgrp1high-1} 
  
  }
  
  \caption[Analyse des gènes séléctionnés dans l'Analyse n°1]{Analyse des gènes séléctionnés dans l'Analyse n°1 : Expression tissulaire des gènes retenus d'après les données du projet de transcriptome Illumina bodyMap. Résumé de l'Analyse 1, quantification du nombre de patients retrouvés mutés sur chacun des gènes retenus ainsi que du degré de confiance accordé à la cause génétique}\label{fig:plotgrp1high}
  \end{figure}
  
  \newpage
  
  \begin{longtable}[t]{llllll}
  \caption{\label{tab:tabgrp1high}List des gènes présents dans la liste ciliome sur lesquels au moins deux patients portent une mutation tronquante homozygote}\\
  \toprule
  Gene & Patient & Consequence & ESP & 1KG & ExAC\\
  \midrule
  ARMC2 & Ghs37 & frameshift & none & none & none\\
  ARMC2 & Ghs93 & splice donor & none & none & none\\
  CCDC146 & Ghs32 & stop gained & 1e-04 & none & 2.47e-05\\
  CCDC146 & Ghs35 & frameshift & none & none & none\\
  CFAP44 & Ghs22 & stop gained & none & none & none\\
  \addlinespace
  CFAP44 & Ghs34 & splice donor & none & none & none\\
  TTC29 & Ghs19 & splice donor & 0.0012 & 5e-04 & 0.000158\\
  TTC29 & Ghs26 & splice donor & 0.0012 & 5e-04 & 0.000158\\
  \bottomrule
  \end{longtable}
  
  \begin{longtable}[t]{llllllll}
  \caption{\label{tab:unnamed-chunk-14}Liste des patients portant un variant non troquant homozygote sur un des gènes suivant : ARMC2, CCDC146, CFAP44  et  TTC29}\\
  \toprule
  Gene & Patient & Consequence & SIFT & PolyPhen & ESP & 1KG & ExAC\\
  \midrule
  CFAP44 & Ghs89 & missense & tolerated & possibly damaging & 0.0012 & 0.0014 & 0.000692\\
  CFAP44 & Ghs89 & missense & tolerated & probably damaging & 0.0012 & 0.0014 & 0.000692\\
  ARMC2 & Ghs107 & missense & deleterious & probably damaging & none & none & none\\
  ARMC2 & Ghs107 & missense & deleterious & probably damaging & none & none & none\\
  \bottomrule
  \end{longtable}
  
  \begin{longtable}[t]{llllllll}
  \caption{\label{tab:unnamed-chunk-15}Liste des patients portant un variant non troquant homozygote sur un des gènes suivant : ARMC2, CCDC146, CFAP44  et  TTC29}\\
  \toprule
  Gene & Patient & Consequence & SIFT & PolyPhen & ESP & 1KG & ExAC\\
  
  
  \bottomrule
  \end{longtable}
  
  \newpage
  
  \subparagraph{Analyse n° 2}\label{analyse-n-2}
  
  Pour rappel, au cours de cette analyse, nous avons étudié les gènes
  \textbf{absents dans la liste cilliome} mais sur lesquls on trouvait
  toujours \textbf{au moins deux} de nos patients présentant au moins 1
  variant tronquant à l'état \textbf{homozygote}.
  
  \begin{enumerate}
  \def\labelenumi{\arabic{enumi}.}
  \item
    \textbf{Étape n°1} : Nous avons ainsi pu identifier 8 gènes sur
    lesquels au moins 2 de nos patients restant et jusqu'à 5 portaient un
    variant homozygote ayant un effet tronquant sur la protéine. Cela nous
    a permis d'obtenir une liste de 17 nouveaux patients avec un candidat
    potentiel (\textbf{Table :} \ref{tab:tabgrp2}, \textbf{Figure :}
    \ref{fig:plotgrp2} - \textbf{A}). Parmis les gènes identifié, 3 se
    démarquent avec une forte expression testiculaire : \emph{CFAP43},
    \emph{FSIP2} et \emph{CCDC129} respectivement retrouvés mutés chez 19,
    4, et 2 des 17 patients séléctionnés. C'est pourquoi nous avons
    focalisé nos analyses sur ceux-ci. (\textbf{Table :
    }\ref{tab:tabgrp2high}, \textbf{Figure :} \ref{fig:plotgrp2} -
    \textbf{B}). L'analyse bibliographique du gène \emph{FSIP2} révèle
    qu'une équipe a démontré en 2003 l'implication de ce gène dans la
    structure de la gaine fibreuse du flagelle spermatique confirmant que
    ce gène est un bon candidat pour expliquer le phénotype MMAF des 4
    patients portant un variant homozygote sur celui-ci (Brown, Miki,
    Harper, \& Eddy, \protect\hyperlink{ref-Brown2003}{2003}).
  \item
    \textbf{Étapes n°2 et n°3} : La recherche de patients portant des
    variants homozygotes non tronquant sur ces 3 gènes ainsi que ceux
    portant au moins deux variants hétérozygotes tronquant ou non nous a
    permis d'identifier 7 nouveaux patients. Il faut cependant noter que
    parmis ceux-ci, certains portent de variant pour lesquels il est
    difficile d'affirmer avec certitude qu'ils sont responsables du
    phénotype. Le patients Ghs25 par exemple porte un variant homozygote
    chevauchant le gènes \emph{CFAP43}. Cependant, le variant est un
    variant intronique situé dans la région d'épissage (et non sur un site
    donneur ou accepteur). Bien que ce type de variants puissent
    effectivement avoir un impact sur l'epissage, il pourrait également
    être sans effet, or, il est difficile de le prédire à ce stade. De
    même pour le patient Ghs131, dont le variant faux-sens chevauchant le
    gène \emph{FSIP2} qu'il porte est prédit comme \emph{benign} par
    PolyPhen alors que SIFT ne fournit aucune prédiction pour celui-ci.
    Pour les patients Ghs40, Ghs92 et Ghs101, au delà du fait qu'il nous
    est impossible de savoir à ce stade si les deux variants hétérozygotes
    qu'ils portent impacte le même allèle ou bien deux allèles différents,
    tous portent des variant faux-sens sur le gène \emph{FSIP2} là encore
    prédit pour la plupart comme sans effet par SIFT et / ou PolyPhen. Le
    patient Ghs132 en revanche semble plus intéréssant, puisque celui-ci
    porte deux variant hétérozygotes sur le gène \emph{CFAP43} parmis
    lesquels un est un indel entrainant un décalage du cadre de lecture
    tandis que l'autre est un faux-sens prédit comme \emph{possibly
    damaging} par Polyphen, bien qu'il soit annoté \emph{tolerated} par
    SIFT (\textbf{Tables : }\ref{tab:tabgrp2moderate} et
    \ref{tab:tabgrp2low}).
  \item
    \textbf{Étape n°4} : Dans cette analyse seuls 9 les patients
    identifiés au cours de l'étape n°1 présentaient des arguments
    génétique suffisement convaiquant pour affirmer avec certitude qu'ils
    étaient responsables de leur phénotype MMAF. Ainsi, seuls leurs
    données furent retirées pour les analyses ulterieures. POur les 7 des
    analyses de biologie moléculaire sont nécéssaires afin de pouvoir
    déterminés siles variants mis en évidences dans les étape n°3 et 4
    sont bien responsable de leur phénotype. En attendant, leurs données
    sont conservés afin de voir si les analyses ulterieures ne
    permettraient pas d'identifier un meilleur candidat pour ces patients.
  \end{enumerate}
  
  Dans cette Analyse, nous avons pu déterminer avec certitude la cause
  génétique du phénotype MMAF de seulement 9 patients, c'est à dire ceux
  identifiés au cours de l'étape n°1. Pour 7 autres patients des analyses
  complémentaires sont nécéssaires. Cela nous a tout de même permis de
  pouvoir impliquer 3 nouveaux gènes au phénotype MMAF (\textbf{Figure :}
  \ref{fig:plotresumegrp2}).
  
  \newpage  
  
  \begin{figure}
  
  {\centering \includegraphics{thesis_files/figure-latex/plotgrp2-1} 
  
  }
  
  \caption[TODOOOOOOOOOOOOOOOOOO]{TODOOOOOOOOOOOOOOOOOO : Dans ces deux graphiques, chaque couleur représente un gène. **A** : Quantification du nombre d'individus présentant un variant homozygote ayant un effet tronquant pour chacun des gènes retrouvés présent dans la liste des gènes du ciliome et retouvés avec un variant tronquant homozygote chez au moins deux patients. **B** : Expression tissulaire des gènes précédents d'après les données du projet de transcriptome Illumina bodyMap}\label{fig:plotgrp2}
  \end{figure}
  
  \newpage  
  
  \begin{longtable}[t]{lll}
  \caption{\label{tab:tabgrp2high}List des gènes sur lesquels au moins deux patients portent une mutation tronquante non présents dans la liste ciliome }\\
  \toprule
  Gene & Patient & Consequence\\
  \midrule
  BAZ1A & Ghs18 & frameshift\\
  BAZ1A & Ghs94 & frameshift\\
  CCDC129 & Ghs91 & frameshift\\
  CCDC129 & Ghs132 & frameshift\\
  CFAP43 & Ghs17 & stop gained\\
  \addlinespace
  CFAP43 & Ghs17 & stop gained\\
  CFAP43 & Ghs17 & stop gained\\
  CFAP43 & Ghs17 & stop gained\\
  CFAP43 & Ghs41 & stop gained\\
  CFAP43 & Ghs41 & stop gained\\
  \addlinespace
  CFAP43 & Ghs41 & stop gained\\
  CFAP43 & Ghs41 & stop gained\\
  CFAP43 & Ghs102 & frameshift\\
  CFAP43 & Ghs102 & frameshift\\
  CFAP43 & Ghs102 & frameshift\\
  \addlinespace
  CFAP43 & Ghs102 & frameshift\\
  CFAP43 & Ghs105 & splice acceptor\\
  CFAP43 & Ghs105 & splice acceptor\\
  CFAP43 & Ghs105 & splice acceptor\\
  CFAP43 & Ghs105 & splice acceptor\\
  \addlinespace
  CFAP43 & Ghs126 & stop gained\\
  CFAP43 & Ghs126 & stop gained\\
  CFAP43 & Ghs126 & stop gained\\
  FSIP2 & Ghs20 & frameshift\\
  FSIP2 & Ghs20 & frameshift\\
  \addlinespace
  FSIP2 & Ghs21 & frameshift\\
  FSIP2 & Ghs21 & frameshift\\
  ICA1 & Ghs17 & frameshift\\
  ICA1 & Ghs40 & frameshift\\
  ICA1 & Ghs105 & frameshift\\
  \addlinespace
  NACA & Ghs39 & frameshift\\
  NACA & Ghs41 & frameshift\\
  SART3 & Ghs31 & splice donor\\
  SART3 & Ghs38 & splice donor\\
  SART3 & Ghs40 & splice donor\\
  \addlinespace
  SART3 & Ghs41 & splice donor\\
  TRAV26-1 & Ghs40 & frameshift\\
  TRAV26-1 & Ghs43 & frameshift\\
  TRAV26-1 & Ghs55 & frameshift\\
  \bottomrule
  \end{longtable}
  
  \begin{longtable}[t]{lllll}
  \caption{\label{tab:tabgrp2moderate}Liste des patients portant un variant non troquant homozygote sur un des gènes suivant : TODOOOOOO}\\
  \toprule
  Gene & Patient & Consequence & SIFT & PolyPhen\\
  \midrule
  CFAP43 & Ghs25 & splice region & No prediction & No prediction\\
  FSIP2 & Ghs131 & missense & No prediction & benign\\
  \bottomrule
  \end{longtable}
  
  \newpage
  
  \begin{longtable}[t]{lllll}
  \caption{\label{tab:tabgrp2low}Liste des patients portantau moins deux variants hétérozygotes sur un des gènes suivant : TODOOOOOO}\\
  \toprule
  Gene & Patient & Consequence & SIFT & PolyPhen\\
  \midrule
  FSIP2 & Ghs40 & missense & No prediction & benign\\
  FSIP2 & Ghs40 & missense & No prediction & probably damaging\\
  FSIP2 & Ghs40 & missense & No prediction & benign\\
  FSIP2 & Ghs92 & missense & No prediction & benign\\
  FSIP2 & Ghs92 & missense & No prediction & benign\\
  \addlinespace
  FSIP2 & Ghs95 & missense & No prediction & benign\\
  FSIP2 & Ghs95 & missense & No prediction & benign\\
  FSIP2 & Ghs101 & missense & No prediction & unknown\\
  FSIP2 & Ghs101 & missense & No prediction & benign\\
  CFAP43 & Ghs132 & frameshift & No prediction & No prediction\\
  CFAP43 & Ghs132 & missense & tolerated & possibly damaging\\
  \bottomrule
  \end{longtable}
  
  \begin{figure}
  
  {\centering \includegraphics{thesis_files/figure-latex/plotresumegrp2-1} 
  
  }
  
  \caption[Résumé des gènes identifiés au cours de l'analyse n°2]{Résumé des gènes identifiés au cours de l'analyse n°2 : Quantification du nombre de patients retrouvés mutés sur chacun des gènes retenus ainsi que du degré de confiance accordé à la cause génétique}\label{fig:plotresumegrp2}
  \end{figure}
  
  \newpage
  
  \subparagraph{Analyse n°3}\label{analyse-n3}
  
  Dans cette un troisième analyse, nous avons séléctionné à nouveau les
  gènes \textbf{présents dans la liste cilliome} en conservanrt cett
  fois-ci ceux sur lesquels \textbf{un seul} de nos patients présentaient
  au moins 1 variant tronquant à l'état \textbf{homozygote}.
  
  \begin{enumerate}
  \def\labelenumi{\arabic{enumi}.}
  \item
    \textbf{Étape n°1} : La recherche de variants homozygote tronquant
    nous a permis d'identifier 36 patients d'un variant sur un des 75 gène
    suivant : NTHL1, SIGLECL1, ZNF528, PLA2R1, PI4K2A, HOGA1, CASP1,
    DPY19L2P2, CROCCP2, PPFIBP2, CCDC66, PSG4, CCL3L3, MTCH1, ZNF862,
    CHST4, NACA, OR2T11, C6orf118, ICA1, FOXD4L5, CASC3, SLC35E1, FCN3,
    CNTLN, SIM2, RP6-206I17.1, TUBB8P7, CCDC9, PSPHP1, ACSBG2,
    EPB41L4A-AS2, LRRC9, SP100, MANEA, LINC00969, SDHAP2, TTLL2, PTGR1,
    ZNF438, CLEC12A, KRT74, RBM45, AP000275.65, TCP10L, C21orf59, PCSK6,
    CCDC80, OR51H1P, TRAP1, PMS1, RRM2B, SDSL, ALDOA, NDUFV2, DRC1,
    ABCA10, CFHR1, NAT16, SLC2A8, MAP4K2, SERPINA10, UNC93A, RNFT2,
    OR13A1, MS4A6A, CCDC65, TMEM59, RYK, SPEF2, LDHAL6A, HIST1H4J, CAV1,
    MCPH1, GPR142. Cependant, l'analyse de l'expression testiculaire de
    ces gènes nous a insitée à n'en garder que CCDC65, C6orf118 d'entre
    eux : CCDC65, C6orf118 (\textbf{Table :} \ref{tab:tabgrp3high},
    \textbf{Figure :} \ref{fig:plotgrp3} - \textbf{A}). En effet,
    \emph{CCDC65} présente une forte et quasi exclusive expression
    testiculaire de plus, la protéine NYD-SP28 (ancien nom de CCDC65)
    avait déjà caractérisé comme faisant partie du flagelle spermatique
    (Y. Zheng et al., \protect\hyperlink{ref-Zheng2006}{2006}). En
    revanche \emph{C6orf118} présente une forte expression à la fois dans
    le testicule mais égallement dans le poumoun de plus, ce gène a
    récemment été décrit comme étant associé au phénotype de tuberculose
    pulmonaire (E. P. Hong, Go, Kim, \& Park,
    \protect\hyperlink{ref-Hong2017}{2017}). Cependant cela n'est en rien
    contradictoire avec le phénotype MMAF du patient, le poumon comprenant
    de nombreuse cellules ciliées, notamment au niveau de l'épithélium
    respiratoire, il n'est donc pas surprenant que des gènes du flagelle
    aient égallement une fonction au sein d'autres organes ciliés.
  \item
    \textbf{Étape n°2} : Nous avons ensuite cherché si des patients
    présentaient des variants homozygotes non tronquants sur au moins 1 de
    ces 2 gènes cependant aucun ne remplissait ces critères.
  \item
    \textbf{Étape n°3} : La recherche de potentiels hétérozygotes
    composites s'est elle révélée plus fructueuse puisque 1 de nos
    patients portait deux variants faux-sens sur le gene \emph{C6orf118} à
    l'état hétérozygote. Le premier de ces variants étant prédit
    \emph{probably damaging} par PolyPhen et \emph{tolerated low
    confidence} par SIFT tandis que le second est prédit \emph{possibly
    damaging} et \emph{tolerated}, il est dificile de se prononcer quant à
    l'effet délétère de ces deux variants (\textbf{Table :}
    \ref{tab:tabgrp3low}).
  \item
    \textbf{Étape n°4} : Comme pour l'analyse n°2, nous avons ici décidé
    de ne retirer de notre liste de variant que ceux présents chez les
    patients identifiés au cours de l'étape n°1. Cela nous a ainsi permis
    de réduire à nouveau cette liste arrivant ainsi à 1065 variants
    cheveauchant 938 gènes différents.
  \end{enumerate}
  
  Analyser les gènes de la liste ciliome sur lesquels \emph{un seul}
  patient portait un variant tronquant à l'état homozygote nous a permis
  d'identifié 75 parmis lesquels 2 présentaient une forte expression
  testiculaire. L'analyse de ces 2 derniers gènes nous a permis
  d'identifier 2 patients chacun portant un variant tronquant homozygote
  sur l'un de ces 2 gènes. Étendre notre recherche à la fois aux variants
  tronquant ainsi qu'aux hétérozygote composites nous a permis
  d'identifier 1 patient pour qui deux faux-sens hétérozygotes sur le gène
  \emph{C6orf118} pourraient expliquer le phénotype (\textbf{Figure :}
  \ref{fig:plotgrp3} - \textbf{B}).
  
  \newpage  
  
  \begin{verbatim}
  ## Warning in RColorBrewer::brewer.pal(n, pal): n too large, allowed maximum for palette Set1 is 9
  ## Returning the palette you asked for with that many colors
  \end{verbatim}
  
  \begin{verbatim}
  ## Warning: Removed 32 rows containing missing values (geom_col).
  \end{verbatim}
  
  \begin{figure}
  
  {\centering \includegraphics{thesis_files/figure-latex/plotgrp3-1} 
  
  }
  
  \caption[Analyse des gènes séléctionnés dans l'Analyse n°3]{Analyse des gènes séléctionnés dans l'Analyse n°3 : Expression tissulaire des gènes retenus dans cette analyse d'après les données du projet de transcriptome Illumina bodyMap. Résumé de l'Analyse 3, quantification du nombre de patients retrouvés mutés sur chacun des gènes retenus ainsi que du degré de confiance accordé à la cause génétique}\label{fig:plotgrp3}
  \end{figure}
  
  \newpage
  
  \begin{longtable}[t]{lll}
  \caption{\label{tab:tabgrp3high}List des gènes sur lesquels au moins deux patients portent une mutation tronquante non présents dans la liste ciliome }\\
  \toprule
  Gene & Patient & Consequence\\
  \midrule
  ABCA10 & Ghs95 & frameshift\\
  ACSBG2 & Ghs52 & splice acceptor\\
  ALDOA & Ghs92 & frameshift\\
  AP000275.65 & Ghs88 & stop gained\\
  C21orf59 & Ghs88 & frameshift\\
  \addlinespace
  C6orf118 & Ghs40 & stop gained\\
  CASC3 & Ghs40 & frameshift\\
  CASP1 & Ghs27 & splice acceptor\\
  CAV1 & Ghs133 & frameshift\\
  CCDC65 & Ghs127 & frameshift\\
  \addlinespace
  CCDC66 & Ghs33 & frameshift\\
  CCDC80 & Ghs90 & frameshift\\
  CCDC9 & Ghs47 & splice donor\\
  CCL3L3 & Ghs36 & stop lost\\
  CFHR1 & Ghs96 & splice donor\\
  \addlinespace
  CHST4 & Ghs38 & stop gained\\
  CLEC12A & Ghs87 & splice donor\\
  CNTLN & Ghs42 & splice acceptor\\
  CROCCP2 & Ghs30 & splice acceptor\\
  DPY19L2P2 & Ghs28 & splice acceptor\\
  \addlinespace
  DRC1 & Ghs94 & frameshift\\
  EPB41L4A-AS2 & Ghs55 & frameshift\\
  FCN3 & Ghs42 & frameshift\\
  FOXD4L5 & Ghs40 & frameshift\\
  GPR142 & Ghs134 & frameshift\\
  \addlinespace
  HIST1H4J & Ghs133 & frameshift\\
  HOGA1 & Ghs25 & stop gained\\
  ICA1 & Ghs40 & frameshift\\
  KRT74 & Ghs87 & splice acceptor\\
  LDHAL6A & Ghs131 & stop gained\\
  \addlinespace
  LINC00969 & Ghs87 & splice donor\\
  LRRC9 & Ghs55 & stop gained\\
  MANEA & Ghs86 & frameshift\\
  MAP4K2 & Ghs97 & splice donor\\
  MCPH1 & Ghs134 & stop gained\\
  \addlinespace
  MS4A6A & Ghs127 & splice acceptor\\
  MTCH1 & Ghs38 & frameshift\\
  NACA & Ghs39 & frameshift\\
  NAT16 & Ghs96 & frameshift\\
  NDUFV2 & Ghs92 & frameshift\\
  \addlinespace
  NTHL1 & Ghs18 & stop gained\\
  OR13A1 & Ghs125 & frameshift\\
  OR2T11 & Ghs40 & frameshift\\
  OR51H1P & Ghs90 & frameshift\\
  PCSK6 & Ghs89 & stop lost\\
  \addlinespace
  PI4K2A & Ghs25 & stop gained\\
  PLA2R1 & Ghs24 & frameshift\\
  PMS1 & Ghs92 & frameshift\\
  PPFIBP2 & Ghs31 & frameshift\\
  PSG4 & Ghs33 & stop gained\\
  \addlinespace
  PSPHP1 & Ghs51 & splice donor\\
  PTGR1 & Ghs87 & splice donor\\
  RBM45 & Ghs88 & frameshift\\
  RNFT2 & Ghs107 & frameshift\\
  RP6-206I17.1 & Ghs43 & splice acceptor\\
  \addlinespace
  RRM2B & Ghs92 & frameshift\\
  RYK & Ghs128 & frameshift\\
  SDHAP2 & Ghs87 & splice donor\\
  SDSL & Ghs92 & frameshift\\
  SERPINA10 & Ghs97 & stop gained\\
  \addlinespace
  SIGLECL1 & Ghs23 & frameshift\\
  SIM2 & Ghs42 & frameshift\\
  SLC2A8 & Ghs96 & stop gained\\
  SLC2A8 & Ghs96 & stop gained\&splice region\\
  SLC35E1 & Ghs40 & stop gained\\
  \addlinespace
  SP100 & Ghs86 & frameshift\\
  SPEF2 & Ghs131 & frameshift\\
  TCP10L & Ghs88 & stop gained\\
  TMEM59 & Ghs128 & frameshift\\
  TRAP1 & Ghs90 & stop gained\\
  \addlinespace
  TTLL2 & Ghs87 & splice donor\\
  TUBB8P7 & Ghs43 & frameshift\\
  UNC93A & Ghs107 & stop gained\\
  ZNF438 & Ghs87 & frameshift\\
  ZNF528 & Ghs23 & frameshift\\
  ZNF862 & Ghs38 & splice donor\\
  \bottomrule
  \end{longtable}
  
  \begin{longtable}[t]{lllll}
  \caption{\label{tab:tabgrp3low}Liste des patients portantau moins deux variants hétérozygotes sur un des gènes suivant : TODOOOOOO}\\
  \toprule
  Gene & Patient & Consequence & SIFT & PolyPhen\\
  \midrule
  C6orf118 & Ghs27 & missense & tolerated low confidence & probably damaging\\
  C6orf118 & Ghs27 & missense & tolerated & possibly damaging\\
  \bottomrule
  \end{longtable}
  
  \newpage
  
  \subparagraph{Analyse n°4}\label{analyse-n4}
  
  \begin{center}\includegraphics{thesis_files/figure-latex/dfresumegrp4-1} \end{center}
  
  \begin{center}\includegraphics{thesis_files/figure-latex/dfresumegrp4-2} \end{center}
  
  \newpage
  
  Pour finir nous avons dans cette analyse séléctionné l'ensemble des
  variants chevauchant des gènes \textbf{absents dans la liste cilliome}
  sur lesquels \textbf{un seul} de nos patients présentaient au moins 1
  variant tronquant à l'état \textbf{homozygote}. Cela nous a permis
  d'obtenir une liste de 0 gènes différents retrouvés mutés chez 0 de nos
  patients.
  
  \begin{enumerate}
  \def\labelenumi{\arabic{enumi}.}
  \tightlist
  \item
    \textbf{Étape n°1} : En raison du grand nombre de gène ayant remplis
    ces critères\\
  \item
    \textbf{Étape n°2} : Nous avons donc cherché des patients portant au
    moins 1 variant homozygote non tronquant sur l'un de ces 0. Nous avons
    ainsi pu identifier 0
  \end{enumerate}
  
  \newpage
  
  \begin{longtable}[t]{lllll}
  \caption{\label{tab:tabgrp4moderate}Liste des patients portant un variant non troquant homozygote sur un des gènes suivant : TODOOOOOO}\\
  \toprule
  Gene & Patient & Consequence & SIFT & PolyPhen\\
  
  
  \bottomrule
  \end{longtable}
  
  \begin{verbatim}
  ## Warning in bind_rows_(x, .id): binding factor and character vector,
  ## coercing into character vector
  \end{verbatim}
  
  \begin{verbatim}
  ## Warning in bind_rows_(x, .id): binding character and factor vector,
  ## coercing into character vector
  \end{verbatim}
  
  \begin{center}\includegraphics{thesis_files/figure-latex/unnamed-chunk-19-1} \end{center}
  
  \newpage 
  
  Cette approche nous a permis de rapidement identifier \ldots{} nouveaux
  acteurs potentiels pour le phénotype MMAF impliquant \ldots{} de nos
  patients qui portaient tous au moins une mutation homozygote sur l'un de
  ces gènes. Comme précédemment nous avons ensuite cherché des éventuels
  hétérozygotes composites nous permettant ainsi mettant ainsi en évidence
  \ldots{} patients portant au moins deux variants hétérozygotes
  différents sur un de ces gènes.
  
  \begin{enumerate}
  \def\labelenumi{\arabic{enumi}.}
  \item
    \textbf{\emph{WDR52}} : Ce gène (récemment renommé \emph{CFAP44}) a
    été le premier à être identifié. En effet, malgrès une expression
    ubiquitaire (\textbf{Figure : }\ref{figb:plotexpcandidat}), ce gène
    fait partie des gènes prédit avec des preuves fortes comme étant
    impliqué dans le cilliome humain. De plus, 0 de nos patients portaient
    des variants homozygotes sur ce gène, tous ayant un effet tronquant
    sur la protéine. La recherche d'hétérozygotes composites sur ce gène
    s'est cependant révélées négatives puisqu'aucun de nos patients ne
    correspondaient aux critères (\textbf{Table : }\ref{tab:tabwdr52}).
  \item
    \textbf{\emph{EFCAB6}}, \textbf{\emph{TTC29}}, \textbf{\emph{CCDC146}}
    : Ces trois gènes ont ensuite été identifié simultanément puisque pour
    tous trois deux patients sont retrouvés avec un variant homozygote.
    Pour \emph{TTC29} les deux patients portent la même variation
    impactant un site donneur d'épissage pouvant donc altérer l'épissage
    du transcrit induisant à la production d'une protéine aberrante. Les
    deux patients \emph{CCDC146} portent chacun un variant induisant
    respectivement un codon stop prématuré et un décalage du cadre de
    lecture. Ces deux entrainant la production d'une protéine tronquée. Et
    enfin, deux variants faux-sens différents sont portés par les patients
    \emph{EFCAB6}. De plus ces trois gènes sont présents dans notre liste
    des gènes du cilliome avec de fortes preuves et les bases de données
    publiques indiquent que ces trois gènes ont une forte (et quasi
    exclusive) expression testiculaire. La recherche d'e potentiels
    hétérozygotes composites s'est en revanche avérées négatives pour les
    gènes \emph{TTC29} et \emph{CCDC146}. Cependant, un patient portait
    deux variants hétérozygotes sur \emph{EFCAB6}, l'un induisant un
    décalage du cadre de lecture et l'autre un faux-sens (\textbf{Table :
    }\ref{tab:tabefcab6ttc29ccdc146}). \newpage
  \item
    \textbf{\emph{LRRC43}} : En poursuivant nos analyses, nous avons
    identifié le gène \emph{LRRC43} sur lequel 4 portaient le même variant
    faux-sens à l'état homozygote. De plus, un autre patient portait ce
    même variant à l'état hétérozygote couplé à un second variant
    faux-sens hétérozygote lui aussi. Ce gène est également connu pour
    avoir une forte expression et exclusive testiculaire. De plus, ce gène
    est inclus dans notre liste de gène du cilliome bien qu'il soit classé
    dans la catégorie \emph{No evidence from previous studies}
    (\textbf{Table : }\ref{tab:tablrrc43}).
  \item
    \textbf{\emph{ARMC2}} : Comme \emph{LRRC43}, \emph{ARMC2} est classé
    dans la catégorie \emph{No evidence from previous studies} des gènes
    du cilliome de notre liste. Cependant son expression testiculaire
    forte et exclusive font de lui un bon candidat pour expliquer le
    phénotype MMAF de 3 portant respectivement à l'état homozygote un
    variant faux-sens, un impactant le site d'épissage et l'autre
    entrainant un décalage du cadre de lecture (\textbf{Table :
    }\ref{tab:tabarmc2}).
  \item
    \textbf{\emph{ANKRD20A3}} : Ce gène sur lequel 8 de nos patients
    portent un variant homozygote (7 d'entre eux portent le même variant
    faux-sens) et 9 au moins deux variants hétérozygotes ne fait pas parti
    de la liste des gènes du cilliome. Cependant sa forte et quasi
    exclusive expression testiculaire fait tout de même de ce gène un très
    bon candidat (\textbf{Table : }\ref{tab:tabankrd20a3}).
  \item
    \textbf{\emph{WDR96}} : Ce gène non plus ne fait pas parti de la liste
    de gène cilliome, cependant, comme \emph{ANKRD20A3} sa forte
    expression spécifique au testicule font de lui un bon candidat. Au
    sein de notre cohorte, 0 patients ont été identifiés car ils portaient
    un variant homozygote sur ce gène. Parmi eux, -1 portaient une
    mutation tronquante, le dernier portait un variant intronique proche
    de la zone d'épissage. En plus de ceux-ci, 0 portait deux variant
    hétérozygotes dont un causait un décalage du cadre de lecture l'autre
    entrainant un faux-sens (\textbf{Table : }\ref{tab:tabwdr96}).
  \end{enumerate}
  
  \newpage
  
  \begin{enumerate}
  \def\labelenumi{\arabic{enumi}.}
  \setcounter{enumi}{6}
  \tightlist
  \item
    \textbf{\emph{FSIP2}} : Bien que ce gène ne soit lui non plus pas
    inclus dans la liste des gène du cilliome, une équipe a démontré en
    2003 l'implication de ce gène dans la structure de la gaine fibreuse
    su flagelle spermatique (Brown et al.,
    \protect\hyperlink{ref-Brown2003}{2003}) faisant également de ce gène
    un excellent candidat dans l'explication du phénotype MMAF de 3
    patients portant respectivement à l'état homozygote deux indels
    différents induisant un décalage du cadre de lecture et un variant
    faux-sens. De même, les doubles ou triples variants faux-sens
    retrouvés à l'état hétérozygotes chez 4 de nos patients pourraient
    aussi expliquer leur phénotype MMAF (\textbf{Table :
    }\ref{tab:tabfsip2}.
  \end{enumerate}
  
  \newpage
  
  \begin{figure}
  
  {\centering \includegraphics{thesis_files/figure-latex/plotexpcandidat-1} 
  
  }
  
  \caption[Expression tissulaire des gènes gènes candidats retenus]{Expression tissulaire des gènes gènes candidats retenus : Données provenant du projet de transcriptome Illumina bodyMap}\label{fig:plotexpcandidat}
  \end{figure}
  
  \newpage
  
  \subsubsection{Discussion}\label{discussion-1}
  
  L'analyse de cette cohorte de 62 patients MMAF nous dans un premier
  temps permis de confirmer l'importance de l'implication du gène
  \emph{DNAH1} dans ce phénotype grâce à \ldots{} patients présentant des
  variants sur ce gène dont \ldots{} à l'état homozygote. Elle nous a
  également permis d'identifier 7 nouveaux gène candidats pouvant
  expliquer le phénotype de 29 de nos patients soit 47 \% de la cohorte.
  parmi ceux-ci, \ldots{} portaient au moins un variant homozygote sur un
  de ces gènes. Pour les autres des études sont nécessaires afin de
  déterminer si les différents variants hétérozygotes qu'ils portent sont
  situés sur leurs deux allèles différents faisant d'eux des hétérozygotes
  composites (\textbf{Figure : }\ref{fig:plotresumebigmmaf} - \textbf{A}).
  
  Parmi cet ensemble de patients, il faut noter que 5 d'entre eux porte
  des variants pouvant expliquer leur phénotype sur plusieurs des gènes
  candidats que nous avons identifiés. En effet, 5 de nos patients portent
  des variants sur deux de nos gènes candidats et 0 sur 3 d'entre eux
  (\textbf{Figure : }\ref{tab:tabresumebigmmaf} - \textbf{B}).
  
  Cependant, parmi ces différents variants certains semblent plus
  probables pour expliquer le phénotype (\textbf{Table :
  }\ref{tab:tabresumebigmmaf}) avec par exemple :
  
  \begin{enumerate}
  \def\labelenumi{\arabic{enumi}.}
  \tightlist
  \item
    Patient Ghs105 : Ce patient porte à la fois un variant homozygote
    affectant le site d'épissage du gène \emph{WDR96} ainsi que deux
    variants hétérozygotes causant tous les deux un faux-sens sur le gène
    \emph{EFCAB6}. Au vu du génotype homozygote et de l'effet tronquant du
    variant impactant \emph{WDR96}, il parait plus probable que celui-ci
    soit responsable du phénotype MMAF au détriment des deux variants
    hétérozygotes chevauchant \emph{EFCAB6}.\\
  \item
    Patient Ghs17 : Deux variants homozygotes ont été retenus pour ce
    patient. L'un causant un faux-sens sur le gène \emph{EFCAB6} l'autre
    créant un codon stop prématuré sur \emph{WDR96}. Ici aussi, au vu de
    l'impact délétère du variant chevauchant \emph{WDR96}, il parait plus
    probable que ce soit celui-ci qui soit la cause du phénotype de ce
    patient.\\
  \item
    Patients Ghs32 et Ghs35 : Ces patients portent tous deux à la fois des
    variants sur le gène \emph{ANKRD20A3} et sur le gène \emph{CCDC146}
    cependant l'effet tronquant de leur variant impactant ce dernier nous
    laisse penser que ceux-ci soient la cause du phénotype MMAF de ces
    deux patients.
  \end{enumerate}
  
  En procédant de la même manière pour les autres patients il est, dans la
  grande majorité des cas, possible de dégager un gène pour lequel
  l'implication dans le phénotype des patients parait plus évidente bien
  que des analyses complémentaires soient nécessaires.
  
  Ainsi, cette analyse révèle l'efficacité de notre pipeline puisqu'elle a
  permis d'identifier au moins un gène candidat pour 47 \% de nos
  patients. Pour les autres des analyses individuelles complémentaires
  sont nécessaires afin d'identifier la cause génétique responsable de
  leur phénotype.
  
  Une partie de ces différents résultats ont déjà été publiés dans deux
  articles dont je suis co-auteur :
  
  \begin{enumerate}
  \def\labelenumi{\arabic{enumi}.}
  \item
    \textbf{Whole exome cohort study and analysis of mouse and Trypanosoma
    models demonstrate the importance of WDR proteins in flagellogenesis
    and male fertility}, \emph{Nat Genet} (soumis) : Cette article
    présente nos différents résultats dans la caractérisation des gènes
    \emph{WDR96} et \emph{WDR52} ainsi que les différentes preuves de leur
    implication dans le phénotype MMAF.
  \item
    \protect\hyperlink{famdnah1}{\textbf{Whole-exome sequencing of
    familial cases of multiple morphological abnormalities of the sperm
    flagella (MMAF) reveals new DNAH1 mutations}} : En plus des résultats
    évoqués précédemment pour la famille MMAF2, cet article inclus ceux de
    \ldots{} patients de cette cohorte présentant des variants sur le gène
    \emph{DNAH1}
  \end{enumerate}
  
  Pour les autres, notre équipe travaille actuellement à la
  caractérisation des différents gènes afin de comprendre les processus
  moléculaires
  
  \newpage
  
  \begin{figure}
  
  {\centering \includegraphics{thesis_files/figure-latex/plotresumebigmmaf-1} 
  
  }
  
  \caption[Conclusion des analyses WES de notre large cohorte MMAF, liste des gènes candidats]{Conclusion des analyses WES de notre large cohorte MMAF, liste des gènes candidats : **A** : Quantification du nombre de patient portant un ou plusieurs variants sur un des gène candidat. La couleur des barres dépend du génotype des patients, la barre rouge indique les patients pour lesquels aucun candidat n'a été identifié. **B** : Nombre de candidat potentiel pour chaque patient (parmi ceux pour lesquels au moins un gène candidat a été identifié)}\label{fig:plotresumebigmmaf}
  \end{figure}
  
  \newpage
  
  \newpage 
  
  \section{Conclusion}\label{conclusion}
  
  Au cours de ces différentes études nous avons pu identifier les variants
  pouvant expliquer les phénotypes de \ldots{} des différents patients que
  nous avons analysé que ce soit au sein d'études familiales ou bien au
  sein de plus large cohorte composés d'individus non apparentés. Bien que
  ces résultats soient satisfaisant, il faut noter que pour \ldots{}
  patients, soit \ldots{} \% d'entre eux aucun candidat n'a pu à ce jour
  être identifié. Pour ces patients, le WES permets désormais de nouvelles
  approches permettant d'identifier de larges variants structuraux
  (insertion ou délétions) pouvant eux aussi être responsable du phénotype
  qui ne sont pas détectés par les analyses classiques. Néanmoins, il
  semble clair que des avancés soient encore nécessaires afin d'améliorer
  l'efficacité de ce genre d'étude notamment en créant de nouveaux filtres
  permettant ainsi d'épurer les listes de variants facilitant ainsi
  l'identification des gènes candidats.
  
  \newpage  
  
  \chapter{MutaScript}\label{mutascript}
  
  \chapter*{Conclusion}\label{conclusion-1}
  \addcontentsline{toc}{chapter}{Conclusion}
  
  \chapter{The First Appendix}\label{the-first-appendix}
  
  \chapter*{References}\label{references}
  \addcontentsline{toc}{chapter}{References}
  
  \hypertarget{refs}{}
  \hypertarget{ref-Adzhubei2010}{}
  Adzhubei, I. A., Schmidt, S., Peshkin, L., Ramensky, V. E., Gerasimova,
  A., Bork, P., \ldots{} Sunyaev, S. R. (2010). A method and server for
  predicting damaging missense mutations. \emph{Nature Methods},
  \emph{7}(4), 248--9. \url{http://doi.org/10.1038/nmeth0410-248}
  
  \hypertarget{ref-Baker2004}{}
  Baker, K. E., \& Parker, R. (2004). Nonsense-mediated mRNA decay:
  terminating erroneous gene expression. \emph{Current Opinion in Cell
  Biology}, \emph{16}(3), 293--9.
  \url{http://doi.org/10.1016/j.ceb.2004.03.003}
  
  \hypertarget{ref-BenKhelifa2014}{}
  Ben Khelifa, M., Coutton, C., Zouari, R., Karaouzène, T., Rendu, J.,
  Bidart, M., \ldots{} Ray, P. F. (2014). Mutations in DNAH1, which
  encodes an inner arm heavy chain dynein, lead to male infertility from
  multiple morphological abnormalities of the sperm flagella.
  \emph{American Journal of Human Genetics}, \emph{94}(1), 95--104.
  \url{http://doi.org/10.1016/j.ajhg.2013.11.017}
  
  \hypertarget{ref-Brown2003}{}
  Brown, P. R., Miki, K., Harper, D. B., \& Eddy, E. M. (2003). A-Kinase
  Anchoring Protein 4 Binding Proteins in the Fibrous Sheath of the Sperm
  Flagellum. \emph{Biology of Reproduction}, \emph{68}(6), 2241--2248.
  \url{http://doi.org/10.1095/biolreprod.102.013466}
  
  \hypertarget{ref-Chang2007}{}
  Chang, Y.-F., Imam, J. S., \& Wilkinson, M. F. (2007). The
  Nonsense-Mediated Decay RNA Surveillance Pathway. \emph{Annual Review of
  Biochemistry}, \emph{76}(1), 51--74.
  \url{http://doi.org/10.1146/annurev.biochem.76.050106.093909}
  
  \hypertarget{ref-DePristo2011}{}
  DePristo, M. A., Banks, E., Poplin, R., Garimella, K. V., Maguire, J.
  R., Hartl, C., \ldots{} Pritchard, E. (2011). A framework for variation
  discovery and genotyping using next-generation DNA sequencing data.
  \emph{Nature Genetics}, \emph{43}(5), 491--498.
  \url{http://doi.org/10.1038/ng.806}
  
  \hypertarget{ref-Firat-Karalar2014}{}
  Firat-Karalar, E. N., Sante, J., Elliott, S., \& Stearns, T. (2014).
  Proteomic analysis of mammalian sperm cells identifies new components of
  the centrosome. \emph{Journal of Cell Science}, \emph{127}(Pt 19),
  4128--33. \url{http://doi.org/10.1242/jcs.157008}
  
  \hypertarget{ref-Hong2017}{}
  Hong, E. P., Go, M. J., Kim, H.-L., \& Park, J. W. (2017). Risk
  prediction of pulmonary tuberculosis using genetic and conventional risk
  factors in adult Korean population. \emph{PloS One}, \emph{12}(3),
  e0174642. \url{http://doi.org/10.1371/journal.pone.0174642}
  
  \hypertarget{ref-Hu2011}{}
  Hu, Y., Yu, H., Shaw, G., Renfree, M. B., \& Pask, A. J. (2011).
  Differential roles of TGIF family genes in mammalian reproduction.
  \emph{BMC Developmental Biology}, \emph{11}, 58.
  \url{http://doi.org/10.1186/1471-213X-11-58}
  
  \hypertarget{ref-Imai2015}{}
  Imai, Y., Morita, H., Takeda, N., Miya, F., Hyodo, H., Fujita, D.,
  \ldots{} Komuro, I. (2015). A deletion mutation in myosin heavy chain 11
  causing familial thoracic aortic dissection in two Japanese pedigrees.
  \emph{International Journal of Cardiology}, \emph{195}, 290--292.
  \url{http://doi.org/10.1016/j.ijcard.2015.05.178}
  
  \hypertarget{ref-Ivliev2012}{}
  Ivliev, A. E., 't Hoen, P. A. C., Roon-Mom, W. M. C. van, Peters, D. J.
  M., \& Sergeeva, M. G. (2012). Exploring the Transcriptome of Ciliated
  Cells Using In Silico Dissection of Human Tissues. \emph{PLoS ONE},
  \emph{7}(4), e35618. \url{http://doi.org/10.1371/journal.pone.0035618}
  
  \hypertarget{ref-Kumar2009}{}
  Kumar, P., Henikoff, S., \& Ng, P. C. (2009). Predicting the effects of
  coding non-synonymous variants on protein function using the SIFT
  algorithm. \emph{Nature Protocols}, \emph{4}(7), 1073--1081.
  \url{http://doi.org/10.1038/nprot.2009.86}
  
  \hypertarget{ref-Lee2011}{}
  Lee, B., Park, I., Jin, S., Choi, H., Kwon, J. T., Kim, J., \ldots{}
  Cho, C. (2011). Impaired spermatogenesis and fertility in mice carrying
  a mutation in the Spink2 gene expressed predominantly in testes.
  \emph{The Journal of Biological Chemistry}, \emph{286}(33), 29108--17.
  \url{http://doi.org/10.1074/jbc.M111.244905}
  
  \hypertarget{ref-Lek2016}{}
  Lek, M., Karczewski, K. J., Minikel, E. V., Samocha, K. E., Banks, E.,
  Fennell, T., \ldots{} Exome Aggregation Consortium, D. G. (2016).
  Analysis of protein-coding genetic variation in 60,706 humans.
  \emph{Nature}, \emph{536}(7616), 285--91.
  \url{http://doi.org/10.1038/nature19057}
  
  \hypertarget{ref-Lunter2011}{}
  Lunter, G., \& Goodson, M. (2011). Stampy: A statistical algorithm for
  sensitive and fast mapping of Illumina sequence reads. \emph{Genome
  Research}, \emph{21}(6), 936--939.
  \url{http://doi.org/10.1101/gr.111120.110}
  
  \hypertarget{ref-McLaren2016}{}
  McLaren, W., Gil, L., Hunt, S. E., Riat, H. S., Ritchie, G. R. S.,
  Thormann, A., \ldots{} Cunningham, F. (2016). The Ensembl Variant Effect
  Predictor. \emph{Genome Biology}, \emph{17}(1), 122.
  \url{http://doi.org/10.1186/s13059-016-0974-4}
  
  \hypertarget{ref-Nielsen2011}{}
  Nielsen, R., Paul, J. S., Albrechtsen, A., \& Song, Y. S. (2011).
  Genotype and SNP calling from next-generation sequencing data.
  \emph{Nature Reviews. Genetics}, \emph{12}(6), 443--51.
  \url{http://doi.org/10.1038/nrg2986}
  
  \hypertarget{ref-Su2014}{}
  Su, Z., Łabaj, P. P., Li, S. S., Thierry-Mieg, J., Thierry-Mieg, D.,
  Shi, W., \ldots{} Shi, L. (2014). A comprehensive assessment of RNA-seq
  accuracy, reproducibility and information content by the Sequencing
  Quality Control Consortium. \emph{Nature Biotechnology}, \emph{32}(9),
  903--14. \url{http://doi.org/10.1038/nbt.2957}
  
  \hypertarget{ref-Zheng2006}{}
  Zheng, Y., Zhang, J., Wang, L., Zhou, Z., Xu, M., Li, J., \& Sha, J.-H.
  (2006). Cloning and characterization of a novel sperm tail protein,
  NYD-SP28. \emph{International Journal of Molecular Medicine},
  \emph{18}(6), 1119--25. Retrieved from
  \url{http://www.ncbi.nlm.nih.gov/pubmed/17089017}


  % Index?

\end{document}

