% This is the Reed College LaTeX thesis template. Most of the work
% for the document class was done by Sam Noble (SN), as well as this
% template. Later comments etc. by Ben Salzberg (BTS). Additional
% restructuring and APA support by Jess Youngberg (JY).
% Your comments and suggestions are more than welcome; please email
% them to cus@reed.edu
%
% See http://web.reed.edu/cis/help/latex.html for help. There are a
% great bunch of help pages there, with notes on
% getting started, bibtex, etc. Go there and read it if you're not
% already familiar with LaTeX.
%
% Any line that starts with a percent symbol is a comment.
% They won't show up in the document, and are useful for notes
% to yourself and explaining commands.
% Commenting also removes a line from the document;
% very handy for troubleshooting problems. -BTS

% As far as I know, this follows the requirements laid out in
% the 2002-2003 Senior Handbook. Ask a librarian to check the
% document before binding. -SN

%%
%% Preamble
%%
% \documentclass{<something>} must begin each LaTeX document
\documentclass[12pt,twoside]{ugathesis}
% Packages are extensions to the basic LaTeX functions. Whatever you
% want to typeset, there is probably a package out there for it.
% Chemistry (chemtex), screenplays, you name it.
% Check out CTAN to see: http://www.ctan.org/
%%
\usepackage{graphicx,latexsym}
\usepackage[french]{babel}
\usepackage{amsmath}
\usepackage{amssymb,amsthm}
\usepackage[dvipsnames]{xcolor} % tk: for more color
\usepackage{xcolor}
\usepackage{eso-pic}
\usepackage{longtable,booktabs,setspace}
\usepackage{chemarr} %% Useful for one reaction arrow, useless if you're not a chem major
\usepackage[hyphens]{url}
\usepackage{pdfpages}
\usepackage{tikz}
\usetikzlibrary{calc}
% Added by CII
\usepackage{hyperref}
\usepackage{lmodern}
\usepackage{float}
\floatplacement{figure}{H}
% End of CII addition
\usepackage{rotating}
\usepackage{upgreek} % tk : pour pouvoir utiliser le symbole µ droit (pas en itallic)
\usepackage{pdfpages} % tk : pour pouvoir insérer des fichiers pdf dans le corp de texte
\usepackage{lscape} % tk : pour pouvoir insérer des images au format paysage
\newcommand{\blandscape}{\begin{landscape}}
\newcommand{\elandscape}{\end{landscape}}
\usepackage[utf8]{inputenc}

% Next line commented out by CII
%%% \usepackage{natbib}
% Comment out the natbib line above and uncomment the following two lines to use the new
% biblatex-chicago style, for Chicago A. Also make some changes at the end where the
% bibliography is included.
%\usepackage{biblatex-chicago}
%\bibliography{thesis}


% Added by CII (Thanks, Hadley!)
% Use ref for internal links
\renewcommand{\hyperref}[2][???]{\autoref{#1}}
\def\chapterautorefname{Chapter}
\def\sectionautorefname{Section}
\def\subsectionautorefname{Subsection}
% End of CII addition

% Added by CII
\usepackage{caption}
\captionsetup{width=5in}
% End of CII addition

% \usepackage{times} % other fonts are available like times, bookman, charter, palatino


% To pass between YAML and LaTeX the dollar signs are added by CII
\title{THÈSE}
\author{Thomas Karaouzene}
\lab{Génétique, Epigénétique et Thérapies de l'Infertilité (GETI) et
Techniques de l'Ingénierie Médicale et de la Complexité - Informatique,
Mathématiques et Applications de Grenoble (TIMC-IMAG)}
\date{07 novembre 2017}
\division{Mathematics and Natural Sciences}
\advisor{Pierre Ray}
%If you have two advisors for some reason, you can use the following
% Uncommented out by CII
\altadvisor{Nicolas Thierry-Mieg}
% End of CII addition
%\institution{}
%\degree{}

%%% Remember to use the correct department!
\department{Ingénierie de la Santé, de la Cognition et Environnement (EDISCE)}
% if you're writing a thesis in an interdisciplinary major,
% uncomment the line below and change the text as appropriate.
% check the Senior Handbook if unsure.
%\thedivisionof{The Established Interdisciplinary Committee for}
% if you want the approval page to say "Approved for the Committee",
% uncomment the next line
%\approvedforthe{Committee}

% Added by CII
%%% Copied from knitr
%% maxwidth is the original width if it's less than linewidth
%% otherwise use linewidth (to make sure the graphics do not exceed the margin)
\makeatletter
\def\maxwidth{ %
  \ifdim\Gin@nat@width>\linewidth
    \linewidth
  \else
    \Gin@nat@width
  \fi
}
\makeatother

\renewcommand{\contentsname}{Table of Contents}
% End of CII addition

\setlength{\parskip}{0pt}

% Added by CII
  %\setlength{\parskip}{\baselineskip}
  \usepackage[parfill]{parskip}

\providecommand{\tightlist}{%
  \setlength{\itemsep}{0pt}\setlength{\parskip}{0pt}}

\Acknowledgements{

}

\Dedication{

}

\Preface{

}

\Abstract{

}

	\usepackage{tikz}
% End of CII addition
%%
%% End Preamble
%%
%

\begin{document}

% Everything below added by CII
  \maketitle

\frontmatter % this stuff will be roman-numbered
\pagestyle{empty} % this removes page numbers from the frontmatter



  \hypersetup{linkcolor=black}
  \setcounter{tocdepth}{3}
  \tableofcontents

  \listoftables

  \listoffigures



\mainmatter % here the regular arabic numbering starts
\pagestyle{fancyplain} % turns page numbering back on

\chapter{Mise en place d'une stratégie pour l'analyse des données
exomiques -- application en recherche
clinique}\label{mise-en-place-dune-strategie-pour-lanalyse-des-donnees-exomiques-application-en-recherche-clinique}

\newpage

En 2011, les bases moléculaires d'environ 3700 pathologies à
transmission mendélienne avaient été élucidées. Cependant, pour une
quantité équivalente de pathologies Mendéliennes (ou suspectées de
l'être) cette cause reste un mystère
{[}\protect\hyperlink{ref-Amberger2011}{1}{]}. Avec plusieurs centaines
de pathologies caractérisées depuis 2010
{[}\protect\hyperlink{ref-Ng}{2}{]}, les séquençages WGS et WES ont,
depuis leur émergence, révolutionnés les méthodes de recherche dans le
cadre d'étude phénotype-génotype en permettant de manière rapide et à
moindre cout le séquençage de la quasi-totalité des gènes humains. Dès
lors, le défi de ces analyses n'est plus le séquençage de l'ADN mais
l'interprétation des données massives produites. En effet, l'un des plus
grands challenges des analyses phénotype-génotype réalisées par WES
réside dans l'analyse de l'importante quantité de variants portée par
chaque individu s'élevant à plusieurs dizaines de milliers lorsque l'on
compare avec le génome de référence. Même après avoir retiré les
variants retrouvés fréquemment dans la population générale, des méthodes
additionnelles sont nécessaires pour prédire, parmi les variants
restant, lesquels induisent des conséquences fonctionnelles sérieuses
afin de les prioriser {[}\protect\hyperlink{ref-Pelak2010}{3}{]}. De
nombreux logiciels tel que Variant Effect Predictor
{[}\protect\hyperlink{ref-McLaren2016}{4}{]}, SnpEff
{[}\protect\hyperlink{ref-Cingolani2012}{5}{]} ou encore ANNOVAR
{[}\protect\hyperlink{ref-Wang2010}{6}{]} permettent d'identifier quels
sont les variants qui ont un effet tronquant sur la protéine. Cependant,
avec en moyenne 165 variants homozygotes ayant un effet tronquant
retrouvés dans chaque exome {[}\protect\hyperlink{ref-Pelak2010}{3}{]}
ces méthodes, bien qu'efficaces sont souvent insuffisantes.

D'autres logiciels tel que Exomiser
{[}\protect\hyperlink{ref-Robinson2014}{7}{]} vont, à partir d'une liste
de variants déjà appelés effectuer les étapes d'annotation, de filtrage
et de priorisation. Malgré l'efficacité de ces logiciels, aucun d'entre
eux ne couvre l'ensemble des étapes allant de l'alignement des
\emph{reads} à la priorisation des variants. La plupart ayant pour point
de départ une liste de variants appelés en amont. Ils ne contrôlent donc
en aucune manière les étapes d'alignement et d'appel des variants. Or,
comme il a été dit plus tôt, ces deux étapes constituent la base de
l'analyse.

Ce chapitre décrit à la fois la constitution d'un pipeline d'analyse des
données de séquençage exomique recouvrant l'ensemble des étapes allant
de l'alignement des séquences à la priorisation des variants ainsi que
son utilisation dans le cadre de la recherche de mutations entrainant
différents phénotypes d'infertilité d'une part au sein de de cas
familiaux et, pour finir, au sein d'une large cohorte d'individus non
apparentés présentant tous un phénotype MMAF.

\newpage

\section{Méthode : Description du
pipeline}\label{methode-description-du-pipeline}

\subsection{\texorpdfstring{L'alignement des
\emph{reads}}{L'alignement des reads}}\label{lalignement-des-reads}

Comme expliqué plus tôt, l'étape d'alignement a pour objectif de
repositionner l'ensemble des \emph{reads} d'un individu le long d'un
génome de référence. Cette étape peut ainsi être comparée à la
reconstruction d'un puzzle dans lequel chaque \emph{read} peut être
assimilé à l'une des pièces tandis que le génome de référence serait ici
le modèle (\textbf{Figure : }\ref{fig:picdnamapping}).

L'ensemble de nos exomes ayant été réalisé en \emph{paired-end}, les
deux extrémités de chaque fragment sont séquencées. Chaque \emph{end}
d'un même \emph{read} peut donc être considérée comme un \emph{read} à
part entière qui est alignée \textbf{indépendamment} le long du génome
de référence. L'information fournie par le \emph{paired-end} n'étant
utilisée qu'à \emph{posteriori} en tant que critère qualité. Au sein de
notre pipeline, cette étape est effectuée par le logiciel MAGIC
{[}\protect\hyperlink{ref-Su2014}{8}{]}, qui dans le cadre de nos
études, s'est basé sur la version hg19 / GHRC37 du génome de référence.
Suite à cet alignement, plusieurs critères sont observés afin de filtrer
les \emph{reads} présentant une faible qualité d'alignement.

Ainsi, le premier de ces filtres consiste à tout d'abord filtrer
l'ensemble des \emph{reads} dupliqués, c'est à dire les \emph{reads}
ayant des séquences parfaitement identiques, ceux-ci étant souvent le
résultat d'un excès d'amplification au moment des PCRs effectuées en
amont. De la même manière, afin d'éviter toute ambiguïté au moment de
l'interprétation des résultats, l'ensemble des \emph{reads} s'étant
alignés sur plusieurs régions du génome est aussi filtré. Une fois cela
fait, nous vérifions la ``compatibilité'' des deux \emph{ends}
composants chacun des \emph{reads} restant. Un \emph{read} est dit
compatible lorsque les deux \emph{ends} qui le composent s'alignent face
à face (une sur le brin sens du génome de référence et l'autre sur le
brin anti-sens) et couvrent une zone ne faisant pas plus de 3 fois la
taille médiane de l'insert. Les \emph{reads} dont les deux \emph{ends}
se sont alignées mais ne remplissant pas ces conditions seront dit ``non
compatible'', ceux dont une seule des deux \emph{ends} s'est alignée
seront appelés ``orphelins'' et enfin ceux pour lesquels aucune des deux
\emph{ends} ne se sont alignées sont appelés ``non-aligné''. L'ensemble
des \emph{reads} ``non-compatibles'', ``orphelins'' et ``non-alignés''
sont, en raison de leur faible qualité, filtrés et donc non considérés
pour les analyses en aval. Les \emph{reads} ayant passé l'ensemble des
critères qualité mentionnés précédemment seront, eux, utilisés pour
effectuer l'appel des variants.

\newpage

\subsection{L'appel des variants}\label{lappel-des-variants}

Si l'alignement des séquences peut être comparé à la reconstruction d'un
puzzle, l'appel des variants pourrait lui être vu comme un jeu des 7
erreurs, au cours duquel, pour chaque position couverte, les différences
entre la séquence de l'individu séquencé et le génome de référence
seront listées et appelées variants. Comme nous l'avons vu plus
\protect\hyperlink{varcall}{tôt}, il est fortement conseillé d'effectuer
l'appel des variants en tenant compte de l'aligneur choisi
{[}\protect\hyperlink{ref-Nielsen2011}{9}--\protect\hyperlink{ref-Lunter2011}{11}{]}.
C'est pourquoi, nous avons développé notre propre algorithme d'appel des
variants spécialement conçu pour l'analyse des données de MAGIC. Ainsi,
l'appel des variants sera directement basé sur quatre comptages
(R\(_+\), R\(_-\), V\(_+\) et V\(_-\)) fournis directement par MAGIC
pour chaque position suffisement couverte :

\begin{enumerate}
\def\labelenumi{\arabic{enumi}.}
\tightlist
\item
  \textbf{R}\(_+\) \textbf{et R}\(_-\) : Ces deux comptages
  correspondent au nombre de \emph{reads} \emph{forward} (+) et
  \emph{reverse} (-) sur lesquels est observé l'allèle de
  \textbf{référence} (R) à une position donnée.\\
\item
  \textbf{V}\(_+\) \textbf{et V}\(_-\) : À l'inverse de R\(_+\) et
  R\(_-\), ces comptages correspondent au nombre de \emph{reads}
  \emph{forward} et \emph{reverse} sur lesquels est observé un allèle
  \textbf{variant} (V) à une position donnée.
\end{enumerate}

Ainsi, les sommes : \(R_+ + V_+\) et \(R_- + V_-\) indiqueront
respectivement la couverture d'une position en ne tenant que des
\emph{reads forward} et \emph{reverse}. En fonction de ces couvertures
nos appels seront classés en trois catégories :

\begin{enumerate}
\def\labelenumi{\arabic{enumi}.}
\item
  \textbf{Les appels \emph{double strand} (DS)} : Qualifient les
  positions ayant une couverture \(\ge\) 10 sur \textbf{les deux}
  \emph{strands}. Ces appels sont ceux sont ceux ayant la meilleure
  qualité.
\item
  \textbf{Les appels \emph{single strand} (SS)} : Ces appels définissent
  les positions pour lesquels \textbf{un des deux} \emph{strands}
  présente une couverture \(\le\) 10. Dans ce cas, ce \emph{strand} est
  ignoré et l'appel est effectué uniquement en utilisant le second
  \emph{strand}.
\item
  \textbf{Les appels \emph{non strand} (NS)} : Les positions NS sont
  celles pour lesquelles la couverture est \(\le\) 10 sur \textbf{les
  deux} \emph{strands}. Aucun appel n'est effectué à ces positions qui
  \textbf{ne sont pas conservés dans la suite des analyses}.
\end{enumerate}

Ensuite, pour chaque position couverte, des appels indépendants seront
effectués pour chaque \emph{strand} de telle sorte que, pour chacune de
ces positions si :

\begin{enumerate}
\def\labelenumi{\arabic{enumi}.}
\item
  0 à 20\% des \emph{reads} portent un variant, la position est appelée
  \textbf{homozygote référence}.
\item
  20 à 40\% des \emph{reads} portent un variant, l'appel sera considéré
  comme \textbf{ambigu bas}. Cette région est elle-même subdivisée en
  deux sous région, 20 à 30\% et 30 à 40\%.
\item
  40 à 75\% des \emph{reads} portent un variant, la position est appelée
  \textbf{hétérozygote}.
\item
  75 à 85\% des \emph{reads} portent un variant, l'appel sera considéré
  comme \textbf{ambigu haut}. Cette région est elle-même subdivisée en
  deux sous région, 75 à 80\% et 80 à 85\%.
\item
  85 à 100\% des \emph{reads} portent un variant, la position est
  appelée \textbf{homozygote variant}.
\end{enumerate}

Pour les positions DS, la concordance des appels fournis par chaque
\emph{end} est ensuite vérifiée. Cette vérification de la concordance
des appels entre les \emph{reads forward} et \emph{reverse} a pour
principal intérêt de filtrer les erreurs systématiques pouvant survenir
lors du processus de séquençage. Par exemple, les séquenceurs Illumina
vont avoir tendance à ``se tromper'' à la position T des motifs GGT
{[}\protect\hyperlink{ref-Robinson2011}{12}{]}. De fait, cette erreur
devrait \emph{a priori} se produire uniquement lors du séquençage dans
un seul des deux sens, celui contenant ce motif. Les \emph{reads}
alignés sur le brin complémentaire contiendront dès lors la séquence
correcte. C'est pourquoi, afin de limiter les erreurs d'appels au
maximum, nous effectuons, pour chaque position DS, des appels
indépendants sur les deux sens. Ainsi, un variant sera considéré
(\textbf{Figure : }\ref{fig:picvarcall}) :

\begin{enumerate}
\def\labelenumi{\arabic{enumi}.}
\item
  \textbf{Homozygote référence} si les deux appels sont homozygotes
  références, ou, un des appels est homozygote référence et l'autre se
  situe dans la sous-région 20-30\% de la région ambigu bas.
\item
  \textbf{Hétérozygote} si les deux appels sont hétérozygotes, ou, si
  l'un des appels est hétérozygote et l'autre se situe dans la
  sous-région 30-40\% de la région ambigu bas ou bien dans la
  sous-région 75-80\% de la région ambigu haut.
\item
  \textbf{Homozygote variant} si les deux appels sont homozygotes
  variants, ou, un des appels est homozygote variant et l'autre dans la
  sous-région 75-85\% de la région ambigu haut
\item
  \textbf{Ambigu} si les deux appels sont ambigus bas ou si ils sont
  tous les deux ambigus haut.
\item
  \textbf{Discordant} pour toutes les combinaisons restantes.
\end{enumerate}

Pour les positions SS, l'appel final correspondra directement à l'appel
effectué sur l'unique \emph{strand} suffisamment couvert. Ces variant,
bien que conservés seront, en raison des erreurs dont ils peuvent être
la source, considérés comme de faible qualité. Les appels ambigus et
discordants seront, eux, filtrés.

\newpage

\begin{figure}

{\centering \includegraphics[scale=0.5]{figure/var_call} 

}

\caption[Détail de l'appel général effectué pour les appels DS]{\textbf{\emph{Détail de l'appel général effectué pour
les appels DS}} : Illustration de l'appel des génotypes effectué en
fonction du pourcentage de \emph{reads} variants observés sur chacun des
deux \emph{strand} de séquençage (\emph{forward} et \emph{reverse}). Le
génotype général est appelé si les génotypes des deux brins sont les
mêmes ou si l'un des deux est dans la zone ambiguë adjacente au premier.
Les zones vertes correspondent aux appels homozygotes références, les
zones orange, hétérozygotes et rouges homozygotes variants. Les zones
grises sont les appels ambigus tandis que les noires sont les appels
discordants. Ces deux derniers appels ne sont pas conservés dans la
suite des analyses.}\label{fig:picvarcall}
\end{figure}













\newpage

\subsection{L'annotation}\label{lannotation}

Chaque variant retenu sera ensuite annoté tout d'abord par le logiciel
\emph{variant effect predictor} (VEP)
{[}\protect\hyperlink{ref-McLaren2016}{4}{]} qui nous indiquera pour
chaque variant la conséquence que celui-ci aura sur la séquence codante
de l'ensemble des transcrits Ensembl qu'il chevauche (\textbf{Figure :
}\ref{fig:pictvepcsq}, \textbf{Table : }\ref{tab:tabvepcsq}). Dans le
cas d'une substitution faux-sens, c'est à dire entrainant le changement
d'un seul acide-aminé de la séquence protéique, nous utilisons les
prédictions fournies par SIFT et PolyPhen afin d'estimer leur
pathogénicité. Ensuite, nous ajoutons, pour chaque gène, son expression
tissulaire en nous basant sur les données Ensembl
{[}\protect\hyperlink{ref-Aken2017}{13}{]} générées par le projet
Illumina BodyMap qui recense les données RNAseq des gènes humains pour
16 tissus différents. Suite à cela nous ajoutons, lorsque celle-ci est
disponible, la fréquence du variant dans les bases de données ExAC
{[}\protect\hyperlink{ref-Lek2016}{14}{]}, ESP600
(\href{http://evs.gs.washington.edu/EVS/}{Exome Variant Server, NHLBI GO
Exome Sequencing Project (ESP), Seattle, WA}) et 1000Genomes
{[}\protect\hyperlink{ref-1000GenomesProjectConsortium2015}{15}{]}
donnant ainsi une estimation de sa fréquence dans la population
générale. De même, la particularité de ce pipeline est que chaque
variant qu'elle a identifié alimente une base de données interne pouvant
par la suite servir de contrôle lors de l'analyse d'individus présentant
un phénotype différent de ceux étudiés précédemment. L'intérêt d'une
telle base, par rapport aux bases de données tel que ExAC, est qu'elle
permet d'utiliser comme contrôle des individus ayant subi le même
protocole de séquençage et la même analyse bioinformatique permettant
ainsi de mieux identifier, et donc filtrer, les erreurs systématiques
pouvant arriver à chacune des étapes.

\begin{figure}

{\centering \includegraphics[scale=.9]{figure/vep_csq} 

}

\caption[Listes des différentes conséquences prédites par VEP et leur positionnement sur le transcrit]{\textbf{\emph{Listes des différentes conséquences
prédites par VEP et leur positionnement sur le transcrit} d'après :
\href{http://www.ensembl.org/info/genome/variation/consequences.jpg}{\emph{Variant
Effect Predictor web site}}}}\label{fig:pictvepcsq}
\end{figure}






\newpage

\blandscape

\begin{longtable}[]{@{}lll@{}}
\caption{\label{tab:tabvepcsq} Liste simplifiée des conséquences prédites
par VEP avec leur description et impact associée}\tabularnewline
\toprule
\begin{minipage}[b]{0.17\columnwidth}\raggedright\strut
VEP consequence\strut
\end{minipage} & \begin{minipage}[b]{0.13\columnwidth}\raggedright\strut
VEP impact\strut
\end{minipage} & \begin{minipage}[b]{0.62\columnwidth}\raggedright\strut
Description\strut
\end{minipage}\tabularnewline
\midrule
\endfirsthead
\toprule
\begin{minipage}[b]{0.17\columnwidth}\raggedright\strut
VEP consequence\strut
\end{minipage} & \begin{minipage}[b]{0.13\columnwidth}\raggedright\strut
VEP impact\strut
\end{minipage} & \begin{minipage}[b]{0.62\columnwidth}\raggedright\strut
Description\strut
\end{minipage}\tabularnewline
\midrule
\endhead
\begin{minipage}[t]{0.17\columnwidth}\raggedright\strut
Splice acceptor / donor\strut
\end{minipage} & \begin{minipage}[t]{0.13\columnwidth}\raggedright\strut
HIGH\strut
\end{minipage} & \begin{minipage}[t]{0.62\columnwidth}\raggedright\strut
A splice variant that changes the 2 base region at the 3' / 5' end of an
intron\strut
\end{minipage}\tabularnewline
\begin{minipage}[t]{0.17\columnwidth}\raggedright\strut
Stop gained\strut
\end{minipage} & \begin{minipage}[t]{0.13\columnwidth}\raggedright\strut
HIGH\strut
\end{minipage} & \begin{minipage}[t]{0.62\columnwidth}\raggedright\strut
A sequence variant whereby at least one base of a codon is changed,
resulting in a premature stop codon, leading to a shortened
transcript\strut
\end{minipage}\tabularnewline
\begin{minipage}[t]{0.17\columnwidth}\raggedright\strut
Frameshift\strut
\end{minipage} & \begin{minipage}[t]{0.13\columnwidth}\raggedright\strut
HIGH\strut
\end{minipage} & \begin{minipage}[t]{0.62\columnwidth}\raggedright\strut
A sequence variant which causes a disruption of the translational
reading frame, because the number of nucleotides inserted or deleted is
not a multiple of three\strut
\end{minipage}\tabularnewline
\begin{minipage}[t]{0.17\columnwidth}\raggedright\strut
Stop lost\strut
\end{minipage} & \begin{minipage}[t]{0.13\columnwidth}\raggedright\strut
HIGH\strut
\end{minipage} & \begin{minipage}[t]{0.62\columnwidth}\raggedright\strut
A sequence variant where at least one base of the terminator codon
(stop) is changed, resulting in an elongated transcript\strut
\end{minipage}\tabularnewline
\begin{minipage}[t]{0.17\columnwidth}\raggedright\strut
Start lost\strut
\end{minipage} & \begin{minipage}[t]{0.13\columnwidth}\raggedright\strut
HIGH\strut
\end{minipage} & \begin{minipage}[t]{0.62\columnwidth}\raggedright\strut
A codon variant that changes at least one base of the canonical start
codo\strut
\end{minipage}\tabularnewline
\begin{minipage}[t]{0.17\columnwidth}\raggedright\strut
Inframe insertion / deletion\strut
\end{minipage} & \begin{minipage}[t]{0.13\columnwidth}\raggedright\strut
MODERATE\strut
\end{minipage} & \begin{minipage}[t]{0.62\columnwidth}\raggedright\strut
An inframe non synonymous variant that inserts / deletes bases into in
the coding sequenc\strut
\end{minipage}\tabularnewline
\begin{minipage}[t]{0.17\columnwidth}\raggedright\strut
Missense\strut
\end{minipage} & \begin{minipage}[t]{0.13\columnwidth}\raggedright\strut
MODERATE\strut
\end{minipage} & \begin{minipage}[t]{0.62\columnwidth}\raggedright\strut
A sequence variant, that changes one or more bases, resulting in a
different amino acid sequence but where the length is preserved\strut
\end{minipage}\tabularnewline
\begin{minipage}[t]{0.17\columnwidth}\raggedright\strut
Splice region\strut
\end{minipage} & \begin{minipage}[t]{0.13\columnwidth}\raggedright\strut
LOW\strut
\end{minipage} & \begin{minipage}[t]{0.62\columnwidth}\raggedright\strut
A sequence variant in which a change has occurred within the region of
the splice site, either within 1-3 bases of the exon or 3-8 bases of the
intron\strut
\end{minipage}\tabularnewline
\begin{minipage}[t]{0.17\columnwidth}\raggedright\strut
Stop retained\strut
\end{minipage} & \begin{minipage}[t]{0.13\columnwidth}\raggedright\strut
LOW\strut
\end{minipage} & \begin{minipage}[t]{0.62\columnwidth}\raggedright\strut
A sequence variant where at least one base in the terminator codon is
changed, but the terminator remains\strut
\end{minipage}\tabularnewline
\begin{minipage}[t]{0.17\columnwidth}\raggedright\strut
Synonymous\strut
\end{minipage} & \begin{minipage}[t]{0.13\columnwidth}\raggedright\strut
LOW\strut
\end{minipage} & \begin{minipage}[t]{0.62\columnwidth}\raggedright\strut
A sequence variant where there is no resulting change to the encoded
amino acid\strut
\end{minipage}\tabularnewline
\begin{minipage}[t]{0.17\columnwidth}\raggedright\strut
5 / 3 prime UTR\strut
\end{minipage} & \begin{minipage}[t]{0.13\columnwidth}\raggedright\strut
MODIFIER\strut
\end{minipage} & \begin{minipage}[t]{0.62\columnwidth}\raggedright\strut
A UTR variant of the 5' / 3' UTR\strut
\end{minipage}\tabularnewline
\begin{minipage}[t]{0.17\columnwidth}\raggedright\strut
Intron\strut
\end{minipage} & \begin{minipage}[t]{0.13\columnwidth}\raggedright\strut
MODIFIER\strut
\end{minipage} & \begin{minipage}[t]{0.62\columnwidth}\raggedright\strut
A transcript variant occurring within an intron\strut
\end{minipage}\tabularnewline
\begin{minipage}[t]{0.17\columnwidth}\raggedright\strut
NMD transcript\strut
\end{minipage} & \begin{minipage}[t]{0.13\columnwidth}\raggedright\strut
MODIFIER\strut
\end{minipage} & \begin{minipage}[t]{0.62\columnwidth}\raggedright\strut
A variant in a transcript that is the target of NMD\strut
\end{minipage}\tabularnewline
\begin{minipage}[t]{0.17\columnwidth}\raggedright\strut
Non coding transcript\strut
\end{minipage} & \begin{minipage}[t]{0.13\columnwidth}\raggedright\strut
MODIFIER\strut
\end{minipage} & \begin{minipage}[t]{0.62\columnwidth}\raggedright\strut
A transcript variant of a non coding RNA gene\strut
\end{minipage}\tabularnewline
\bottomrule
\end{longtable}

\elandscape
\newpage

\subsection{Le filtrage des variants}\label{le-filtrage-des-variants}

L'étape de filtrage est extrêmement importante si l'on souhaite analyser
de manière efficace les données provenant de WES. C'est pourquoi elle
occupe une place importante dans notre pipeline. L'intégralité des
paramètres de cette étape peut être modifiée par l'utilisateur de sorte
à faire correspondre les critères de filtres aux besoins de l'étude.
Afin de rendre son utilisation la plus efficace possible, nous avons
souhaité définir des paramètres par défauts pertinent dans la plupart
des études de séquençage exomique de sorte qu'à moins que le contraire
ne soit spécifié les filtres suivants seront appliqués :

\begin{enumerate}
\def\labelenumi{\arabic{enumi}.}
\item
  \textbf{Filtre 1 : L'union des variants} : Quand des individus
  présentent un lien de parenté et présentent le même phénotype, seuls
  les variants observés chez l'ensemble des individus sont conservés.
\item
  \textbf{Filtre 2 : Génotype des variants} : Ce pipeline d'analyse a
  avant tout été développé pour la recherche de variants impliqué dans
  des pathologies à transmission récessive. C'est pourquoi, dans le
  cadre d'étude d'individus présentant un historique de consanguinité,
  l'ensemble des variants hétérozygotes sont filtrés. En revanche, dans
  le cas d'individus issus d'unions non consanguin nous procédons à la
  recherche de variants hétérozygotes composites, c'est à dire
  \textbf{au moins deux variants hétérozygotes différents situés sur
  chacun des deux allèles du même gène d'un patient}. Dès lors, bien que
  les variants soient différents, les deux allèles sont altérés rendant
  possible l'apparition de phénotype récessif. Malheureusement, dans le
  cadre des séquençages WES et WGS, il est impossible de connaitre le
  ``phasage'' de ces variants, c'est à dire que l'on ne peut déterminer
  si deux variants hétérozygotes sont situés sur le même allèle ou sur
  deux allèles différents (\textbf{Figure : }\ref{fig:piccompositehet}).
  Pour cela, une analyse familiale permettant de suivre la ségrégation
  des variants est nécessaire.
\item
  \textbf{Filtre 3 : Les transcrits ``non pertinents''} : Au cours de
  nos analyses nous nous sommes concentré uniquement sur les transcrits
  codant pour une protéine. Ainsi, l'ensemble des transcrits annotés
  comme étant non codant furent filtrés. De même pour les transcrits
  annotés comme étant NMD (\emph{nonsense-mediated decay}). En effet, ce
  mécanisme a pour but de contrôler la qualité des ARNm cellulaires chez
  les eucaryotes {[}\protect\hyperlink{ref-Chang2007}{16}{]} en
  éliminant les ARNm qui comportent un codon stop prématuré
  {[}\protect\hyperlink{ref-Baker2004}{17}{]} pouvant être le résultat
  d'une erreur de transcription, d'une mutation ou encore d'une erreur
  d'épissage. Il est donc peu probable que les variants présents sur des
  transcrits annotés NMD soient responsables du phénotype. Dès lors, ces
  transcrits ont été également filtrés. Ainsi, l'ensemble des variants
  impactant \textbf{uniquement} des transcrits non codant et / ou annoté
  NMD sont filtrés.
\item
  \textbf{Filtre 4 : Impact du variant} : Afin de ne conserver que les
  variants ayant le plus de risque d'avoir un effet délétère sur la
  protéine, seuls sont conservés ceux impactant la séquence codante d'un
  transcrit. De plus les variants synonymes ne sont pas conservés
  (excepté ceux se trouvant proches des régions d'épissage) car ceux-ci
  n'ont aucun effet sur la séquence protéique. Pour les variants
  faux-sens (changement d'un seul acide-aminé de la séquence protéique)
  il est plus difficile de trancher, dès lors, seuls ceux étant prédit
  comme \emph{tolerated} par SIFT
  {[}\protect\hyperlink{ref-Kumar2009}{18}{]} \textbf{et} comme
  \emph{benign} par Polyphen
  {[}\protect\hyperlink{ref-Adzhubei2010}{19}{]} sont filtrés.
\item
  \textbf{Filtre 5 : Fréquence des variants} : La fréquence d'un variant
  dans la population générale est un moyen rapide d'avoir une prédiction
  fiable de l'effet délétère ou non de celui-ci. En effet, il est peu
  probable qu'un variant retrouvé fréquemment dans la population
  générale soit causal d'une pathologie sévère. C'est pourquoi,
  l'ensemble des variants ayant une fréquence \(\ge\) 1\% dans l'une des
  trois bases de données que sont ExAC, ESP et 1KG sont filtrés.
\item
  \textbf{Filtre 6 : Présence des variants dans la cohorte contrôle} :
  Ce filtre utilise les variants composant la base de données interne du
  pipeline et permet de filtrer l'intégralité des variants homozygotes
  retrouvés chez les patients séquencés ne présentant \textbf{pas} le
  même phénotype que le patient analysé. Comme dit plus tôt, ce filtre
  se révèle particulièrement intéressant lorsque plusieurs patients
  porteurs de phénotypes différents ont subi le même protocole de
  séquençage ainsi l'ensemble des variants faux-positifs résultant
  d'artéfacts liés aux différentes étapes en amont de l'analyse
  bioinformatique pourront alors être filtrés. De même ce filtre permet
  de mettre en évidence les variants propres à une population lorsque
  des patients provenant de la même région géographique et ne présentant
  toujours pas le même phénotype sont comparés.
\end{enumerate}

\begin{figure}

{\centering \includegraphics[scale=0.35]{figure/hetero_composites} 

}

\caption[Représentation schématique des phasages de deux variants avec les génotypes associés]{\textbf{\emph{Représentation schématique des
phasages de deux variants avec les génotypes associés}} : Un variant est
homozygote lorsque le \textbf{même} variant est présent sur les deux
allèles d'un gène et hétérozygote lorsqu'il est présent sur \textbf{un
seul} des deux allèles. On parle d'hétérozygotes \emph{cis} lorsque deux
variants hétérozygotes \textbf{différents} sont positionnés sur
\textbf{le même allèle} et d'hétérozygote \emph{trans} (ou composite)
lorsque ces deux variants hétérozygotes sont positionnés sur
\textbf{deux allèles différents}. En WES et en WGS il est impossible de
différentier les hétérozygotes \emph{cis} des hétérozygotes \emph{trans}}\label{fig:piccompositehet}
\end{figure}












\newpage

\section{Résultats 1 : Analyse de 3 phénotypes par des cas
familiaux}\label{resultats-1-analyse-de-3-phenotypes-par-des-cas-familiaux}

Cette partie, se concentre sur l'analyse bioinformatique des résultats
des séquençages exomiques de 13 individus infertiles provenant de 7
familles. Pour 6 familles 2 frères atteints ont été analysés et pour la
septième, un seul des frères atteint a été séquencé et l'ADN du deuxième
frère a été disponible à postériori \ref{tab:tabfam}.

\begin{enumerate}
\def\labelenumi{\arabic{enumi}.}
\item
  \textbf{Famille FAM} : Cette famille est composée de 2 frères 2
  azoospermes. Comme nous avons pu le voir, l'azoospermie est un
  phénotype d'infertilité masculine caractérisé par l'absence de
  spermatozoïde dans l'éjaculat. Des 13 patients de cette étude, les
  frères Ghs44 et Ghs45 sont les deux seuls à ne pas avoir été séquencés
  au Génopole d'Évry.
\item
  \textbf{Famille FF} : Les spermatozoïdes des 2 frères de cette famille
  sont caractérisés par leur incapacité à féconder l'ovocyte malgré leur
  morphologie et leur mobilité normales.
\item
  \textbf{Famille MMAF1-5} : Ici nous avons 5 familles dont l'ensemble
  des membres séquencés ici présentent un phénotype MMAF (\emph{multiple
  morphological abnormalities of the sperm flagella}). Ce syndrome se
  caractérise par la présence d'une majorité de spermatozoïdes
  présentant une mosaïque d'anomalie morphologique du flagelle.
\end{enumerate}

\begin{longtable}[t]{lllrl}
\caption{\label{tab:tabfam}Tableau récapitulatif des familles séquencées et de leur phénotype}\\
\toprule
Family & Individuals & Phenotype & Year & Place\\
\midrule
AZ & Ghs44, Ghs45 & Azoospermia & 2012 & Mount Sinai Institut\\
FF & Ghs113, Ghs117 & Fertilization failure & 2014 & Genoscope (Evry)\\
MMAF1 & Ghs56, Ghs58 & MMAF & 2014 & Genoscope (Evry)\\
MMAF2 & Ghs59, Ghs60 & MMAF & 2014 & Genoscope (Evry)\\
MMAF3 & Ghs62, Ghs130 & MMAF & 2014 & Genoscope (Evry)\\
\addlinespace
MMAF4 & Ghs119, Ghs63 & MMAF & 2014 & Genoscope (Evry)\\
MMAF5 & Ghs131 & MMAF & 2014 & Genoscope (Evry)\\
\bottomrule
\end{longtable}

\newpage 

\subsection{Résultats des différentes étapes de
l'analyse}\label{resultats-des-differentes-etapes-de-lanalyse}

\subsubsection{Résultat de l'alignement}\label{resultat-de-lalignement}

Pour rappel, l'\protect\hyperlink{lalignement}{alignement} consiste à
repositionner l'ensemble des \emph{reads} générés au cours de l'étape de
séquençage le long d'un génome de référence.

La quantité de \emph{reads} composant les exomes de chaque individu peut
varier en fonction de plusieurs paramètres et n'est donc pas égale pour
chaque patient bien que l'ordre de grandeur reste le même avec une
médiane de 87747080 \emph{reads}. Seuls les deux frères \emph{Ghs44} et
\emph{Ghs45} de la famille AZ se distinguent avec près de 3 fois plus de
\emph{reads} que les autres patients. Cette différence peut être
expliquée car ces deux patients sont les deux seuls à voir été séquencés
au Mount Sinaï Institut. Or leur protocole d'amplification précédant le
séquençage contient un nombre de cycles de PCR supérieur à ceux
appliqués au Génopole d'Évry où ont été séquencés les autres patients.
Il faut noter que ce nombre plus important de \emph{reads} n'est en rien
le reflet d'une meilleure qualité. En effet, celui-ci est causé par une
grande quantité de \emph{reads} dupliqués qui seront pour la plupart
filtrés au cours des analyses ultérieures (\textbf{Table :}
\ref{tab:tabfam}, \textbf{Figure : }\ref{fig:plotfammapping} -
\textbf{A}).

La première étape du contrôle qualité des \emph{reads} consiste à
filtrer ceux ne s'étant pas alignés sur le génome. Ces \emph{reads} sont
extrêmement minoritaires puisqu'ils ne représentent qu'entre 1.2 et 5.5
\% des \emph{reads} de nos individus (\textbf{Figure :
}\ref{fig:plotfammapping} - \textbf{B}).

Parmi les \emph{reads} s'étant correctement alignés sur le génome, seuls
les \emph{reads} présentant des \emph{ends} compatibles sont conservés
Pour rappel, ces \emph{reads} représentent ceux pour lesquels une des
deux \emph{ends} s'est alignée sur le \emph{strand forward} tandis que
l'autre s'est alignée sur le \emph{strand reverse} et que la distance
qui les sépare n'est pas supérieure à trois fois celle de l'insert. Dans
nos données, les \emph{reads} remplissant ces conditions sont
majoritaires puisqu'ils représentent environ 91.3 \% des \emph{reads}
s'étant correctement alignés tandis que les \emph{reads} non-compatibles
et orphelins représentent respectivement 6.2 et 2.8 \% de ces même
\emph{reads} (\textbf{Figure : }\ref{fig:plotfammapping} - \textbf{C}).

La dernière étape de ce contrôle-qualité consiste à analyser le nombre
de sites sur lesquels se sont alignés les \emph{reads}. En effet,
certaine zones du génome étant dupliquées, l'une des problématiques des
\emph{short-reads} est qu'il est possible que ceux-ci s'alignent à
plusieurs régions différentes du génome. Afin d'éviter toute ambiguïté,
seuls ceux s'étant alignés sur un site unique sont conservés pour la
suite des analyses. Ces \emph{reads} représente entre 92.3 et 96.9 \%
des \emph{reads} ayant passé les précédents filtres (\textbf{Figure :
}\ref{fig:plotfammapping} - \textbf{D}).

Ainsi, à la fin de ces trois étapes de contrôles qualité des
\emph{reads}, environ 84.8 \% d'entre eux sont conservés soit un total
d'environs 74409523 \emph{reads} par patients.

\newpage 

\begin{figure}

{\centering \includegraphics{thesis_files/figure-latex/plotfammapping-1} 

}

\caption[Processus simplifié du contrôle qualité des
\emph{reads}\\]{\textbf{\emph{Processus simplifié du contrôle
qualité des \emph{reads}}} : Pour chacun des graphiques, les
\emph{reads} représentés en vert sont conservés tandis que ceux en rouge
sont filtrés. \textbf{A} : Quantité de \emph{reads} bruts générés pour
chaque patient au cours de l'étape de séquençage. La médiane des
\emph{reads} est représentée en bleue. \textbf{B} : Pourcentage pour
chaque individu de \emph{reads} s'étant aligné correctement et ne
s'étant pas alignés sur le génome de référence. \textbf{C} :
Distribution pour chaque patient des \emph{reads} compatibles (Comp),
non compatibles (Non comp) et orphelins (Orphans). \textbf{D} :
Présentation pour chaque \emph{reads} du nombre de site auxquels ils
s'alignent}\label{fig:plotfammapping}
\end{figure}
















\newpage

\subsubsection{L'appel des variants}\label{lappel-des-variants-1}

Une fois l'alignement effectué, il faut identifier les positions
présentant des différences avec le génome de référence et leur assigner
un génotype. Ce génotype étant dépendant du pourcentage de \emph{reads}
variants à une position donnée, il est nécessaire, afin de calibrer les
bornes de notre algorithme d'appel, de connaitre la répartition globale
de ces proportions. Comme attendue, nous observons des pics à 0, 50 et
100\% de \emph{reads} variants pour une position donnée. Ces trois
pourcentages correspondant respectivement aux appels ``homozygote
référence'', ``hétérozygote'' et ``homozygote variant''. La démarcation
séparant les appels hétérozygote et homozygote variant est relativement
claire. La distinction de ces deux génotypes ne pose donc pas de
problème, la région d'ambiguïté allant de 75 à 85\% de \emph{reads}
variants ne concernant qu'une minorité de positions. Ce n'est cependant
pas le cas pour celle séparant les appels homozygote référence et
hétérozygote. En effet, on peut constater que la zone d'ambiguïté allant
de 20 à 40\% de \emph{reads} variants concerne une part non négligeable
des positions couvertes. Les positions SS situées dans cette région
seront systématiquement filtrées. Pour les positions DS, celles pour
lesquels \textbf{un seul} des \emph{strand} se situe dans cette région
alors que l'autre se trouve dans une des régions adjacentes (homozygote
référence ou hétérozygote) seront conservées. Les autres seront filtrées
elles aussi (\textbf{Figure : }\ref{fig:plotdensityvar}).

Dans nos données, les appels SS sont majoritaires et représentent
environ 46.7 \% de nos appels (contre 37 \% d'appels DS). Au vu de
l'importance de ces appels, nous avons fait le choix de les conserver
afin de ne pas filtrer une quantité trop importante de données. Ces
appels seront cependant considérés comme étant de faible qualité, de
fait, leurs analyses et interprétation seront plus précautionneuses. En
revanche, au vu de la trop grande incertitude de l'appel des variants
NS, ceux-ci sont systématiquement filtrés éliminant ainsi entre 10.3 et
23.7 \% des positions appelées pour chaque patient (\textbf{Figure :
}\ref{fig:plotvarcall} - \textbf{A}).

De même, les appels discordant et ambigus sont filtrés, soit environ
13.9 \% des variants DS. Il est intéressant de noter que bien que les
variants \emph{single strand} (SS) soient conservés, on peut s'attendre
à ce qu'environ 8.8\%, soit le pourcentage de DS discordants, soient
aberrants, ceux-ci n'ayant pu subir le même contrôle que les SS
(\textbf{Figure : }\ref{fig:plotvarcall} - \textbf{B}).

Pour l'ensemble des variants ayant passé les filtres énoncés ci-dessus,
c'est à dire les variants SS et les variants DS avec appels concordants,
le génotype est déterminé en fonction du pourcentage de \emph{reads}
portant le variant à cette position. Ainsi, pour chaque individu nous
avons pu établir une liste de SNVs et d'indels avec leur génotype
associé. Pour chacun de nos 13 patients les ordres de grandeur du nombre
de variants appelés sont identiques. Ainsi pour chaque patient nous
avons appelés environ 43246 variants hétérozygotes (40721 SNVs et 2525
indels) et 34290 variants homozygotes (32497 SNVs et 1793 indels)
(\textbf{Figure : }\ref{fig:plotvarcall} - \textbf{C}).

\newpage

\begin{figure}

{\centering \includegraphics{figure/density_plot} 

}

\caption[Densité de répartition du pourcentage de *reads* variants pour chaque position couverte]{\textbf{\emph{Densité de répartition du pourcentage
de \emph{reads} variants pour chaque position couverte}} : Les
homozygotes références évidents ne nous intéressant pas, les positions
représentées sont restreinte à celles contenant un minimum de 7\% de
\emph{reads} variants. L'arrière-plan montre les bornes utilisées par
notre algorithme pour chaque appel. Les lignes pointillées séparent en
deux les régions d'ambiguïté basse et haute.}\label{fig:plotdensityvar}
\end{figure}









\newpage

\begin{figure}

{\centering \includegraphics{thesis_files/figure-latex/plotvarcall-1} 

}

\caption[Contrôle qualité des variants appelés]{\textbf{\emph{Contrôle qualité des variants appelés}}
: Pour chacun des graphiques, les variants représentés en vert et en
orange sont conservés tandis que ceux en rouge sont filtrés. \textbf{A}
: Distribution du \emph{stranding} des appels pour chaque patient.
\textbf{B} : Comparaison des appels entre les deux \emph{ends} des
variants appelés DS. \textbf{C} : Distribution des SNVs et indels en
fonction de leur génotype pour chaque patient (représentés par une
barre).}\label{fig:plotvarcall}
\end{figure}










\newpage

\subsubsection{L'annotation des
variants}\label{lannotation-des-variants}

Après avoir annoté nos variants, nous avons pu constater que pour chaque
patient 24794 gènes sont en moyenne affecté par au moins un variant
homozygote pour en moyenne 121843 transcrits (soit environ 5 transcrits
par gènes) (\textbf{Figure : }\ref{fig:plotannotation} - \textbf{A}). Il
faut noter que parmi ces gènes se trouvent à la fois des gènes codant
pour des protéine \textbf{et} d'autres non codant.

Chaque variant affectera l'ensemble des transcrits qu'il chevauche,
ainsi un même variant pourra impacter plusieurs transcrits. Ces impacts
sont ensuite classés par VEP en quatre catégories qui sont, de la plus
délétère à la moins délétère : \emph{HIGH}, \emph{MODERATE}, \emph{LOW},
\emph{MODIFIER} (\textbf{Table : }\ref{tab:tabvepcsq}).

Comme attendu, les variants ayant un impact tronquant se retrouvent être
les moins fréquent chez chacun de nos patients. Ceci est d'autant plus
flagrant pour l'impact \emph{HIGH} qui regroupe, entre autres, les
variants créant un codon stop ou causant un décalage du cadre de lecture
(\textbf{Table : }\ref{tab:tabvepcsq}). Ceux-ci se retrouvent, par
rapport aux autres impacts, en quantité extrêmement faible puisqu'ils ne
représentent en moyenne que 0.23 \% des variants. Cependant, bien que ce
pourcentage soit faible, cela représente tout de même une moyenne de 131
variants \emph{HIGH} hétérozygotes par patients et 114 variants
\emph{HIGH} homozygotes par patient) (\textbf{Figure :
}\ref{fig:plotannotation} - \textbf{B}).

\newpage

\begin{figure}

{\centering \includegraphics{thesis_files/figure-latex/plotannotation-1} 

}

\caption[Annotation des variants]{\textbf{\emph{Annotation des variants}} :
\textbf{A} : Quantification du nombre de gènes (en bleu) / transcrits
(en rose) impactés par au moins un variant pour chaque patient chacun
représentés par une barre. \textbf{B} : Distribution des impacts
\emph{HIGH MODERATE LOW} et \emph{MODIFIER} en fonction des patients et
du statut du variant.}\label{fig:plotannotation}
\end{figure}








\subsubsection{Le filtrage des
variants}\label{le-filtrage-des-variants-1}

Les étapes précédentes nous ont permis de mettre en évidence pour chaque
patient une liste de variants passant l'ensemble de nos critères
qualités. Ces variants ont dès lors pu être annotés nous permettant
notamment d'avoir connaissance de leurs impacts sur les différents
transcrits qu'ils chevauchent ou encore leur fréquence dans la
population générale. Désormais, afin de ne conserver que les variants
ayant la plus forte probabilité d'être responsable du phénotype de ces
patients, nous avons appliqué successivement les six filtres
précédemment décrits.

\begin{enumerate}
\def\labelenumi{\arabic{enumi}.}
\item
  \textbf{Filtre 1 : L'union des variants} : Dans cette étude nous
  analysons les données génétiques de 7 familles composées de 1 à 2
  frères. Nous avons donc émis l'hypothèse que \textbf{le phénotype de
  chacun de frères d'une même famille était dû à une cause génétique
  commune}. C'est pourquoi, seul les variants observés chez l'ensemble
  des membres d'une même famille furent conservés. Ainsi ce filtre a
  permis de filtrer entre 17703 et 41944 variants pour chacun des
  patients (\textbf{Figure : }\ref{fig:plotcomparefilter} - \textbf{A}).
\item
  \textbf{Filtre 2 : Génotype des variants} : Ici, nous avons émis
  l'hypothèse d'une transmission récessive du phénotype. Ainsi, seuls
  les variants homozygotes ont été conservés. filtrant en moyenne 45089
  variant par individu soit une moyenne de 56.5\% de leurs variants
  (\textbf{Figures : }\ref{fig:plotvarcall} - \textbf{C},
  \ref{fig:plotcomparefilter} - \textbf{A}).
\item
  \textbf{Filtre 3 : Impact du variant} : Ce filtre se base à la fois
  sur les prédictions VEP mais aussi, dans le cas de variants faux-sens,
  sur les prédictions SIFT et PolyPhen permet de ne conserver que les
  variants ayant les effets les plus délétères. Ce filtre est, de prime
  abord le plus efficace puisqu'il permet de filtrer à lui seul environs
  88.3\% des variants de chaque individu.
\item
  \textbf{Filtre 4 : Les transcrits ``non pertinents''} : Cette étape de
  filtre permet de filtrer systématiquement entre 13712 et 17992
  transcrits différents par patients. Sont considérés comme
  non-pertinents les transcrits ne codant pas pour une protéine et ceux
  annoté étant dégradés (\emph{nonsense mediated decay} (NMD)).
  Cependant, un même variant pouvant impacter à la fois des transcrits
  ``non pertinents'' \textbf{et} des transcrits ``pertinents'', seuls
  ceux impactant \textbf{uniquement} des transcrits ``non pertinents''
  sont filtrés, soit une moyenne de 1797 variants par individu
  (\textbf{Figure : }\ref{fig:plotfilternonpertinanttr}).
\end{enumerate}

\newpage 

\begin{figure}

{\centering \includegraphics{thesis_files/figure-latex/plotfilternonpertinanttr-1} 

}

\caption[Filtrage des transcrits jugés "non pertinents" et des variants les chevauchant]{\textbf{\emph{Filtrage des transcrits
jugés ``non pertinents'' et des variants les chevauchant}} : Pour chaque
patient nous avons filtré les transcrits jugés ``non pertinents'' pour
l'analyse, c'est à dire ceux ne codant pas pour une protéine et ceux
annoté NMD (boite rouge). Dès lors, l'intégralité des variants
chevauchant \textbf{uniquement} des transcrits non pertinents sont
systématiquement filtrés (boites bleue). Les autres sont conservés.}\label{fig:plotfilternonpertinanttr}
\end{figure}









\begin{enumerate}
\def\labelenumi{\arabic{enumi}.}
\setcounter{enumi}{4}
\item
  \textbf{Fréquence des variants :} Filtrer systématiquement les
  variants retrouvés avec une fréquence \(\ge\) 0.01 dans l'une des
  trois bases de données que sont ExAC, 1KG et ESP6500 permet de filtrer
  entre 25126 et 29797 variants par patients.
\item
  \textbf{Présence des variants dans la cohorte contrôle :} L'ensemble
  des variants répertoriés au sein de ces 3 phénotypes a été confrontés
  à des listes de variants identifiés chez d'autres patients analysés
  par notre pipeline et présentant des phénotypes différents. Dès lors,
  l'ensemble des variants retrouvés à l'état homozygote chez au moins un
  des individus de la cohorte contrôle sera filtré de notre liste de
  variants. On peut cependant noter que les variants retrouvés chez les
  patients \emph{Ghs44} et \emph{Ghs45} de la famille AZ n'ont pas été
  confrontés à ceux observés dans notre cohorte de 15 femmes présentant
  des anomalies de développement ovocytaire. En effet, ces deux
  phénotypes peuvent être causés par un même gène impliqué dans la
  division méiotique (\textbf{Figure :} \ref{fig:plotsamplectrl}).\\
  \newpage
\end{enumerate}

\begin{figure}

{\centering \includegraphics{thesis_files/figure-latex/plotsamplectrl-1} 

}

\caption[Nombre d'individus et leur phénotypes composant la cohorte contrôle de chaque famille]{\textbf{\emph{Nombre d'individus et leur phénotypes
composant la cohorte contrôle de chaque famille}} : Ici, chaque barre
représente une famille et sa hauteur est déterminée par le nombre
d'individus composant la cohorte contrôle à laquelle elle a été
confronté. Le nombre total de contrôles utilisés par famille est inscrit
au-dessus de chaque barre. Les couleurs déterminent les phénotypes des
individus de la cohorte contrôle. Chaque individu de cette cohorte a été
séquencé en WES par notre équipe. Afin d'être considéré comme
``contrôle'' et intégrer cette cohorte, un individu doit être sain ou
présenter un phénotype d'infertilité suffisamment différent de la
famille étudiée. Par exemple, un individus MMAF pourra servir de
contrôle aux familles AZ et FF mais pas aux familles MMAF1-5.}\label{fig:plotsamplectrl}
\end{figure}














\newpage

Comme on pouvait s'y attendre, ces six filtres ont un pouvoir
discriminant extrêmement différent. En effet, tandis que le filtre
``\emph{Transcript relevance}'' (filtre n°4) élimine en moyenne 3.9\%
des variants de chaque individu, le filtre ``Variant impact'' (filtre n°
3) élimine, lui, jusqu'à 90.1\% de ces mêmes variants. Cette différence
n'est pas surprenante. En effet, comme nous l'avions vu plus tôt, les
variants considérés comme \emph{MODIFIER} par VEP qui regroupent entre
autres les variants chevauchant les séquences UTRs et introniques
(\textbf{Table :} \ref{tab:tabvepcsq}) représentent en moyenne 81\% des
variants de nos patients, or, ceux-ci sont tous filtrés. On peut
également constater l'importance de la cohorte contrôle puisque elle
permet retirer entre 76.5 et 88.4\% des variants de chaque individus
(\textbf{Figure :} \ref{fig:plotcomparefilter} - \textbf{A}).

Cependant, regarder uniquement le pourcentage de variants filtrés par
chaque filtre révèle une information partielle. En effet, dans ce cas de
figure, on observe la quantité de variant éliminée par chaque filtre
indépendamment les uns des autres. Ainsi, un même variant peut donc être
filtré par plusieurs filtres. Dès lors, il faut également analyser la
quantité de variants filtrée \textbf{spécifiquement} par chaque filtre.
Ainsi, on peut constater que le classement des filtres en fonctions de
leur stringence reste quasiment identique. Il est tout de même
intéressant de noter que désormais le filtre ``Variant impact'' apparait
moins efficace que les filtres ``Ctrl'' et ``Genotype'' en filtrant
spécifiquement une moyenne de 253 variants par individu contre 423 pour
le filtre génotype et 882 pour le filtre ``Ctrl''. Ainsi, ce dernier
devient celui filtrant spécifiquement le plus de variants avec entre 364
et 1060 variants spécifiquement filtrés par patients confirmant ainsi
l'importance de ce filtre dans nos analyses. Aussi, les filtres
``Transcript relevance'', ``Union'' et ``Frequency'' apparaissent
désormais comme étant anecdotiques en comparaison aux trois autres
filtres puisqu'ils filtrent au maximum 43 variants spécifiques
(\textbf{Figure :} \ref{fig:plotcomparefilter} - \textbf{B}).

Après avoir appliqué l'ensemble de ces filtres, seuls quelques variants
subsistent nous permettant d'obtenir une liste de gènes restreinte pour
chaque famille et ainsi de tirer des conclusions quant au variant(s)
responsable(s) du phénotype de chacune d'entre elles. Ces travaux ont
ainsi pu mener à l'écriture de trois articles dont je suis co-auteur.

\newpage

\begin{figure}

{\centering \includegraphics{thesis_files/figure-latex/plotcomparefilter-1} 

}

\caption[Comparaison de l'efficacité de chacun des six filtres utilisés]{\textbf{\emph{Comparaison de l'efficacité de
chacun des six filtres utilisés}} : \textbf{A} : Comparaison du
pourcentage de variants filtrés par chacun des six filtres
indépendamment les uns des autres pour chaque patient (représenté par
les points. Dès lors, un même variant peut-être filtré par plusieurs
filtres. \textbf{B} : Comparaison du nombre de variant filtrés
spécifiquement par chacun des filtres. Ici, un variant ne peut-être
filtré que par un seul filtre. .}\label{fig:plotcomparefilter}
\end{figure}










\newpage

\newpage  

\subsection{Article n° 3}\label{article-n-3}

\textbf{SPINK2 deficiency causes infertility by inducing sperm defects
in heterozygotes and azoospermia in homozygotes}

Kherraf ZE\textsuperscript{*}, Christou-Kent M\textsuperscript{*},
\textbf{Karaouzène T}, Amiri-Yekta A, Martinez G, Vargas AS, Lambert E,
Borel C, Dorphin B, Aknin-Seifer I, Mitchell MJ, Metzler-Guillemain C,
Escoffier J, Nef S, Grepillat M, Thierry-Mieg N, Satre V, Bailly M,
Boitrelle F, Pernet-Gallay K, Hennebicq S, Fauré J, Bottari SP, Coutton
C, Ray PF, Arnoult C

\textsuperscript{*} Co-premiers auteurs

EMBO Molecular Medicine, Mai 2017

\newpage

\subsubsection{Contexte et objectifs}\label{contexte-et-objectifs}

L'oligospermie, comme l'azoospermie sont des phénotypes d'infertilités
masculines liées à la quantité de spermatozoïdes présent dans
l'éjaculat. Les différentes études publiées ces dernières années
montrent que les microdélétions du chromosome Y sont retrouvées chez
10\% des hommes avec une azoospermie non-obstructives et chez 5\% des
patients avec une oligozoospermie sévères
{[}\protect\hyperlink{ref-Hotaling2014}{20}{]}. Ces taux bien qu'élevé
ne représente qu'une infime partie des cas d'azoospermie et
d'oligospermie suggérant l'implication de nombreux autres gènes dans ce
phénotype.

Entre 2005 et 2014 deux frères issus d'une union consanguine ont demandé
des conseils médicaux auprès de différentes cliniques d'infertilité
après deux ans d'échec reproductif. Ces deux frères étant mariés à des
femmes non-apparentées une cause féminine fût exclue et les recherches
ont été concentrées sur l'analyse des deux frères. Tous deux
présentaient de sévères défauts de production de spermatozoïdes, l'un
des frères présentant une azoospermie non-obstructive et l'autre une
oligozoospermie extrême (\textless{}1 Million de spermatozoïde/ ml). Au
vu de la similarité du phénotype et de leur lien de parenté, l'hypothèse
d'une cause génétique commune fut émise. L'analyse de leur caryotype et
du locus AZF du chromosome Y ne révélant aucune anomalie, la procédure
d'un séquençage WES fut décidée.

DDans ce contexte, l'objectif de mon travail a été d'effectuer
l'ensemble des analyses des données WES obtenues après leur séquençage
afin de mettre en évidence une mutation homozygote commune pouvant
expliquer le déficit spermatique des deux frères. Dans un second temps,
j'ai mis en place le protocole de génotypage des souris au locus du gène
\emph{Spink2} permettant d'identifier les souris sauvages
\emph{Spink2}\textsuperscript{+/+} des souris KO
\emph{Spink2}\textsuperscript{-/-}. Pour finir, afin d'estimer
l'importance des variants du gène \emph{SPINK2} comme cause
d'infertilité masculine chez l'humain, j'ai également contribué au
séquençage Sanger de la séquence codante de \emph{SPINK2} d'une partie
des 611 patients séquencés dans cette étude.

\newpage

\includepdf[pages=-]{bib/SPINK2_2017.pdf}

\newpage

\subsubsection{Principaux résultats}\label{principaux-resultats}

L'analyse et le filtrage des données de séquençage des deux frères par
notre pipeline a permis de retenir seulement 2 variants répondant à tous
les critères établis (\textbf{Table : }\ref{tab:tabrecapaz}). Le premier
de ces variants chevauchait la séquence codante du gène \emph{GUF1} et
entrainait la substitution d'une sérine par une isoleucine. Le second
impactait le gène \emph{SPINK2} et se situait trois nucléotides en amont
du deuxième exon pouvant ainsi entrainer des erreurs d'épissage pouvant
par la suite soit entrainer un décalage du cadre de lecture menant à la
formation d'un codon stop prématuré ou bien sauter completement la
transcription de cet exon créant un codon stop au début de l'exon 3.

Parmi ces deux gènes, seul \emph{SPINK2} présentaient une forte
expression testiculaire dans les données Ensembl (\textbf{Figure :
}\ref{fig:plotexpfamaz}) et nous avons confirmé cette surexpression par
RT-PCR dans cette étude. De plus, un KO partiel de \emph{Spink2} chez la
souris avait déjà été décrit comme induisant des défauts partiels de la
spermatogenèse {[}\protect\hyperlink{ref-Lee2011}{21}{]}. Ces arguments
ont ainsi fait de \emph{SPINK2} le candidat évident pour expliquer le
phénotype de ces deux frères. Après avoir confirmé en séquençage Sanger
la présence de la mutation de ce gène à l'état homozygote pour les deux
frères et hétérozygotes pour les parents, nous avons, afin de continuer
nos investigations, développés un modèle murin KO
\emph{Spink2}\textsuperscript{-/-} confirmant une azoospermie complète
pour les souris mâle causé par un arrêt de la spermatogenèse au stage
des spermatides rondes. De plus, malgré une fertilité normale, nous
avons pu noter un taux élevé d'anomalies morphologiques du spermatozoïde
ainsi qu'une motilité spermatique réduite chez les souris mâles
hétérozygotes \emph{Spink2}\textsuperscript{+/-}. Les femelles, elles ne
présentaient aucun phénotype apparent. L'étude de la localisation de la
protéine Spink2 chez la souris et SPINK2 chez l'humain a révélés que ces
deux protéines localisaient dans la vésicule acrosomale depuis le début
de la biogénèse de l'acrosome jusqu'au spermatozoïde mature. Nous avons
par ailleurs démontré que Spink2, une antiprotéase à sérine, permettait
de neutraliser l'acrosine (Acr) pendant son transit par l'appareil de
golgi, et qu'en son absence, Acr déstructurait le golgi, empêchait la
formation acrosomale et entrainait le blocage au stade de spermatide
ronde.

Suite à cela, afin d'évaluer l'importance des variants du gène
\emph{SPINK2} dans l'infertilité humaine, nous avons effectué le
séquençage Sanger de 611 patient parmi lesquels 210 étaient azoospermes,
393 oligozoospermes et 8 dont la cause n'étaient pas spécifiée. Parmi
ces patients, seul 1 s'est révélé porter un variant non répertorié dans
ExAC sur le gène \emph{SPINK2}. Ce patient présentant un phénotype
d'oligozoospermie porte à l'état hétérozygote un variant altérant le
codon start du gène \emph{SPINK2}. Ces résultats indiquent donc que chez
l'homme, comme chez la souris, la présence de mutations homozygotes sur
le gène \emph{SPINK2} induit un phénotype d'azoospermie ou
d'oligozoospermie sévère tandis que la présence d'une mutation
hétérozygote pouvait entrainer un phénotype d'oligozoospermie avec un
taux élevé d'anomalies morphologique du spermatozoïde. Le fait que même
les hétérozygotes puissent subir une pression de sélection négative
pourrait expliquer la rareté observée des mutations \emph{SPINK2}.

\begin{longtable}[t]{llllll}
\caption{\label{tab:tabrecapaz}Liste des variants ayant passé l'ensemble des filtres pour les deux fères de la famille AZ}\\
\toprule
\multicolumn{1}{c}{ } & \multicolumn{2}{c}{Impact} & \multicolumn{3}{c}{Frequency} \\
\cmidrule(l{2pt}r{2pt}){2-3} \cmidrule(l{2pt}r{2pt}){4-6}
Gene & HGVSc, HGVSp & Consequence & ExAC & ESP & 1KG\\
\midrule
GUF1 & c.443A>T ; p.Ser148Ile & missense & 0.00207 & 0.0028 & 9e-04\\
SPINK2 & c.56-3C>G ; . & splice region & . & . & .\\
\bottomrule
\end{longtable}

\begin{figure}

{\centering \includegraphics{thesis_files/figure-latex/plotexpfamaz-1} 

}

\caption[Expression tissulaire des gènes \emph{SPINK2} et
\emph{GUF1}]{\textbf{\emph{Expression tissulaire des gènes
\emph{SPINK2} et \emph{GUF1}}} : Données provenant du projet de
transcriptome Illumina bodyMap. Contrairement au gène \emph{GUF1} (en
rouge) qui a une expression relativement ubiquitaire, \emph{SPINK2} (en
bleu) a une expression quasi spécifique au testicule.}\label{fig:plotexpfamaz}
\end{figure}










\newpage

\subsection{Article n° 4}\label{article-n-4}

\textbf{Homozygous mutation of PLCZ1 leads to defective human oocyte
activation and infertility that is not rescued by the WW-binding protein
PAWP}

Jessica Escoffier J\textsuperscript{*}, Lee HC\textsuperscript{*},
Yassine S\textsuperscript{*}, Zouari R, Martinez G, \textbf{Karaouzène
T}, Coutton C, Kherraf ZE, Halouani L, Triki C, Nef S, Thierry-Mieg N,
Savinov SN, Fissore R, Ray PF, Arnoult C

\textsuperscript{*} Co-premiers auteurs

Human Molecular Genetics, Décembre 2015

\newpage

\subsubsection{Contexte et objectifs}\label{contexte-et-objectifs-1}

L'activation ovocytaire regroupe une série de processus intervenant lors
de la fécondation de l'ovocyte par le spermatozoïde. En 1990, plusieurs
études ont démontrées que chez les mammifères ces processus reposent
principalement sur le relargage par le spermatozoïde de ``facteurs
spermatiques'' qui déclenchent un signal constitué d'oscillations
Ca\textsuperscript{2+} {[}todo : ref{]}. Plus tard, la protéine PLCZ1
fut identifiée comme la molécule responsable de l'induction de ces
oscillations calciques. Cependant, l'incapacité à produire des modèles
animaux \emph{PLCZ1} KO capable de produire des spermatozoïdes mature a
empêché d'attribuer l'exclusivité de ce rôle à \emph{PLCZ1} laissant
ouverte la possibilité que l'activation ovocytaire puisse être
tributaire d'autres facteurs spermatiques. C'est ainsi qu'en 2014 fut
proposé la protéine PAWP comme facteur spermatique alternatif ou
complémentaire à PLCZ1 {[}\protect\hyperlink{ref-Aarabi2014}{22},
\protect\hyperlink{ref-Aarabi2014a}{23}{]}.

Les travaux ci-dessous décrivent les analyses effectuées sur deux frères
issus d'une union consanguine ayant tous deux été dans l'incapacité de
concevoir un enfant spontanément. Des fécondations in vitro avec
injection directe d'un spermatozoïde (ICSI) ont par la suite été
réalisées et n'ont pas permis la fécondation ovocytaires.\\
Comme dans l'étude précédente, en raison de l'historique de
consanguinité de la famille des deux frères et du non-apparentement de
leurs femmes respectives nous avons exclu l'hypothèse d'une cause
féminine et nous avons recherché un variant homozygote commun aux deux
frères par séquençage WES. Comme précédemment, j'ai été en charge de
l'ensemble des analyses des données issues du séquençage des deux
frères.

\newpage

\includepdf[pages=-]{bib/PLCZ1_2016}

\newpage

\subsubsection{Principaux résultats}\label{principaux-resultats-1}

Suite à l'analyse bioinformatique de ces deux frères, un seul variant
subsistait après l'application de l'ensemble des filtres. Celui-ci était
recensé uniquement dans la base de données ExAC avec une fréquence de
8.24e-06 et entrait un faux-sens prédit comme \emph{deleterious} par
SIFT et \emph{possibly damaging} par PolyPhen sur la séquence du gène
\emph{PLCZ1}. La forte expression testiculaire de ce gène
(\textbf{Figure : }\ref{fig:plotexpfamff}) couplée à l'implication déjà
connu de celui-ci dans l'activation ovocytaire, ont fait de ce variant
le candidat évident pour expliquer le phénotype de ces deux frères. De
plus, aucun variant n'a été retrouvé sur la séquence du gène
\emph{WBP2NL} codant pour la protéine PAWP (l'autre gène candidat à la
fonction d'activateur ovocytaire) bien que l'intégralité de la séquence
codante de \emph{WBP2NL} ait une couverture \(\ge\) 40x (les zones moins
couvertes du début de l'exon 1 et de la fin de l'exon 6 correspondant
aux régions UTR) (\textbf{Figure : }\ref{fig:plotcovplcz}). Ces
résultats suggérant une parfaite fonctionnalité de la protéine PAWP ont
pu être confirmée par \emph{Western blot}, de même, la bonne
localisation de la protéine PAWP a pu être observée chez les deux
patients par Immunofluorescence alors que PLCZ1 était absent du sperme
de nos patients.

Cette étude confirme le rôle primordial de PLCZ1 dans l'activation
ovocytaire et démontre que la présence de PAWP seul ne permet pas cette
activation.

\begin{longtable}[t]{llllll}
\caption{\label{tab:tabrecapff}Liste des variants ayant passé l'ensemble des filtres pour les deux fères de la famille FF}\\
\toprule
\multicolumn{1}{c}{ } & \multicolumn{4}{c}{Impact} & \multicolumn{1}{c}{Frequency} \\
\cmidrule(l{2pt}r{2pt}){2-5} \cmidrule(l{2pt}r{2pt}){6-6}
Gene & HGVSc, HGVSp & Consequence & SIFT & PolyPhen & ExAC\\
\midrule
PLCZ1 & c.1465G>T ; p.Ile489Phe & missense & deleterious & possib damaging & 8.24e-06\\
\bottomrule
\end{longtable}

\newpage

\begin{figure}

{\centering \includegraphics{thesis_files/figure-latex/plotexpfamff-1} 

}

\caption[Expression tissulaire du gène PLCZ1]{\textbf{\emph{Expression tissulaire du gène
\emph{PLCZ1}}} : Données provenant du projet de transcriptome Illumina
bodyMap}\label{fig:plotexpfamff}
\end{figure}





\begin{figure}

{\centering \includegraphics[scale=.37]{figure/pawp_coverage} 

}

\caption[Couverture des 6 éxons de WBP2NL pour les deux frères de la famille FF]{\textbf{\emph{Couverture des 6 éxons de \emph{WBP2NL}
pour les deux frères de la famille FF}} : Les barres rouges représentent
la couverture moyenne de 10 nucléotides, les bleues représentent les
bornes de chaque éxon.}\label{fig:plotcovplcz}
\end{figure}






\newpage

\subsection{Article n° 5}\label{article-n-5}

\subsubsection{Whole-exome sequencing of familial cases of multiple
morphological abnormalities of the sperm flagella (MMAF) reveals new
DNAH1
mutations}\label{whole-exome-sequencing-of-familial-cases-of-multiple-morphological-abnormalities-of-the-sperm-flagella-mmaf-reveals-new-dnah1-mutations}

Amiri-Yekta A\textsuperscript{*}, Coutton C\textsuperscript{*}, Kherraf
ZE, \textbf{Karaouzène T}, Le Tanno P, Sanati MH, Sabbaghian M, Almadani
N, Sadighi Gilani MA, Seyedeh Hanieh Hosseini, Bahrami S, Daneshipour A,
Bini M, Arnoult C, Colombo R, Gourabi H, Ray PF

\textsuperscript{*} Co-premiers auteurs

Human Reproduction, Octobre 2016

\newpage

\subsubsection{Contexte et objectifs}\label{contexte-et-objectifs-2}

Dans une étude précédente non détaillée dans ce manuscrit (en
\protect\hyperlink{dnah12014}{annexe}), notre équipe a pu identifier
\emph{DNAH1} comme le premier gène codant pour une dynéine axonèmale
responsable uniquement d'infertilité masculine. Dans cette première
étude, 5 de nos 18 patients non-apparentés, soit environ 28\% d'entre
eux, étaient porteur d'une mutation homozygote sur le gène \emph{DNAH1}
responsable de leur phénotype MMAF. Ces résultats ont ainsi démontrés
l'importance de l'implication de ce gène dans ce phénotype.

Dans cette nouvelle étude, nous nous concentrons sur l'analyse des
données génétiques de 5 familles iraniennes et d'une famille italienne
soit un total de 12 individus parmi lesquels 10 ont été séquencés en
WES. Cependant, pour des raisons techniques, seules les données de 9
d'entre eux ont été analysées sur notre pipeline décrit précédemment.

Dans ce contexte j'ai effectué les l'ensemble des analyses
bioinformatiques de ces patients.

\newpage

\includepdf[pages=-]{bib/Fam_DNAH1_2016.pdf}

\newpage

\subsubsection{Principaux résultats}\label{principaux-resultats-2}

Après avoir été séquencés en WES, les données des 9 patients
\emph{Ghs119}, \emph{Ghs130}, \emph{Ghs131}, \emph{Ghs56}, \emph{Ghs58},
\emph{Ghs59}, \emph{Ghs60}, \emph{Ghs62} et \emph{Ghs63} ont été
analysés au sein de notre pipeline. Compte tenu de l'historique de
consanguinité de ces familles, l'ensemble des variants hétérozygotes ont
été filtrés de même que l'ensemble des variants observés fréquemment
dans la population générale et ceux n'ayant aucun impact sur la séquence
codante. Aussi, dans les cas où les données de WES de plusieurs frères
d'une même famille étaient disponibles, seuls les variants partagés par
l'ensemble des frères ont été gardés. À l'issus de cette étape de
filtrage, seuls quelques variants subsistaient pour chacune des familles
impactant entre 1 et 23 gènes différents en fonction de celles-ci
(\textbf{Tables :} \ref{tab:tabmmaf1}, \ref{tab:tabmmaf2},
\ref{tab:tabmmaf3}, \ref{tab:tabmmaf4}, \ref{tab:tabmmaf5},
\textbf{Figure : }\ref{fig:plotremaininggenes}).

Parmi cette liste de gènes, \emph{DNAH1} fut retrouvé mutés chez les
deux frères de la famille MMAF3 ainsi que chez les deux frères de la
famille MMAF6 (non analysés par notre pipeline). De même, un variant
entrainant un décalage du cadre de lecture dans la séquence du gène
\emph{SPEF2} (codant pour la protéine \emph{Sperm flagellar 2} (SPEF2))
a été retrouvé chez le patient P10 de la famille MMAF5. Aucun autre
candidat évident n'a pu être identifié pour les individus composant les
3 autres familles. Ensuite, bien que le gène \emph{SPEF2} ait déjà été
caractérisé comme ayant un rôle dans la biogénèse du flagelle
spermatique {[}\protect\hyperlink{ref-Lehti2017}{24}{]} nous nous sommes
dans un premiers temps concentrés sur la caractérisation des deux
variants retrouvés sur \emph{DNAH1}.

\begin{enumerate}
\def\labelenumi{\arabic{enumi}.}
\item
  \textbf{Famille MMAF3 :} Les deux frères P5 et P6 analysés en WES
  était tout deux porteur de la même mutation c.8626-1G \textgreater{} A
  qui fut par la suite confirmée en Sanger pour ces deux patients ainsi
  que pour leur troisième frère (P7) non analysé en WES. Cette mutation,
  absente des différentes bases de données, impacte le dernier
  nucléotide de l'intron 54 de \emph{DNAH1} soit l'une des deux bases
  composant le site accepteur consensus d'épissage. Afin d'évaluer
  l'impact de cette mutation sur le transcrit de \emph{DANH1}, nous
  avons étudiés par RT-PCR l'ARNm provenant de ces trois frères ainsi
  que de 2 individus contrôles. Cette étude a révélés qu'aucune
  amplification du transcrit de \emph{DNAH1} n'était observée chez les
  trois frères contrairement aux deux contrôles tandis que
  l'amplification ciblant \emph{GAPDH} était positif pour les 5
  individus confirmant ainsi l'intégrité de l'ARNm de l'ensemble des
  individus testés. Ces résultats suggèrent donc que les transcrits
  produits par les trois frères mutés ont été soumis au mécanisme de
  dégradation spécifique par \emph{mRNA decay}. Afin de valider la
  pathogénicité de ce variant, la protéine DNAH a ensuite été localisée
  par immunofluorescence à la fois chez les patients et les contrôles
  révélant que contrairement aux contrôles, la protéine DNAH1 était
  absente chez les trois frères, renforçant ainsi l'hypothèse d'une
  dégradation spécifique des ARNm.
\item
  \textbf{Famille MMAF6 :} Le variant c.3860 T \textgreater{} G
  (p.Val1287Gly) induisant une mutation faux sens dans la séquence de
  l'éxon 23 de \emph{DNAH1} a été retrouvé dans les données WES des deux
  frères P11 et P12 (non analysés par notre pipeline) et a par la suite
  été confirmée en Sanger. Malheureusement, par manque de materiel,
  aucune étude par RT-PCR ou immunofluorescence n'a pu être effectuée
  sur ces patients. Cependant, le fait que ce variant faux-sens soit
  absent des bases de données et qu'il soit prédit comme \emph{probably
  dammaging} par PolyPhen et \emph{deleterious} par SIFT renforce
  l'hypothèse de la pathogénicité de ce variant.
\end{enumerate}

Ainsi, à l'issus de l'analyse de ces 6 familles présentant un phénotype
MMAF (dont seulement 5 ont été analysées sur notre pipeline), le variant
responsable a pu être déterminé pour deux d'entre elles grâce à
l'identification de deux nouveaux variants impactant la séquence codante
du gène \emph{DNAH1} confirmant ainsi l'importance de l'implication de
ce gène dans le phénotype MMAF. De même un indel entrainant un décalage
du cadre de lecture dans la séquence du gène \emph{SPEF2} fait de ce
gène, déjà connu comme ayant un rôle dans la formation des flagelles, un
excellent candidat pour expliquer le phénotype du patient P10, notre
équipe travaille à l'heure actuelle à la caractérisation de ce gène.
Cependant, aucun candidat évident n'a pu être identifié pour les 3
familles restante laissant supposer que le variant responsable de leur
phénotype ait été éliminé par un de nos filtres. Afin d'identifier la
cause génétique de leur phénotype soit n'a pas été détecté soit a été
éliminé par un de nos filtres. Afin d'identifier la cause génétique de
leur phénotype, nous ré-analysons actuellement les données de ces
patients en appliquant des filtres moins stringents, en conservant les
gènes sur lesquels deux variants hétérozygotes sont retrouvés, ce qui
pourrait être la signature d'une hétérozygotie composite.

\newpage

\begin{figure}

{\centering \includegraphics{thesis_files/figure-latex/plotremaininggenes-1} 

}

\caption[Nombre de gènes passant l'ensemble des filtres par famille]{\textbf{\emph{Nombre de gènes passant
l'ensemble des filtres par famille}} : Chaque barre représente une des
familles analysées. La hauteur de cette barre correspond au nombre de
gènes ayant passé l'ensemble des filtres pour chaque famille. Les barres
vertes caractérisent les familles pour lesquelles le gène responsable de
la pathologie a été identifié parmi la liste de gènes (dans ce cas le
symbole du gène est écrit au-dessus de la barre). La barre orange
caractérise la famille pour laquelle un candidat potentiel a été
identifié (le symbole du gène est écrit au-dessus suivit d'un ``?'').
Les barres rouges indiquent qu'aucun des gènes ayant passé les filtres
pour ne semble expliquer le phénotype (dans ce cas il est écrit ``???''
au-dessus de la barre).}\label{fig:plotremaininggenes}
\end{figure}














\newpage

\section{Résultats 2 : Étude d'une cohorte de femmes
infertiles}\label{resultats-2-etude-dune-cohorte-de-femmes-infertiles}

\subsection{Article n° 6}\label{article-n-6}

\textbf{PATL2 Gene Mutation Causes Oocyte Meiotic Deficiency and Female
Infertility}

Christou-Kent M, Amiri-Yekta A, Kherraf ZE, \textbf{Karaouzène T},
Escoffier J, Guttin A, Martinez G, Le Blévec E, Lambert E, Fourati Ben
Mustapha S, Cedrin-Durnerin I, Halouani L, Marrakchi O, Makni M, Latrous
H, Kharouf M, Bottari S, Thierry-Mieg N, Coutton C, Zouari R, Issartel
JP, Ray PF, Arnoult C

New England Journal of Medicine, 07 Juillet 2017 (soummis)

\newpage

\subsubsection{Contexte et objectifs}\label{contexte-et-objectifs-3}

Entre 2013 et 2014 notre équipe a pris en charge l'étude génétique de 23
femmes nord africaines présentant toutes une déficience méiotique
ovocytaire (DMO) caractérisé par un blocage de la méiose au stade M1 et
entrainant une infertilité. À l'heure actuelle, seul le gène TUBB8
retrouvé muté à l'état hétérozygote chez des patientes chinoise avait pu
être lié à ce phénotype. Cette étude a donc pour objectif de
caractériser la cause génétique responsable du phénotype DMO de ces 23
femmes. Parmi celles-ci, 15 ont été analysées par séquençage haut-débit.
Dans ce contexte, j'ai, au cours de ma thèse, été en charge de
l'ensemble des analyses bioinformatiques de ces 15 femmes.

\newpage

\includepdf[pages=-]{bib/PATL2_2017.pdf}

\newpage

\subsubsection{Principaux résultats}\label{principaux-resultats-3}

L'application de notre pipeline d'analyse sur les données de ces femmes
nous a permis d'obtenir une liste de 316 variants impactant 299 gènes
différents. Parmi ces variants, aucun n'impactait le gène \emph{TUBB8}.
Afin de restreindre à nouveau la liste de gènes, nous nous sommes
concentrés sur ceux retrouvés mutés à l'état homozygote chez au moins 2
femmes. Seul 3 gènes ont passés ce nouveau critère : \emph{FAM58A},
\emph{MGAM} et \emph{PATL2}. Aucune investigation n'a pour l'instant été
effectuée sur les 296 autres. En raison de la fréquence élevée du
variant retrouvé sur \emph{FAM58A} et de l'impact peu délétère du
variant chevauchant \emph{MGAM}, ces deux gènes ont été considéré comme
de mauvais candidats.

Ainsi, nous nous sommes dans un premier temps concentrés sur la
caractérisation de \emph{PATL2} dont l'orthologue \emph{xpat1a} chez le
xénope a été décrit comme étant exprimé au cours du développement de
l'ovocyte {[}\protect\hyperlink{ref-Marnef2010}{25},
\protect\hyperlink{ref-Nakamura2010}{26}{]} faisant de ce gène un
excellent candidat. Nous avons observé que 5 de nos patientes (33.3\%)
étaient porteuses de la même mutation homozygote :
c.478G\textgreater{}T, p.Arg160Ter induisant un codon stop prématuré
dans la séquence codante du transcrit canonique \emph{PATL2}:
ENST00000434130. Au vu de ces résultats un séquençage Sanger de la
séquence codante de ce gène fut réalisé pour ces 5 femmes afin de
confirmer cette mutation, ainsi que sur 8 femmes supplémentaires
souffrant du même phénotype. Parmi ces dernières, une s'est révélée
porter la même mutation à l'état homozygote. Au final six femmes sur 23
(26\%) étaient homozygotes pour la mutation stop de \emph{PATL2}.

Dans un second temps, l'étude du modèle murin KO
\emph{Patl2}\textsuperscript{-/-} nous a permis de mettre en évidence
une subfertilité importante chez les souris femelles tandis qu'aucun
phénotype n'était observable chez les mâles.

Pour finir, \emph{xpat1a}, l'orthologue de \emph{PATL2} chez le xénope,
ayant été décrit comme réprimant la traduction de l'ARNm dans l'ovocyte
nous avons cherché à savoir si les souris femelles KO
\emph{Patl2}\textsuperscript{-/-} présentait des dérégulation de leur
transcriptome ovocytaire. Pour cela, nous avons procédés à une étude
comparative des transcriptome ovocytaires aux stades GC et MII murins
sur puces Affymetrix mesurant les valeurs d'expression d'environ 66,000
transcrits différents. Ainsi, nous avons pu mettre en évidence 134
transcrits différentiellement exprimés au stade GV parmi lesquels 95
étaient sous-exprimés tandis que 39 étaient sur-exprimés. Au stade MII,
ces dérégulation se révélèrent être plus impressionnantes puisque 124
étaient sous-exprimées et 122 sur-exprimées démontrant ainsi une forte
implication de \emph{Patl2} dans la transcription ovocytaire des gènes
murins.

\newpage  

\section{Résultats 3 : Étude d'une large cohorte de patients
MMAF}\label{resultats-3-etude-dune-large-cohorte-de-patients-mmaf}

\subsection{Article n° 7}\label{article-n-7}

\textbf{Whole exome cohort study and analysis of mouse and Trypanosoma
models demonstrate the importance of WDR proteins in flagellogenesis and
male fertility}

Coutton C, Vargas A, Amiri-Yekta A, Kherraf ZE, Fourati Ben Mustapha S,
Le Tanno P, Wambergue-Legrand C, \textbf{Karaouzène T}, Martinez G,
Daneshipour A, Hanieh Hosseini S, Mitchell V, Halouani L, Marrakchi O,
Makni M, Latrous H, Kharouf M, Deleuze JF, Boland A, Hennebicq S, Satre
V, Jouk PS, Bottari SP, Thierry-Mieg N, Conne B, Dacheux-Deschamps D,
Schmitt A, Stouvenel L, Lorès P, El Khouri E, Fauré J, Wolf JP,
Escoffier J, Gourabi H, Robinson DR, Nef S, Dulioust E, Zouari R,
Bonhivers M, Touré A, Arnoult C, Ray PF

Nature Communications. En révision

\newpage

\subsubsection{Contexte et objectifs}\label{contexte-et-objectifs-4}

Après avoir mis en évidence l'implication du gène \emph{DNAH1} dans le
phénotype MMAF notre équipe s'est en partie spécialisé dans la
caractérisation ce syndrome. Ainsi, entre 2012 et 2016, notre équipe a
effectué le séquençage de 78 individus présentant tous ce phénotype afin
d'en établir la cause génétique. Ces séquençages ont été effectués dans
4 centres différents que sont le Genoscope, Integragen, Novogene et la
plateforme de séquençage de l'IGBMC de Strasbourg. La plupart de ces
séquençages ont été effectués sur des Illumina HiSeq2000 sauf les 25
plus récents qui ont été effectués sur un Illumina HiSeq4000
(\textbf{Table : }\ref{tab:tabrunbigmmaf}).

\begin{longtable}[t]{lrlr}
\caption{\label{tab:tabrunbigmmaf}Liste des différents individus présentant un phénotype MMAF séquencés en WES}\\
\toprule
Place & Year & Platform & Nb of individuals\\
\midrule
IGBMC & 2012 & Illumina HiSeq2000 & 12\\
Genoscope & 2013 & Illumina HiSeq2000 & 12\\
Genoscope & 2014 & Illumina HiSeq2000 & 25\\
Genoscope & 2015 & Illumina HiSeq2000 & 3\\
Integragen & 2016 & Illumina HiSeq4000 & 15\\
Novogene & 2016 & Illumina HiSeq4000 & 10\\
\bottomrule
\end{longtable}

\includepdf[pages=-]{bib/CFAP_2017.pdf}

\newpage

\subsubsection{Principaux résultats}\label{principaux-resultats-4}

L'application de de notre pipeline d'analyse sur les données de ces 78
patients nous a permis d'obtenir une liste de 3630 variant distincts
ayant passés l'ensemble de nos filtres (2903 SNPs et 727 indels),
ceux-ci impactant un total de 2780 gènes différents.

Le gène \emph{DNAH1} étant le candidat évident, nous avons cherché en
priorité l'ensemble des variants retrouvés sur ce gène. Ainsi, nous
avons obtenus une liste de 5 patients portant tous soit au moins un
variant homozygote sur le gène \emph{DNAH1} soit deux variants
hétérozygotes sur ce même gène.

Suite à cela, au vu du nombre important de gènes restant et afin
d'étudier en priorité ceux pouvant expliquer le phénotype d'un maximum
de patients nous avons, limité nos recherches aux gènes sur lesquels
\textbf{au moins 3 patients portaient un variant homozygote tronquant}.
Cela nous a ainsi permis de mettre en évidence les gènes \emph{CFAP43}
et \emph{CFAP44} sur lesquels des variants homozygotes ont été retrouvés
chez respectivement 9 et 6 patients auxquels viennent s'ajouter 1
patients portant deux variants hétérozygotes sur le gène \emph{CFAP43}
(\textbf{Tables : }\ref{tab:tabcfap43} et \ref{tab:tabcfap44}). Ces deux
gènes CFAP (pour \emph{Cilia and Flagella Associated Protein}) avaient
déjà été répertoriés dans les bases de données publiques comme ayant une
forte expression testiculaire, et comme étant probablement impliqués
dans la structure et / ou fonction du flagelle spermatique
{[}\protect\hyperlink{ref-Ivliev2012}{27}{]}. De plus, ces deux gènes
codent tous deux pour des protéines appartenant à la famille des WDR et
comportent tous deux neuf répétitions WD (tryptophane - acide
aspartique) {[}\protect\hyperlink{ref-Smith2008}{28}{]}. Ainsi, en
tenant compte du nombre important de patients portant des variants sur
un de ces deux gènes et le fait qu'ils codent tous deux pour des
protéines appartenant à la même famille, nous avons décidé de nous
concentrer dans un premiers temps à la caractérisation de ces deux seuls
gènes, ceux-ci étant les meilleurs candidats pour expliquer le phénotype
d'infertilité de 16 de nos patients.

Un total de 9 variants différents ont ainsi put être identifiés sur le
gène \emph{CFAP43} impactant 10 de nos patients n'ayant, à notre
connaissance aucun lien de parenté. Huit de ces variants ont un effet
tronquant évident et le dernier est un variants intronique localisé 5
nucléotides après l'exon 16, non listé dans ExAC et prédit par Human
Splicing Finder (\url{http://www.umd.be/HSF3}) comme altérant l'épissage
de l'exon 16 de CFAP43.

Pour \emph{CFAP44}, 6 de nos patients portaient des variants homozygotes
ayant tous un effet tronquant.

L'analyse au microscope électronique à transmission des cellules
spermatiques d'un patient portant un variant sur \emph{CFAP43} et d'un
autre portant un variant sur \emph{CFAP44} révéla des défauts au niveau
de l'axonème ainsi qu'une gaine fibreuse désorganisée pour chacun des
deux patients.

Ensuite, afin de compenser l'absence d'anticorps anti-CFAP43 et
anti-CFAP44 fiables chez l'humain comme chez la souris nous avons décidé
de caractériser leur orthologues chez \emph{Trypanosoma brucei}
(\emph{T. brucei}), un protozoaire flagellé utilisé comme organisme
modèle dans l'études des flagelles chez qui, les protéines
\emph{Tb}CFAP43 et \emph{Tb}CFAP44 respectivement orthologues de CFAP43
et CFAP44 avaient déjà été identifiées comme des protéines du flagelles
{[}\protect\hyperlink{ref-Broadhead2006}{29},
\protect\hyperlink{ref-Subota2014}{30}{]}. Ensuite, l'utilisation d'ARN
interférence nous a permis de produire des organismes \emph{knock-down}
pour ces deux gènes \emph{TbCFAP43}\textsuperscript{RAi} et
\emph{TbCFAP44}\textsuperscript{RNAi} nous permettant ainsi d'évaluer la
fonction de ces deux gènes au sein du flagelle du trypanosome. Cela nous
a permis d'observer un arrêt de la prolifération cellulaire au bout de
24h ainsi que de nombreux défauts au niveau des flagelles pour
l'ensemble des lignées cellulaires \emph{TbCFAP43}\textsuperscript{RAi}
et \emph{TbCFAP44}\textsuperscript{RNAi}. Cet arrêt de prolifération est
typique des problèmes flagellaire chez le Trypanosome qui dépends de son
mouvement pour sa survie.

Pour finir, l'impact de l'absence des protéines CFAP43 et 44 sur la
spermatogénèse murine a été déterminé grâce à la génération de modèle KO
utilisant la technologie CRISPR-Cas9 qui nous a permis d'obtenir des
phénotypes reproductibles pour nos deux modèles de souris KO. Ces
modèles nous ont permis d'observer que les mâles
\emph{Cfap43}\textsuperscript{-/-} et \emph{Cfap44}\textsuperscript{-/-}
présentaient tous deux de nombreuses anomalies au niveau des flagelles
tandis que les femelles \emph{Cfap43}\textsuperscript{-/-} et
\emph{Cfap44}\textsuperscript{-/-} étaient parfaitement fertiles.

Pour conclure, cette étude portant sur la caractérisation du phénotype
MMAF nous a permis d'identifiée la cause génétique de 21 (26.9\%) de nos
patients, ceci n'ayant aucun lien de parenté. L'utilisation de notre
pipeline pour l'analyse des données NGS nous a permis à la fois de
confirmer l'importance du gène \emph{DNAH1} dans la structure du
flagelle et son implication dans ce phénotype, mais aussi d'identifier
deux nouveaux gènes, \emph{CFAP43} et \emph{CFAP44} respectivement
responsables du phénotype de 10 et 6 de nos patients soit 12.8 et 7.7\%
de notre cohorte.

\appendix

\hypertarget{dnah12014}{\chapter{Mutations in DNAH1, which Encodes an
Inner Arm Heavy Chain Dynein, Lead to Male Infertility from Multiple
Morphological Abnormalities of the Sperm Flagella}\label{dnah12014}}

Ben Khelifa M\textsuperscript{*}, Coutton C\textsuperscript{*}, Zouari
Raoudha, \textbf{Karaouzène T}, Rendu J, Bidart M, Yassine S, Pierre V,
Delaroche J, Hennebicq S, Grunwald D, Escalier D, Pernet-Gallay K, Jouk
PS, Thierry-Mieg N, Touré A, Arnoult C, Ray PF

American Journal of Human Genetics, Janvier 2014

\textsuperscript{*} Co-premiers auteurs

\newpage

\includepdf[pages=-]{bib/DNAH1_2014.pdf}

\newpage

\newpage

\chapter{Tables des variants restant après application des filtres pour
les cas
familliaux}\label{tables-des-variants-restant-apres-application-des-filtres-pour-les-cas-familliaux}

\begin{longtable}[t]{lll}
\caption{\label{tab:tabmmaf1}Liste des variants ayant passé l'ensemble des filtres pour les deux fères P1 et P2 de la famille MMAF1}\\
\toprule
\multicolumn{1}{c}{ } & \multicolumn{2}{c}{Variant impact} \\
\cmidrule(l{2pt}r{2pt}){2-3}
SYMBOL & HGVSc, HGVSp & Consequence\\
\midrule
PLA2G4B & c.1710-6delA ; . & splice region\\
\bottomrule
\end{longtable}

\begin{longtable}[t]{lll}
\caption{\label{tab:tabmmaf2}Liste des variants ayant passé l'ensemble des filtres pour les deux fères P3 et P4 de la famille MMAF1}\\
\toprule
\multicolumn{1}{c}{ } & \multicolumn{2}{c}{Variant impact} \\
\cmidrule(l{2pt}r{2pt}){2-3}
SYMBOL & HGVSc, HGVSp & Consequence\\
\midrule
ZNF469 & . ; p.Arg1864Lys & missense\\
HYDIN & c.14857>T ; p.Arg4953Trp & missense\\
DAPK1 & c.2431T>A ; p.Val811Met & missense\\
CCDC37 & c.1234\_1236delTGG ; p.Glu412del & inframe deletion\\
SEMA5B & c.658C>A ; p.Glu220Lys & missense\\
\addlinespace
MMP9 & c.820G>A ; p.Glu274Lys & missense\\
MTSS1L & . ; p.Arg609Trp & missense\\
TMEM231 & . ; p.Ala71Val & missense\\
TGIF2 & c.496C>A ; p.Leu166Met & missense\\
\bottomrule
\end{longtable}

\begin{longtable}[t]{lll}
\caption{\label{tab:tabmmaf3}Liste des variants ayant passé l'ensemble des filtres pour les deux fères P5 et P6 de la famille MMAF3}\\
\toprule
\multicolumn{1}{c}{ } & \multicolumn{2}{c}{Variant impact} \\
\cmidrule(l{2pt}r{2pt}){2-3}
SYMBOL & HGVSc, HGVSp & Consequence\\
\midrule
MYH11 & c.4625G>A ; p.Arg1542Gln & missense\\
DNAH1 & . ; . & splice acceptor\\
\bottomrule
\end{longtable}

\begin{longtable}[t]{lll}
\caption{\label{tab:tabmmaf4}Liste des variants ayant passé l'ensemble des filtres pour les deux fères P8 et P9 de la famille MMAF4}\\
\toprule
\multicolumn{1}{c}{ } & \multicolumn{2}{c}{Variant impact} \\
\cmidrule(l{2pt}r{2pt}){2-3}
SYMBOL & HGVSc, HGVSp & Consequence\\
\midrule
WEE2 & . ; p.Pro92Leu & missense\\
GBP2 & c.412T>A ; p.Ala138Thr & missense\\
ZFYVE28 & c.1729C>A ; p.Val577Met & missense\\
FCGR3A & c.133T>C ; p.Ala45Pro & missense\\
\bottomrule
\end{longtable}

\newpage

\begin{longtable}[t]{lll}
\caption{\label{tab:tabmmaf5}Liste des variants ayant passé l'ensemble des filtres pour le patient P10 de la famille MMAF5}\\
\toprule
\multicolumn{1}{c}{ } & \multicolumn{2}{c}{Variant impact} \\
\cmidrule(l{2pt}r{2pt}){2-3}
SYMBOL & HGVSc, HGVSp & Consequence\\
\midrule
ALOX15 & c.268T>C ; p.Asp90His & missense\\
NFATC4 & . ; p.Glu577Gly & missense\\
MYH7 & c.77G>T ; p.Ala26Val & missense\\
LDHAL6A & c.334A>T ; p.Arg112Ter & stop gained\\
GBF1 & c.1588G>T ; p.Arg530Cys & missense\\
\addlinespace
TRBV6-6 & c.5A>T ; p.Ser2Ile & missense\\
MROH2A & . ; p.Ile180Thr & missense\\
FSIP2 & . ; p.Ala86Val & missense\\
ZSWIM2 & c.676G>A ; p.Leu226Met & missense\\
CPZ & c.1236A>T ; p.Lys412Asn & missense\\
\addlinespace
CCDC73 & c.851A>C ; p.Phe284Ser & missense\\
NXPE2 & c.1595>T ; p.Thr532Met & missense\\
NKX2-1 & c.843G>T ; p.Gln281His & missense\\
CASP8 & c.74A>G ; p.Pro25Arg & missense\\
OR6S1 & . ; p.Leu116Met & missense\\
\addlinespace
PDZD7 & . ; p.Ser324Asn & missense\\
COL27A1 & c.5063G>A ; p.Arg1688Gln & missense\\
SPEF2 & c.3240delG ; p.Phe1080LeufsTer2 & frameshift\\
ACAP1 & c.1597A>T ; p.Arg533Trp & missense\\
C6 & c.2312C>A ; p.Gly771Asp & missense\\
\addlinespace
SLIT1 & . ; . & splice region\\
ARHGAP19-SLIT1 & c.*1721-7A>T ; . & splice region\\
DNAH2 & . ; p.Arg2538Gln & missense\\
\bottomrule
\end{longtable}

\chapter{\texorpdfstring{Table des variants retrouvés sur le gène
\emph{PATL2}}{Table des variants retrouvés sur le gène PATL2}}\label{table-des-variants-retrouves-sur-le-gene-patl2}

\begin{longtable}[t]{llll}
\caption{\label{tab:tabpatl2}Table des variants retrouvés sur le gène *PATL2*}\\
\toprule
\multicolumn{2}{c}{ } & \multicolumn{2}{c}{Variant impact} \\
\cmidrule(l{2pt}r{2pt}){3-4}
Patient & Geno & HGVSc, HGVSp & Consequence\\
\midrule
Ghs103 & Homo & c.478G>T ; p.Arg160Ter & stop gained\\
Ghs104 & Homo & c.478G>T ; p.Arg160Ter & stop gained\\
Ghs114 & Homo & c.478G>T ; p.Arg160Ter & stop gained\\
Ghs5 & Homo & c.478G>T ; p.Arg160Ter & stop gained\\
Ghs98 & Homo & c.478G>T ; p.Arg160Ter & stop gained\\
\bottomrule
\end{longtable}

\newpage

\chapter{Variants retrouvés au sein de notre cohorte de patients
MMAF}\label{variants-retrouves-au-sein-de-notre-cohorte-de-patients-mmaf}

\newpage

\begin{landscape}
\begin{longtable}[t]{llllll}
\caption{\label{tab:tabcfap43}Variants homozygotes retrouvés sur le gène *CFAP43*}\\
\toprule
\multicolumn{2}{c}{ } & \multicolumn{2}{c}{Variant impact} & \multicolumn{1}{c}{Variant frequency} \\
\cmidrule(l{2pt}r{2pt}){3-4} \cmidrule(l{2pt}r{2pt}){5-5}
Patient & Geno & HGVSc, HGVSp & Consequence & SIFT ; PolyPhen & ExAC ; ESP ; 1KG\\
\midrule
Ghs162 & Homo & c.1240\_1241delGC ; p.Val414LeufsTer46 & frameshift & . ; . & . ; . ; .\\
Ghs164 & Homo & c.1240\_1241delGC ; p.Val414LeufsTer46 & frameshift & . ; . & . ; . ; .\\
Ghs25 & Homo & c.2141+5T>A ; p.Lys714Val*11 & splice region & . ; . & . ; . ; .\\
Ghs17 & Homo & c.2658G>A ; p.Trp886Ter & stop gained & . ; . & 9.88e-05 ; 2e-04 ; .\\
Ghs41 & Homo & c.2680C>T ; p.Arg894Ter & stop gained & . ; . & 8.24e-06 ; . ; .\\
\addlinespace
Ghs126 & Homo & c.3352C>T ; p.Arg1118Ter & stop gained & . ; . & 3.29e-05 ; . ; .\\
Ghs105 & Homo & c.3541-2A>C ; p.Ser1181Lysfs*4 & splice acceptor & . ; . & . ; . ; .\\
Ghs160 & Homo & c.3541-2A>C ; p.Ser1181Lysfs*4 & splice acceptor & . ; . & . ; . ; .\\
Ghs102 & Homo & c.3882delA ; p.Glu1294AspfsTer47 & frameshift & . ; . & . ; . ; .\\
Ghs132 & Hete & c.1040G>C ; p.Val347Ala & missense & tolerated ; possibly damaging & 7.41e-05 ; 2e-04 ; .\\
Ghs132 & Hete & c.1300\_1301insT ; p.Leu435SerfsTer26 & frameshift & . ; . & . ; . ; .\\
\bottomrule
\end{longtable}
\end{landscape}

\newpage

\begin{longtable}[t]{llll}
\caption{\label{tab:tabcfap44}Variants homozygotes retrouvés sur le gène *CFAP44*}\\
\toprule
\multicolumn{1}{c}{ } & \multicolumn{2}{c}{Variant impact} & \multicolumn{1}{c}{Variant frequency} \\
\cmidrule(l{2pt}r{2pt}){2-3} \cmidrule(l{2pt}r{2pt}){4-4}
Patient & HGVSc, HGVSp & Consequence & ExAC ; ESP ; 1KG\\
\midrule
Ghs155 & c.1387G>T ; p.Glu463Ter & stop gained & . ; . ; .\\
Ghs177 & c.1387G>T ; p.Glu463Ter & stop gained & . ; . ; .\\
Ghs34 & c.1890+1G>A ; p.Pro631Ile*22 & splice donor & . ; . ; .\\
Ghs168 & c.2818\_2819insG ; p.Glu940GlyfsTer19 & frameshift & 8.24e-06 ; . ; .\\
Ghs22 & c.3175C>T ; p.Arg1059Ter & stop gained & . ; . ; .\\
Ghs181 & c.4767delC ; p.Ile1589MetfsTer6 & frameshift & . ; . ; .\\
\bottomrule
\end{longtable}

\hypertarget{refs}{}
\hypertarget{ref-Amberger2011}{}
1. J. Amberger, C. Bocchini, and A. Hamosh: ``A new face and new
challenges for Online Mendelian Inheritance in Man (OMIM).'' \emph{Human
Mutation}. vol. 32, no. 5, pp. 564--567, 2011.

\hypertarget{ref-Ng}{}
2. S.B. Ng, K.J. Buckingham, C. Lee, A.W. Bigham, H.K. Tabor, K.M. Dent,
C.D. Huff, P.T. Shannon, E.W. Jabs, D.A. Nickerson, J. Shendure, and
M.J. Bamshad: ``Exome sequencing identifies the cause of a Mendelian
disorder.''

\hypertarget{ref-Pelak2010}{}
3. K. Pelak, K.V. Shianna, D. Ge, J.M. Maia, M. Zhu, J.P. Smith, E.T.
Cirulli, J. Fellay, S.P. Dickson, C.E. Gumbs, E.L. Heinzen, A.C. Need,
E.K. Ruzzo, A. Singh, C.R. Campbell, L.K. Hong, K.A. Lornsen, A.M.
McKenzie, N.L.M. Sobreira, J.E. Hoover-Fong, J.D. Milner, R. Ottman,
B.F. Haynes, J.J. Goedert, and D.B. Goldstein: ``The characterization of
twenty sequenced human genomes.'' \emph{PLoS genetics}. vol. 6, no. 9,
pp. e1001111, 2010.

\hypertarget{ref-McLaren2016}{}
4. W. McLaren, L. Gil, S.E. Hunt, H.S. Riat, G.R.S. Ritchie, A.
Thormann, P. Flicek, and F. Cunningham: ``The Ensembl Variant Effect
Predictor.'' \emph{Genome biology}. vol. 17, no. 1, pp. 122, 2016.

\hypertarget{ref-Cingolani2012}{}
5. P. Cingolani, A. Platts, L.L. Wang, M. Coon, T. Nguyen, L. Wang, S.J.
Land, X. Lu, and D.M. Ruden: ``A program for annotating and predicting
the effects of single nucleotide polymorphisms, SnpEff.'' \emph{Fly}.
vol. 6, no. 2, pp. 80--92, 2012.

\hypertarget{ref-Wang2010}{}
6. K. Wang, M. Li, and H. Hakonarson: ``ANNOVAR: functional annotation
of genetic variants from high-throughput sequencing data.''
\emph{Nucleic Acids Research}. vol. 38, no. 16, pp. e164--e164, 2010.

\hypertarget{ref-Robinson2014}{}
7. P.N. Robinson, S. Köhler, A. Oellrich, S.M.G. Sanger Mouse Genetics
Project, K. Wang, C.J. Mungall, S.E. Lewis, N. Washington, S. Bauer, D.
Seelow, P. Krawitz, C. Gilissen, M. Haendel, and D. Smedley: ``Improved
exome prioritization of disease genes through cross-species phenotype
comparison.'' \emph{Genome research}. vol. 24, no. 2, pp. 340--8, 2014.

\hypertarget{ref-Su2014}{}
8. Z. Su, P.P. Łabaj, S.S. Li, J. Thierry-Mieg, D. Thierry-Mieg, W. Shi,
C. Wang, G.P. Schroth, R. a Setterquist, J.F. Thompson, W.D. Jones, W.
Xiao, W. Xu, R.V. Jensen, R. Kelly, J. Xu, A. Conesa, C. Furlanello,
H.H. Gao, H. Hong, N. Jafari, S. Letovsky, Y. Liao, F. Lu, E.J. Oakeley,
Z. Peng, C.A. Praul, J. Santoyo-Lopez, A. Scherer, T. Shi, G.K. Smyth,
F. Staedtler, P. Sykacek, X.-X. Tan, E.A. Thompson, J. Vandesompele,
M.D. Wang, J.J.J. Wang, R.D. Wolfinger, J. Zavadil, S.S. Auerbach, W.
Bao, H. Binder, T. Blomquist, M.H. Brilliant, P.R. Bushel, W. Cai, J.G.
Catalano, C.-W. Chang, T. Chen, G. Chen, R. Chen, M. Chierici, T.-M.
Chu, D.-A. Clevert, Y. Deng, A. Derti, V. Devanarayan, Z. Dong, J.
Dopazo, T. Du, H. Fang, Y. Fang, M. Fasold, A. Fernandez, M. Fischer, P.
Furió-Tari, J.C. Fuscoe, F. Caimet, S. Gaj, J. Gandara, H.H. Gao, W. Ge,
Y. Gondo, B. Gong, M. Gong, Z. Gong, B. Green, C. Guo, L.-W.L. Guo,
L.-W.L. Guo, J. Hadfield, J. Hellemans, S. Hochreiter, M. Jia, M. Jian,
C.D. Johnson, S. Kay, J. Kleinjans, S. Lababidi, S. Levy, Q.-Z. Li, L.
Li, P. Li, Y. Li, H. Li, J. Li, S.S. Li, S.M. Lin, F.J. López, X. Lu, H.
Luo, X. Ma, J. Meehan, D.B. Megherbi, N. Mei, B. Mu, B. Ning, A. Pandey,
J. Pérez-Florido, R.G. Perkins, R. Peters, J.H. Phan, M. Pirooznia, F.
Qian, T. Qing, L. Rainbow, P. Rocca-Serra, L. Sambourg, S.-A. Sansone,
S. Schwartz, R. Shah, J. Shen, T.M. Smith, O. Stegle, N. Stralis-Pavese,
E. Stupka, Y. Suzuki, L.T. Szkotnicki, M. Tinning, B. Tu, J. van Delft,
A. Vela-Boza, E. Venturini, S.J. Walker, L. Wan, W. Wang, J.J.J. Wang,
J.J.J. Wang, E.D. Wieben, J.C. Willey, P.-Y. Wu, J. Xuan, Y. Yang, Z.
Ye, Y. Yin, Y. Yu, Y.-C. Yuan, J. Zhang, K.K. Zhang, W.W. Zhang, W.W.
Zhang, Y. Zhang, C. Zhao, Y. Zheng, Y. Zhou, P. Zumbo, W. Tong, D.P.
Kreil, C.E. Mason, and L. Shi: ``A comprehensive assessment of RNA-seq
accuracy, reproducibility and information content by the Sequencing
Quality Control Consortium.'' \emph{Nature Biotechnology}. vol. 32, no.
9, pp. 903--14, 2014.

\hypertarget{ref-Nielsen2011}{}
9. R. Nielsen, J.S. Paul, A. Albrechtsen, and Y.S. Song: ``Genotype and
SNP calling from next-generation sequencing data.'' \emph{Nature
reviews. Genetics}. vol. 12, no. 6, pp. 443--51, 2011.

\hypertarget{ref-DePristo2011}{}
10. M.A. DePristo, E. Banks, R. Poplin, K.V. Garimella, J.R. Maguire, C.
Hartl, A.A. Philippakis, G. del Angel, M.A. Rivas, M. Hanna, A. McKenna,
T.J. Fennell, A.M. Kernytsky, A.Y. Sivachenko, K. Cibulskis, S.B.
Gabriel, D. Altshuler, M.J. Daly, S. Keenan, M. Komorowska, E. Kulesha,
I. Longden, T. Maurel, W. McLaren, M. Muffato, R. Nag, B. Overduin, M.
Pignatelli, B. Pritchard, and E. Pritchard: ``A framework for variation
discovery and genotyping using next-generation DNA sequencing data.''
\emph{Nature Genetics}. vol. 43, no. 5, pp. 491--498, 2011.

\hypertarget{ref-Lunter2011}{}
11. G. Lunter and M. Goodson: ``Stampy: A statistical algorithm for
sensitive and fast mapping of Illumina sequence reads.'' \emph{Genome
Research}. vol. 21, no. 6, pp. 936--939, 2011.

\hypertarget{ref-Robinson2011}{}
12. J.T. Robinson, H. Thorvaldsdóttir, W. Winckler, M. Guttman, E.S.
Lander, G. Getz, J.P. Mesirov, L. Pachter, J. Aach, E. Leproust, K.
Eggan, G. Church, H. Li, Y. Lu, X. Fang, H. Liang, Z. Du, D. Li, Y.
Zhao, Y. Hu, Z. Yang, H. Zheng, I. Hellmann, M. Inouye, J. Pool, X. Yi,
J. Zhao, J. Duan, Y. Zhou, and J. Qin: ``Integrative genomics viewer.''
\emph{Nature Biotechnology}. vol. 29, no. 1, pp. 24--26, 2011.

\hypertarget{ref-Aken2017}{}
13. B.L. Aken, P. Achuthan, W. Akanni, M.R. Amode, F. Bernsdorff, J.
Bhai, K. Billis, D. Carvalho-Silva, C. Cummins, P. Clapham, L. Gil, C.G.
Girón, L. Gordon, T. Hourlier, S.E. Hunt, S.H. Janacek, T. Juettemann,
S. Keenan, M.R. Laird, I. Lavidas, T. Maurel, W. McLaren, B. Moore, D.N.
Murphy, R. Nag, V. Newman, M. Nuhn, C.K. Ong, A. Parker, M. Patricio,
H.S. Riat, D. Sheppard, H. Sparrow, K. Taylor, A. Thormann, A. Vullo, B.
Walts, S.P. Wilder, A. Zadissa, M. Kostadima, F.J. Martin, M. Muffato,
E. Perry, M. Ruffier, D.M. Staines, S.J. Trevanion, F. Cunningham, A.
Yates, D.R. Zerbino, and P. Flicek: ``Ensembl 2017.'' \emph{Nucleic
acids research}. vol. 45, no. D1, pp. D635--D642, 2017.

\hypertarget{ref-Lek2016}{}
14. M. Lek, K.J. Karczewski, E.V. Minikel, K.E. Samocha, E. Banks, T.
Fennell, A.H. O'Donnell-Luria, J.S. Ware, A.J. Hill, B.B. Cummings, T.
Tukiainen, D.P. Birnbaum, J.A. Kosmicki, L.E. Duncan, K. Estrada, F.
Zhao, J. Zou, E. Pierce-Hoffman, J. Berghout, D.N. Cooper, N. Deflaux,
M. DePristo, R. Do, J. Flannick, M. Fromer, L. Gauthier, J. Goldstein,
N. Gupta, D. Howrigan, A. Kiezun, M.I. Kurki, A.L. Moonshine, P.
Natarajan, L. Orozco, G.M. Peloso, R. Poplin, M.A. Rivas, V.
Ruano-Rubio, S.A. Rose, D.M. Ruderfer, K. Shakir, P.D. Stenson, C.
Stevens, B.P. Thomas, G. Tiao, M.T. Tusie-Luna, B. Weisburd, H.-H. Won,
D. Yu, D.M. Altshuler, D. Ardissino, M. Boehnke, J. Danesh, S. Donnelly,
R. Elosua, J.C. Florez, S.B. Gabriel, G. Getz, S.J. Glatt, C.M. Hultman,
S. Kathiresan, M. Laakso, S. McCarroll, M.I. McCarthy, D. McGovern, R.
McPherson, B.M. Neale, A. Palotie, S.M. Purcell, D. Saleheen, J.M.
Scharf, P. Sklar, P.F. Sullivan, J. Tuomilehto, M.T. Tsuang, H.C.
Watkins, J.G. Wilson, M.J. Daly, D.G. MacArthur, and D.G. Exome
Aggregation Consortium: ``Analysis of protein-coding genetic variation
in 60,706 humans.'' \emph{Nature}. vol. 536, no. 7616, pp. 285--91,
2016.

\hypertarget{ref-1000GenomesProjectConsortium2015}{}
15. T.1.G.P. 1000 Genomes Project Consortium, A. Auton, L.D. Brooks,
R.M. Durbin, E.P. Garrison, H.M. Kang, J.O. Korbel, J.L. Marchini, S.
McCarthy, G.A. McVean, and G.R. Abecasis: ``A global reference for human
genetic variation.'' \emph{Nature}. vol. 526, no. 7571, pp. 68--74,
2015.

\hypertarget{ref-Chang2007}{}
16. Y.-F. Chang, J.S. Imam, and M.F. Wilkinson: ``The Nonsense-Mediated
Decay RNA Surveillance Pathway.'' \emph{Annual Review of Biochemistry}.
vol. 76, no. 1, pp. 51--74, 2007.

\hypertarget{ref-Baker2004}{}
17. K.E. Baker and R. Parker: ``Nonsense-mediated mRNA decay:
terminating erroneous gene expression.'' \emph{Current opinion in cell
biology}. vol. 16, no. 3, pp. 293--9, 2004.

\hypertarget{ref-Kumar2009}{}
18. P. Kumar, S. Henikoff, and P.C. Ng: ``Predicting the effects of
coding non-synonymous variants on protein function using the SIFT
algorithm.'' \emph{Nature protocols}. vol. 4, no. 7, pp. 1073--1081,
2009.

\hypertarget{ref-Adzhubei2010}{}
19. I.A. Adzhubei, S. Schmidt, L. Peshkin, V.E. Ramensky, A. Gerasimova,
P. Bork, A.S. Kondrashov, and S.R. Sunyaev: ``A method and server for
predicting damaging missense mutations.'' \emph{Nature methods}. vol. 7,
no. 4, pp. 248--9, 2010.

\hypertarget{ref-Hotaling2014}{}
20. J. Hotaling and D.T. Carrell: ``Clinical genetic testing for male
factor infertility: current applications and future directions.''
\emph{Andrology}. vol. 2, no. 3, pp. 339--350, 2014.

\hypertarget{ref-Lee2011}{}
21. B. Lee, I. Park, S. Jin, H. Choi, J.T. Kwon, J. Kim, J. Jeong, B.-N.
Cho, E.M. Eddy, and C. Cho: ``Impaired spermatogenesis and fertility in
mice carrying a mutation in the Spink2 gene expressed predominantly in
testes.'' \emph{The Journal of biological chemistry}. vol. 286, no. 33,
pp. 29108--17, 2011.

\hypertarget{ref-Aarabi2014}{}
22. M. Aarabi, H. Balakier, S. Bashar, S.I. Moskovtsev, P. Sutovsky,
C.L. Librach, and R. Oko: ``Sperm content of postacrosomal WW binding
protein is related to fertilization outcomes in patients undergoing
assisted reproductive technology.'' \emph{Fertility and Sterility}. vol.
102, no. 2, pp. 440--447, 2014.

\hypertarget{ref-Aarabi2014a}{}
23. M. Aarabi, H. Balakier, S. Bashar, S.I. Moskovtsev, P. Sutovsky,
C.L. Librach, and R. Oko: ``Sperm-derived WW domain-binding protein,
PAWP, elicits calcium oscillations and oocyte activation in humans and
mice.'' \emph{FASEB journal : official publication of the Federation of
American Societies for Experimental Biology}. vol. 28, no. 10, pp.
4434--40, 2014.

\hypertarget{ref-Lehti2017}{}
24. M.S. Lehti, F.-P. Zhang, N. Kotaja, and A. Sironen: ``SPEF2
functions in microtubule-mediated transport in elongating spermatids to
ensure proper male germ cell differentiation.'' \emph{Development}. vol.
144, no. 14, 2017.

\hypertarget{ref-Marnef2010}{}
25. A. Marnef, M. Maldonado, A. Bugaut, S. Balasubramanian, M. Kress, D.
Weil, and N. Standart: ``Distinct functions of maternal and somatic Pat1
protein paralogs.'' \emph{RNA}. vol. 16, no. 11, pp. 2094--2107, 2010.

\hypertarget{ref-Nakamura2010}{}
26. Y. Nakamura, K.J. Tanaka, M. Miyauchi, L. Huang, M. Tsujimoto, and
K. Matsumoto: ``Translational repression by the oocyte-specific protein
P100 in Xenopus.'' \emph{Developmental biology}. vol. 344, no. 1, pp.
272--83, 2010.

\hypertarget{ref-Ivliev2012}{}
27. A.E. Ivliev, P.A.C. 't Hoen, W.M.C. van Roon-Mom, D.J.M. Peters, and
M.G. Sergeeva: ``Exploring the Transcriptome of Ciliated Cells Using In
Silico Dissection of Human Tissues.'' \emph{PLoS ONE}. vol. 7, no. 4,
pp. e35618, 2012.

\hypertarget{ref-Smith2008}{}
28. T.F. Smith: ``Diversity of WD-Repeat proteins.'' The coronin family
of proteins. pp. 20--30. \emph{Springer New York}, New York, NY (2008).

\hypertarget{ref-Broadhead2006}{}
29. R. Broadhead, H.R. Dawe, H. Farr, S. Griffiths, S.R. Hart, N.
Portman, M.K. Shaw, M.L. Ginger, S.J. Gaskell, P.G. McKean, and K. Gull:
``Flagellar motility is required for the viability of the bloodstream
trypanosome.'' \emph{Nature}. vol. 440, no. 7081, pp. 224--227, 2006.

\hypertarget{ref-Subota2014}{}
30. I. Subota, D. Julkowska, L. Vincensini, N. Reeg, J. Buisson, T.
Blisnick, D. Huet, S. Perrot, J. Santi-Rocca, M. Duchateau, V. Hourdel,
J.-C. Rousselle, N. Cayet, A. Namane, J. Chamot-Rooke, and P. Bastin:
``Proteomic Analysis of Intact Flagella of Procyclic
\textless{}i\textgreater{}Trypanosoma brucei\textless{}/i\textgreater{}
Cells Identifies Novel Flagellar Proteins with Unique Sub-localization
and Dynamics.'' \emph{Molecular \& Cellular Proteomics}. vol. 13, no. 7,
pp. 1769--1786, 2014.


% Index?

\end{document}

