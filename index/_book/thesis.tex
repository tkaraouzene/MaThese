% This is the Reed College LaTeX thesis template. Most of the work
% for the document class was done by Sam Noble (SN), as well as this
% template. Later comments etc. by Ben Salzberg (BTS). Additional
% restructuring and APA support by Jess Youngberg (JY).
% Your comments and suggestions are more than welcome; please email
% them to cus@reed.edu
%
% See http://web.reed.edu/cis/help/latex.html for help. There are a
% great bunch of help pages there, with notes on
% getting started, bibtex, etc. Go there and read it if you're not
% already familiar with LaTeX.
%
% Any line that starts with a percent symbol is a comment.
% They won't show up in the document, and are useful for notes
% to yourself and explaining commands.
% Commenting also removes a line from the document;
% very handy for troubleshooting problems. -BTS

% As far as I know, this follows the requirements laid out in
% the 2002-2003 Senior Handbook. Ask a librarian to check the
% document before binding. -SN

%%
%% Preamble
%%
% \documentclass{<something>} must begin each LaTeX document
\documentclass[12pt,twoside]{reedthesis}
% Packages are extensions to the basic LaTeX functions. Whatever you
% want to typeset, there is probably a package out there for it.
% Chemistry (chemtex), screenplays, you name it.
% Check out CTAN to see: http://www.ctan.org/
%%
\usepackage{graphicx,latexsym}
\usepackage[french]{babel} 
\usepackage{amsmath}
\usepackage{amssymb,amsthm}
\usepackage[dvipsnames]{xcolor} % tk: for more color
\usepackage{xcolor}
\usepackage{eso-pic}
\usepackage{longtable,booktabs,setspace}
\usepackage{chemarr} %% Useful for one reaction arrow, useless if you're not a chem major
\usepackage[hyphens]{url}
\usepackage{tikz}
\usetikzlibrary{calc}
\newcommand\HRule{\rule{\textwidth}{1pt}}
% Added by CII
\usepackage{hyperref}
\usepackage{lmodern}
\usepackage{float}
\floatplacement{figure}{H}
% End of CII addition
\usepackage{rotating}
\usepackage{upgreek} % tk : pour pouvoir utiliser le symbole µ droit (pas en itallic)
\usepackage{lscape}
\newcommand{\blandscape}{\begin{landscape}}
\newcommand{\elandscape}{\end{landscape}}

% Next line commented out by CII
%%% \usepackage{natbib}
% Comment out the natbib line above and uncomment the following two lines to use the new
% biblatex-chicago style, for Chicago A. Also make some changes at the end where the
% bibliography is included.
%\usepackage{biblatex-chicago}
%\bibliography{thesis}


% Added by CII (Thanks, Hadley!)
% Use ref for internal links
\renewcommand{\hyperref}[2][???]{\autoref{#1}}
\def\chapterautorefname{Chapter}
\def\sectionautorefname{Section}
\def\subsectionautorefname{Subsection}
% End of CII addition

% Added by CII
\usepackage{caption}
\captionsetup{width=5in}
% End of CII addition

% \usepackage{times} % other fonts are available like times, bookman, charter, palatino


% To pass between YAML and LaTeX the dollar signs are added by CII
\title{THÈSE}
\author{Thomas Karaouzene}
\labo{}
% The month and year that you submit your FINAL draft TO THE LIBRARY (May or December)
\date{31 octobre 2017}
\division{}
\advisor{Pierre Ray}
%If you have two advisors for some reason, you can use the following
% Uncommented out by CII
\altadvisor{Nicolas Thierry-Mieg}
% End of CII addition

%%% Remember to use the correct department!
\department{Ingénierie de la Santé, de la Cognition et Environnement (EDISCE)}
% if you're writing a thesis in an interdisciplinary major,
% uncomment the line below and change the text as appropriate.
% check the Senior Handbook if unsure.
%\thedivisionof{The Established Interdisciplinary Committee for}
% if you want the approval page to say "Approved for the Committee",
% uncomment the next line
%\approvedforthe{Committee}

% Added by CII
%%% Copied from knitr
%% maxwidth is the original width if it's less than linewidth
%% otherwise use linewidth (to make sure the graphics do not exceed the margin)
\makeatletter
\def\maxwidth{ %
  \ifdim\Gin@nat@width>\linewidth
    \linewidth
  \else
    \Gin@nat@width
  \fi
}
\makeatother

\renewcommand{\contentsname}{Table of Contents}
% End of CII addition

\setlength{\parskip}{0pt}

% Added by CII
  %\setlength{\parskip}{\baselineskip}
  \usepackage[parfill]{parskip}

\providecommand{\tightlist}{%
  \setlength{\itemsep}{0pt}\setlength{\parskip}{0pt}}

\Acknowledgements{

}

\Dedication{

}

\Preface{
This is an example of a thesis setup to use the reed thesis document
class (for LaTeX) and the R bookdown package, in general.
}

\Abstract{

}

	\usepackage{tikz}
% End of CII addition
%%
%% End Preamble
%%
%

\usepackage{amsthm}
\newtheorem{theorem}{Theorem}[section]
\newtheorem{lemma}{Lemma}[section]
\theoremstyle{definition}
\newtheorem{definition}{Definition}[section]
\newtheorem{corollary}{Corollary}[section]
\newtheorem{proposition}{Proposition}[section]
\theoremstyle{definition}
\newtheorem{example}{Example}[section]
\theoremstyle{remark}
\newtheorem*{remark}{Remark}
\begin{document}

% Everything below added by CII
      \maketitle
  
  \frontmatter % this stuff will be roman-numbered
  \pagestyle{empty} % this removes page numbers from the frontmatter

  
      \begin{preface}
      This is an example of a thesis setup to use the reed thesis document
      class (for LaTeX) and the R bookdown package, in general.
    \end{preface}
  
      \hypersetup{linkcolor=black}
    \setcounter{tocdepth}{3}
    \tableofcontents
  
      \listoftables
  
      \listoffigures
  
  
  
  \mainmatter % here the regular arabic numbering starts
  \pagestyle{fancyplain} % turns page numbering back on

  \chapter{Delete line 6 if you only have one
  advisor}\label{delete-line-6-if-you-only-have-one-advisor}
  
  \chapter*{Remerciements}\label{remerciements}
  \addcontentsline{toc}{chapter}{Remerciements}
  
  \chapter*{Résumé}\label{resume}
  \addcontentsline{toc}{chapter}{Résumé}
  
  \chapter{Introduction}\label{introInf}
  
  \chapter{Investigation génétique et physiologique de la
  globozoospermie}\label{globo}
  
  \chapter{Mise en place d'une stratégie pour l'analyse des données
  exomiques -- application en recherche
  clinique}\label{mise-en-place-dune-strategie-pour-lanalyse-des-donnees-exomiques-application-en-recherche-clinique}
  
  \section{Intro}\label{intro}
  
  Comme vu précédemment, l'émergence du séquençage haut débit, avec
  notamment le WGS et le WES, a révolutionné les méthodes de recherche
  dans le cadre d'étude phénotype-génotype en permettant de manière rapide
  et à moindre coup le séquençage de la quasi totalité des gènes humains.
  Les causes de plusieurs centaines de pathologies ont pu être identifiées
  grâce à ces technique depuis leur premier succès pubilié en 2010 (Ng et
  al., n.d.). Dès lors, l'analyse des données issues du séquençage est
  devenu la clef dans la réussite de ces études.
  
  Il existe de nombreux logiciels qui à partir des variants appelés
  effectuent les étapes d'annotation et de filtrage. C'est par exemple le
  cas d'Exomiser {[}TODO: insert ref and Exomiser describtion{]} ou encore
  de {[}TODO: insert at least one other soft{]}. La plupart de ces
  logiciels fonctionnent très bien, cependant tous prennent pour point de
  départ des variants appelés en amont. Ils ne contrôlent donc en aucune
  manière les étapes d'alignement et d'appel des variants. Or, comme il a
  été dit plus tôt, ces deux étapes constituent la bases de l'analyse
  {[}TODO insert ref{]} et les résultats
  
  Dans ce chapitre, je détaillerai les résultats de 4 articles dont je
  suis coauteur :
  
  \begin{enumerate}
  \def\labelenumi{\arabic{enumi}.}
  \tightlist
  \item
    \protect\hyperlink{famdnah1}{\textbf{Whole-exome sequencing of
    familial cases of multiple morphological abnormalities of the sperm
    flagella (MMAF) reveals new DNAH1 mutations}} : {[}todo{]}
  \item
    \protect\hyperlink{plcz}{\textbf{Homozygous mutation of PLCZ1 leads to
    defective human oocyte activation and infertility that is not rescued
    by the WW-binding protein PAWP}} : Dans cet article j'ai, comme
    précédemment, effectué l'integralité des analyses bioinformatiques des
    données d'exomes effectués sur deux frères infertiles présentant des
    échecs de fécondation.\\
  \item
    \protect\hyperlink{spink2}{\textbf{SPINK2 deficiency causes
    infertility by inducing sperm defects in heterozygotes and azoospermia
    in homozygotes}} : Dans cet article j'ai effectuer non seulement
    l'intégralité des analyses bioinformatiques des données d'exomes de
    deux frères infertiles présentant un phénotype d'azoospermie mais
    aussi séquencer en Sanger les séquences codantes du gène \emph{SPINK2}
    pour une parie des 611 individus analyser ainsi que contribué à
    l'extraction de l'ARN testiculaire des souris pour l'analyse
    fonctionelle du gène \emph{Spink2} sur le modèle murin.\\
  \item
    \protect\hyperlink{cohortemmah}{****} : {[}todo{]}
  \end{enumerate}
  
  \section{Résultats}\label{resultats}
  
  \subsection{Description de la
  pipeline}\label{description-de-la-pipeline}
  
  Notre pipeline d'analyse effectue l'ensemble des étapes allant de
  l'alignement des données jusqu'au filtrage des variants
  
  \begin{enumerate}
  \def\labelenumi{\arabic{enumi}.}
  \tightlist
  \item
    \textbf{L'alignement} : L'alignement des \emph{reads} le long du
    génome de référence est effectué par le logiciel MAGIC (Su et al.,
    \protect\hyperlink{ref-Su2014}{2014}). Celui-ci l'intégralité pour
    l'ensemble des analyses en aval l'ensemble des \emph{reads} dupliqués
    et / ou s'alignant à plusieurs zone du génome. Au cours de cette
    étape, MAGIC va produire également quatre comptages pour chaque
    position couverte du génome : R+, V+, R- et V- :
  
    \begin{enumerate}
    \def\labelenumii{\alph{enumii}.}
    \tightlist
    \item
      \textbf{R+ et R-} : Ces deux comptages correspondent au nombres de
      \emph{reads} \emph{forward} (+) et \emph{reverse} (-) sur lesquels
      est observé l'allere de \textbf{référence} (R) à une position
      donnée.\\
    \item
      \textbf{V+ et V-} : À l'inverse de R+ et R-, ces comptages
      correspondent au nombres de \emph{reads} \emph{forward} et
      \emph{reverse} sur lesquels est observé un allele de
      \textbf{variant} (V) à une position donnée.\\
    \end{enumerate}
  \item
    \textbf{L'appel des variants} : Comme nous l'avons vu plus
    \protect\hyperlink{varcall}{tôt}, il est fortement conseillé
    d'effectuer l'appel des variants en tenant compte de l'aligneur choisi
    (Nielsen, Paul, Albrechtsen, \& Song,
    \protect\hyperlink{ref-Nielsen2011}{2011}, M. A. DePristo et al.
    (\protect\hyperlink{ref-DePristo2011}{2011}), Lunter \& Goodson
    (\protect\hyperlink{ref-Lunter2011}{2011})). C'est pourquoi, nous
    avons conçu notre propre algorithme d'appel des variants spécialement
    conçu pour l'analyse des données de MAGIC. Ainsi, l'appel des variants
    sera directement basé sur les quatre comptages vu précédement. Tout
    d'abord, les positions ayant une couverture \textless{} 10 sur l'un
    des deux \emph{strands} sera considérée comme de faible qualité,
    celles aynant une couverture \textless{} 10 sur les deux
    \emph{strands} seront exclus. Ensuite pour chaque variant, des appels
    indépendant seront effectués pour chaque \emph{strand}. L'appel final
    sera une synthèse de ces deux appels où seul les cas où ces deux
    appels sont concordants seront considérés comme de bone qualité.\\
  \item
    \textbf{L'annotation} : Chaque variant retenu sera ensuite annoté tout
    d'abord par le logiciel \emph{variant effect predictor} (VEP) (W.
    McLaren et al., \protect\hyperlink{ref-McLaren2016}{2016}) qui nous
    indiquera pour chaque variant la conséquence que celui-ci aura sur la
    séquence codante de l'ensemble des transcrits Ensembl qu'il chevauche
    (\textbf{Figure : }\ref{fig:figvepcsq}) (\textbf{Table :
    }\ref{tab:tabvepcsq}). Suite à cela nous ajoutons, lorsque celle-ci
    est disponible, la fréquence du variant dans les bases de données ExAC
    (Lek et al., \protect\hyperlink{ref-Lek2016}{2016}), ESP600 {[}TODO{]}
    et 1000Genomes {[}TODO{]} donnant ainsi une estimation de sa fréquence
    dans la population générale. De même, la particularité de cette
    pipeline est qu'elle conserve l'ensemble des variants identifiés dans
    les études effectués précédement permettant d'ajouter aux annotations
    la fréquences d'un variant chez les individus déjà séquencé et donc la
    fréquence d'un variant dans chaque phénotype étudié créant ainsi une
    base de données interne qui pourra servir de contrôle dans les études
    ulterieur.
  \end{enumerate}
  
  \begin{figure}
  
  {\centering \includegraphics[scale=.9]{figure/vep_csq} 
  
  }
  
  \caption[Listes des différentes conséquences prédites par VEP et leurs positionement sur le transcrit]{Listes des différentes conséquences prédites par VEP et leurs positionement sur le transcrit d'après [VEP site](http://www.ensembl.org/info/genome/variation/consequences.jpg)}\label{fig:figvepcsq}
  \end{figure}
  
  \begin{enumerate}
  \def\labelenumi{\arabic{enumi}.}
  \setcounter{enumi}{3}
  \tightlist
  \item
    \textbf{Le filtrage des variants} : L'étape de filtrage est
    extremement importante si l'on souhaite analyser de manière efficace
    les données provenant de WES. C'est pourquoi elle occupe une place
    importante dans notre pipeline. L'intégralité des paramètres de cette
    étape peuvent être modifier par l'utilisateur de sorte à faire
    correspondre les critères de filtre aux bsoins de l'étude. Afin de
    rendre son utilisation le plus efficace possibe, nous avons souhaité
    définir des paramètres par défauts pertinent dans la plupart des étude
    de séquençage exomique de sorte que à moins que le contraire ne soit
    spécifié, seul les variants impactant les transcrits codant pour une
    protéine sont conservés. De même les variants synonymes ou affectant
    les séquences UTRs sont filtrés ainsi que les variants ayant une
    fréquence \(\ge\) 1\% dans les bases dans l'une des bases données
    (ExAC, ESP6500 ou 1KH). Aussi, pour un phénotype donné, l'ensemble des
    variants observés chez les individus étudiés présentant un phénotype
    différent sont de même enlevés de la liste finale.
  \end{enumerate}
  
  \newpage        
  
  \begin{table}
  
  \caption{\label{tab:tabvepcsq}Liste des conséquences prédites par VEP avec leur description et impact associée (trouver comment la faire tenir}
  \centering
  \begin{tabular}[t]{lll}
  \toprule
  Consequence & Description & Impact\\
  \midrule
  Transcript ablation & A feature ablation whereby the deleted region includes a transcript feature & HIGH\\
  Splice acceptor variant & A splice variant that changes the 2 base region at the 3' end of an intron & HIGH\\
  Splice donor variant & A splice variant that changes the 2 base region at the 5' end of an intron & HIGH\\
  Stop gained & A sequence variant whereby at least one base of a codon is changed, resulting in a premature stop codon, leading to a shortened transcript & HIGH\\
  Frameshift variant & A sequence variant which causes a disruption of the translational reading frame, because the number of nucleotides inserted or deleted is not a multiple of three & HIGH\\
  \addlinespace
  Stop lost & A sequence variant where at least one base of the terminator codon (stop) is changed, resulting in an elongated transcript & HIGH\\
  Start lost & A codon variant that changes at least one base of the canonical start codo & HIGH\\
  Transcript amplification & A feature amplification of a region containing a transcript & HIGH\\
  Inframe insertion & An inframe non synonymous variant that inserts bases into in the coding sequenc & MODERATE\\
  Inframe deletion & An inframe non synonymous variant that deletes bases from the coding sequenc & MODERATE\\
  \addlinespace
  Missense variant & A sequence variant, that changes one or more bases, resulting in a different amino acid sequence but where the length is preserved & MODERATE\\
  Protein altering variant & A sequence\_variant which is predicted to change the protein encoded in the coding sequence & MODERATE\\
  Splice region variant & A sequence variant in which a change has occurred within the region of the splice site, either within 1-3 bases of the exon or 3-8 bases of the intron & LOW\\
  Incomplete terminal codon variant & A sequence variant where at least one base of the final codon of an incompletely annotated transcript is changed & LOW\\
  Stop retained variant & A sequence variant where at least one base in the terminator codon is changed, but the terminator remains & LOW\\
  \addlinespace
  Synonymous variant & A sequence variant where there is no resulting change to the encoded amino acid & LOW\\
  Coding sequence variant & A sequence variant that changes the coding sequence & MODIFIER\\
  Mature miRNA variant & A transcript variant located with the sequence of the mature miRNA & MODIFIER\\
  5 prime UTR variant & A UTR variant of the 5' UTR & MODIFIER\\
  3 prime UTR variant & A UTR variant of the 3' UTR & MODIFIER\\
  \addlinespace
  Non coding transcript exon variant & A sequence variant that changes non-coding exon sequence in a non-coding transcript & MODIFIER\\
  Intron variant & A transcript variant occurring within an intron & MODIFIER\\
  NMD transcript variant & A variant in a transcript that is the target of NMD & MODIFIER\\
  Non coding transcript variant & A transcript variant of a non coding RNA gene & MODIFIER\\
  Upstream gene variant & A sequence variant located 5' of a gene & MODIFIER\\
  \addlinespace
  Downstream gene variant & A sequence variant located 3' of a gene & MODIFIER\\
  TFBS ablation & A feature ablation whereby the deleted region includes a transcription factor binding site & MODIFIER\\
  TFBS amplification & A feature amplification of a region containing a transcription factor binding site & MODIFIER\\
  TF binding site variant & A sequence variant located within a transcription factor binding site & MODIFIER\\
  Regulatory region ablation & A feature ablation whereby the deleted region includes a regulatory region & MODERATE\\
  \addlinespace
  Regulatory region amplification & A feature amplification of a region containing a regulatory region & MODIFIER\\
  Feature elongation & A sequence variant located within a regulatory region & MODIFIER\\
  Regulatory region variant & A sequence variant located within a regulatory region & MODIFIER\\
  Feature truncation & A sequence variant that causes the reduction of a genomic feature, with regard to the reference sequence & MODIFIER\\
  Intergenic variant & A sequence variant located in the intergenic region, between genes & MODIFIER\\
  \bottomrule
  \end{tabular}
  \end{table}
  
  \newpage       
  
  \subsection{Utilisation de la pipeline dans des cas familiaux
  :}\label{utilisation-de-la-pipeline-dans-des-cas-familiaux}
  
  \subsubsection{Description des familles}\label{description-des-familles}
  
  Dans cette partie, je me concentre sur l'analyse bioinformatique des
  résultats des séquençages exomiques effectués entre 2012 et 2014 de 13
  individus infertiles provenant de 6 familles différentes. Parmi
  celles-ci, 3 phénotypes différents ont été observés :
  
  \begin{enumerate}
  \def\labelenumi{\arabic{enumi}.}
  \tightlist
  \item
    \textbf{\protect\hyperlink{infquant}{L'Azoospermie} :} Comme nous
    avons pu le voir, l'azoospermie est un phénotype d'infertilité
    masculine caractérisé par l'absence de spermatozoïde dans
    l'éjaculat.\\
  \item
    \textbf{Echec de fécondation :} Ce phénotype d'infertilité se
    caractérise par l'incapacité des spermatozoïdes à féconder
    l'ovocyte.\\
  \item
    \textbf{MMAF} : Le syndrome MMAF (\emph{multiple morphological
    abnormalities of the sperm flagella}) caractérise comme son nom
    l'indique les patients présentant une majorité de spermatozoïdes
    atteins par une mosaïque d'anomalie morphologique du flagelle.
  \end{enumerate}
  
  Un récapitulatif des familles et de leur phénotype est disponible dans
  la table \ref{tab:recapfam}.
  
  \begin{longtable}[t]{lrlrll}
  \caption{\label{tab:recapfam}Tableau recapitulatif des familles séquencées et de leur phénotype}\\
  \toprule
  Familly & Individuals & Phenotype & Year & Plateform & Place\\
  \midrule
  Az & 2 & Azoospermia & 2012 & Illumina HiSeq2000 & Mount Sinai Institut\\
  FF & 2 & Fertilization failure & 2014 & Illumina HiSeq2000 & Genoscope (Evry)\\
  MMAF1 & 2 & MMAF & 2014 & Illumina HiSeq2000 & Genoscope (Evry)\\
  MMAF2 & 2 & MMAF & 2014 & Illumina HiSeq2000 & Genoscope (Evry)\\
  MMAF3 & 2 & MMAF & 2014 & Illumina HiSeq2000 & Genoscope (Evry)\\
  MMAF4 & 3 & MMAF & 2014 & Illumina HiSeq2000 & Genoscope (Evry)\\
  \bottomrule
  \end{longtable}
  
  \subsubsection{Resultats des exomes}\label{resultats-des-exomes}
  
  \paragraph{Résultat de l'alignement}\label{resultat-de-lalignement}
  
  L'ensemble de nos exomes ayant été réalisés en \emph{paired-end}, les
  deux extrémités de chaques fragments sont séquencés créant ainsi deux
  \emph{reads}. Après avoir aligné indépendemment les deux \emph{ends}
  d'un même \emph{read}, seul ceux pour lesquels les deux \emph{ends}
  présentent un alignement ``compatible'' sont conservés. Un alignement
  est dit ``compatible'' lorsque les deux \emph{ends} s'alignent face à
  face (une sur le \emph{strand} + et l'autre sur le \emph{strand} - et
  couvrent une zone ne faisant pas plus de 3 fois la taille médiane de
  l'insert.
  
  \begin{center}\includegraphics{thesis_files/figure-latex/plotrunstrandness-1} \end{center}
  
  qsswxfdcsdfgds dfsqfdqfg fdrgsfdgh fdgzrfg fgdg s
  
  \begin{verbatim}
  ## <ggproto object: Class ScaleDiscretePosition, ScaleDiscrete, Scale>
  ##     aesthetics: x xmin xmax xend
  ##     axis_order: function
  ##     break_info: function
  ##     break_positions: function
  ##     breaks: waiver
  ##     call: call
  ##     clone: function
  ##     dimension: function
  ##     drop: TRUE
  ##     expand: waiver
  ##     get_breaks: function
  ##     get_breaks_minor: function
  ##     get_labels: function
  ##     get_limits: function
  ##     guide: none
  ##     is_discrete: function
  ##     is_empty: function
  ##     labels: AZ1 AZ2 FF1 FF2 MMAF1.1 MMAF1.2 MMAF2.1 MMAF2.2 MMAF3.1  ...
  ##     limits: NULL
  ##     make_sec_title: function
  ##     make_title: function
  ##     map: function
  ##     map_df: function
  ##     n.breaks.cache: NULL
  ##     na.translate: TRUE
  ##     na.value: NA
  ##     name: waiver
  ##     palette: function
  ##     palette.cache: NULL
  ##     position: bottom
  ##     range: <ggproto object: Class RangeDiscrete, Range>
  ##         range: NULL
  ##         reset: function
  ##         train: function
  ##         super:  <ggproto object: Class RangeDiscrete, Range>
  ##     range_c: <ggproto object: Class RangeContinuous, Range>
  ##         range: NULL
  ##         reset: function
  ##         train: function
  ##         super:  <ggproto object: Class RangeContinuous, Range>
  ##     reset: function
  ##     scale_name: position_d
  ##     train: function
  ##     train_df: function
  ##     transform: function
  ##     transform_df: function
  ##     super:  <ggproto object: Class ScaleDiscretePosition, ScaleDiscrete, Scale>
  \end{verbatim}
  
  \begin{figure}
  
  {\centering \includegraphics{thesis_files/figure-latex/plotreadsfate-1} 
  
  }
  
  \caption[Résultats du mapping des reads pour chaque Patients]{Résultats du mapping des reads pour chaque Patients : **A** : Cette boite à moustache montre le nombre de *reads* bruts générés pour chaque patients (représentés par les points) au cours de l'étape de séquençage. On constate que ce nombre reste pour chaque patient dans le même ordre de grandeur sauf pour un des frères de la famille AZ  qui contient presque 35 millions de *reads* en plus que la mediane. **B** : Ce diagramme en barre montre le nombre de *reads* correctement allignés pour chaque patients. On peut donc constater que la quasi totalité de *reads* générés ont été correctement allignés pour l'ensemble des patients. **C** : Ces deux boites à moustaches montrent .... On constate que la grande majorité des *reads* ne mappent qu'à un seul site (boite bleue), les autres (boite rouge) seront d'ailleur écartés des analyses à suivre}\label{fig:plotreadsfate}
  \end{figure}
  
  \paragraph{Résultat de l'appel des
  variants}\label{resultat-de-lappel-des-variants}
  
  \paragraph{Résultats de l'annotation}\label{resultats-de-lannotation}
  
  plot du comptage des impact / patients
  
  plot du nombre de variants ayant un match dans une des base de donnée
  (ExAC\ldots{}) plot de la distribution des MAF associées des MA
  
  \paragraph{Résultats du filtrage}\label{resultats-du-filtrage}
  
  Pour l'ensemble des individus de ces quatre familles nous avons appliqué
  notre pipeline d'analyse de sorte à obtenir pour chaque patient une
  liste de SNV et d'indel avec leur génotype associé (\textbf{Figure :
  }\ref{fig:resvarcall}).
  
  \begin{figure}
  
  {\centering \includegraphics{thesis_files/figure-latex/resvarcall-1} 
  
  }
  
  \caption[Comptage des SNVs et indels retrouvés par patients avec leur génotypes associés]{Comptage des SNVs et indels retrouvés par patients avec leur génotypes associés}\label{fig:resvarcall}
  \end{figure}
  
  Ensuite, afin de ne conserver que les variants ayant la plus forte
  probabilité d'être responsable du phénotype nous avons appliqué
  succesivement six filtres :
  
  \begin{enumerate}
  \def\labelenumi{\arabic{enumi}.}
  \tightlist
  \item
    \textbf{L'union des variants :} Dans ces différentes études, nos
    patients ont à chaque fois au moins un frère présentant le même
    phénotype. Ainsi nous avons pu formuler l'hypothèse d'une cause
    génétique commune entre les différents frères d'une même famille et
    donc filtrer l'ensemble des variants qui ne sont pas partagés par
    l'ensemble des membre de la fraterie.\\
  \item
    \textbf{Genotype des variants :} Dans ces études, nous avons emmis
    l'hypothèse d'une transmission recessive du phénotype. Ainsi, seul les
    variants homozygotes ont été conservés. Ce filtre est le plus efficace
    du pipeline en permettant de filtrer entre 38814 et 53448 variants par
    individus (\textbf{Figure : }\ref{fig:resvarcall},
    \ref{fig:comparefilter}).\\
  \item
    \textbf{Impact du variant :} Afin de ne conserver que les variants
    ayant un effet potentiellement tronquant sur la protéine, nous avons
    filtré les variants intonique et ceux tombant dans les sequences UTRs.
    De même les variants synonymes ne sont pas conservés (exeptés ceux se
    trouvant proches des régions d'épissage) car ceux-ci n'ont aucun effet
    sur séquences protéique. Pour les variants faux sens (changement d'un
    seul aa de la séquence protéique) il est plus difficile de se décider
    {[}TODO insert citation{]} nous avons donc utilisé les logiciels SIFT
    et Polyphen et filtré l'ensemble des fauxsens prédit comme
    \emph{tolerated} par SIFT et \emph{benign} par Polyphen.\\
  \item
    \textbf{Transcrits NMD :} Le mécanisme NMD (\emph{nonsense-mediated
    decay}) a pour but de controler la qualité des ARNm cellulaires chez
    les eucaryotes (Y.-F. Chang, Imam, \& Wilkinson,
    \protect\hyperlink{ref-Chang2007}{2007}) en éliminant les ARNm qui
    comportent un codon stop prématuré (Baker \& Parker,
    \protect\hyperlink{ref-Baker2004}{2004}), pouvant être le résultat
    d'une erreur de transcription, d'une mutation ou encore d'une erreur
    d'épissage. Il est donc peu probable que les variants présents sur
    transcrits annotés NMD soient responsables du phénotype. Nous avons
    donc filtré l'ensemble des variants chevauchant \textbf{uniquement}
    des transcrits annotés NMD. Cette étape de filtre permet à elle seule
    de filtrer systematiquement les variants de 2587 à 3212 transcrits
    (\textbf{Figure : }\ref{fig:nmdtranscripts}) en fonction des individus
    soit, entre 7261 et 10872 variants différents par individus
    (\textbf{Figure : }\ref{fig:comparefilter}).
  \end{enumerate}
  
  \begin{figure}
  
  {\centering \includegraphics{thesis_files/figure-latex/nmdtranscripts-1} 
  
  }
  
  \caption[Nombre de transcrits filtrés car ils sont annotés NMD]{Nombre de transcrits filtrés car ils sont annotés NMD : Chaque point représente un individu séquencé, la couleur et la forme du point dépend de la famille d'origine de l'individu}\label{fig:nmdtranscripts}
  \end{figure}
  
  \begin{enumerate}
  \def\labelenumi{\arabic{enumi}.}
  \setcounter{enumi}{4}
  \tightlist
  \item
    \textbf{Frequence des variants :} La fréquence d'un variant dans la
    population générale est un moyen rapide d'avoir un avis sur l'effet
    délétère de celui-ci. En efft, il est peu probable qu'un retrouvé
    fréquement dans la population générale soit causal d'une pathologie
    sévère. Ainsi nous avons filtré pour l'ensemble de nos patients
    l'ensemble des variants ayant une fréquence \(\ge\) 0.01 dans l'une
    des trois bases de données que sont ExAC, ESP et 1KG.\\
  \item
    \textbf{Présence des variants dans la cohorte contrôle :} Au cours de
    nos différentes études, nous avons été ammené à séquencé 134.
    L'ensemble de ces individus peuvent être soit sains soit présenter
    l'un des 6 phénotypes étudié au cours de nos différentes études
    (\textbf{Table : }\ref{tab:TODO}). Ces phénotypes étant très
    différent, il n'est pas abérant d'emmetre l'hypothèse qu'ils que leurs
    causes génétiques soient diffrentes. De même, les variants recherché
    étant rares, il est peu probable qu'un individu porte les variants de
    deux phénotypes différents. Ainsi, pour chacune des 6 familles, nous
    avons pu constituer une cohorte contrôle composée dans l'ensemble des
    patients précédemment analysés et ne présentant pas le même phénotype
    que celui étudié dans la famille (\textbf{Figure :
    }\ref{fig:plotsamplectrl}). Dès lors, nous avons put filtrer
    l'ensemble des variants retrouvés à la fois chez nos patients et
    observés à l'état homozygote dans la cohorte contrôle.
  \end{enumerate}
  
  \begin{figure}
  
  {\centering \includegraphics{thesis_files/figure-latex/plotsamplectrl-1} 
  
  }
  
  \caption[Nombre d'individus la cohorte contrôle constituée pour chaque famille de l'analyse]{Nombre d'individus la cohorte contrôle constituée pour chaque famille de l'analyse : Ici, chaque barre représente une famille et sa hauteur est déterminée par le nombre d'individus composant la cohorte contrôle à laquelle elle a été confronté. Chaque individus de la cohorte contrôle a été séquencés en WES par notre équipe. Afin d'être considérer comme "contrôle" un individus doit être sain ou présenter un phénotype d'infertilité différent de la famille étudiée. Par exemple, un individus MMAF pourra servir de contrôle aux familles AZ et FF mais pas aux familles MMAF1-4}\label{fig:plotsamplectrl}
  \end{figure}
  
  Afin de comparer le pouvoir discriminant de chacun de ces filtres, nous
  avons compté le nombre de variant filtrés par chacun d'entre eux
  indépendamments des autres (\textbf{Figure : }\ref{fig:comparefilter}).
  
  \begin{figure}
  
  {\centering \includegraphics{thesis_files/figure-latex/comparefilter-1} 
  
  }
  
  \caption[Comparaison du pouvoir discriminant de chaque filtre employé TODO mettre union]{Comparaison du pouvoir discriminant de chaque filtre employé TODO mettre union : Les boites à moustache représentent la quantité de variants filtrés par chacun des 6 filtres utilisés sur l'ensemble des patients. Chaque point représente un patient et sa hauteur informe du nombre exact de variant filtrés par uin filtre donné. Comme on peut le voir, le filtre le plus discriminant est celui consistant à filtrer l'ensemble des variants hétérozygotes ce qui n'est pas surprenant puisqu'ils représentent entre ... et ... pourcents des variants observés pour chaque patients}\label{fig:comparefilter}
  \end{figure}
  
  Après avoir effectuer l'ensemble de ces filtres, seuls quelques variants
  subsistent nous permettant d'obtenir unle liste de gènes restrainte pour
  chaque famille (\textbf{Table : }\ref{tab:tablegene}). Ainsi, la cause
  génétique expliquant le phénotype d'une famille a pu être mis en
  évidence dans \ldots{} familles sur \ldots{} {[}TODO{]} (\textbf{Figure
  : }\ref{fig:plotremaininggenes}). Il est a noté que l'ensemble des
  familles pour lesquelles la cause génétique a été identifiée présente un
  historique consanguin {[}figure arbre{]} ce qui n'était pas le cas pour
  les \ldots{} autres. Cette consanguinité observée dans une partie des
  famille nous a permi de justifier l'exclusion des variants
  hétérozygotes. En revanche pour les \ldots{} autres fa milles, rien ne
  justifiait un tel filtre. Ainsi, pour celles-ci il est probable que les
  variants responsables se soient vu exclus par ce filtre. C'est pourquoi,
  notre équipe se concentre actuellement sur les variants hétérozygotes de
  ces familles.
  
  \begin{figure}
  
  {\centering \includegraphics{thesis_files/figure-latex/plotremaininggenes-1} 
  
  }
  
  \caption[Nombre de gènes passant l'ensemble des filtres par famille]{Nombre de gènes passant l'ensemble des filtres par famille  :  Chaque barre représente une des familles analysées. La hauteure de cette barre correspond au nombre de gènes ayant passé l'ensemble des filtres pour chaque famille. Les barres bleues caractérisent les familles pour lesquelles le gène responsable de la pathologie a été identifié parmi la liste de gène (dans ce cas le symbole du gène est écrit au dessus de la barre). Les barres rouges indique qu'aucun des gènes ayant passé les filtres pour ne semble expliquer le phénotype (dans ce cas il est écrit "???" au dessus de la barre)}\label{fig:plotremaininggenes}
  \end{figure}
  
  \begin{table}
  
  \caption{\label{tab:tablegene}Tableau des gènes ayant passé l'ensemble des filtres pour les fifférentes familles}
  \centering
  \begin{tabular}[t]{llllll}
  \toprule
  AZ & FF & MMAF1 & MMAF2 & MMAF3 & MMAF4\\
  \midrule
  GUF1 &  &  &  &  & \\
  SPINK2 &  &  &  &  & \\
   & PLCZ1 &  &  &  & \\
   &  & PLA2G4B &  &  & \\
   &  & JMJD7-PLA2G4B &  &  & \\
  \addlinespace
   &  &  & MYH11 &  & \\
   &  &  & DNAH1 &  & \\
   &  &  &  & WEE2 & \\
   &  &  &  & PCSK5 & \\
   &  &  &  & ZFYVE28 & \\
  \addlinespace
   &  &  &  & GBP2 & \\
   &  &  &  & FCGR3A & \\
   &  &  &  &  & MMP9\\
   &  &  &  &  & TGIF2\\
   &  &  &  &  & ZNF469\\
  \addlinespace
   &  &  &  &  & HYDIN\\
   &  &  &  &  & MTSS1L\\
   &  &  &  &  & CDH23\\
   &  &  &  &  & CCDC37\\
   &  &  &  &  & DAPK1\\
  \addlinespace
   &  &  &  &  & SEMA5B\\
   &  &  &  &  & SLC13A3\\
   &  &  &  &  & TMEM231\\
   &  &  &  &  & ZNF276\\
  \bottomrule
  \end{tabular}
  \end{table}
  
  \hypertarget{cohortemmah}{\subsection{Etude d'une large cohorte de
  patients MMAF}\label{cohortemmah}}
  
  Dans cette partie, nous allons détailler les analyses effectuées sur une
  cohorte de 62 individus présentant le phénotype MMAF pour lesquels nous
  avons effectués un séquençage WES. Nous avons ainsi pu appliquer notre
  pipeline d'analyse afin d'appeler et annoter les variants de ces 62
  individus (\textbf{Figure : }\ref{fig:largemmafcall}).
  
  \begin{figure}
  
  {\centering \includegraphics{thesis_files/figure-latex/largemmafcall-1} 
  
  }
  
  \caption[Comptage des variants pour chaque individus avec leur génotype et l'impact prédite par VEP]{Comptage des variants pour chaque individus avec leur génotype et l'impact prédite par VEP  :  VEP possède quatre niveaux d'impact pour ses variants : **HIGH** : variant ayant une forte probabilité de causer des dommages sévères à la protéine, **MODERATE** : Variants non-tronquant pouvant tout de même affecté la protéine, **LOW** : variant ayant peu de chance d'alterer la protéine, **MODIFIER** :  Variants affectant les régions non codantes du transcrits et dont l'impact sur la protéine est difficile à prévoir. Chaque barre représente le comptage pour un individus}\label{fig:largemmafcall}
  \end{figure}
  
  Les filtres utilisés ont été les mêmes que ceux détaillés dans l'études
  des cas familiaux, c'est à dire que seul les variants homozygotes ayant
  une fréquence \(\le\) 0.01 dans la population générales et n'étant pas
  observés dans la cohorte contrôle de 63 individus furent conservés. De
  même, les variants synonymes, impactant la séquences UTR ou chevauchant
  uniquement des transcrits annotés NMD par VEP on été filtrés. Ainsi, ces
  différents filtres nous ont permis d'obtenir une liste de 1369 SNVs
  (entre 1 et 77 différents par patients) et de 211 indels (entre 1 et 11
  différents par patients) (\textbf{Figure : }\ref{fig:plotvarperrun}).
  Cet ensemble de variant nous a alors permis d'obtenir des listes
  comprenant entre 1 et 74 gènes différents par patients (\textbf{Figure :
  }\ref{fig:plotpassinggenes} - \textbf{A}) constituant ainsi un total de
  1316 gènes différents pour l'ensemble des patients. Parmi ceux là, 1121
  (85\%) ont été retrouvés mutés chez un seul patients tandis que 195
  (15\%) ont été retrouvé chez au moins 2 patients (\textbf{Figure :
  }\ref{fig:plotpassinggenes} - \textbf{B}).
  
  \begin{figure}
  
  {\centering \includegraphics{thesis_files/figure-latex/plotvarperrun-1} 
  
  }
  
  \caption[Comptage des variants filtrés]{Comptage des variants filtrés  :  **A**: Comptage des SNVs et Indels ayant été filtrés (FILTERED) et ayant passé les filtres (PASS), **B**: Pourcentage des SNVs et indels ayant passé les filtres, **C**: Comptage pour chaque individus du nombre de SNVs et d'indels ayant passé les filtres. Chaque point représente le comptage pour un individus}\label{fig:plotvarperrun}
  \end{figure}
  
  \begin{figure}
  
  {\centering \includegraphics{thesis_files/figure-latex/plotpassinggenes-1} 
  
  }
  
  \caption[Analyse des gènes passant les filtres]{Comptage des variants filtrés  :  **A**: Comptage des SNVs et Indels ayant été filtrés (FILTERED) et ayant passé les filtres (PASS), **B**: Pourcentage des SNVs et indels ayant passé les filtres, **C**: Comptage pour chaque individus du nombre de SNVs et d'indels ayant passé les filtres. Chaque point représente le comptage pour un individus}\label{fig:plotpassinggenes}
  \end{figure}
  
  Le flagelle spermatique pouvant s'apparenté à un cil, nous avons ensuite
  comparés ces gènes avec une liste de 371 gènes prédits comme faisant
  parti du ciliome {[}TODO: insert ref{]}. Ainsi, sur notre ensemble de
  gène ayant passé les filtres 31 sont prédit comme faisant parti du
  ciliome (\textbf{Table :} \ref{tab:tabciliome}).
  
  \begin{longtable}[t]{lrl}
  \caption{\label{tab:tabciliome}Gènes ayant passé les filtres et annotés comme faisant partie du ciliome}\\
  \toprule
  Gene & Patient carrying a variant on the gene &  Ciliome evidence\\
  \midrule
  LRRC43 & 4 & No evidence from previous studies\\
  ARMC2 & 3 & No evidence from previous studies\\
  WDR52 & 3 & Strong evidence from previous studies\\
  AK7 & 2 & Strong evidence from previous studies\\
  EFCAB6 & 2 & Strong evidence from previous studies\\
  \addlinespace
  CCDC146 & 2 & Strong evidence from previous studies\\
  TTC29 & 2 & Strong evidence from previous studies\\
  KIAA0556 & 1 & No evidence from previous studies\\
  KIF9 & 1 & Strong evidence from previous studies\\
  FBXO15 & 1 & Weak evidence from previous studies\\
  \addlinespace
  C21orf59 & 1 & Strong evidence from previous studies\\
  FAM81B & 1 & Strong evidence from previous studies\\
  WDR16 & 1 & Strong evidence from previous studies\\
  CCDC147 & 1 & Strong evidence from previous studies\\
  KIF6 & 1 & Strong evidence from previous studies\\
  \addlinespace
  SPAG17 & 1 & Weak evidence from previous studies\\
  C6orf118 & 1 & Strong evidence from previous studies\\
  RSPH9 & 1 & Strong evidence from previous studies\\
  KIAA0319 & 1 & No evidence from previous studies\\
  SPEF2 & 1 & Strong evidence from previous studies\\
  \addlinespace
  C6 & 1 & Weak evidence from previous studies\\
  ZMYND10 & 1 & Strong evidence from previous studies\\
  MIPEP & 1 & Weak evidence from previous studies\\
  PROM1 & 1 & Strong evidence from previous studies\\
  DLEC1 & 1 & Strong evidence from previous studies\\
  \addlinespace
  CCDC65 & 1 & Strong evidence from previous studies\\
  HYDIN & 1 & Strong evidence from previous studies\\
  C21orf58 & 1 & No evidence from previous studies\\
  SLFN13 & 1 & Weak evidence from previous studies\\
  ACYP1 & 1 & No evidence from previous studies\\
  STK33 & 1 & Strong evidence from previous studies\\
  \bottomrule
  \end{longtable}
  
  Ces analyses nous ont permis de mettre en évidence certains candidats
  évidents (TODO: insert table avec candidats évidents) nous permettant
  ainsi d'identifier la cause génétique de \ldots{} patients soit
  \ldots{}\% de notre cohorte (TODO).
  
  \chapter{MutaScript}\label{mutascript}
  
  \chapter*{Conclusion}\label{conclusion}
  \addcontentsline{toc}{chapter}{Conclusion}
  
  \chapter{The First Appendix}\label{the-first-appendix}
  
  \chapter*{References}\label{references}
  \addcontentsline{toc}{chapter}{References}
  
  \hypertarget{refs}{}
  \hypertarget{ref-Baker2004}{}
  Baker, K. E., \& Parker, R. (2004). Nonsense-mediated mRNA decay:
  terminating erroneous gene expression. \emph{Current Opinion in Cell
  Biology}, \emph{16}(3), 293--9.
  \url{http://doi.org/10.1016/j.ceb.2004.03.003}
  
  \hypertarget{ref-Chang2007}{}
  Chang, Y.-F., Imam, J. S., \& Wilkinson, M. F. (2007). The
  Nonsense-Mediated Decay RNA Surveillance Pathway. \emph{Annual Review of
  Biochemistry}, \emph{76}(1), 51--74.
  \url{http://doi.org/10.1146/annurev.biochem.76.050106.093909}
  
  \hypertarget{ref-DePristo2011}{}
  DePristo, M. A., Banks, E., Poplin, R., Garimella, K. V., Maguire, J.
  R., Hartl, C., \ldots{} Pritchard, E. (2011). A framework for variation
  discovery and genotyping using next-generation DNA sequencing data.
  \emph{Nature Genetics}, \emph{43}(5), 491--498.
  \url{http://doi.org/10.1038/ng.806}
  
  \hypertarget{ref-Lek2016}{}
  Lek, M., Karczewski, K. J., Minikel, E. V., Samocha, K. E., Banks, E.,
  Fennell, T., \ldots{} Exome Aggregation Consortium, D. G. (2016).
  Analysis of protein-coding genetic variation in 60,706 humans.
  \emph{Nature}, \emph{536}(7616), 285--91.
  \url{http://doi.org/10.1038/nature19057}
  
  \hypertarget{ref-Lunter2011}{}
  Lunter, G., \& Goodson, M. (2011). Stampy: A statistical algorithm for
  sensitive and fast mapping of Illumina sequence reads. \emph{Genome
  Research}, \emph{21}(6), 936--939.
  \url{http://doi.org/10.1101/gr.111120.110}
  
  \hypertarget{ref-McLaren2016}{}
  McLaren, W., Gil, L., Hunt, S. E., Riat, H. S., Ritchie, G. R. S.,
  Thormann, A., \ldots{} Cunningham, F. (2016). The Ensembl Variant Effect
  Predictor. \emph{Genome Biology}, \emph{17}(1), 122.
  \url{http://doi.org/10.1186/s13059-016-0974-4}
  
  \hypertarget{ref-Ng}{}
  Ng, S. B., Buckingham, K. J., Lee, C., Bigham, A. W., Tabor, H. K.,
  Dent, K. M., \ldots{} Bamshad, M. J. (n.d.). Exome sequencing identifies
  the cause of a Mendelian disorder. \url{http://doi.org/10.1038/ng.499}
  
  \hypertarget{ref-Nielsen2011}{}
  Nielsen, R., Paul, J. S., Albrechtsen, A., \& Song, Y. S. (2011).
  Genotype and SNP calling from next-generation sequencing data.
  \emph{Nature Reviews. Genetics}, \emph{12}(6), 443--51.
  \url{http://doi.org/10.1038/nrg2986}
  
  \hypertarget{ref-Su2014}{}
  Su, Z., Łabaj, P. P., Li, S. S., Thierry-Mieg, J., Thierry-Mieg, D.,
  Shi, W., \ldots{} Shi, L. (2014). A comprehensive assessment of RNA-seq
  accuracy, reproducibility and information content by the Sequencing
  Quality Control Consortium. \emph{Nature Biotechnology}, \emph{32}(9),
  903--14. \url{http://doi.org/10.1038/nbt.2957}


  % Index?

\end{document}

