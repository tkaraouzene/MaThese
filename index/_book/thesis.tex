% This is the Reed College LaTeX thesis template. Most of the work
% for the document class was done by Sam Noble (SN), as well as this
% template. Later comments etc. by Ben Salzberg (BTS). Additional
% restructuring and APA support by Jess Youngberg (JY).
% Your comments and suggestions are more than welcome; please email
% them to cus@reed.edu
%
% See http://web.reed.edu/cis/help/latex.html for help. There are a
% great bunch of help pages there, with notes on
% getting started, bibtex, etc. Go there and read it if you're not
% already familiar with LaTeX.
%
% Any line that starts with a percent symbol is a comment.
% They won't show up in the document, and are useful for notes
% to yourself and explaining commands.
% Commenting also removes a line from the document;
% very handy for troubleshooting problems. -BTS

% As far as I know, this follows the requirements laid out in
% the 2002-2003 Senior Handbook. Ask a librarian to check the
% document before binding. -SN

%%
%% Preamble
%%
% \documentclass{<something>} must begin each LaTeX document
\documentclass[12pt,twoside]{ugathesis}
% Packages are extensions to the basic LaTeX functions. Whatever you
% want to typeset, there is probably a package out there for it.
% Chemistry (chemtex), screenplays, you name it.
% Check out CTAN to see: http://www.ctan.org/
%%
\usepackage{graphicx,latexsym}
\usepackage[french]{babel}
\usepackage{amsmath}
\usepackage{amssymb,amsthm}
\usepackage[dvipsnames]{xcolor} % tk: for more color
\usepackage{xcolor}
\usepackage{eso-pic}
\usepackage{longtable,booktabs,setspace}
\usepackage{chemarr} %% Useful for one reaction arrow, useless if you're not a chem major
\usepackage[hyphens]{url}
\usepackage{pdfpages}
\usepackage{tikz}
\usetikzlibrary{calc}
% Added by CII
\usepackage{hyperref}
\usepackage{lmodern}
\usepackage{float}
\floatplacement{figure}{H}
% End of CII addition
\usepackage{rotating}
\usepackage{upgreek} % tk : pour pouvoir utiliser le symbole µ droit (pas en itallic)
\usepackage{pdfpages} % tk : pour pouvoir insérer des fichiers pdf dans le corp de texte
\usepackage{lscape} % tk : pour pouvoir insérer des images au format paysage
\newcommand{\blandscape}{\begin{landscape}}
\newcommand{\elandscape}{\end{landscape}}
\usepackage[utf8]{inputenc}

% Next line commented out by CII
%%% \usepackage{natbib}
% Comment out the natbib line above and uncomment the following two lines to use the new
% biblatex-chicago style, for Chicago A. Also make some changes at the end where the
% bibliography is included.
%\usepackage{biblatex-chicago}
%\bibliography{thesis}


% Added by CII (Thanks, Hadley!)
% Use ref for internal links
\renewcommand{\hyperref}[2][???]{\autoref{#1}}
\def\chapterautorefname{Chapter}
\def\sectionautorefname{Section}
\def\subsectionautorefname{Subsection}
% End of CII addition

% Added by CII
\usepackage{caption}
\captionsetup{width=5in}
% End of CII addition

% \usepackage{times} % other fonts are available like times, bookman, charter, palatino



%%% Mise en page des titres (TK)
\xdefinecolor{royalblue}{named}{RoyalBlue}
\titleformat{\chapter}[display]
{\vfill}
{{%
   \normalfont\fontsize{48pt}{48pt}\sffamily\color{royalblue}\MakeUppercase{\chaptername}
   \fontsize{48pt}{48pt}\selectfont\thechapter%
 }%
}
{5pt}
{\Huge\normalfont
 \parbox{\textwidth-\widthof{\LARGE\sffamily\MakeUppercase{\chaptername}}}%
}[\vfill\clearpage]
\titlespacing*{\chapter}{0pt}{0pt}{0pt}
%%%





% To pass between YAML and LaTeX the dollar signs are added by CII
\title{THÈSE}
\author{Thomas Karaouzene}
\lab{Génétique, Epigénétique et Thérapies de l'Infertilité (GETI) et
Techniques de l'Ingénierie Médicale et de la Complexité - Informatique,
Mathématiques et Applications de Grenoble (TIMC-IMAG)}
\date{29 novembre 2017}
\division{Mathematics and Natural Sciences}
\advisor{Pierre Ray}
%If you have two advisors for some reason, you can use the following
% Uncommented out by CII
\altadvisor{Nicolas Thierry-Mieg}
% End of CII addition
%\institution{}
%\degree{}

%%% Remember to use the correct department!
\department{Ingénierie de la Santé, de la Cognition et Environnement (EDISCE)}
% if you're writing a thesis in an interdisciplinary major,
% uncomment the line below and change the text as appropriate.
% check the Senior Handbook if unsure.
%\thedivisionof{The Established Interdisciplinary Committee for}
% if you want the approval page to say "Approved for the Committee",
% uncomment the next line
%\approvedforthe{Committee}

% Added by CII
%%% Copied from knitr
%% maxwidth is the original width if it's less than linewidth
%% otherwise use linewidth (to make sure the graphics do not exceed the margin)
\makeatletter
\def\maxwidth{ %
  \ifdim\Gin@nat@width>\linewidth
    \linewidth
  \else
    \Gin@nat@width
  \fi
}
\makeatother

\renewcommand{\contentsname}{Table of Contents}
% End of CII addition

\setlength{\parskip}{0pt}

% Added by CII
  %\setlength{\parskip}{\baselineskip}
  \usepackage[parfill]{parskip}

\providecommand{\tightlist}{%
  \setlength{\itemsep}{0pt}\setlength{\parskip}{0pt}}

\Acknowledgements{

}

\Dedication{

}

\Preface{

}

\Abstract{

}

	\usepackage{tikz}
% End of CII addition
%%
%% End Preamble
%%
%

\begin{document}

% Everything below added by CII
  \maketitle

\frontmatter % this stuff will be roman-numbered
\pagestyle{empty} % this removes page numbers from the frontmatter



  \hypersetup{linkcolor=black}
  \setcounter{tocdepth}{3}
  \tableofcontents

  \listoftables

  \listoffigures



\mainmatter % here the regular arabic numbering starts
\pagestyle{fancyplain} % turns page numbering back on

\chapter{If you are creating a PDF you'll need to write your preliminary
content here
or}\label{if-you-are-creating-a-pdf-youll-need-to-write-your-preliminary-content-here-or}

\chapter*{Remerciements}\label{remerciements}
\addcontentsline{toc}{chapter}{Remerciements}

\newpage

Je remercie Messieurs le Professeurs Jacques VAN HELDEN et le Docteur
Michaël MITCHELL d'avoir accepté la charge d'être rapporteurs de ma
thèse.

Je remercie Madame le Professeur Christel THAUVIN et Messieurs le
Docteur Julien THÉVENON d'avoir accepté de participer à mon jury et
juger ce travail.

À Pierre pour m'avoir fait confiance toutes ces années et pour m'avoir
permis de réaliser cette thèse.

À Nicolas, qui as du s'arracher les yeux sur certains de mes codes, pour
m'avoir inculqué, avec le temps, sa méthode et sa rigeur.

À l'ensemble des l'équipes GETI, en particulier Zine pour toutes nos
discussions, les tacos et les bons moments.

À l'ensemble de l'équipe BCM, en particulier, Florient, Keurcien (mon
fils adoptif), Kévin et Thomas, les fous du
\includegraphics[width=0.05000\textwidth]{figure/r_logo.jpg} pour toutes
les discussions (du coup déterminisme ou pas ?), les barbecues, le
\emph{Fight Club}\ldots{} À quand la \emph{start-up} de la \emph{Data
Team} ?

Carlos et johnny Cage pour tous les fous rires que nous avons eu
ensemble pendant ces dernières années.

À Agnès et Gérard pour avoir pris le temps de relire mon manuscrit et
corrigé mes nombreuses fautes.

À mes amis, toujours présents malgré la distance.

À l'ensemble du Budo club de Malesherbes et plus particulièrement à
Maître Hiram et Fanfan, qui m'ont vu grandir et qui, à travers leurs
enseignements du Karaté, ont réussi à me transmettre des valeurs et un
code de vie.

À ma famille pour son unité et sa solidarité.

À Dadette et Marco pour m'avoir accueilli pendant ces cinq années, et
tous les débats parfois endiablés que nous avons pu avoir ensemble.
``\emph{Hakuna Matata}''.

À Aurélien pour tous ces moments de ``luronage'', les grecs, les parties
de tarots, nos années \emph{Bell \& Barksdale}. N'oublie jamais,
``\emph{le lion ne pactise pas avec les hommes}''.

À mes parents, pour avoir toujours été à mes côtés quelques soit mes
choix. Pour votre amour et votre dévouement indéfectible. Votre mission
est désormais accomplie, profitez bien de votre vie de grands-parents.

À ma sœur pour toutes ces années de complicité. Bonne chance dans ta
nouvelle vie qui commence.

À Estelle, mon garde-fou. Tu as su pendant ces années me conseiller et
me soutenir sans jamais faillir. Tu as toujours été là pour moi. Ta
capacité à me supporter est la preuve de ton courage et ta force. Tu
m'as donné le plus beau des cadeaux en me permettant de construire, avec
toi, une famille. Tu es la clef de voute sur laquelle je me repose (trop
souvent). J'espère sincèrement que ces dix années ne sont que les
prémisses de notre Histoire et sache que tu pourras toujours compter sur
moi comme je compte sur toi.

À Noham, mon fils, ma plus grande fierté et source de bonheur. Tu as
bouleversé ma vie . Ton seul rire à le pouvoir de me rendre heureux et
de me faire oublier tous mes tracas. Puisses-tu mener ta vie comme tu
l'entends, faire tes propres choix, tes propres erreurs en sachant que
je te soutiendrai toujours. Je t'aime.

\chapter*{Résumé}\label{resume}
\addcontentsline{toc}{chapter}{Résumé}

\chapter{Introduction}\label{introInf}

\section{La spermatogenèse}\label{la-spermatogenese}

\subsection{Rappels sur le testicule}\label{rappels-sur-le-testicule}

\subsection{La phase de
multiplication}\label{la-phase-de-multiplication}

\subsection{La méiose}\label{meiose}

\subsection{La spermiogenèse}\label{spermiogenese}

\section{Structure et fonction du
spermatozoïde}\label{structure-et-fonction-du-spermatozoide}

\subsection{La tête}\label{la-tete}

\subsection{Le flagelle}\label{le-flagelle}

\section{L'infertilité masculine}\label{linfertilite-masculine}

\subsection{Les différents phénotypes d'infertilité
masculine}\label{les-differents-phenotypes-dinfertilite-masculine}

\subsubsection{Anomalies liées à la quantité
spermatique}\label{infquant}

\subsubsection{Anomalies liées à la
morphologie}\label{anomalies-liees-a-la-morphologie}

\subsubsection{Anomalies liées à la
mobilité}\label{anomalies-liees-a-la-mobilite}

\subsection{La génétique de
l'infertilité}\label{la-genetique-de-linfertilite}

\subsubsection{Les causes fréquentes}\label{les-causes-frequentes}

\subsubsection{Les nouveaux gènes}\label{les-nouveaux-genes}

\section{Les techniques d'analyses
génétiques}\label{les-techniques-danalyses-genetiques}

\subsection{\texorpdfstring{Approche ``gènes
candidats''}{Approche gènes candidats}}\label{approche-genes-candidats}

\subsection{Les puces}\label{les-puces}

\subsubsection{Les puces à expression}\label{les-puces-a-expression}

\subsubsection{Les puces à SNP, plateforme
génotypage}\label{les-puces-a-snp-plateforme-genotypage}

\subsubsection{Les puces à indels}\label{les-puces-a-indels}

\subsubsection{Limitation}\label{limitation}

\subsection{Le séquençage NGS}\label{ngs}

\subsubsection{La capture des parties à séquencer, avantages et
inconvénients}\label{la-capture-des-parties-a-sequencer-avantages-et-inconvenients}

\subsubsection{L'amplification}\label{lamplification}

\subsubsection{La réaction de séquence}\label{la-reaction-de-sequence}

\section{L'analyse bioinformatique des données de
NGS}\label{lanalyse-bioinformatique-des-donnees-de-ngs}

\subsection{Les données fournies par le
NGS}\label{les-donnees-fournies-par-le-ngs}

\subsubsection{\texorpdfstring{Un \emph{read}, c'est quoi
?}{Un read, c'est quoi ?}}\label{un-read-cest-quoi}

\subsubsection{Le format FASTQ}\label{fastq}

\subsection{L'alignement}\label{lalignement}

\subsection{L'appel des variants}\label{varcall}

\subsection{L'annotation des variants}\label{lannotation-des-variants}

\subsection{Le filtrage des variants}\label{le-filtrage-des-variants}

\subsection{Conclusion NGS}\label{conclusion-ngs}

\chapter{Mise en place d'une stratégie pour l'analyse des données
exomiques -- application en recherche
clinique}\label{mise-en-place-dune-strategie-pour-lanalyse-des-donnees-exomiques-application-en-recherche-clinique}

\section{Méthode : Description du
pipeline}\label{methode-description-du-pipeline}

\subsection{\texorpdfstring{L'alignement des
\emph{reads}}{L'alignement des reads}}\label{lalignement-des-reads}

\subsection{L'appel des variants}\label{lappel-des-variants}

\subsection{L'annotation}\label{lannotation}

\subsection{Le filtrage des variants}\label{le-filtrage-des-variants-1}

\section{Résultats 1 : Analyse de 3 phénotypes par des cas
familiaux}\label{resultats-1-analyse-de-3-phenotypes-par-des-cas-familiaux}

\subsection{Résultats des différentes étapes de
l'analyse}\label{resultats-des-differentes-etapes-de-lanalyse}

\subsubsection{Résultat de l'alignement}\label{resultat-de-lalignement}

\subsubsection{L'appel des variants}\label{lappel-des-variants-1}

\subsubsection{L'annotation des
variants}\label{lannotation-des-variants-1}

\subsubsection{Le filtrage des
variants}\label{le-filtrage-des-variants-2}

\subsection{Article n°1}\label{article-n1}

\subsubsection{Contexte et objectifs}\label{contexte-et-objectifs}

\subsubsection{Principaux résultats}\label{principaux-resultats}

\subsection{Article n°2}\label{article-n2}

\subsubsection{Contexte et objectifs}\label{contexte-et-objectifs-1}

\subsubsection{Principaux résultats}\label{principaux-resultats-1}

\subsection{Article n°3}\label{article-n3}

\subsubsection{Contexte et objectifs}\label{contexte-et-objectifs-2}

\subsubsection{Principaux résultats}\label{principaux-resultats-2}

\section{Résultats 2 : Étude d'une cohorte de femmes
infertiles}\label{resultats-2-etude-dune-cohorte-de-femmes-infertiles}

\subsection{Article n°4}\label{article-n4}

\subsubsection{Contexte et objectifs}\label{contexte-et-objectifs-3}

\subsubsection{Principaux résultats}\label{principaux-resultats-3}

\section{Résultats 3 : Étude d'une large cohorte de patients
MMAF}\label{resultats-3-etude-dune-large-cohorte-de-patients-mmaf}

\subsection{Article n°5}\label{article-n5}

\subsubsection{Contexte et objectifs}\label{contexte-et-objectifs-4}

\subsubsection{Principaux résultats}\label{principaux-resultats-4}

\chapter{Investigation génétique et physiologique de la
globozoospermie}\label{globo}

\section{Introduction sur la
globozoospermie}\label{introduction-sur-la-globozoospermie}

\section{\texorpdfstring{Résultats 1 : Les mécanismes mutationnels
entraînant la délétion au locus de \emph{DPY19L2} chez
l'humain}{Résultats 1 : Les mécanismes mutationnels entraînant la délétion au locus de DPY19L2 chez l'humain}}\label{mecamut}

\subsection{Article n°6:}\label{article-n6}

\subsubsection{Contexte et objectifs}\label{contexte-et-objectifs-5}

\subsubsection{Principaux résultats}\label{principaux-resultats-5}

\section{Résultat 2 : La transcriptomique}\label{transcriptome}

\subsection{Article n°7:}\label{article-n7}

\subsubsection{Contexte et objectifs}\label{contexte-et-objectifs-6}

\subsubsection{Principaux résultats :}\label{principaux-resultats-6}

\chapter{This chunk ensures that the thesisdown package
is}\label{this-chunk-ensures-that-the-thesisdown-package-is}

\chapter{Article annexe}\label{dnah12014}

\chapter*{References}\label{references}
\addcontentsline{toc}{chapter}{References}


% Index?

\end{document}

