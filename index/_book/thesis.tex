% This is the Reed College LaTeX thesis template. Most of the work
% for the document class was done by Sam Noble (SN), as well as this
% template. Later comments etc. by Ben Salzberg (BTS). Additional
% restructuring and APA support by Jess Youngberg (JY).
% Your comments and suggestions are more than welcome; please email
% them to cus@reed.edu
%
% See http://web.reed.edu/cis/help/latex.html for help. There are a
% great bunch of help pages there, with notes on
% getting started, bibtex, etc. Go there and read it if you're not
% already familiar with LaTeX.
%
% Any line that starts with a percent symbol is a comment.
% They won't show up in the document, and are useful for notes
% to yourself and explaining commands.
% Commenting also removes a line from the document;
% very handy for troubleshooting problems. -BTS

% As far as I know, this follows the requirements laid out in
% the 2002-2003 Senior Handbook. Ask a librarian to check the
% document before binding. -SN

%%
%% Preamble
%%
% \documentclass{<something>} must begin each LaTeX document
\documentclass[12pt,twoside]{reedthesis}
% Packages are extensions to the basic LaTeX functions. Whatever you
% want to typeset, there is probably a package out there for it.
% Chemistry (chemtex), screenplays, you name it.
% Check out CTAN to see: http://www.ctan.org/
%%
\usepackage{graphicx,latexsym}
\usepackage[french]{babel} 
\usepackage{amsmath}
\usepackage{amssymb,amsthm}
\usepackage[dvipsnames]{xcolor} % tk: for more color
\usepackage{xcolor}
\usepackage{eso-pic}
\usepackage{longtable,booktabs,setspace}
\usepackage{chemarr} %% Useful for one reaction arrow, useless if you're not a chem major
\usepackage[hyphens]{url}
\usepackage{tikz}
\usetikzlibrary{calc}
\newcommand\HRule{\rule{\textwidth}{1pt}}
% Added by CII
\usepackage{hyperref}
\usepackage{lmodern}
\usepackage{float}
\floatplacement{figure}{H}
% End of CII addition
\usepackage{rotating}
\usepackage{upgreek} % tk : pour pouvoir utiliser le symbole µ droit (pas en itallic)
\usepackage{lscape}
\newcommand{\blandscape}{\begin{landscape}}
\newcommand{\elandscape}{\end{landscape}}
\usepackage[utf8]{inputenc}




% Next line commented out by CII
%%% \usepackage{natbib}
% Comment out the natbib line above and uncomment the following two lines to use the new
% biblatex-chicago style, for Chicago A. Also make some changes at the end where the
% bibliography is included.
%\usepackage{biblatex-chicago}
%\bibliography{thesis}


% Added by CII (Thanks, Hadley!)
% Use ref for internal links
\renewcommand{\hyperref}[2][???]{\autoref{#1}}
\def\chapterautorefname{Chapter}
\def\sectionautorefname{Section}
\def\subsectionautorefname{Subsection}
% End of CII addition

% Added by CII
\usepackage{caption}
\captionsetup{width=5in}
% End of CII addition

% \usepackage{times} % other fonts are available like times, bookman, charter, palatino


% To pass between YAML and LaTeX the dollar signs are added by CII
\title{THÈSE}
\author{Thomas Karaouzene}
\labo{}
% The month and year that you submit your FINAL draft TO THE LIBRARY (May or December)
\date{31 octobre 2017}
\division{}
\advisor{Pierre Ray}
%If you have two advisors for some reason, you can use the following
% Uncommented out by CII
\altadvisor{Nicolas Thierry-Mieg}
% End of CII addition

%%% Remember to use the correct department!
\department{Ingénierie de la Santé, de la Cognition et Environnement (EDISCE)}
% if you're writing a thesis in an interdisciplinary major,
% uncomment the line below and change the text as appropriate.
% check the Senior Handbook if unsure.
%\thedivisionof{The Established Interdisciplinary Committee for}
% if you want the approval page to say "Approved for the Committee",
% uncomment the next line
%\approvedforthe{Committee}

% Added by CII
%%% Copied from knitr
%% maxwidth is the original width if it's less than linewidth
%% otherwise use linewidth (to make sure the graphics do not exceed the margin)
\makeatletter
\def\maxwidth{ %
  \ifdim\Gin@nat@width>\linewidth
    \linewidth
  \else
    \Gin@nat@width
  \fi
}
\makeatother

\renewcommand{\contentsname}{Table of Contents}
% End of CII addition

\setlength{\parskip}{0pt}

% Added by CII
  %\setlength{\parskip}{\baselineskip}
  \usepackage[parfill]{parskip}

\providecommand{\tightlist}{%
  \setlength{\itemsep}{0pt}\setlength{\parskip}{0pt}}

\Acknowledgements{

}

\Dedication{

}

\Preface{
This is an example of a thesis setup to use the reed thesis document
class (for LaTeX) and the R bookdown package, in general.
}

\Abstract{

}

	\usepackage{tikz}
% End of CII addition
%%
%% End Preamble
%%
%

\usepackage{amsthm}
\newtheorem{theorem}{Theorem}[section]
\newtheorem{lemma}{Lemma}[section]
\theoremstyle{definition}
\newtheorem{definition}{Definition}[section]
\newtheorem{corollary}{Corollary}[section]
\newtheorem{proposition}{Proposition}[section]
\theoremstyle{definition}
\newtheorem{example}{Example}[section]
\theoremstyle{remark}
\newtheorem*{remark}{Remark}
\begin{document}

% Everything below added by CII
      \maketitle
  
  \frontmatter % this stuff will be roman-numbered
  \pagestyle{empty} % this removes page numbers from the frontmatter

  
      \begin{preface}
      This is an example of a thesis setup to use the reed thesis document
      class (for LaTeX) and the R bookdown package, in general.
    \end{preface}
  
      \hypersetup{linkcolor=black}
    \setcounter{tocdepth}{3}
    \tableofcontents
  
      \listoftables
  
      \listoffigures
  
  
  
  \mainmatter % here the regular arabic numbering starts
  \pagestyle{fancyplain} % turns page numbering back on

  \chapter{Delete line 6 if you only have one
  advisor}\label{delete-line-6-if-you-only-have-one-advisor}
  
  \chapter*{Conclusion et discussion}\label{conclusion-et-discussion}
  \addcontentsline{toc}{chapter}{Conclusion et discussion}
  
  Répétition de chiffre sur l'infertilité masculine redéfinition de la
  globo\\
  identification de \emph{DPY19L2} comme principal responsable (énnoner et
  décrire la méthode utilisée, carto)\\
  Résultats des différentes études (tailles des LCR, point de cassure,
  dérégulation de certains gènes) Remise en cause des techniques utilisées
  (puce), introduction au NGS
  
  C'est pourquoi nous avons
  
  \newpage
  
  Entre \ldots{} et \ldots{}, notre équipe a effectué le séquençage
  exomique de \ldots{} hommes infertiles. Parmi ces hommes, ,deux frères
  présentaient un phénotype d'échec de fécondation de l'ovocyte, deux
  autres frères étaient azoospermiques et quatre autre familles étaient
  composées d'individus présentant un phénotype MMAF. Les \ldots{}
  patients résetent présentaient eux aussi tous un phénotype MMAF
  maisétaient cette fois-ci non apparentés. Bien que le séquençage de ces
  patients ait été effectués dans 5 centres différents, l'analyse
  bioinformatique des données obtenues à quant à elle été homogène puisque
  elle a été effectué au sein de notre laboratoire par le pipeline
  d'analyse que nous avons spécifiquement dévelopé. Celui-ci,
  contrairement à la plupart des pipelines d'analyse de données WES
  existant tel que \ldots{} et \ldots{} prend en charge l'ensemble des
  étapes de l'analyse allant de l'alignement des \emph{short-reads} sur le
  génome de référence jusqu'à la priorisation des variants en passant bien
  évidemment par l'appel des variants et leur annotation. L'alignements
  des \emph{reads} est effectué par le logiciel MAGIc tandis que l'appel
  des variant est effectué par un algorithme développé dans notre
  laboratoir soécifiquement concu pour analyser les informations fournies
  par MAGIC. Pour l'annotation des variants nous utilisons plusieurs
  ressources exterieurs tel que le logiciel Variant Effect Predictor qui
  va nous informer de l'effet d'un variant sur 'ensemble des transcrits
  qu'il chevauche. De même, les bases de données ExAC ESP6500 ou encore
  1KG nous donne une indication de la fréquence des différents variants
  observés dans la population générale. Afin de ne conserver que les
  variants ayant le plus de risque d'être responsable de la pathologie du
  patient nous avons mis en place une succession de filtre s'appuyant à la
  fois sur les critères qualité des résultats de séquençage, le génotype
  des variants ou encore son l'impact sur la protéine.
  
  Dans le cas des analyses familiales, un autre filtre consistant à
  conserver uniquement l'intersections des variants ayant passé l'ensemble
  des précédents filtres chez l'ensemble des membr'e d'une même famille a
  également été appliqué. Ainsi, dans le cas des analyses familiales, un
  gène candidat a pu être identifié dans la moitié des familles. Parmi ces
  gènes on peut noter le gène \emph{DNAH1} responsable du phénotype MMAF
  d'une de nos 6 familles confirmant ainsi l'importance de l'implication
  de ce gène dans les pathologies flagellaires. De la même manière, la des
  mutations dans le gène \emph{PLC}\(\zeta\)\emph{1} ont pu être reliées
  au phénotype des frères subissant échecs de fécondations confirmant
  l'importance de la protéine PLC\(\zeta 1\) dans le processus
  d'activation ovocytaire. L'étude de la famille azoosperme a permi elle
  de mettre en évidence le gène \emph{SPINK2} jusqu'à présent jamais relié
  à la spermatogénèse chez l'humain bien que ce phénotype avait déjà été
  observé chez les souris KO \emph{SPINK2}\textsuperscript{-/-}.
  
  L'analyse de notre cohorte de \ldots{} patients MMAF non apparentés nous
  a, enraison du nombre important de variants ayant passé les filtres chez
  au moins 1 patient, contraint à mettre en place une stratégie visant à
  prioriser es gènes ayant la plus forte probabilité d'être responsables
  du phénotype MMAF de ces patients. Les gènes candidats identifiés ont
  ensuite pu être classé en fonction d'un niveau de confiance à 3 degrés :
  \emph{High} pour les variants homozygotes ayant un effet tronquant sur
  la protéine, \emph{Moderate} pour les variants homozygotes n'ayant
  \textbf{pas} un effet tronquant et \emph{Low}, pour les gènes sur
  lesquels le patients portait au moins deux variants hétérozygotes. Les
  différentes analyses effectuées nous on ainsi permis de trouver un gène
  candidat pour \ldots{} de nos patients parmi lesquels \ldots{} avait un
  niveau de confiance \emph{High}, \ldots{} \emph{Moderate} et \ldots{}
  \emph{Low}. Parmi l'ensemble de nos patients, il y en a \ldots{} pour
  lesquels deux gènes candidats ont été identifiés, cependant, en nous
  fiant au niveaux de confiance que nous avons instauré, il nous a
  toujours été possible d'en faire ressortir un ayant plus de risque
  d'être le \textbf{vrai} responsable du phénotype. L'étude de cette
  cohorte nous a ainsi permis d'identifier \ldots{} nouveaux gènes
  candidats pour le phénotype MMAF que sont : \ldots{} . De même, nous
  avons également pu retrouver des variants dans le gènes \emph{DNAH1} sur
  lequel \ldots{} de nos patients portent un variant tronquant homozygote,
  \ldots{} un variant homozygote et \ldots{} au moins deux variants
  hétérozygotes. Ainsi, notre pipeline d'analyse nous a permis
  d'identifier un candidat pour \ldots{} de nos patients parmis lesquels
  \ldots{} présentaient des preuves génétiques de forte confiance
  démontrant ainsi son efficacité.
  
  Ces résultats sont cependant à relativiser puisque pour \ldots{} de nos
  patients, aucun candidat n'a pu être identifié soit \ldots{} \% de nos
  patients. Plusieurs raison peuvent expliquer cela. Tout d'abord, au
  cours des analyses décrites dans ce manuscrits nous nous concentrons
  uniquements sur les SNPs et les indels. Cependant de nombreux logiciels
  tel que \ldots{}, \ldots{} ou encore \ldots{} permettent de détecter des
  CNVs à partir de données WES et / ou WGS. Les stratégies de prédictions
  de ces logiciels pouvant être êxtremement différents, le profil des CNVs
  détécté ou non le sera aussi {[}ref {]}. Ainsi, dans des analyses non
  décrites ici, notre équipe a cherchée à identifier des CNVs à partir de
  nos données d'exome à l'aide du logiciel ExomeDepth. Cette approche a
  été extrêmement concluante puisqu'elle a permi d'identifier une délétion
  homozygote sur les gène \emph{WDR66} de \ldots{} de nos patiens pour
  lesquels aucun candidat n'avait été alors identifié. Ces délétions ont
  ensuite pu être confirmées par PCR et la caractérisation de ce gène est
  actuellement en coursq au sein de notre équipe. Au vu de cette réussite,
  il est désormai prévu d'intégrer ce genre d'analyse de manière
  automatique et systématique au sein de notre pipeline. Ensuite, il est
  possible que le choix de la stratégie d'un séquençage exomique plutôt
  que du génome entier ait masqué la cause génétique du phénotype de
  certains de nos patients. En effet, dans ces analyses, nous nous sommes
  concentrés sur l'analyse des variants situés dans les parties codantes
  \textbf{uniquement}. Ainsi les variants situés par exemple dans les
  microARN n'ont pu être observés. Or, les microARN jouent un rôle
  important dans la régulation génique principalement en influant sur la
  stabilité d'ARNm cibles et sont présent en grande quantité au sein des
  cellules germinales et leur importance dans la spermatogénèse a déjà été
  démontrée chez la souris (Comazzetto et al.,
  \protect\hyperlink{ref-Comazzetto2014}{2014}) ainsi que plus récemment
  chez d'autres mammifères dont l'Humain (Chen, Li, Guo, Zhang, \& Zeng,
  \protect\hyperlink{ref-Chen2017}{2017}) laissant penser que des defauts
  altérants ces microARN pourraient entrainer des dysfonctionnement de la
  spermatogénèse. Aussi, il faut noter que les analyses WES \textbf{et}
  WGS ne permettent pas d'observer les défauts épigénétiques, or, ceux-ci
  représentent une part croissante des causes impliquées dans les cas
  d'infertilité masculine (Carrell \& Aston,
  \protect\hyperlink{ref-Carrell2011}{2011}, R Dada, Shamsi, \& Kumar
  (\protect\hyperlink{ref-Dada2011}{2011}), Rima Dada et al.
  (\protect\hyperlink{ref-Dada2012}{2012})). Aussi, au vus du grand nombre
  de gène impliqué dans la spermatogénèse il est très possible que les
  causes génétique responsables d'un même phénotype puissent être très
  hétérogène. Or, à la suite de nos \ldots{} analyses effectués dans
  l'étude de la cohorte MMAF, \ldots{} variants différents étaient
  observés chez les \ldots{} patients pour lesquels aucun candidat n'était
  identifié. Ces variants impactaient \ldots{} gènes différents parmi
  lesquels \ldots{} étaient retrouvé muté chez uniquement 1 seul de ces
  patients. Au vu de se nombre important de gène, il est très compliqué
  d'effectué des analyses poussées sur l'ensemble d'entre eux mettant
  ainsi en évidence la nécéssité de créer de nouveaux filtres afin de
  pouvoir réduire encore cette liste.
  
  C'est dans ce but que notre équipe travail actuellement au développement
  du score MutaScript. Ce score a pour but de classer l'ensemble des
  transcrit codant en fonction de leur charge mutationnelle avec l'idée
  sous-jacente que les transcrits les plus mutés dans la population
  générale ne sont probablement pas impliqués dans des pathologies sévères
  à transmission Mendélienne, \emph{a contrario} ceux retrouvés comme
  n'étant pas / peu mutés le sont probablement. Pour ce faire, le score
  MutaScript repose sur trois (\ldots{}). La première étant le jeu de
  transcrit fournit par Ensembl (Aken et al.,
  \protect\hyperlink{ref-Aken2017}{2017}). Afin de connaitre la charge
  mutationnelle de ces transcrits, nous nous sommes basées sur les
  variants mis à disposition par ExAC (Lek et al.,
  \protect\hyperlink{ref-Lek2016}{2016}) qui réunit les données d'exome de
  60.706 individus non apparentés que nous avons ensuite annoté grâce au
  logiciel \emph{variant effect predictor} (VEP) (McLaren et al.,
  \protect\hyperlink{ref-McLaren2016}{2016}) afin de prédire l'impact de
  chaque variant sur l'ensemble des transcrits qu'ils chevauchent de sorte
  à ce que les variants ayant un impact prédit comme étant délétère aient
  une plus grosse contribution au score MutaScript que ceux ayant un
  impact faible. Ces dernières années, des scores tel que le
  \emph{residual variance intolerance score} (RVIS) (Petrovski et al.,
  \protect\hyperlink{ref-Petrovski2013}{2013}) ou encore \emph{the the
  Probability of loss-of-function Incoherency} (pLI) (Lek et al.,
  \protect\hyperlink{ref-Lek2016}{2016}) ont vu le jour. MutaScript se
  présente comme une alternative à ces derniers scores et, bien que sa
  fonction soit similaire, il diffère de ceux-ci sur de nombreux points.
  Tout d'abord, MutaScript donne un score à l'ensemble des transcrits
  codant pour une protéine là où pLI donne un score seulement au transcrit
  consensus de chaque gène et RVIS qui agrège les séquences codantes de
  l'ensemble des transcrits d'un même gène créant ainsi un transcrit
  ``chimérique''. Ce procédé, bien qu'il facilite l'interprétation du
  score, engendre une perte d'information puisque l'on se retrouve avec un
  seul score par gène et non par transcrits. De plus, dans la conception
  de leur score, RVIS et pLI ne considère que les variants dit
  \emph{loss-of-function} (LoF), c'est à dire les variants impactant
  l'épissage, engendrant un codon stop ou un décalage du cadre de lecture.
  Cependant, ces variants ne représentent que \ldots{}\% des variants
  fournit par la base de données ExAC. C'est pourquoi, MutaScript prend en
  compte l'ensemble des variants, peu importe leur impact sur les
  différents transcrits qu'ils chevauchent, et leur attribue un poids en
  fonction de cet impact de sorte à ce que les variants délétères
  contribuent plus au score d'un transcrits que les autres. Aussi, l'étude
  des scores RVIS et pLI nous a permis de mettre en évidence une
  corrélation forte entre le score qu'ils attribuent à un gène et la
  taille de la séquence codante (CDS) de ce même gène. Cette corrélation
  étant due à un biais causé par leur manière de calculer leur score et
  non à une réalité biologique, MutaScript fut construit de sorte à éviter
  cette corrélation qui peut mener à des erreurs d'interprétations. Le
  développement de ce score étant encore
  
  La Science et le Technique, ont pendant de nombreuses années été
  considérées comme des disiplines distinctes pratiquées, dans la grande
  majorité des cas, des activités de manière indépendantes l'une de
  l'autre et surtout par des personnes différentes n'entretennant aucune
  intéraction. Bien que la distinction entre Science et Technique soit
  réelle, la première pouvant être définit comme la quête de la
  connaissance et de la compréhension du monde tandis que la seconde met
  en oeuvre un ensemble de moyen afin de modifier celui-ci d'une manière
  déterminée à l'avance. Désormais, l'interdépendance entre ces deux
  notions n'a jamais été aussi forte de sorte qu'elles sont souvent
  confondues \ldots{}.. Ainsi, si la Science n'est pas la Technique, elle
  est dans de nombreux cas dépendante de celle-ci. En effet, comme nous
  avons pu le voir, l'étude et la connaissance du génome ont du attendre
  les progrès techniques permettant le séquençage de l'ADN. La Technique,
  elle, n'a pas nécessairement besoin de savoirs scientifiques pour être
  conçue : des savoirs empiriquement acquis suffisent à l'application
  d'une technique. En effet, bien qu'ils n'aient eu aucune conscience des
  mécanismes scientifiques sous-jacent, les premiers Hommes ont su
  maitriser la technique de la production et de l'entretien du feu.
  Cependant la Technique utilise de plus en plus des connaissances
  scientifiques et a ainsi finit par beaucoup dépendre d'Elle donnant
  ainsi naissance à la Technologie qui met en application des découvertes
  scientifiques. Les travaux décrits dans ce manuscrits illustre
  parfaitement cette relation d'interdépendance entre la Science et le
  Technique / Technologie. En effet, la connaissance du génome a été
  permis par l'émmergence des différentes techniques de séquençage qui
  s'appuies elles aussi sur de nombreuses connaissances scientifiques. On
  peut dès lors s'attendre à ce que Science et la Techniques continuent
  d'évoluer de manière concomitantes en s'entre allimentant permettant aux
  prochains progrès technologiques d'êtres à l'origine de découvertes
  scientifiques qui serviront elles même à la fois de socle mais aussi de
  guide aux évolutions technologiques futures.
  
  \newpage
  
  Petit essai sur technologie et recherche
  
  \hypertarget{refs}{}
  \hypertarget{ref-Aken2017}{}
  Aken, B. L., Achuthan, P., Akanni, W., Amode, M. R., Bernsdorff, F.,
  Bhai, J., \ldots{} Flicek, P. (2017). Ensembl 2017. \emph{Nucleic Acids
  Research}, \emph{45}(D1), D635--D642.
  \url{http://doi.org/10.1093/nar/gkw1104}
  
  \hypertarget{ref-Carrell2011}{}
  Carrell, D. T., \& Aston, K. I. (2011). The search for SNPs, CNVs, and
  epigenetic variants associated with the complex disease of male
  infertility. \emph{Systems Biology in Reproductive Medicine},
  \emph{57}(1-2), 17--26.
  \url{http://doi.org/10.3109/19396368.2010.521615}
  
  \hypertarget{ref-Chen2017}{}
  Chen, X., Li, X., Guo, J., Zhang, P., \& Zeng, W. (2017). The roles of
  microRNAs in regulation of mammalian spermatogenesis. \emph{Journal of
  Animal Science and Biotechnology}, \emph{8}, 35.
  \url{http://doi.org/10.1186/s40104-017-0166-4}
  
  \hypertarget{ref-Comazzetto2014}{}
  Comazzetto, S., Di Giacomo, M., Rasmussen, K. D., Much, C., Azzi, C.,
  Perlas, E., \ldots{} O'Carroll, D. (2014). Oligoasthenoteratozoospermia
  and Infertility in Mice Deficient for miR-34b/c and miR-449 Loci.
  \emph{PLoS Genetics}, \emph{10}(10), e1004597.
  \url{http://doi.org/10.1371/journal.pgen.1004597}
  
  \hypertarget{ref-Dada2012}{}
  Dada, R., Kumar, M., Jesudasan, R., Fernández, J. L., Gosálvez, J., \&
  Agarwal, A. (2012). Epigenetics and its role in male infertility.
  \emph{Journal of Assisted Reproduction and Genetics}, \emph{29}(3),
  213--223. \url{http://doi.org/10.1007/s10815-012-9715-0}
  
  \hypertarget{ref-Dada2011}{}
  Dada, R., Shamsi, M., \& Kumar, K. (2011). Genetic and epigenetic
  factors: Role in male infertility. \emph{Indian Journal of Urology},
  \emph{27}(1), 110. \url{http://doi.org/10.4103/0970-1591.78436}
  
  \hypertarget{ref-Lek2016}{}
  Lek, M., Karczewski, K. J., Minikel, E. V., Samocha, K. E., Banks, E.,
  Fennell, T., \ldots{} Exome Aggregation Consortium, D. G. (2016).
  Analysis of protein-coding genetic variation in 60,706 humans.
  \emph{Nature}, \emph{536}(7616), 285--91.
  \url{http://doi.org/10.1038/nature19057}
  
  \hypertarget{ref-McLaren2016}{}
  McLaren, W., Gil, L., Hunt, S. E., Riat, H. S., Ritchie, G. R. S.,
  Thormann, A., \ldots{} Cunningham, F. (2016). The Ensembl Variant Effect
  Predictor. \emph{Genome Biology}, \emph{17}(1), 122.
  \url{http://doi.org/10.1186/s13059-016-0974-4}
  
  \hypertarget{ref-Petrovski2013}{}
  Petrovski, S., Wang, Q., Heinzen, E. L., Allen, A. S., Goldstein, D. B.,
  Davydov, E., \ldots{} Lisacek, F. (2013). Genic Intolerance to
  Functional Variation and the Interpretation of Personal Genomes.
  \emph{PLoS Genetics}, \emph{9}(8), e1003709.
  \url{http://doi.org/10.1371/journal.pgen.1003709}


  % Index?

\end{document}

