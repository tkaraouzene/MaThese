% This is the Reed College LaTeX thesis template. Most of the work
% for the document class was done by Sam Noble (SN), as well as this
% template. Later comments etc. by Ben Salzberg (BTS). Additional
% restructuring and APA support by Jess Youngberg (JY).
% Your comments and suggestions are more than welcome; please email
% them to cus@reed.edu
%
% See http://web.reed.edu/cis/help/latex.html for help. There are a
% great bunch of help pages there, with notes on
% getting started, bibtex, etc. Go there and read it if you're not
% already familiar with LaTeX.
%
% Any line that starts with a percent symbol is a comment.
% They won't show up in the document, and are useful for notes
% to yourself and explaining commands.
% Commenting also removes a line from the document;
% very handy for troubleshooting problems. -BTS

% As far as I know, this follows the requirements laid out in
% the 2002-2003 Senior Handbook. Ask a librarian to check the
% document before binding. -SN

%%
%% Preamble
%%
% \documentclass{<something>} must begin each LaTeX document
\documentclass[12pt,twoside]{reedthesis}
% Packages are extensions to the basic LaTeX functions. Whatever you
% want to typeset, there is probably a package out there for it.
% Chemistry (chemtex), screenplays, you name it.
% Check out CTAN to see: http://www.ctan.org/
%%
\usepackage{graphicx,latexsym}
\usepackage[french]{babel} 
\usepackage{amsmath}
\usepackage{amssymb,amsthm}
\usepackage[dvipsnames]{xcolor} % tk: for more color
\usepackage{xcolor}
\usepackage{eso-pic}
\usepackage{longtable,booktabs,setspace}
\usepackage{chemarr} %% Useful for one reaction arrow, useless if you're not a chem major
\usepackage[hyphens]{url}
\usepackage{tikz}
\usetikzlibrary{calc}
\newcommand\HRule{\rule{\textwidth}{1pt}}
% Added by CII
\usepackage{hyperref}
\usepackage{lmodern}
\usepackage{float}
\floatplacement{figure}{H}
% End of CII addition
\usepackage{rotating}
\usepackage{upgreek} % tk : pour pouvoir utiliser le symbole µ droit (pas en itallic)
\usepackage{pdfpages}
\usepackage{lscape}
\newcommand{\blandscape}{\begin{landscape}}
\newcommand{\elandscape}{\end{landscape}}
\usepackage[utf8]{inputenc}




% Next line commented out by CII
%%% \usepackage{natbib}
% Comment out the natbib line above and uncomment the following two lines to use the new
% biblatex-chicago style, for Chicago A. Also make some changes at the end where the
% bibliography is included.
%\usepackage{biblatex-chicago}
%\bibliography{thesis}


% Added by CII (Thanks, Hadley!)
% Use ref for internal links
\renewcommand{\hyperref}[2][???]{\autoref{#1}}
\def\chapterautorefname{Chapter}
\def\sectionautorefname{Section}
\def\subsectionautorefname{Subsection}
% End of CII addition

% Added by CII
\usepackage{caption}
\captionsetup{width=5in}
% End of CII addition

% \usepackage{times} % other fonts are available like times, bookman, charter, palatino


% To pass between YAML and LaTeX the dollar signs are added by CII
\title{THÈSE}
\author{Thomas Karaouzene}
\labo{}
% The month and year that you submit your FINAL draft TO THE LIBRARY (May or December)
\date{31 octobre 2017}
\division{}
\advisor{Pierre Ray}
%If you have two advisors for some reason, you can use the following
% Uncommented out by CII
\altadvisor{Nicolas Thierry-Mieg}
% End of CII addition

%%% Remember to use the correct department!
\department{Ingénierie de la Santé, de la Cognition et Environnement (EDISCE)}
% if you're writing a thesis in an interdisciplinary major,
% uncomment the line below and change the text as appropriate.
% check the Senior Handbook if unsure.
%\thedivisionof{The Established Interdisciplinary Committee for}
% if you want the approval page to say "Approved for the Committee",
% uncomment the next line
%\approvedforthe{Committee}

% Added by CII
%%% Copied from knitr
%% maxwidth is the original width if it's less than linewidth
%% otherwise use linewidth (to make sure the graphics do not exceed the margin)
\makeatletter
\def\maxwidth{ %
  \ifdim\Gin@nat@width>\linewidth
    \linewidth
  \else
    \Gin@nat@width
  \fi
}
\makeatother

\renewcommand{\contentsname}{Table of Contents}
% End of CII addition

\setlength{\parskip}{0pt}

% Added by CII
  %\setlength{\parskip}{\baselineskip}
  \usepackage[parfill]{parskip}

\providecommand{\tightlist}{%
  \setlength{\itemsep}{0pt}\setlength{\parskip}{0pt}}

\Acknowledgements{

}

\Dedication{

}

\Preface{
This is an example of a thesis setup to use the reed thesis document
class (for LaTeX) and the R bookdown package, in general.
}

\Abstract{

}

	\usepackage{tikz}
% End of CII addition
%%
%% End Preamble
%%
%

\usepackage{amsthm}
\newtheorem{theorem}{Theorem}[section]
\newtheorem{lemma}{Lemma}[section]
\theoremstyle{definition}
\newtheorem{definition}{Definition}[section]
\newtheorem{corollary}{Corollary}[section]
\newtheorem{proposition}{Proposition}[section]
\theoremstyle{definition}
\newtheorem{example}{Example}[section]
\theoremstyle{remark}
\newtheorem*{remark}{Remark}
\begin{document}

% Everything below added by CII
      \maketitle
  
  \frontmatter % this stuff will be roman-numbered
  \pagestyle{empty} % this removes page numbers from the frontmatter

  
      \begin{preface}
      This is an example of a thesis setup to use the reed thesis document
      class (for LaTeX) and the R bookdown package, in general.
    \end{preface}
  
      \hypersetup{linkcolor=black}
    \setcounter{tocdepth}{3}
    \tableofcontents
  
      \listoftables
  
      \listoffigures
  
  
  
  \mainmatter % here the regular arabic numbering starts
  \pagestyle{fancyplain} % turns page numbering back on

  \chapter*{Résumé}\label{resume}
  \addcontentsline{toc}{chapter}{Résumé}
  
  \newpage
  
  L'infertilité masculine est définie comme étant l'incapacité d'aboutir à
  une grossesse après 12 mois ou plus de rapports sexuels réguliers non
  protégés. Dix à quinze pour cent des couples font face à des problèmes
  d'infertilité soit plus de 70 millions de personnes dans le monde,
  faisant de cette pathologie un véritable enjeu de santé publique. Bien
  que multifactorielle, l'infertilité masculine a une composante génétique
  importante et encore peu explorée.
  
  Cependant, l'investigation des phénotypes ayant une composante
  génétique, et donc de l'infertilité a été bouleversée ces dix dernières
  années par l'émergence des techniques de séquençage haut-débit. Ces
  nouvelles technologies permettent désormais de ne plus séquencer les
  gènes un par un, mais de pouvoir avoir accès à l'intégralité de la
  séquence génomique d'un individu dans un temps et pour un coût de plus
  en plus réduits. Cependant, les masses de données générées lors de ces
  procédés sont désormais trop importantes pour pouvoir être analysées
  uniquement par l'humain, rendant dès lors indispensable l'utilisation de
  l'outil informatique. Une fois les données de séquençage obtenues,
  plusieurs étapes sont nécessaires pour arriver à une liste de variants
  suffisamment restreinte pour pouvoir être interprétée. Chacune de ces
  étapes doit donc être pensée pour correspondre au mieux aux besoins de
  l'étude. Cependant, chacune d'entre elles peut être source d'erreurs et
  de biais, complexifiant ainsi l'interprétation des données. Dans ce
  contexte, l'objectif de ma thèse a été double.
  
  Dans un premier temps, j'ai pu poursuivre les investigations génétiques
  et moléculaires menées au sein du laboratoire. Ainsi, j'ai contribué à
  la caractérisation du gène \emph{DPY19L2} dont la délétion homozygote
  est responsable de la majeure partie des cas de globozoospermie, une
  teratozoospermie caractérisée entre autres par la présence de
  spermatozoïdes à tête ronde et dépourvus d'acrosome dans l'éjaculat.
  Pour cela, j'ai tout d'abord contribué à caractériser les mécanismes
  responsables de cette délétion récurrente. Ensuite, le modèle murin
  \emph{Dpy19l2} KO mimant le phénotype humain, j'ai réalisé une étude
  comparative des transcriptomes testiculaires de souris sauvages
  \emph{Dpy19l2}\textsuperscript{+/+} et de souris KO
  \emph{Dpy19l2}\textsuperscript{-/-}. Cette étude effectuée sur puce à
  ADN avait pour objectif principal de mettre en évidence des
  dérégulations transcriptomiques chez les souris KO pouvant notamment
  expliquer le faible taux de réussite des fécondations \emph{in vitro}
  effectuées avec des spermatozoïdes globozoocéphales, même dans les cas
  où celles-ci sont effectuées par \emph{intra cytoplasmic sperm
  injection} (ICSI). Cette étude a ainsi permis de mettre en évidence la
  dérégulation de 76 gènes chez la souris KO. Parmi ceux-ci, 23 sont
  impliqués dans la liaison d'acides nucléiques et de protéines, pouvant
  ainsi expliquer les défauts d'ancrage de l'acrosome au noyau chez les
  spermatozoïdes globozoocéphales.
  
  Dans un second temps j'ai développé un pipeline d'analyse de données
  issues de séquençage haut-débit prenant en compte l'intégralité des
  étapes effectuées en aval du processus de séquençage. Ainsi, ce pipeline
  effectue l'alignement en utilisant le logiciel MAGIC qui fournit, une
  fois l'alignement terminé, un comptage du nombre de \emph{reads} portant
  l'allèle référence et variant pour chaque position couverte. Ce comptage
  est ensuite utilisé par un algorithme développé au sein du laboratoire
  pour effectuer l'appel des variants à l'issu duquel, une liste de
  variant ainsi que leur génotype est obtenu pour chaque patient. Ces
  variants sont ensuite annotés, principalement grâce au logiciel
  \emph{Variant Effect Predictor} (VEP) fournira l'impact de chacun
  d'entre eux sur les transcrits qu'ils chevauchent. Suite à cela,
  plusieurs paramètres de filtres ont été implémentés de sorte à ne
  conserver que les variants les plus intéressant dans le contexte de
  l'étude. Chacune de ces étapes a été calibrée afin de correspondre aux
  critères standards de recherche de variants impliqués dans des
  pathologies à transmission mendélienne, tout en tirant parti des
  spécificités des études que nous réalisons au laboratoire. Le pipeline a
  été utilisé dans cinq études au sein du laboratoire, parmi lesquelles
  trois se basent sur des cas familiaux et deux sur des cohortes
  d'individus non apparentés, présentant différents phénotypes
  d'infertilité. Dans ces cinq projets il a permis l'identification de
  gènes candidats prometteurs, qui ont ensuite pu être validés
  expérimentalement.
  
  \newpage
  
  \chapter*{Abstract}\label{abstract}
  \addcontentsline{toc}{chapter}{Abstract}
  
  \newpage
  
  Même chose en anglais
  
  \chapter{Introduction}\label{introInf}
  
  \section{La spermatogénèse}\label{la-spermatogenese}
  
  La spermatogenèse des mammifères est un processus long et complexe
  contrôlé par plusieurs mécanismes étroitement liés
  {[}\protect\hyperlink{ref-Gnessi1997}{1}--\protect\hyperlink{ref-KIERSZENBAUM1994}{3}{]}.
  C'est au cours de celle-ci qu'à partir de cellules germinales, seront
  produits les spermatozoïdes matures. Ce processus est divisé en trois
  phases principales : La phase de multiplication, la phase de division
  (appelée la \protect\hyperlink{meiose}{méiose}) et la phase de
  maturation. Chez les hommes, ces étapes se déroulent en continu dans la
  paroi des tubes séminifères du testicule depuis la puberté jusqu'à la
  mort et impliquent trois types de cellules germinales : les
  spermatogonies, les spermatocytes et les spermatides. Le temps
  nécessaire pour obtenir un spermatozoïde mature à partir de cellules
  germinales est de 74 jours et la production quotidienne de spermatozoïde
  s'élève environ à 45 million par testicules
  {[}\protect\hyperlink{ref-Johnson1980}{4}{]}. Le cycle spermatogénétique
  est défini comme la succession chronologique des différents stades de
  différenciation d'une génération de cellules germinales (depuis la
  spermatogonie jusqu'au spermatozoïde). Chacune des étapes du cycle
  spermatogénétique a une durée fixe et constante selon les espèces
  (\textbf{Table : }\ref{tab:spermatotime}).
  
  \begin{longtable}[t]{ll}
  \caption{\label{tab:spermatotime}Durée de vie moyenne des cellules germinales humaines}\\
  \toprule
  Cellules germinales & Durée de vie moyenne (jours)\\
  \midrule
  Spermatogonies Ap & 16-18\\
  Spermatogonie B & 7.5-9\\
  Spermatocytes primaires & 23\\
  Spermatocytes secondaires & 1\\
  Spermatides & 1\\
  \bottomrule
  \end{longtable}
  
  \newpage
  
  \subsection{Rappels sur le testicule}\label{rappels-sur-le-testicule}
  
  Les testicules sont les organes sexuels masculins. Ils possèdent deux
  fonctions principales plus ou moins exprimées selon les périodes de la
  vie de l'individu : une fonction endocrine caractérisée par la synthèse
  des hormones stéroïdes sexuelles masculines (la stéroïdogenèse) et une
  fonction exocrine au cours de laquelle seront produits les gamètes
  masculins. Chez un individu adulte en bonne santé, le testicule présente
  une forme ovoïde ayant un volume moyen de 18 cm\textsuperscript{3}. Chez
  l'homme, comme chez la plupart des mammifères terrestres, ils sont
  localisés sous le pénis dans une poche de peau appelée scrotum et reliés
  à l'abdomen par le cordon spermatique (\textbf{Figure :}
  \ref{fig:testicule}). Cette externalisation des testicules permet leur
  maintien à une température plus basse que celle du reste du corps
  nécessaire à la spermatogenèse.\\
  L'intérieur du testicule contient des tubes séminifères enroulés ainsi
  que du tissu entre les tubules appelé espace interstitiel. Les tubes
  séminifères sont de longs tubes compactés sous forme de boucles et dont
  les deux extrémités débouchent sur le \emph{rete testis} (\textbf{Figure
  :} \ref{fig:testicule}). C'est le long des parois du tube séminifère que
  se déroulera l'ensemble des étapes de la spermatogenèse.
  
  \begin{figure}
  
  {\centering \includegraphics[scale=0.65]{figure/coupe_testicule2} 
  
  }
  
  \caption{Schéma anatomique du testicule humain}\label{fig:testicule}
  \end{figure}
  
  \newpage
  
  \subsection{La phase de
  multiplication}\label{la-phase-de-multiplication}
  
  La phase de multiplication est la phase au cours de laquelle les
  spermatogonies se divisent par mitoses pour aboutir au stade de
  spermatocytes primaires. Les spermatogonies sont des cellules diploïdes
  à l'origine de l'ensemble des autres cellules germinales humaines. Pour
  cela, elles vont s'auto-renouveler par mitoses successives afin de
  maintenir une production continue de spermatozoïdes tout au long de la
  vie de l'individu. Ces cellules sont localisées dans le compartiment
  basal des tubes séminifères. Les analyses histologiques ont permis de
  distinguer trois types de spermatogonies en fonction de leur contenu en
  hétérochromatine
  {[}\protect\hyperlink{ref-Clermont1963}{5}--\protect\hyperlink{ref-Goossens2013}{7}{]}
  : Les spermatogonies de type A dark (ou Ad), les spermatogonies de type
  A pale (ou Ap) et les spermatogonies de type B.
  
  Chez l'Homme, les spermatogonies Ad ont une activité mitotique au cours
  de la spermatogénèse et servent de réserve. Elles vont au cours d'une
  première mitose former une spermatogonie Ad et un spermatogonie Ap
  (\textbf{Figure :} \ref{fig:spermatogenese}). Cette propriété permet à
  la fois de se différencier en spermatocytes tout en constituant un
  compartiment de réserve de spermatogonies Ad pour la régénération de la
  population de cellules germinales au sein de l'épithélium séminifère.
  L'entrée en division des spermatogonies Ap se fait par groupes
  cellulaire tous les 16 jours. Les cellules d'une même génération
  maintiennent entre elles des ponts cytoplasmiques jusqu'à la
  \protect\hyperlink{spermiogenese}{spermiogénèse} ce qui permet la
  synchronisation parfaite du développement gamétique de toutes les
  cellules filles issues d'un groupe de spermatogonies Ap. Ce phénomène
  est appelé onde spermatogénétique. Chaque spermatogonie Ap va,
  lorsqu'elle se divise par mitose, former deux spermatogonies B qui
  elles-mêmes se diviseront en deux spermatocytes primaires diploïdes
  (\textbf{Figure :} \ref{fig:spermatogenese}).
  
  \newpage
  
  \begin{figure}
  
  {\centering \includegraphics[scale=0.35]{figure/spermatogenese2} 
  
  }
  
  \caption{Les différentes phases de la spermatogénèse d'après [medizin-kompakt](http://www.medizin-kompakt.de/spermatogenese) : description à écrire !!!}\label{fig:spermatogenese}
  \end{figure}
  
  \newpage 
  
  \hypertarget{meiose}{\subsection{La méiose}\label{meiose}}
  
  La méiose, ou phase de maturation, est l'étape au cours de laquelle, à
  partir de cellules diploïdes (les spermatogonies B) vont se former des
  cellules haploïdes, les spermatocytes secondaires (spermatocytes II). Ce
  résultat est le fruit de deux divisions successives (\textbf{Figure :
  }\ref{fig:meiose}) appelée respectivement méiose réductionnelle ou
  méiose I (MI) et méiose équationnelle ou méiose II (MII). La MI va
  séparer les chromosomes homologues, produisant deux cellules et
  réduisant la ploïdie de diploïde à haploïde (d'où son non
  \emph{réductionnelle}). En plus de son rôle de division vu précédemment,
  la méiose joue un rôle clef dans le brassage génétique (mélange des
  gènes) et ce, grâce à deux mécanismes de brassage : le brassage
  inter-chromosomique, lorsque les chromosomes sont séparés et le brassage
  intra-chromosomique impliquant notamment des enjambements chromosomiques
  (crossing-over) (\textbf{Figure : }\ref{fig:crossingover}).
  
  La méiose est initiée dès la fin de la phase de multiplication à partir
  des spermatocytes primaires issus de la division des spermatogonies de
  type B. Ces cellules nouvellement formées se situent dans le
  compartiment basal du tube séminifère. C'est là qu'ils vont tout d'abord
  subir une interphase (stade préleptotène) durant entre 2 et 4 jours. Au
  cours de cette phase a lieu la réplication de l'ADN. Cette réplication
  se fait lorsque l'ADN est à l'état de chromatine, pendant la phase S
  (pour synthèse) de l'interphase. À l'issue de cette phase, chaque
  chromosome sera composé de deux chromatides reliées entre elles par le
  centromère, le matériel génétique de chaque cellule ayant donc été
  multiplié par 2. Par la suite, ces cellules vont subir deux divisions
  méiotiques, chacune composées de 4 étapes distinctes (\textbf{Figure :
  }\ref{fig:meiose}) :
  
  \begin{figure}
  
  {\centering \includegraphics[scale=0.33]{figure/Meiosis_Stages} 
  
  }
  
  \caption[Les différentes étapes de la méiose gamétique masculine]{Les différentes étapes de la méiose gamétique masculine d'après Sasaki et Matsui, 2008}\label{fig:meiose}
  \end{figure}
  
  \newpage
  
  \begin{enumerate}
  \def\labelenumi{\arabic{enumi}.}
  \tightlist
  \item
    \textbf{Méiose réductionnelle} : (\textbf{Figure : }\ref{fig:meiosei})
  
    \begin{enumerate}
    \def\labelenumii{\alph{enumii}.}
    \item
      \textbf{La prophase I} : Cette longue étape dure 23 jours chez
      l'homme et peut être subdivisée en 5 phases successives : leptotène,
      zygotène, pachytène, diplotène et diacinèse.
  
      \begin{enumerate}
      \def\labelenumiii{\roman{enumiii}.}
      \tightlist
      \item
        \textbf{Leptotène} : condensation de la chromatine et formation
        des chromosomes.\\
      \item
        \textbf{Zygotène} : Appariement des chromosomes homologues par
        paires appelées bivalents grâce l'intermédiaire d'une structure
        multi-protéique : le complexe synaptonémal.\\
      \item
        \textbf{Pachytène} : Ce stade dure 16 jours et est le plus long de
        la prophase I. C'est au cours de celui-ci qu'à lieu l'échange de
        matériel génétique par le biais des crossing-over entre les
        chromatides non-sœurs appelés nodules de recombinaison
        (\textbf{Figure : }\ref{fig:crossingover}).\\
      \item
        \textbf{Diplotène} : La dissociation du complexe synaptonémal va
        permettre aux chromosomes homologues d'initier leur séparation.
        Certains sites d'appariement étroits nommés chiasmas demeurent
        néanmoins liés permettant une séparation plus progressive des
        chromosomes et réduisant ainsi le risque d'aneuploïdies (nombre
        anormal de chromosomes)
        {[}\protect\hyperlink{ref-Handyside2012}{8}{]}.\\
      \item
        \textbf{Diacinèse} : Cette étape marque la fin de la méiose I et
        fait office de transition avec la méiose II. Elle est caractérisée
        par une condensation maximale des chromosomes et la disparition de
        la membrane nucléaire et du nucléole. Le fuseau méiotique commence
        à s'assembler, les centromères des chromosomes homologues
        s'éloignent et les chiasmas glissent progressivement vers les
        télomères.\\
      \end{enumerate}
    \item
      \textbf{La métaphase I} : phase au cours de laquelle les chromosomes
      vont s'aligner à l'équateur de la cellule pour former la plaque
      équatoriale.
    \item
      \textbf{L'anaphase I} : les chromatides sœurs (ou les chromosomes
      homologues en fonction de la phase méiotique) vont se séparer et
      migrer aux pôles opposés de la cellule.\\
    \item
      \textbf{La télophase I} : qui est l'étape finale, les chromosomes se
      décondensent et l'enveloppe nucléaire se reforme autours des
      chromosomes. La cellule mère se sépare alors en deux cellules filles
      appelées spermatocytes secondaires.
    \end{enumerate}
  \end{enumerate}
  
  \newpage
  
  \begin{enumerate}
  \def\labelenumi{\arabic{enumi}.}
  \setcounter{enumi}{1}
  \tightlist
  \item
    \textbf{Méiose équationnelle} : (\textbf{Figure : }\ref{fig:meioseii})
    La MII est similaire à une division mitotique et peut se décomposer en
    4 parties distinctes :
  
    \begin{enumerate}
    \def\labelenumii{\alph{enumii}.}
    \tightlist
    \item
      \textbf{La prophase II} : Contrairement à la prophase I, la prophase
      II est très courte. Les chromosomes alors formés de deux chromatides
      sœurs se dirigent vers la plaque équatoriale.\\
    \item
      \textbf{La métaphase II} : À ce stade, les chromosomes sont alignés
      le long de la plaque équatoriale au niveau de leur centromère.\\
    \item
      \textbf{L'anaphase II} : Les centromères de chaque chromosome se
      séparent permettant aux chromatides sœurs de se diriger vers les
      pôles opposés des spermatocytes II.\\
    \item
      \textbf{La télophase II} : Comme en télophase I, les cellules mères
      se séparent en deux cellules filles haploïdes appelées spermatides,
      contenant chacune n chromosomes.
    \end{enumerate}
  \end{enumerate}
  
  \begin{figure}
  
  {\centering \includegraphics[scale=0.35]{figure/crossingover} 
  
  }
  
  \caption{Schéma simplifié d'un enjambement chromosomique (crossing-over)}\label{fig:crossingover}
  \end{figure}
  
  \newpage 
  
  \begin{figure}
  
  {\centering \includegraphics[scale=0.43]{figure/MeiosisI} 
  
  }
  
  \caption[Les différentes étapes de la première division méiotique masculine adapté]{Les différentes étapes de la première division méiotique masculine adapté d'après [Wikipédia](https://en.wikipedia.org/wiki/Meiosis)}\label{fig:meiosei}
  \end{figure}
  
  \begin{figure}
  
  {\centering \includegraphics[scale=0.43]{figure/MeiosisII} 
  
  }
  
  \caption[Les différentes étapes de la deuxième division méiotique masculine adapté]{Les différentes étapes de la deuxième division méiotique masculine adapté d'après [Wikipédia](https://en.wikipedia.org/wiki/Meiosis)}\label{fig:meioseii}
  \end{figure}
  
  \newpage
  
  \hypertarget{spermiogenese}{\subsection{La
  spermiogénèse}\label{spermiogenese}}
  
  La spermiogénèse est la phase finale de la spermatogénèse. Elle dure
  environ 23 jours chez l'humain et peut être subdivisée en sept étapes
  (\textbf{Figure : }\ref{fig:spermiogenese}). La spermiogénèse définie la
  cytodifférentiation des spermatides en spermatozoïdes. C'est au cours de
  cette phase que les caractéristiques morphologiques et fonctionnelles du
  spermatozoïde seront déterminées
  {[}\protect\hyperlink{ref-YvesClermontRichardOko1993}{9}{]}. Elle est
  caractérisée par 3 évènements majeurs : la formation de l'acrosome, la
  compaction de l'ADN nucléaire et la formation du flagelle. Le
  développement de l'acrosome et la formation du flagelle commencent au
  niveau des spermatides rondes
  {[}\protect\hyperlink{ref-Escalier1991}{10}{]}. Pendant l'élongation de
  la spermatide, le noyau se condense et devient hautement polarisé
  {[}\protect\hyperlink{ref-Hamilton1987}{11}{]}.\\
  Les spermatides sont situées dans le compartiment adluminal, à proximité
  de la lumière du tube séminifère. Ce sont de petites cellules (8 à 10
  \(\upmu\)m) que l'on peut schématiquement diviser en trois classes :
  
  \begin{enumerate}
  \def\labelenumi{\arabic{enumi}.}
  \item
    \textbf{Les spermatides rondes} (\textbf{Figure :
    }\ref{fig:spermiogenese} - \textbf{1} et \textbf{2}) :
    L'identification de ces cellules représente une difficulté technique.
    Elles ont cependant pu être décrites en détail par différentes
    techniques de coloration sous microscope optique
    {[}\protect\hyperlink{ref-Clermont1963}{5},
    \protect\hyperlink{ref-Papic}{12}--\protect\hyperlink{ref-WorldHealthOrganization1992}{15}{]}.
    Plusieurs études animales ont pu démontrer le potentiel des
    spermatides rondes à donner la vie à des individus sains et fertiles,
    {[}\protect\hyperlink{ref-Ogura1994}{16}--\protect\hyperlink{ref-Sasagawa}{18}{]},
    la même chose ayant été également observée plus récemment chez l'homme
    {[}\protect\hyperlink{ref-Tanaka2015}{19}{]} bien que le taux de
    fécondation et d'implantation soit extrêmement faible
    {[}\protect\hyperlink{ref-Asimakopoulos2003}{20}{]}. Ils possèdent un
    noyau rond avec une chromatine pâle et homogène. C'est à partir de ces
    étapes que démarre la biogenèse de l'acrosome avec la production par
    l'appareil de Golgi des vésicules pro-acrosomales (phase de Golgi).
    Les deux centrioles contenus dans le cytoplasme vont se déplacer au
    futur pôle caudal. Le centriole proximal est inactif alors que le
    centriole distal donne naissance à un ensemble de microtubules à
    l'origine de l'axonème du futur flagelle.
  \item
    \textbf{Les spermatides en élongation} (\textbf{Figure :
    }\ref{fig:spermiogenese} - \textbf{3} et \textbf{4}) : À ce stade,
    l'acrosome va s'étendre le long du noyau lui donnant une forme plus
    allongée et la chromatine devient plus sombre. Un réseau de
    microtubule se forment autours du noyau créant ainsi la manchette qui
    participera également à l'allongement de la tête du spermatozoïde et
    permettra la migration des mitochondries vers la pièce intermédiaire
    du flagelle former le manchon de mitochondries
    {[}\protect\hyperlink{ref-Moreno2006}{21}{]}. Les spermatides en
    élongation peuvent aussi permettre la fécondation et d'initier des
    grossesses avec un meilleur taux que les spermatides rondes et
    engendreraient théoriquement moins de risques d'anomalies génétiques
    {[}\protect\hyperlink{ref-Asimakopoulos2003}{20}{]}.
  \end{enumerate}
  
  \newpage
  
  \begin{enumerate}
  \def\labelenumi{\arabic{enumi}.}
  \setcounter{enumi}{2}
  \tightlist
  \item
    \textbf{Les spermatides en condensation} (\textbf{Figure :
    }\ref{fig:spermiogenese} - \textbf{5} et \textbf{7}) : C'est le stade
    final de la différentiation de la spermatide en spermatozoïde. À ce
    stade le noyau est très allongé, avec une partie caudale globulaire et
    une partie antérieure saillante. La chromatine est sombre et
    condensée. L'axonème va continuer à s'allonger pour former le flagelle
    mature. Les différentes organelles inutiles pour la physiologique
    spermatique et l'excès de cytoplasme vont former la gouttelette
    cytoplasmique qui va se détacher et donner le corps résiduel qui va
    ensuite être phagocyté par les cellules de Sertoli
    {[}\protect\hyperlink{ref-Hermo2010}{22}{]}.
  \end{enumerate}
  
  Une fois ces étapes de différentiation finies, les spermatides sont
  relâchées en tant que spermatozoïdes dans la lumière du tube séminifère.
  Ce procédé est appelé spermiation.
  
  \begin{figure}
  
  {\centering \includegraphics[scale=0.3]{figure/spermiogenese} 
  
  }
  
  \caption[Principales étapes et modifications structurales lors de la spermiogénèse]{Principales étapes et modifications structurales lors de la spermiogénèse : 1. La spermatide immature avec un gros noyau arrondi. La vésicule acrosomale est attachée au noyau, l’ébauche du flagelle n’atteint pas le noyau. 2. La vésicule acrosomale a augmenté de taille et apparaît aplatie au niveau du noyau. Le flagelle entre en contact avec le noyau. 3-7. Formation de l’acrosome, condensation du noyau et développement des structures flagellaires. Ac, acrosome ; Ax, axonème ; CC, corps chromatoïdes ; CR, corps résiduel ; FD, fibres denses ; GF, gaine fibreuse ; M, mitochondrie ; Ma, manchette. d'après [@Toure2011]}\label{fig:spermiogenese}
  \end{figure}
  
  \newpage  
  
  \section{Structure et fonction du
  spermatozoïde}\label{structure-et-fonction-du-spermatozoide}
  
  Le spermatozoïde est une cellule hautement différenciée dont la taille,
  l'orientation et la symétrie sont déterminées. La morphologie générale
  du spermatozoïde éjaculé est similaire à celle du spermatozoïde
  testiculaire. Le spermatozoïde humain normal mature mesure environ 60
  \(\upmu\)m de long et est essentiellement constitué de deux parties : la
  tête et le flagelle (\textbf{Figure : }\ref{fig:spz}). En plus d'être
  unique dans sa morphologie, le spermatozoïde l'est aussi dans sa
  fonction puisque c'est la seule cellule produite de manière endogène et
  dont l'action est exercée de manière exogène. La fécondation d'un
  ovocyte par un spermatozoïde formera un zygote diploïde qui pourrase
  développer ensuite en embryon dans l'uterus féminin.
  
  \begin{figure}
  
  {\centering \includegraphics[scale=.75]{figure/sperm_anatomy} 
  
  }
  
  \caption[Anatomie simplifiée du spermatozoïde]{Anatomie simplifiée du spermatozoïde}\label{fig:spz}
  \end{figure}
  
  \newpage
  
  \subsection{La tête}\label{la-tete}
  
  \begin{enumerate}
  \def\labelenumi{\arabic{enumi}.}
  \tightlist
  \item
    \textbf{L'acrosome} : C'est une vésicule de sécrétion géante située
    dans la moitié supérieure de la tête du spermatozoïde. Elle se
    développe à partir de l'appareil de Golgi lors de la spermiogénèse. Au
    cours de sa formation, l'acrosome forme tout d'abord un granule
    sphérique qui se colle sur la partie apicale du noyau. En
    s'aplatissant contre celui-ci, l'acrosome va prendre une forme
    hémisphérique recouvrant la membrane nucléaire formant la coiffe
    céphalique. Le rôle de l'acrosome est fondamental dans le processus de
    fécondation puisqu'il permet d'excréter notamment l'acrosine, une
    enzyme de digestion permettant au spermatozoïde de traverser la zone
    pellucide qui entoure les ovocytes. Ce processus de relargage est
    appelé réaction acrosomale.\\
  \item
    \textbf{L'acroplaxome} : L'acroplaxome est une structure cytosquelette
    composée de microfilaments d'actine (F- actine) et de kératine 5.
    Cette structure est positionnée en face de l'appareil de golgi et
    contre le noyau et sert de point d'attachement ainsi que de guide aux
    vésicules pro-acrosomales
    {[}\protect\hyperlink{ref-Kierszenbaum2004}{23}{]}. C'est une
    structure transitoire qui disparaît pour être remplacée par la thèque
    périnucléaire dans le spermatozoïde mature.\\
  \item
    \textbf{Le noyau} : C'est une structure cellulaire présente dans la
    majorité des cellules eucaryotes. Il contient l'essentiel du matériel
    génétique. Le noyau du spermatozoïde est caractérisé par une
    compaction extrêmement importante de l'ADN. Dans les cellules
    somatiques l'ADN est enroulé par unité de 146 paires de bases autour
    d'un octamère d'histones dit de cœur (H2A, H2B, H3 et H4) afin
    d'organiser les 3 milliards de paires de bases du génome humain dans
    un noyau de quelques microns (\textbf{Figure : }\ref{fig:noyau}).
    L'ADN des spermatides va subir une réorganisation chromatinienne plus
    importante au cours de la spermatogénèse afin d'augmenter sa
    compaction. Ainsi, les octamères d'histones présents dans les cellules
    somatiques sont remplacés par les protéines de transition (TPN1, TPN2)
    puis par les protamines (PRM1, PRM2) deux protéines riches en arginine
    et en cystéine (\textbf{Figure : }\ref{fig:noyau}). L'intégrité des
    deux protéines composant ce dimère est nécessaire pour la procréation
    {[}\protect\hyperlink{ref-Cho2001}{24}{]}. Cette compaction extrême
    permet de réduire la taille du noyau, mais aussi de protéger l'ADN
    d'agents de dégradation comme l'oxydation des bases. Parallèlement à
    cette condensation chromatinienne se produit un arrêt des processus de
    transcription cellulaire
    {[}\protect\hyperlink{ref-Kierszenbaum1978}{25}{]}. Le noyau du
    spermatozoïde est donc un noyau au repos, transcriptionnellement
    inactif {[}\protect\hyperlink{ref-Ward1994}{26}{]}
  \end{enumerate}
  
  \newpage
  
  \begin{figure}
  
  {\centering \includegraphics[scale=.55]{figure/noyau} 
  
  }
  
  \caption[Schéma de la compaction de l’ADN dans les cellules somatiques et dans les spermatozoïdes]{Schéma de la compaction de l’ADN dans les cellules somatiques et dans les spermatozoïdes : D'après Braun (2001)}\label{fig:noyau}
  \end{figure}
  
  \newpage
  
  \subsection{Le flagelle}\label{le-flagelle}
  
  Le flagelle représente la queue du spermatozoïde. Celui-ci permet, par
  mouvement d'oscillation à haute vitesse, le déplacement du
  spermatozoïde. Cette mobilité est générée par un cytosquelette interne
  extrêmement conservé durant l'évolution appelé l'axonème. Celui-ci est
  composé de neuf doublets de microtubules périphériques et de deux
  doublets internes {[}\protect\hyperlink{ref-Inaba2003}{27}{]}
  (\textbf{Figure : }\ref{fig:axoneme}), on parle alors de structure ``9 +
  2''. Les doublets externes sont reliés entre eux par des ponts de nexine
  et au doublet central par des ponts radiaires. Les doublets externes
  sont également reliés entre eux par les complexes protéiques qui forment
  les dynéines externes et internes. Ce sont ces protéines qui en exerçant
  une contraction alternée permettent le mouvement du spermatozoïde.
  
  \begin{figure}
  
  {\centering \includegraphics[scale=.3]{figure/axoneme} 
  
  }
  
  \caption[Structure simplifiée de l'axonème]{Structure simplifiée de l'axonème d'après [@Inaba2003] : L'axonème est constitué de neuf doublets de microtubules périphériques reliés entre eux par des liens de nexine et d'un doublet central relié aux doublets périphériques par des ponts radiaires}\label{fig:axoneme}
  \end{figure}
  
  Le flagelle du spermatozoïde peut être divisé en trois partie distinctes
  (\textbf{Figure : }\ref{fig:flagelle}) :
  
  \begin{enumerate}
  \def\labelenumi{\arabic{enumi}.}
  \tightlist
  \item
    \textbf{La pièce intermédiaire} : Elle fait jonction avec la tête du
    spermatozoïde et est composée de la gaine de mitochondrie qui fournira
    une partie de de l'énergie nécessaire au battement flagellaire (grâce
    à la phosphorylation oxydative qui produit de l'ATP). L'axonème qui se
    prolonge dans la pièce principale et un ensemble de neuf faisceaux de
    fibres denses.
  \end{enumerate}
  
  \newpage
  
  \begin{enumerate}
  \def\labelenumi{\arabic{enumi}.}
  \setcounter{enumi}{1}
  \tightlist
  \item
    \textbf{La pièce principale} : Ici, la gaine de mitochondrie a
    disparue ainsi que deux des faisceaux de fibres denses présents dans
    la pièce intermédiaire. On note cependant la présence d'une structure
    supplémentaire, la gaine fibreuse. Cette gaine entoure l'axonème et
    comporte deux épaississements diamétralement opposés, appelés colonnes
    longitudinales sur lesquelles s'insèrent les fibres denses 3 et 8.
    C'est le long de la gaine fibreuse qu'est produit la majorité de
    l'énergie nécessaire au glissement des microtubules
    {[}\protect\hyperlink{ref-Eddy2007}{28}{]}.\\
  \item
    \textbf{La pièce terminale} : Elle est située au niveau de l'extrémité
    distale du flagelle et ne contient que l'axonème
    {[}\protect\hyperlink{ref-Inaba2003}{27}{]}.
  \end{enumerate}
  
  \begin{figure}
  
  {\centering \includegraphics[scale=.55]{figure/sperm2} 
  
  }
  
  \caption[Structure du flagelle d’un spermatozoïde]{Structure du flagelle d’un spermatozoïde d'après Borg et al. (2010) : Coupes transversales en microscopie électronique. Le flagelle se compose de trois parties : la pièce intermédiaire, contenant les mitochondries, la pièce principale et la pièce terminale. L’axonème, en position centrale, parcours tout le flagelle. Des structures périaxonèmales sont observables : les fibres denses dans la pièce intermédiaire et principale, et la gaine fibreuse dans la pièce principale seulement.}\label{fig:flagelle}
  \end{figure}
  
  \newpage
  
  \section{L'infertilité masculine}\label{linfertilite-masculine}
  
  L'organisation mondiale de la santé définie l'infertilité comme étant :
  ``\emph{une pathologie du système reproductif définie par l'échec d'une
  grossesse clinique après 12 mois ou plus de rapports sexuels réguliers
  non protégés}''
  (\href{http://www.who.int/reproductivehealth/topics/infertility/definitions/en/}{\texttt{Who.int.\ 2013-03-19.\ Retrieved\ 2013-06-17}}).
  L'étude de l'infertilité représente un des enjeux scientifique et
  médical majeur de ces dernières années. On estime qu'environ 10 à 15\%
  des couples humains font face à des problèmes d'infertilité soit plus de
  70 millions de personnes dans le monde
  {[}\protect\hyperlink{ref-Boivin2007a}{29}{]}. Dans la moitié des cas,
  la cause sous-jacente serait masculine. On estime que les facteurs
  causaux sous-jacents de l'infertilité masculine peuvent être attribués à
  des toxines environnementales, des troubles systémiques tels que la
  maladie hypothalamo-hypophysaire, les cancers testiculaires et l'aplasie
  des cellules germinales. Les facteurs génétiques, y compris les
  aneuploïdies et les mutations de gènes uniques, contribuent également à
  l'infertilité masculine. Cependant, aucune cause n'est identifiée dans
  près de la moitié des cas. Comme nous avons pu le voir, la
  spermatogénèse est une succession de processus complexes qui s'effectue
  de manière coordonnée, de fait la moindre altération génétique affectant
  une seule de ces étapes est susceptible d'entrainer un phénotype
  d'infertilité {[}\protect\hyperlink{ref-Grudzinskas1995}{30}{]}.
  
  \subsection{Les différents phénotypes d'infertilité
  masculine}\label{les-differents-phenotypes-dinfertilite-masculine}
  
  Chez l'homme, l'infertilité est associée à une altération quantitative
  et / ou qualitative des spermatozoïdes présents dans l'éjaculat.
  L'ensemble de ces altérations peuvent être détectées et quantifiées dans
  des laboratoires spécialisés par réalisation d'un spermogramme. Au cours
  de celui-ci, plusieurs critères tels que le volume de sperme sécrété,
  son pH, la quantité et la vitalité des spermatozoïdes qu'il contient
  seront évalués. La proportion de cellules immatures sera elle aussi
  analysée. Ces cellules rondes, se retrouvent à la fois dans l'éjaculat
  des individus ayant une quantité de spermatozoïdes ``normale''
  {[}\protect\hyperlink{ref-Michael1937}{31}{]}, chez les individus
  présentant une quantité basse de spermatozoïdes
  {[}\protect\hyperlink{ref-MacLeod1970}{33},
  \protect\hyperlink{ref-Tomlinson1993}{34}{]} ou en étant dépourvu
  {[}\protect\hyperlink{ref-Kurilo}{35}{]}. Cependant, leur nombre
  augmente tandis que la quantité de spermatozoïde diminue
  {[}\protect\hyperlink{ref-SPERLING1971}{36}{]}.
  
  \newpage
  
  \subsubsection{Anomalies liées à la quantité
  spermatique}\label{infquant}
  
  Chez l'humain, l'arrêt de la spermatogénèse est défini comme
  l'incapacité des cellules spermatogénétiques à devenir des
  spermatozoïdes matures. Elle peut survenir à n'importe quelle étape de
  la formation des cellules germinales. Les blocages méiotiques, au stade
  de spermatocyte I sont les plus fréquents, suivis par l'arrêt au niveau
  des spermatides et moins fréquemment au niveau des spermatogonies
  {[}\protect\hyperlink{ref-Girgis}{37}{]}.
  
  \begin{enumerate}
  \def\labelenumi{\arabic{enumi}.}
  \tightlist
  \item
    \textbf{L'oligozoospermie} : L'oligozoospermie est définie comme un
    phénotype d'infertilité masculine caractérisé par une production
    inférieure à 15 millions de spermatozoïdes par ml de sperme
    {[}\protect\hyperlink{ref-Cooper2010}{38}{]}. Un arrêt de la
    spermatogénèse a été observé dans 4 à 30\% des biopsies testiculaires
    des hommes présentant une oligospermie sévère
    {[}\protect\hyperlink{ref-Colgan1980}{39}--\protect\hyperlink{ref-WONG1973}{42}{]}.
    Cet arrêt a longtemps été considéré comme sans espoir pour les couples
    désirant concevoir, jusqu'à l'émergence de l'injection mécanique d'un
    spermatozoïde dans l'ovocyte appelé \emph{intracytoplasmic sperm
    injection} (ICSI) {[}\protect\hyperlink{ref-Palermo1992}{43}{]}\\
  \item
    \textbf{L'azoospermie} : Comme l'oligozoospermie, l'azoospermie est un
    phénotype d'infertilité masculine cette fois-ci caractérisé par
    l'absence totale de spermatozoïdes dans l'éjaculat. On distingue des
    causes excrétoires empêchant l'excrétion des spermatozoïdes, on parle
    alors d'azoospermie obstructive et des causes sécrétoires, les plus
    fréquentes, accompagnées d'un défaut de la spermatogenèse, on parle
    alors d'azoospermie non-obstructive.
  \end{enumerate}
  
  \subsubsection{Anomalies liées liée à la
  morphologie}\label{anomalies-liees-liee-a-la-morphologie}
  
  Ces anomalies sont observables en effectuant un spermocytogramme.
  Plusieurs classifications ont été établies, cependant, c'est la
  classification de David modifiée (\textbf{Table :}
  \ref{fig:anomaliemorphosperm}) qui est la plus rependue en France. Pour
  ce faire, on procède généralement à une observation de 100
  spermatozoïdes au cours de laquelle l'ensemble des anomalies observées
  sont relevées et quantifiées permettant ainsi de définir un index
  d'anomalies multiple (nombre total d'anomalies/nombre de spermatozoïdes
  anormaux) révélant le nombre moyen d'anomalies par spermatozoïdes.
  
  \newpage  
  
  \begin{figure}
  
  {\centering \includegraphics[scale=.75]{figure/tab_sperm_defect} 
  
  }
  
  \caption[Classification morphologique de spermatozoïdes humains normaux et anormaux adapté]{Classification morphologique de spermatozoïdes humains normaux et anormaux adapté d'après [@Auger2001]}\label{fig:anomaliemorphosperm}
  \end{figure}
  
  \newpage
  
  \subsubsection{Anomalies liées à la
  mobilité}\label{anomalies-liees-a-la-mobilite}
  
  Le succès du passage du spermatozoïde le long du tractus génitale
  féminin dépend en grande partie de la mobilité et de la vitesse du
  spermatozoïde {[}\protect\hyperlink{ref-Lindholmer1974}{44},
  \protect\hyperlink{ref-Bjorndahl2010}{45}{]}. La vitesse moyenne d'un
  spermatozoïde étant de 25 \(\upmu\)m/s. Une mauvaise mobilité observée
  dans plus de 50\% des spermatozoïdes éjaculés se révèle être un
  prédicteur de l'échec de la fécondation
  {[}\protect\hyperlink{ref-Aitken1985}{46}{]}.
  
  \subsection{La génétique de
  l'infertilité}\label{la-genetique-de-linfertilite}
  
  Comme il a déjà été dit, il est estimé que 10 à 15\% des couples humain
  font face à des problèmes d'infertilité. Par ailleurs, 30\% des
  infertilités restent inexpliquées et près de 40\% ont des causes
  incertaines. Ainsi, l'infertilité masculine d'origine génétique pourrait
  concerner près de 1 homme sur 40
  {[}\protect\hyperlink{ref-Tuttelmann2011}{47}{]}.
  
  \subsubsection{Les causes fréquentes}\label{les-causes-frequentes}
  
  \begin{enumerate}
  \def\labelenumi{\arabic{enumi}.}
  \tightlist
  \item
    \textbf{Les microdélétions du chromosome Y} : Le chromosome Y est un
    petit chromosome atteignant une taille d'environ 53 Mb et est porteur
    de 78 gènes principalement impliqués dans la différentiation sexuelle
    masculine et la spermatogénèse
    {[}\protect\hyperlink{ref-Skaletsky2003}{48}{]}. De fait, le
    chromosome Y représente une région d'intérêt évidente dans l'étude de
    facteur génétique liés à l'infertilité masculine. L'évolution des
    technologies a permis de mettre en évidence des délétions invisibles
    au caryotype dans la région du facteur AZF (\emph{Azoospermia
    Factor}). Cette région peut être subdivisée en trois sous-parties,
    AZFa, AZFb et AZFc (\textbf{Figure :} \ref{fig:chry}). Depuis
    plusieurs années, de nombreuses séries de patients azoospermiques ou
    oligozoospermiques ont été étudiées et publiées et tendent à montrer
    que les microdélétions du chromosome Y seraient responsables de 10\%
    des cas d'azoospermie non-obstructive et chez 5\% des cas
    d'oligozoospermie sévère (\textless{}5 millions de spermatozoïdes/ml)
    {[}\protect\hyperlink{ref-Hotaling2014}{49}{]}.
  \end{enumerate}
  
  \newpage
  
  \begin{figure}
  
  {\centering \includegraphics[scale=.45]{figure/chromozomeY} 
  
  }
  
  \caption[Représentation schématique du chromosome Y adapté]{Représentation schématique du chromosome Y adapté d'après [@OFlynnOBrien2010] : Visualisation la région AZF ainsi que des trois sous-régions AZF a, b, c et des principaux gènes compris dans chacune des sous-régions}\label{fig:chry}
  \end{figure}
  
  \begin{enumerate}
  \def\labelenumi{\arabic{enumi}.}
  \setcounter{enumi}{1}
  \tightlist
  \item
    \textbf{Anomalies chromosomiques} : Des anomalies chromosomiques de
    nombre ou de structure impliquant les autosomes ou, le plus souvent,
    les gonosomes, peuvent être impliquées dans des cas d'infertilité
    masculine. Le pourcentage d'individus concernés varie entre 2 et 8\%
    et peut atteindre 15\% pour les patients azoospermiques soit 10 à 20
    fois la fréquence retrouvée dans la population générale
    {[}\protect\hyperlink{ref-Ravel2006}{50}{]}.
  
    \begin{enumerate}
    \def\labelenumii{\alph{enumii}.}
    \item
      \textbf{Syndrome de Klinefelter} : Le syndrome de Klinefelter (ou
      46, XXY) fut décrit pour la première fois en 1942 par Harry F.
      Klinefelter et décrit une affection due à la présence d'un
      chromosome X supplémentaire suite à une erreur de ségrégation des
      chromosomes au moment de la méiose. Sa prévalence dans la population
      générale est estimée à environ 1 sur 1200 (1 homme sur 600)
      {[}\protect\hyperlink{ref-Bojesen2011}{51}{]} mais elle est environ
      50 fois supérieure chez les patients infertiles azoospermiques
      {[}\protect\hyperlink{ref-Gekas2001}{52}{]}.
    \item
      \textbf{Les anomalies de structure} : Les translocations et les
      inversions sont les anomalies de structures retrouvées le plus
      fréquemment chez les patients infertiles.
  
      \begin{enumerate}
      \def\labelenumiii{\roman{enumiii}.}
      \tightlist
      \item
        La translocation est définie comme l'échange de matériel génétique
        entre deux chromosomes non homologues. On en distingue deux types,
        les translocations réciproques et les translocations
        robertsonniennes. Les premières (\textbf{Figure :}
        \ref{fig:figtranslocation} - \textbf{A}) décrivent un échange
        équilibré entre deux mêmes segments chromosomiques de deux
        chromosomes différents. Elles sont retrouvées 4 à 10 fois plus
        fréquemment chez les patients infertiles que dans la population
        générale {[}\protect\hyperlink{ref-Elliott1997}{53}{]}. Les
        secondes (\textbf{Figure :} \ref{fig:figtranslocation} -
        \textbf{B}) impliquent deux chromosomes acrocentriques et sont
        caractérisées par la fusion entre les brins longs de deux
        chromosomes, les brins courts étant perdus. Elles sont retrouvées
        chez 1.6\% des patients oligozoospermiques et 0.09\% des patients
        azoospermiques {[}\protect\hyperlink{ref-OFlynnOBrien2010}{54}{]}.
      \end{enumerate}
    \end{enumerate}
  \end{enumerate}
  
  \begin{figure}
  
  {\centering \includegraphics[scale=.55]{figure/translocation} 
  
  }
  
  \caption[Les différents types de translocation.]{Les différents types de translocation. d'après [embryology.ch](http://www.embryology.ch/francais/kchromaber/abweichende03.html) :  **A** : La translocation réciproque. **B** : La translocation robertsonniènne}\label{fig:figtranslocation}
  \end{figure}
  
  \begin{enumerate}
  \def\labelenumi{\roman{enumi}.}
  \setcounter{enumi}{1}
  \item
    Les inversions chromosomiques caractérisent le mécanisme de cassure
    d'un fragment de chromosome suivi de son retournement à 180° et sa
    réintégration à la même position. Ces inversions vont gêner
    l'appariement des chromosomes homologues (formation d'une boucle
    d'inversion) pendant la méiose et sont, comme les translocations,
    retrouvées plus fréquemment chez les patients infertiles que dans la
    population générale {[}\protect\hyperlink{ref-Krausz2000}{55}{]}.
  
    \begin{enumerate}
    \def\labelenumii{\alph{enumii}.}
    \setcounter{enumii}{2}
    \tightlist
    \item
      \textbf{Autres anomalies chromosomiques} : Parmi les anomalies
      chromosomiques responsables d'infertilité masculine, on peut par
      exemple citer les hommes de formule 46,XX. Ces patients sont
      généralement totalement infertiles et présentent une azoospermie par
      absence des sous- régions AZF a, b et c
      {[}\protect\hyperlink{ref-Vorona2007}{56}{]} bien qu'ils aient un
      phénotype masculin normal. Ces anomalies sont souvent le fait de la
      translocation du gène SRY sur un des chromosomes X du patient.
    \end{enumerate}
  \end{enumerate}
  
  \begin{enumerate}
  \def\labelenumi{\arabic{enumi}.}
  \setcounter{enumi}{2}
  \tightlist
  \item
    \textbf{Mutations du gène \emph{CFTR} }: L'identification du gène
    \emph{CFTR} (\emph{Cystic Fibrosis Transmembrane conductance
    Regulator}) chez les patients atteints de mucoviscidose et présentant
    une agénésie bilatérale des canaux déférents (ABCD) a permis
    d'associer ce gène au phénotype d'azoospermie obstructive. Cette
    malformation serait responsable de 2\% des cas d'infertilité masculine
    et de 25\% des cas d'azoospermie obstructive
    {[}\protect\hyperlink{ref-Yu2012}{57}{]}.
  \end{enumerate}
  
  Bien que la prévalence de ces anomalies génétiques varie en fonction du
  phénotype concerné, il est estimé que ces défauts soient seulement
  retrouvés chez 5\% des cas d'infertilité masculine tous phénotypes
  confondus. Cette observation suggère fortement l'implication de nombreux
  autres gènes encore inconnus dans les différents phénotypes
  d'infertilité masculine.
  
  \newpage
  
  \subsubsection{Les nouveaux gènes}\label{les-nouveaux-genes}
  
  \begin{enumerate}
  \def\labelenumi{\arabic{enumi}.}
  \item
    \textbf{Les anomalies quantitatives} : une analyse de trois familles
    par séquençage \protect\hyperlink{ngs}{haut-débit} a permit
    d'identifier trois gènes \emph{MEIOB}, \emph{TEX14} et \emph{DNAH6}
    impliqué dans un phénotype d'azoospermie, de même une étude de 2016
    démontre l'association de trois variants dans la séquence codante du
    gène \emph{RAD21L} en se basant sur une étude statistique effectuée
    sur 38 japonais présentant un arrêt de la fertilité et 200 contrôles
    {[}\protect\hyperlink{ref-Minase2017}{58}{]}. De même, plusieurs
    variants dans le gène \emph{TEX111} et \emph{SYC1} on été décrit comme
    entrainant un arret de la méiose
    {[}\protect\hyperlink{ref-Yatsenko2015}{59}--\protect\hyperlink{ref-Maor-Sagie2015}{61}{]}.
  \item
    \textbf{Les anomalies morphologiques liées à la tête du spermatozoïde}
    :
  
    \begin{enumerate}
    \def\labelenumii{\alph{enumii}.}
    \tightlist
    \item
      \textbf{La macrozoospermie} : Ce phénotype d'infertilité masculine
      rare est caractérisé par la présence de 100\% des spermatozoïdes de
      l'éjaculat présentant une tête anormalement grosse ainsi que
      plusieurs flagelles. Il fut observé pour la première fois en 1978
      {[}\protect\hyperlink{ref-Nistal}{62}{]}, mais ce n'est qu'en 2007
      qu'une explication génétique fut enfin trouvée. Une étude portant
      sur 14 patients nord Africains a permis d'identifier la délétion
      c144delC du gène \emph{AURKC} (\emph{Aurora kinase C}) comme
      responsable du phénotype de l'ensemble des individus de l'étude
      {[}\protect\hyperlink{ref-Dieterich2007}{63}{]}. Depuis, d'autres
      études ont permis d'associer d'autre variants sur ce même gène à ce
      phénotype {[}\protect\hyperlink{ref-BenKhelifa2011}{64}{]}. Des
      anomalies du gène \emph{AURKC} seraient ainsi responsables d'environ
      83.7\% des cas macrozoospermie chez des patients non apparentés
      {[}INSERT REF{]}. Le gène AURKC, étant impliqué dans la méiose,
      conduit lorsqu'il est muté à un blocage de la première division
      méiotique entrainant la production de spermatozoïdes tétraploïdes,
      c'est à dire, portant une quantité de matériel génétique quatre fois
      supérieure à la normale
      {[}\protect\hyperlink{ref-Dieterich2009}{65}{]}.\\
    \item
      \textbf{La globozoospermie} : La globozoospermie est aussi un
      phénotype rare d'infertilité dont la prévalence est estimée à de
      0,1\%. Il fut dentifié pour la première fois en 1971 et est
      caractérisé par la présence dans l'éjaculat d'une majorité de
      spermatozoïde dépourvue d'acrosome empêchant ainsi le spermatozoïde
      de franchir la zone pellucide de l'ovocyte et compromettant ainsi la
      fécondation
      {[}\protect\hyperlink{ref-Dam2006}{66}--\protect\hyperlink{ref-Holstein1973}{68}{]}.
      En 2007, une étude familiale a permis de lier ce phénotype à la
      mutation c.848G\textgreater{}A dans le gène \emph{SPATA16}
      (\emph{spermatogenesis-associated protein 16})
      {[}\protect\hyperlink{ref-Dam2007a}{69}{]} dont la protéine va, au
      cours de la spermatogénèse fusionner avec les vésicules
      proacrosomales pour former l'acrosome
      {[}\protect\hyperlink{ref-Dam2007a}{69},
      \protect\hyperlink{ref-Lu2006}{70}{]}. Plus tard, en 2011, une étude
      portant sur 20 patients tunisiens permit d'identifier une délétion
      homozygote de 200 kb emportant la totalité du gène \emph{DPY19L2}
      (\emph{Dpy-19 Like 2}) chez 15 des 20 patients {[}{\textbf{???}}{]}.
      cf \protect\hyperlink{globo}{globo}\\
    \item
      \textbf{Spermatozoïdes acéphaliques} : Ce phénotype rapporté
      plusieurs fois
      {[}\protect\hyperlink{ref-Chemes2010}{71}--\protect\hyperlink{ref-Chemes1987}{73}{]}
      caractérise les patients présentant des spermatozoïdes dépourvus de
      tête dans leur éjaculat. Une étude récente a pu lier ce phénotype à
      une mutation c.824C\textgreater{}T homozygote ainsi qu'à deux
      variants hétérozygotes composites c.1006C\textgreater{}T et
      c.485T\textgreater{}A dans le gène \emph{SUN5}
      {[}\protect\hyperlink{ref-Zhu2016}{74}{]} qui avait précédemment été
      décrit comme localisant à la jonction noyau / flagelle du
      spermatozoïde {[}\protect\hyperlink{ref-Yassine2015}{75}{]}.
    \end{enumerate}
  \item
    \textbf{Le phénotype MMAF} : Le phénotype MMAF (\emph{Multiple
    morphological abnormalities of the sperm flagella}) décrit les
    patients atteints d'asthenozoospermie dont les spermatozoïdes
    présentent de multiples anomalies morphologiques touchant en
    particulier les flagelles. Plus précisément, ce phénotype décrit les
    asthenozoospermie résultant d'une mosaïque d'anomalies morphologiques
    au niveau du flagelle tel que l'absence totale de flagelle, des
    flagelles enroulés, courts, anguleux\ldots{}
    {[}\protect\hyperlink{ref-Coutton2015}{76},
    \protect\hyperlink{ref-BenKhelifa2014}{77}{]}. Récemment, le gène
    \emph{DNAH1} (\emph{Dynein Axonemal Heavy Chain 1}) codant pour une
    dynéine de la chaine lourde de l'axonème a été retrouvé muté chez près
    d'un patient sur trois dans sa cohorte comportant 18 patients
    {[}\protect\hyperlink{ref-BenKhelifa2014}{77}{]}. Deux autres études
    ont retrouvé des mutations dans le gène \emph{DNAH1} chez des patients
    venant de Chine, d'Iran et d'Italie, laissant suggérer que ce gène est
    l'un des acteurs majeurs dans le syndrome MMAF
    {[}\protect\hyperlink{ref-Wang2017}{78},
    \protect\hyperlink{ref-Amiri-Yekta2016}{79}{]}.
  \item
    \textbf{Les échecs de fécondation du spermatozoïde} : Au moment de la
    fécondation, l'activation ovocytaire repose sur le relargage par le
    spermatozoïde de ``facteurs spermatiques'' qui déclenchent un signal
    de calcium, constitué d'oscillations Ca\(^{2+}\). Ce processus est
    médié par une protéine spécifique du spermatozoïde, \emph{la
    phospholipase C Zeta 1} (PLC\(\zeta 1\)) codée par le gène
    \emph{PLCZ1} {[}\protect\hyperlink{ref-Nomikos2013}{80},
    \protect\hyperlink{ref-Amdani2013}{81}{]}. Plusieurs cas d'échec
    d'activation ovocytaire ont été liés à l'absence ou la mauvaise
    localisation de la protéine PLC\(\zeta1\). Malgré cela, aucune preuve
    génétique directe n'avait été reportée jusque récemment où deux
    mutations au sein du gène \emph{PLC}\(\zeta\)\emph{1} furent retrouvés
    chez un patient {[}\protect\hyperlink{ref-Heytens2009}{82}{]} et un
    peu plus tard une mutation homozygote chez deux frères consanguins
    {[}\protect\hyperlink{ref-Escoffier2016}{83}{]}.
  \end{enumerate}
  
  \newpage  
  
  \section{Les techniques d'analyses
  génétiques}\label{les-techniques-danalyses-genetiques}
  
  L'acide désoxyribonucléique (ADN) a été identifié comme étant le porteur
  de l'information génétique par Oswald Theodore Avery en 1944. Sa
  structure en double hélice composée par quatre bases, la thymine (T),
  l'adénine (A), la guanine (G) et la cytosine (C) fut caractérisée en
  1953 par James D. Watson et Francis Crick. Cependant, l'existence
  ``d'entités d'information génétiques discrètes'' que sont les gènes fut
  suggéré dès la deuxième moitié du XIX\(^{ième}\) siècle grâce aux
  travaux de Gregor Mendel portant sur l'hérédité de certains traits chez
  le pois. Depuis, de nombreuses méthodes permettant de lier le phénotype
  d'un individu à son génotype ont vu le jour au gré des améliorations
  technologiques.
  
  \subsection{\texorpdfstring{Approche ``gènes
  candidats''}{Approche gènes candidats}}\label{approche-genes-candidats}
  
  L'approche gène candidat consiste à rechercher des mutations chez un
  patient dans un ou plusieurs gènes cibles. Le choix des gènes cibles se
  fera en fonction de plusieurs critères. Le premier d'entre eux est
  l'étude de gènes reliés à des phénotypes proche du phénotype étudié dans
  différents modèles animaux et notamment murins. Dans ce cas, les
  mutations seront recherchées sur le gène orthologue humain
  {[}\protect\hyperlink{ref-DeBoer2015}{84}{]}. Une autre possibilité
  consiste à rechercher des variants dans des gènes paralogues à un gène
  précédemment identifié avec l'idée sous-jacente que leur structure
  proche implique une fonction similaire. Enfin la dernière méthode
  consiste à étudier des gènes connus comme étant des partenaires de gènes
  déjà identifiés dans cette pathologie en supposant que si un variant
  dans un gène donné entraîne une pathologie, un variant dans un
  partenaire de ce gène pourrait entrainer le même phénotype. Cette
  approche est bien souvent infructueuse dû en grande partie à
  l'hétérogénéité génétique des phénotypes étudiés, au nombre limité de
  patients testés {[}\protect\hyperlink{ref-ElInati2012}{85}{]} et aux
  connaissances souvent incomplètes sur le phénotype. De fait, cette
  approche a quasiment disparu au profit des méthodes à haut débit que
  sont les puces et le séquençage nouvelle génération (NGS), néanmoins,
  cette méthode compte à son actif plusieurs succès retentissants avec
  dans le domaine de l'infertilité masculine, les gènes \emph{SYCP3},
  \emph{SOHLH1} et \emph{NR5A1} entrainant tout trois un phénotype
  d'azoospermie
  {[}\protect\hyperlink{ref-Miyamoto2003}{86}--\protect\hyperlink{ref-Bashamboo2010}{88}{]}.
  
  \newpage
  
  \subsection{Les puces}\label{les-puces}
  
  Les puces à ADN furent initialement conçues dans le but de mesurer le
  niveau de transcription des transcrits provenant de plusieurs milliers
  de gènes lors d'une seule et unique expérience. Cette technologie a
  ainsi permis de des patterns d'expression de gènes à un statu
  physiologique donné. L'analyse des ``signatures'' d'expression a ainsi
  permis de caractériser plusieurs cancer
  {[}\protect\hyperlink{ref-Alon1999}{89}--\protect\hyperlink{ref-VantVeer2002}{92}{]},
  mais aussi la réponse physiologique à plusieurs type de stimuli tel que
  la prise de certains médicaments
  {[}\protect\hyperlink{ref-Brachat2002}{93}{]}.
  
  Suite à cela, l'usage des puces à ADN dans le domaine biomédical s'est
  étendu pour ne plus être limité à la simple quantification de
  l'expression génique. Ainsi, cette technologie a également été utilisé
  afin de détecter des \emph{single nucleotide polymorphisms} (SNPs) au
  sein de notre génome permettant notamment l'émergence du Hap Map project
  qui recense les SNPs de plusieurs milliers d'individus
  {[}\protect\hyperlink{ref-Cutler2001}{94}{]}. De même, l'utilisation des
  puces à ADN a permis la détection de \emph{copy number variation}
  (CNVs).
  
  Pendant plus de 10 ans, la grande qualité des puces, l'existence de
  protocoles d'hybridation standardisés ainsi que des algorithmes
  d'analyses robustes ont fait des puces à ADN l'outil d'analyse génomique
  le plus puissant avant l'arrivée du \protect\hyperlink{ngs}{séquençage
  haut débit}
  
  \newpage
  
  \subsubsection{Les puces à expression}\label{les-puces-a-expression}
  
  L'utilisation principale des puces à ADN a été de mesuré l'expression
  des gènes dans un tissus donné. Dans cette application, l'ARN est
  extrait des cellules d'intérêt puis est généralement convertit en ADNc.
  L'ADNc est ensuite hybridé à la puce qui subira ensuite une étape de
  lavage. L'intensité de fluorescence est ensuite mesurée à chaque spot de
  la puce. l'intensité du signal sera ensuite le reflet du niveau
  d'expression d'un gène.
  
  \begin{figure}
  
  {\centering \includegraphics[scale=.9]{figure/expression_array} 
  
  }
  
  \caption[Représentation schématique des méthodes d'analyse d'expression génique par puce à ADN]{Représentation schématique des méthodes d'analyse d'expression génique par puce à ADN d'après [@Trevino2007] : Présentation des méthodes à double et à simple colorant, respectivement à gauche et à droite. Pour les analyses à double colorant, une seule puce et nécessaire, les échantillons de la référence et du test sont mis en compétition sur la même puce, un signal de sortie vert indiquera une surexpression chez le test tandis qu'un signal rouge indiquera une sous-expression. Pour celles à simple colorant, deux puces sont nécessaires, une première pour la référence et une seconde pour le test. Les données des deux puces sont ensuite comparées pour déterminé quels sont les gènes différentiellement exprimés. Dans le cas de la CGH array, le principe est similaire, en remplaçant simplement l'ARNm par de l'ADNg}\label{fig:figexparray}
  \end{figure}
  
  \newpage
  
  \subsubsection{Les puces à SNP, plateforme
  génotypage}\label{les-puces-a-snp-plateforme-genotypage}
  
  Bien que leur utilité principale ait été d'analyser l'expression des
  gènes, les puces à ADN ont également été extrêmement utilisées comme
  moyen de génotyper les SNP (\emph{single-nucleotide-polymorphism}). De
  nombreuses méthodes ont été mises en place pour cela, cependant la plus
  employée est la méthode de discrimination allélique par hybridation
  telle qu'elle est utilisée par Affymetrix
  {[}\protect\hyperlink{ref-Wang1998}{95}{]} malgré le ``bruit de fond''
  causé par l'hybridation non spécifique dont elle souffre (\textbf{Figure
  :} \ref{fig:figallelicdisc}).
  
  \begin{figure}
  
  {\centering \includegraphics[scale=.7]{figure/allelic_discrimination} 
  
  }
  
  \caption[Méthode de génotypage par discrimination allélique par hybridation]{Méthode de génotypage par discrimination allélique par hybridation d'après [@Bumgarner2013] : Des sondes complémentaires à chacun des allèles sont positionnées sur la puce. L'ADN génomique fragmenté et labélisé est mis en contact de la puce. Après nettoyage de la puce, l'analyse du signal émis par l'ADN génomique permettra de déterminé si l'individu est homozygote pour cette allèle (hybridation à une seule des deux sondes) ou bien hétérozygote pour cette allèle (hybridation aux deux sondes) .}\label{fig:figallelicdisc}
  \end{figure}
  
  \newpage
  
  \subsubsection{Les puces à indels}\label{les-puces-a-indels}
  
  L'implication de réarrangements génomiques tel que des duplications,
  translocations ou délétions dans divers pathologies est bien connu.
  C'est afin de détecter ces réarrangements que la \emph{Comparative
  Genomic Hybridization array} (CGH array) a été développée dès 1999
  {[}\protect\hyperlink{ref-Brown1999}{96}{]}. Son principe est très
  similaire à celui utilisé dans les puces à expression (\textbf{Figure :}
  \ref{fig:figexparray}) en remplaçant simplement l'ARN messager (ARNm)
  par de l'ADN génomique (ADNg). Ainsi, la présence d'un CNV sera
  facilement détectée en comparant le signal émis par un individu test
  avec celui émis par un contrôle.
  
  \subsubsection{Limitation}\label{limitation}
  
  Bien que cette technologie ait été largement utilisée dans divers champs
  d'applications, elle présente deux limitations principales. La première
  est que pour les génome complexes (tel que les mammifères), il est
  difficile, si ce n'est impossible de \emph{designer} un puce ne
  permettant pas de l'hybridation non spécifique. En effet, la séquence
  d'une puce prévue pour détecter le gène ``A'' pourra également détecter
  les gènes ``B'', ``C'' et ``D'' si ceux-ci présentent une forte
  homologie avec ``A'' Ce qui est particulièrement problématique dans le
  cas d'analyse de gènes d'une même famille.
  
  Pour finir, la plus grande de ses limitations est que les puces
  détectent uniquement ce pour quoi elles ont été \emph{designer}. Ainsi,
  si la solution que l'on hybride sur la puce contient des séquence d'ADN
  ou d'ARN pour lesquelles il n'y a aucune sonde complémentaire sur la
  puce, celles-ci ne seront pas détectées. Cela peut avoir de grandes
  répercutions puisque par exemple dans le cas des à expression, les gènes
  qui n'ont pas encore été annotés risques de ne pas être représenté sur
  la puce.
  
  \newpage
  
  \hypertarget{ngs}{\subsection{Le séquençage NGS}\label{ngs}}
  
  Le terme séquençage de l'ADN fait référence à l'ensemble des techniques
  permettant de déterminer l'ordre des nucléotides A, T, C et G de
  l'intégralité ou d'une partie d'une molécule d'ADN. Avant de parler des
  nouvelles technologies de séquençage (NGS) faisons un bref historique du
  séquençage de l'ADN. En 1977 Frederick Sanger développe une technologie
  de séquençage d'ADN basée sur la méthode \emph{chain-termination}. Ce
  procédé est désormais connu sous le nom de séquençage Sanger. D'autres
  méthodes furent développées à la même période, notamment celle de Walter
  Gilbert basée sur la modification chimique de l'ADN, cependant sa grande
  efficience et sa faible utilisation de la radioactivité permirent au
  séquençage Sanger de s'imposer comme référence dans la ``première
  génération'' de séquenceur à application commerciale et de recherche.
  Apparu en 1998, les instruments de séquençage automatique ainsi que les
  logiciels associés utilisant le séquençage par capillarité et la
  technologie Sanger furent les outils principaux qui permirent la
  complétion du \emph{human genome project} en 2001
  {[}\protect\hyperlink{ref-Collins2003}{97}{]}.
  
  Contrairement à la méthode Sanger, le NGS ``\emph{lit}'' des fragments
  d'ADN, provenant d'un génome \textbf{entier}. On parle alors de
  séquençage de génomes entiers ou \emph{whole genome sequencing} (WGS).
  Pour cela, la molécule d'ADN est ``coupée'' en plusieurs fragments d'une
  taille donnée. Ce sont ensuite ces fragments qui seront, après une étape
  d'amplification spécifique aux différentes plateformes, séquencés
  simultanément. C'est pourquoi on parle souvent de séquençage parallèle
  massif pour décrire le NGS. Le produit de ce séquençage est appelé
  \emph{read}. Cette technologie est avantageuse de par la masse de
  \emph{reads} qu'elle produit et par son faible coût par bases séquencées
  {[}\protect\hyperlink{ref-Metzker2010}{98}{]}. Ces caractéristiques ont
  permis au séquençage Haut-débit d'être couramment utilisé dans le
  domaine de la recherche clinique.
  
  La taille des \emph{reads} obtenus par séquençage NGS est nettement
  inférieure à celle atteinte par le séquençage Sanger. À l'heure
  actuelle, les \emph{reads} obtenus par séquençage NGS ont une taille
  comprise entre 50 et 500 pb pour la plupart des plateforme contre une
  taille d'environ 800 nucléotides obtenus par Sanger (\textbf{Figure :}
  \ref{fig:readPerRun}), c'est pour cela que les résultats du séquençage
  NGS sont appelés des \emph{reads} courts ou \emph{short reads}.\\
  Étant donné que le NGS produit à l'heure actuelle des \emph{reads}
  courts la notion de couverture est importante et représente l'un des
  critères majeur à considérer dans l'analyse des données
  {[}\protect\hyperlink{ref-Sims2014}{99}{]}. La couverture est définie
  comme le nombre de \emph{reads} qui, après l'étape
  \protect\hyperlink{lalignement}{d'alignement}, se chevauchent les uns
  les autres au sein d'une région génomique spécifique. Par exemple, une
  couverture de 30x pour le gène XXXX signifie que chaque nucléotide de ce
  gène est chevauché par au moins 30 \emph{reads} distincts.
  
  \newpage
  
  \begin{figure}
  
  {\centering \includegraphics[scale=.55]{figure/read_per_run} 
  
  }
  
  \caption[Présentation de la taille des reads et du nombre de reads par run en fonction de la technologie de séquençage utilisée]{Présentation de la taille des reads et du nombre de reads par run en fonction de la technologie de séquençage utilisée d'après [@Hodkinson2015] : Chaque point représente une plateforme de séquençage, la couleur détermine la marque du séquenceur}\label{fig:readPerRun}
  \end{figure}
  
  \subsubsection{La capture des parties à séquencer, avantages et
  inconvénients}\label{la-capture-des-parties-a-sequencer-avantages-et-inconvenients}
  
  Pour de nombreuses applications, il peut être intéressant de ne
  séquencer qu'une partie du génome et non pas son intégralité. Dans cette
  sous partie de génome ciblé on peut trouver par exemple : une région
  génomique spécifique à laquelle une pathologie a déjà été associée,
  l'ensemble des exons de certains gènes candidats, ou encore
  l'intégralité des exons de l'ensemble des gènes codant pour une
  protéine. Dans ce dernier cas on parle alors de séquençage exomique ou
  \emph{whole exome sequencing} (WES). Les principaux avantages du WES par
  rapport au WGS sont son coût réduit ainsi qu'une masse de données moins
  importantes à stocker et à analyser. En effet, l'ensemble de l'exome ne
  représente qu'environ 1\% du génome entier. On considère cependant que
  ces parties codantes contiennent plus de 90\% des anomalies responsables
  de pathologies génétiques chez l'homme. Pour ces raisons, le WES est
  considéré comme le standard dans le cadre de recherche sur des
  pathologies génétiques et révèle être un outil puissant pour
  l'identification de variants associés à des pathologies
  {[}\protect\hyperlink{ref-Ng2010}{100}{]}. Le procédé de séquençage est
  identique au WGS, il est simplement précédé d'une étape d'enrichissement
  au cours de laquelle les exons sont capturés par hybridation à des
  sondes. De fait les exons capturés sont donc dépendants du kit de
  capture utilisé, cette technique permet donc de séquencer uniquement les
  exons connus et ciblés par les sondes. Il faut également noter que
  depuis quelques années, plusieurs études ont remis en cause l'intérêt du
  WES au profit du WGS, notamment car le WGS fournit une meilleure
  couverture sur l'exome que le WES
  {[}\protect\hyperlink{ref-Lelieveld2015}{101},
  \protect\hyperlink{ref-Meienberg2016}{102}{]}. De plus le WES montre une
  plus grande sensibilité au pourcentage de GC contenu dans la région à
  séquencer et à la sélection des kits de capture utilisés
  {[}\protect\hyperlink{ref-Meienberg2016}{102}{]}. Ainsi, bien que le WES
  soit encore à l'heure actuelle le choix privilégié dans la majorité des
  études, la réduction des coûts de séquençage et du stockage des données,
  pourraient permettre prochainement au WGS de remplacer totalement le WES
  ainsi que l'ensemble des techniques impliquant la capture de séquences
  ciblées {[}\protect\hyperlink{ref-Meienberg2016}{102}{]}.
  
  \subsubsection{L'amplification}\label{lamplification}
  
  Dans la plupart des technologies, la phase de séquençage est précédée
  par une étape d'amplification de l'ADN. Cette amplification se fait dans
  la grande majorité des cas sur une surface solide exceptée pour la PCR
  en émulsion qui s'effectue en phase aqueuse. Elle permet d'obtenir dans
  une région définie plusieurs milliers de copies du même fragment d'ADN,
  appelés des clones. Cette étape assure que le signal émis lors du
  séquençage pourra être distingué du bruit. Chacun de ces \emph{spots}
  d'amplification appelés aussi centre de réaction, se retrouve donc être
  le représentant d'un unique fragment d'ADN. Ceux-ci seront ensuite
  séquencés parallèlement aux autres \emph{spots}. Une plateforme de
  séquençage peut gérer plusieurs millions de ces centres de réactions
  simultanément, séquençant ainsi plusieurs millions de molécules d'ADN en
  parallèle, donnant ainsi le nom de séquençage massif en parallèle à ces
  techniques. Cette étape d'amplification est généralement précédée d'une
  phase de fragmentation de l'ADN. Cette fragmentation peut être physique,
  enzymatique ou bien chimique. Ce sont les résidus d'ADN résultant de
  cette fragmentation qui seront ensuite amplifiés. Il existe quatre
  stratégies utilisées pour le clonage de l'ADN dans le cadre du NGS :
  
  \begin{enumerate}
  \def\labelenumi{\arabic{enumi}.}
  \tightlist
  \item
    \textbf{La PCR en émulsion ou emPCR} (\textbf{Figure :
    }\ref{fig:ngsampli} - \textbf{a}) : Le patron d'ADN fragmenté simple
    brin est lié à une séquence adaptatrice complémentaire et est capturé
    par une gouttelette aqueuse appelée micelle contenant une bille
    recouverte d'adaptateur complémentaire à celui fixé sur le fragment
    d'ADN ainsi que tous les composants nécessaires à la réaction de PCR.
    En respectant un ratio nombre de molécules d'ADN / nombre de billes,
    on va fixer un seul fragment d'ADN sur chaque bille. Chacune de ces
    billes seront donc, en fin de réaction, recouverte par plusieurs
    milliers de copies de la même séquence d'ADN.\\
  \item
    \textbf{L'amplification par pont sur face solide} (\textbf{Figure :
    }\ref{fig:ngsampli} - \textbf{b}) : Les fragments d'ADN sont liés à
    des séquences adaptatrices et lié par une de leurs extrémités à une
    amorce fixée sur un support solide. Du fait de la dilution, les
    molécules d'ADN se trouvent éloignées les unes des autres. L'extrémité
    libre du fragment interagit avec les amorces situées à proximité
    formant une structure en pont, d'où le nom de PCR en pont ou
    \emph{bridge-PCR}. La PCR va alors synthétiser un deuxième brin
    complémentaire aux fragments immobilisés sur le support. En procédant
    à des cycles de température comme pour une réaction PCR classique, on
    obtient à l'emplacement de chaque molécule d'ADN un massif de
    molécules fixé sur la plaque, toutes identiques à la molécule
    initiale.\\
  \item
    \textbf{Amplification par modèle mobile ou \emph{walking-template}
    }(\textbf{Figure : }\ref{fig:ngsampli} - \textbf{c}) : L'ADN fragmenté
    est lié à un adaptateur et lié à une amorce complémentaire fixée sur
    un support solide. Le brin complémentaire du fragment sera synthétisé
    par PCR à partir de l'amorce fixée. La molécule double brin
    nouvellement formée sera ensuite partiellement dénaturée permettant à
    l'extrémité libre de se fixer à une séquence amorce voisine. Des
    amorces \emph{reverse} sont ensuite utilisées pour resynthétiser un
    fragment d'ADN libre à partir des fragments fixés sur le support.\\
  \item
    (\textbf{Figure : }\ref{fig:ngsampli} - \textbf{d}) : \textbf{PAS DU
    TOUT COMPRIS LE MECHANISME !!! }
  \end{enumerate}
  
  \begin{figure}
  
  {\centering \includegraphics[scale=.455]{figure/ngs_amplification} 
  
  }
  
  \caption[Présentation des différentes stratégies d'amplification de l'ADN dans le cadre du NGS]{Présentation des différentes stratégies d'amplification de l'ADN dans le cadre du NGS d'après [@Goodwin2016] : **a** : PCR en émulsion. **b** : amplification par pont. **c** : Amplification par modèle mobile. **d** : }\label{fig:ngsampli}
  \end{figure}
  
  \newpage
  
  \subsubsection{La réaction de séquence}\label{la-reaction-de-sequence}
  
  La réaction de séquence est l'étape suivant l'amplification et consiste
  à déterminer l'ordre dans lequel se succèdent les nucléotides de
  l'ensemble des clones générés dans la phase d'amplification. Il existe
  deux technologies principales permettant le séquençage de \emph{reads}
  courts :\\
  
  \begin{enumerate}
  \def\labelenumi{\arabic{enumi}.}
  \tightlist
  \item
    \textbf{Séquençage par synthèse} (SBS) : Ce type de séquençage
    regroupe l'ensemble des méthodes utilisant l'ADN polymérase pour
    synthétiser de l'ADN. En 2016, Sahra Goodwin et ses collègues ont
    différentiés deux catégories de séquençage par synthèse
    {[}\protect\hyperlink{ref-Goodwin2016}{103}{]} :
  
    \begin{enumerate}
    \def\labelenumii{\alph{enumii}.}
    \tightlist
    \item
      \textbf{Terminaison par cycle réversible}, \emph{cyclic reversible
      termination} (CRT) (\textbf{Figure : }\ref{fig:crtSeq}) : Cette
      méthode est caractérisée par l'utilisation de molécule dîtes
      terminatrices auxquelles le groupement \(\mathrm{3'-OH}\) est
      modifié de sorte à éviter l'élongation
      {[}\protect\hyperlink{ref-Guo2008}{104}{]}, on parlera de groupement
      \(\mathrm{3'-bloqué}\). Une amorce liée au fragment d'ADN permettra
      l'initialisation du processus de polymérisation. À chaque cycle, un
      mix comprenant l'ensemble des quatre désoxynucléotides (dNTPs),
      préalablement labélisés par un fluorophore \(\mathrm{3'-bloqué}\),
      est mis en contact du fragment. Après l'incorporation d'un unique
      dNTP au fragment, les dNTPs non liés sont éliminés et la nature du
      dNTP ajouté est identifiée grâce à son fluorophore. Le fluorophore
      et le groupement \(\mathrm{3'-bloqué}\) sont retirés permettant
      ainsi à un nouveau cycle de commencer.
    \end{enumerate}
  \end{enumerate}
  
  \begin{figure}
  
  {\centering \includegraphics[scale=.24]{figure/CRT_seq_illumina} 
  
  }
  
  \caption[Séquençage CRT tel qu'il est effectué par Illumina]{Séquençage CRT tel qu'il est effectué par Illumina d'après [@Goodwin2016] : **a** : Ajout d'un dNTP labellisé par un fluorophore 3'-bloqué. **b** : Identification du dNTP ajouté grâce au fluorophore. **c** : Le fluorophore est clivé du dNTP et le groupement 3'-OH est reformé à partir du groupement 3'-bloqué permettant ainsi l'élongation}\label{fig:crtSeq}
  \end{figure}
  
  \newpage
  
  \begin{enumerate}
  \def\labelenumi{\alph{enumi}.}
  \setcounter{enumi}{1}
  \tightlist
  \item
    \textbf{Addition de nucléotide unique}, \emph{single nucleotide
    addition} (SNA) (\textbf{Figure : }\ref{fig:snaSeq}) :
    L'initialisation de la méthode SNA est identique à celle de la méthode
    CRT. La différence se fait donc au moment de la phase d'élongation.
    Contrairement à la méthode CRT, le mix contenant les dNTPs ne contient
    qu'un seul type de dNTP. Quatre mixs différents sont donc présentés
    successivement au fragment d'ADN à séquencer, ceux-ci se fixeront
    uniquement s'ils sont complémentaires à la séquence. Ces dNTPs n'ont
    donc pas besoin d'être \(\mathrm{3'-bloqué}\) puisqu'un seul dNTP est
    ajouté à chaque itération. Après avoir présenté un mix, on vérifie si
    un dNTP s'est lié au fragment. Lors des séquences homopolymériques
    (plusieurs nucléotides identiques successifs dans la séquence),
    plusieurs dNTPs sont donc liés simultanément, cela sera détecté car le
    signal émis est proportionnel au nombre de nucléotides ajoutés.
  \end{enumerate}
  
  \begin{figure}
  
  {\centering \includegraphics[scale=.26]{figure/SNA_seq_ionTorrent} 
  
  }
  
  \caption[Séquençage SNA tel qu'il est effectué par Ion Torrent]{Séquençage SNA tel qu'il est effectué par Ion Torrent d'après [@Goodwin2016] : **a** : Mise en présence du patron d'ADN à séquencer avec un mix contenant un seul type de dNTP, si le dNTP est complémentaire au patron, il se fixe et libère un proton permettant d'identifier la liaison. **b** : Dans le cas d'homopolymère, autant de proton sont relâchés de bases constituant l'homopolymère, le signal émit est donc plus fort permettant d'identifier le nombre des dNTPs liés}\label{fig:snaSeq}
  \end{figure}
  
  \newpage
  
  \begin{enumerate}
  \def\labelenumi{\arabic{enumi}.}
  \setcounter{enumi}{1}
  \tightlist
  \item
    \textbf{Séquençage par ligation} (SBL) : Par définition, cette méthode
    est basée sur l'hybridation et la ligation de l'ADN à une sonde liée à
    un fluorophore {[}\protect\hyperlink{ref-Tomkinson2006}{105}{]}. Ce
    processus utilise les caractéristiques de la ligase, une enzyme qui a
    pour fonction de catalyser la liaison de deux brins d'ADN par des
    liaisons phosphodiester. La sonde est constituée d'une ou deux bases
    connues, on parle alors de \emph{one-base-encoded probes} ou de
    \emph{two-bases-encoded probes} suivis d'une succession de bases
    ``dégénérées'' ou universelle, c'est à dire, des bases capables de
    s'apparier avec n'importe laquelle des quatre bases de l'ADN.
  \end{enumerate}
  
  \begin{figure}
  
  {\centering \includegraphics[scale=.26]{figure/SBL_seq_solid} 
  
  }
  
  \caption[Séquençage SBL tel qu'il est effectué par SOLiD]{Séquençage SBL tel qu'il est effectué par SOLiD d'après [@Goodwin2016] : }\label{fig:sblSeq}
  \end{figure}
  
  \newpage  
  
  \section{L'analyse bioinformatique des données de
  NGS}\label{lanalyse-bioinformatique-des-donnees-de-ngs}
  
  La stratégie consistant à séquencer en parallèle plusieurs milliers de
  \emph{reads} courts a engendré plusieurs nouveaux défis bioinformatique
  dans l'analyse et l'interprétation des données de séquençage et la
  recherche de variants dans le génome humain
  {[}\protect\hyperlink{ref-Wold2007}{106},
  \protect\hyperlink{ref-Yang2009}{107}{]}. Ces techniques ont été
  appliquées dans différents contextes, notamment la métagénomique
  {[}\protect\hyperlink{ref-Qin2010}{108}{]}, la détection de SNPs
  {[}\protect\hyperlink{ref-VanTassell2008}{109}{]} et de variants
  structuraux {[}\protect\hyperlink{ref-Alkan2010}{110},
  \protect\hyperlink{ref-Medvedev2009}{111}{]} mais également dans des
  études portant sur la méthylation de l'ADN
  {[}\protect\hyperlink{ref-Taylor2007}{112}{]}, l'analyse de l'expression
  des ARNs messagers {[}\protect\hyperlink{ref-Sultan2008}{113}{]}, dans
  la génétique du cancer {[}\protect\hyperlink{ref-Guffanti2009}{114}{]}
  et la médecine personnalisée
  {[}\protect\hyperlink{ref-Auffray2009}{115}{]}. Cependant, pour
  l'ensemble de ces applications, la grande quantité de données générées
  par chaque analyse pose plusieurs défis informatiques
  {[}\protect\hyperlink{ref-Horner2009}{116}{]}. En effet, les progrès
  techniques des dernières décennies ont rendu possible le séquençage de
  plusieurs millions des \emph{reads} d'ADN en un temps relativement court
  et à coûts raisonnable. Ainsi, l'émergence du séquençage haut débit et
  notamment du WGS et du WES a permis de réunir une quantité jusqu'à
  présent inégalée d'informations sur les variations génétiques, et sur
  les gènes et leurs fonctions {[}\protect\hyperlink{ref-Mardis2008}{117},
  \protect\hyperlink{ref-Bentley2006}{118}{]}. Cependant, de par leur
  nature et leur quantité, l'acquisition de ces nouvelles données a
  engendrée de nouvelles problématiques qui freinent les biologistes dans
  leurs recherches.
  
  \subsection{Les données fournies par le
  NGS}\label{les-donnees-fournies-par-le-ngs}
  
  \subsubsection{\texorpdfstring{Un \emph{read} c'est quoi
  ?}{Un read c'est quoi ?}}\label{un-read-cest-quoi}
  
  Après la phase d'amplification, chaque clone est analysé puis, la
  séquence composant chacun de ce clone est déterminée. La taille de cette
  séquence varie en fonction des plateformes de séquençage mais est
  généralement comprise entre 40 et 150 pb pour le NGS (\textbf{Figure :
  }\ref{fig:readPerRun}). Depuis quelques années, un nouveau type de
  \emph{read} est apparu, le \emph{read} \emph{paired-end}. Contrairement
  au \emph{reads} classiques (single-end), les deux extrémités (les
  \emph{ends}) du fragment d'ADN sont désormais séquencées. La distance
  séparant les deux extrémités du \emph{read} étant connue, cela permet
  aux aligneurs d'utiliser cette information afin d'améliorer leur
  précision, notamment dans les zones répétées
  {[}\protect\hyperlink{ref-Li2008}{119}{]}. En plus de SNP, ce format
  permet de mettre en évidence des variants structuraux
  {[}\protect\hyperlink{ref-Korbel2009}{120}{]}.
  
  \newpage
  
  \subsubsection{Le format FASTQ}\label{fastq}
  
  Le format FASTQ (\textbf{Figure : }\ref{fig:fastqformat}) est
  actuellement le format de donnée le plus couramment utilisé dans le
  cadre du séquençage haut-débit. Sa création est cependant antérieure à
  l'émergence du NGS puisqu'il fut inventé à la fin du XX\(^{ième}\) par
  Jim Mullikin au
  \href{https://fr.wikipedia.org/wiki/Wellcome_Trust_Sanger_Institute}{Wellcome
  Trust Sanger Institute} alors que le séquençage commençait à prendre de
  l'ampleur grâce à des projets tels que le
  \href{https://fr.wikipedia.org/wiki/Projet_G\%C3\%A9nome_Humain}{Projet
  Génome Humain}. La quantité de données générées par ces programmes a
  nécessité une analyse automatisée, c'est ainsi que chaque base séquencée
  s'est vue associer un score de qualité appelé \emph{Phred-score}. Chaque
  séquence générait ainsi deux fichiers, un fichier FASTA contenant les
  séquences et un fichier QUAL contenant les scores \emph{Phred} associés
  à chaque base du fichier FASTA Cock2009. Plus tard, afin de n'avoir à
  manipuler qu'un seul fichier, les fichiers FASTA et QUAL furent
  fusionnés en ce que l'on appelle désormais le fichier FASTQ. Ce format
  est aujourd'hui le plus utilisé par les différents séquenceurs on peut
  cependant noter certaines différences dans les formats FASTQ provenant
  des différentes plateformes puisqu'à l'époque, aucune spécification
  officielle n'avait été donnée
  {[}\protect\hyperlink{ref-Cock2009}{121}{]}.
  
  \begin{figure}
  
  {\centering \includegraphics[scale=.55]{figure/fastq} 
  
  }
  
  \caption[présentation d'un fichier FASTQ (FIGURE A CHANGER)]{présentation d'un fichier FASTQ (FIGURE A CHANGER) : **a** : identifiant du *read*. **b** : séquence du *read*. **c** : score de qualité associé}\label{fig:fastqformat}
  \end{figure}
  
  \newpage
  
  \hypertarget{lalignement}{\subsection{L'alignement}\label{lalignement}}
  
  L'alignement constitue la première étape de l'analyse des données de NGS
  lorsqu'un génome de référence est disponible. L'objectif de l'alignement
  est de déterminer la position correcte de chacun des \emph{reads}
  séquencés le long du génome de référence. Cette référence est souvent
  construite à partir des données de séquençage de plusieurs donneurs et
  ne représente donc pas la séquence d'un individu en particulier mais est
  censée représenter la séquence consensus d'une espèce donnée. Par
  exemple, la séquence de référence humaine GRCh37 (\emph{Genome Reference
  Consortium human build 37}) a été créé à partir de 13 volontaires
  anonymes New-Yorkais. Dès lors, cette référence servira de patron aux
  aligneurs afin qu'ils replacent correctement les différents \emph{reads}
  des individus séquencés. Cette étape peut être comparée à la
  reconstruction d'un puzzle dans laquelle les \emph{reads} seraient les
  pièces et le génome de référence le modèle (\textbf{Figure :
  }\ref{fig:picdnamapping}). Elle constitue probablement l'étape la plus
  importante de l'analyse des données issues du séquençage haut débit
  {[}\protect\hyperlink{ref-Flicek2009}{122}{]} car elle est la base sur
  laquelle repose l'ensemble des étapes effectuées en aval, notamment
  l'appel des variants {[}\protect\hyperlink{ref-Nielsen2011}{123}{]}.
  Cependant, l'étape d'alignement est sujette à de nombreuses erreurs dont
  certaines proviennent directement des erreurs survenues lors de l'étape
  de séquençage, d'autres, sont dues aux caractéristiques des régions
  séquencées comme par exemple les séquences répétées
  {[}\protect\hyperlink{ref-Langmead2012}{124}{]} qui pourront entraîner
  l'alignement d'un même \emph{read} à plusieurs régions du génome
  {[}\protect\hyperlink{ref-Treangen2013}{125}{]}. De nombreux aligneurs
  ont émergés afin de répondre au mieux à cette problématique tel que
  Bowtie {[}\protect\hyperlink{ref-Langmead2009}{126}{]}, Bowtie2
  {[}\protect\hyperlink{ref-Langmead2012}{124}{]}, BWA, NovoAlign, MAGIC
  {[}\protect\hyperlink{ref-Su2014}{127}{]}. De nombreuses études ont
  cependant montrées de grandes différences entre ces aligneurs, au niveau
  du temps de calcul, de leur coût en mémoire et de leur taux d'erreur
  {[}\protect\hyperlink{ref-Ruffalo2011}{128}--\protect\hyperlink{ref-Bao2011}{130}{]}.
  
  \newpage
  
  \begin{figure}
  
  {\centering \includegraphics[scale=0.35]{figure/dna_mapping} 
  
  }
  
  \caption[Représentation schématique de l'alignement de *reads paired-end*]{Représentation schématique de l'alignement de *reads paired-end* : **A** : Représentation du génome de référence ainsi que de *reads paired end* avant l'étape d'alignement. Les *reads paired-end* sont composés d'une extrémité *forward* (en vert) complémentaire du brin sens du génome de référence et d'une extrémité réverse (en jaune), complémentaire du brin anti-sens du génome de référence. Chacune de ces extrémités est séparées par un insert de taille connue mais de séquence inconnu. **B** : Après l'étape d'alignement, chaque *read* est repositioné sur la région du génome avec laquelle il présente la plus grande homologie de séquence. Le nombre de *reads* différents recouvrant une même position du génome de référence est appellée couverture}\label{fig:picdnamapping}
  \end{figure}
  
  \newpage
  
  \hypertarget{varcall}{\subsection{L'appel des variants}\label{varcall}}
  
  L'appel des variants, ou \emph{variant calling}, fait référence à
  l'ensemble des méthodes permettant d'identifier des SNVs ou des indels à
  partir des résultats de l'alignement. Cette étape est souvent
  différenciée de l'alignement, cependant, les résultats de l'appel étant
  extrêmement dépendants de l'alignement, il est conseillé d'effectuer son
  appel en tenant compte de l'aligneur choisi
  {[}\protect\hyperlink{ref-Nielsen2011}{123},
  \protect\hyperlink{ref-DePristo2011}{131},
  \protect\hyperlink{ref-Lunter2011}{132}{]}. On appellera variant toute
  différence de séquence observée entre un individu et la séquence de
  référence utilisée. Pour reprendre la comparaison avec la construction
  d'un puzzle, cette étape consiste à détecter quelles sont les pièces qui
  présentent des différences avec le modèle. De nombreux logiciels
  d'appels des variants, ou \emph{caller}, basés sur des algorithmes
  différents ont émergés ces dernières années pour répondre à cette
  problématique. Parmi les plus connus on note SAMtools
  {[}\protect\hyperlink{ref-Li2009}{133}{]}, Genome Analysis Tool Kit -
  HaplotypeCaller (GATK-HC)
  {[}\protect\hyperlink{ref-McKenna2010}{134}{]}, Freebayes, SOAPindel et
  TVC. Les quatre premiers cités, peuvent être utilisés pour analyser des
  données provenant de tout type de plateforme de séquençage contrairement
  à TVC qui a été développé spécifiquement pour les données provenant de
  Ion Proton. Les données issues de NGS peuvent présenter un taux d'erreur
  important. Ce taux d'erreur est multifactoriel et inclus notamment les
  erreurs de l'alignement. L'un des éléments clef à prendre en compte pour
  pouvoir effectuer un appel de qualité est la couverture de la position
  appelée {[}\protect\hyperlink{ref-Sims2014}{99}{]}. Cependant, malgré la
  prise en compte de cet élément, l'appel de variants reste un processus
  difficile souvent lié à plusieurs erreurs. Plusieurs de ces erreurs sont
  même directement liées à la plateforme de séquençage utilisée en amont,
  et les différents logiciels ne présentent pas les mêmes performances en
  fonction de ces différentes plateforme
  {[}\protect\hyperlink{ref-Hwang2015}{135}{]}, c'est pourquoi il convient
  d'adapter le logiciel d'appel en fonction de la plateforme de séquençage
  utilisée préalablement. Les erreurs d'appel sont généralement classées
  en trois catégories et certains aligneurs auront tendance à être plus
  sujets à l'un de ces types d'erreur qu'à l'autre (\textbf{Figure :
  }\ref{fig:snperror}) :
  
  \begin{enumerate}
  \def\labelenumi{\arabic{enumi}.}
  \tightlist
  \item
    Oubli de l'allèle de référence (\textbf{IR}, \emph{ignore the
    reference allele}) : représente un variant appelé homozygote
    correspondant en réalité à un variant hétérozygote composé de l'allèle
    de référence et d'un allèle variant.\\
  \item
    Ajout de l'allèle de référence (\textbf{AR}, \emph{adding the
    reference allele}) : représente un variant appelé hétérozygote composé
    de l'allèle de référence et d'un allèle variant correspondant en
    réalité à un variant homozygote composé de deux allèles variants.\\
  \item
    Autres : incluent l'ensemble des autres types d'appel erronés
    indépendamment de l'allèle de référence
  \end{enumerate}
  
  \newpage
  
  \begin{figure}
  
  {\centering \includegraphics[scale=.50]{figure/snp_error_type} 
  
  }
  
  \caption[Représentation des erreurs d'appel de type IR et AR en fonction de la plateforme de séquençage et du logiciel d'appel]{Représentation des erreurs d'appel de type IR et AR en fonction de la plateforme de séquençage et du logiciel d'appel d'après [@Hwang2015] : Pour la plateforme Illumina, on peut voir que Freebayes favorise les appels variant-homozygote tandis que GATK-HC et Samtools favorisent les appels hétérozygotes. Pour la plateforme Ion Proton, les 4 logiciels entraînent des erreurs de type IR}\label{fig:snperror}
  \end{figure}
  
  De même que pour l'aligneur, le choix du logiciel d'appel est crucial
  car il existe de nombreuses différences dans les variants appelés par
  différents logiciels se basant sur les mêmes données brutes
  {[}\protect\hyperlink{ref-Baes2014}{136}--\protect\hyperlink{ref-Rosenfeld2012}{138}{]}.
  En effet, en 2013, une étude comparant les résultats de 5 \emph{callers}
  montraient que seulement 57,4\% des variants étaient appelés par les 5
  \emph{callers} et que 80,7\% des variants étaient appelés par au moins 3
  d'entre eux. Ce taux chutait drastiquement pour les indels puisque la
  concordance était cette fois seulement de 26,8\% pour les indels non
  retrouvés par les 3 \emph{callers}
  {[}\protect\hyperlink{ref-ORawe2013}{137}{]}. Ces résultats sont
  cependant à pondérer avec une étude de 2015 comparant 4 \emph{callers}
  et montrant que 91,7\% des SNVs séquencés sur une plateforme Illumina
  étaient appelés par 3 \emph{callers}, cependant, pour les variants
  séquencés sur Ion Proton, seulement 27,3\% des variants étaient appelés
  par au moins 3 \emph{callers} et 57,4\% des variants n'étaient appelés
  que par un seul des \emph{callers}
  {[}\protect\hyperlink{ref-Hwang2015}{135}{]}.
  
  \newpage
  
  \subsection{L'annotation des variants, filtrage et
  priorisation}\label{lannotation-des-variants-filtrage-et-priorisation}
  
  Traditionnellement, les scientifiques et les laboratoires dans lesquels
  ils travaillaient développaient leur expertise dans un nombre de
  pathologies et de gènes associés limité. L'émergence du NGS est en train
  de remettre en cause cette pratique car la totalité de l'exome ou du
  génome peut permettre de couvrir tous les gènes quelque en une seule et
  même analyse. De nombreux praticiens maintiennent cependant une
  spécialisation pour certains groupes de pathologies qui est précieuse
  pour l'analyse des données et l'obtention d'un diagnostic. En effet il
  est courant de retrouver entre 20.000 et 25.000 variants différents par
  exome {[}\protect\hyperlink{ref-Gonzaga-Jauregui2012}{139}{]}. Afin de
  pouvoir lier un variant à une pathologie, il est indispensable d'annoter
  cet ensemble de variants, c'est à dire d'associer à ces variants
  l'ensemble des informations qui les caractérisent afin de pouvoir les
  replacer dans leur contexte biologique. Ces informations serviront
  ensuite d'indicateur afin filtrer ou prioriser un variant. Cette
  dernière étape de l'analyse est elle aussi cruciale puisqu'elle permet
  de réduire le nombre de variants à considérer. On peut généralement
  distinguer deux niveaux d'annotations d'un variant :
  
  \begin{enumerate}
  \def\labelenumi{\arabic{enumi}.}
  \tightlist
  \item
    \textbf{Au niveau du variant} : Ce niveau d'annotation regroupe
    l'ensemble des informations \textbf{spécifiques} à un variant
  
    \begin{enumerate}
    \def\labelenumii{\alph{enumii}.}
    \tightlist
    \item
      \textbf{Informations issues des résultats du séquençage} : la
      couverture du variant ainsi que la qualité qui lui est associée
      peuvent permettre de considérer un variant comme étant fiable ou
      non. Le génotype associé à ce variant est également une information
      importante.\\
    \item
      \textbf{La fréquence du variant dans la population générale} :
      l'émergence du séquençage haut-débit a permis de de gros consortium
      tel que ESP6500 (\href{http://evs.gs.washington.edu/EVS/}{Exome
      Variant Server, NHLBI GO Exome Sequencing Project (ESP), Seattle,
      WA}), 1KG
      {[}\protect\hyperlink{ref-1000GenomesProjectConsortium2015}{140}{]}.
      Ces consortiums ont pu mettre à disposition du public de données de
      séquençage exomique de 6503 individus pour ESP et de 2504 pour la
      phase 3 du 1000Genomes. On peut également noter l'\emph{Exome
      Aggregate Consortium} (ExAC)
      {[}\protect\hyperlink{ref-Lek2016}{141}{]} qui n'a effectué aucun
      séquençage mais qui a regroupé les données de plusieurs gros jeux
      (notamment 1000Genome et ESP) afin de leur appliquer la même analyse
      bioinformatique harmonisant ainsi les données provenant de 60.706
      individus non apparentés. Cette masse d'information permet de se
      faire une idée de la fréquence d'un variant dans la population
      générale et même au sein de sous population humaine. On considère
      qu'un variant fréquent ne peut pas être délétère, sinon il aurait
      été contre-sélectionné au cours de l'évolution.
    \end{enumerate}
  \end{enumerate}
  
  \newpage
  
  \begin{enumerate}
  \def\labelenumi{\alph{enumi}.}
  \setcounter{enumi}{2}
  \tightlist
  \item
    \textbf{Son impact sur le transcrit} : Dans la plupart des analyses
    phénotype-génotype, les chercheurs se limitent aux variants
    chevauchant des transcrits codants pour une protéine. Il est donc
    important de savoir l'impact d'un variant sur ce transcrit, c'est à
    dire si le variant va causer une mutation synonyme, un faux-sens ou
    une mutation tronquante. Des logiciels tel que \emph{Variant Effect
    Predictor} (VEP) {[}\protect\hyperlink{ref-McLaren2016}{142}{]},
    SnpEff {[}\protect\hyperlink{ref-Cingolani2012}{143}{]} ou encore
    ANNOVAR {[}\protect\hyperlink{ref-Wang2010}{144}{]} vont prédire
    l'impact qu'aura un variant sur les différents transcrits qu'il
    chevauche. D'autres logiciels tel que SIFT
    {[}\protect\hyperlink{ref-Kumar2009}{145}{]}, PROVEAN
    {[}\protect\hyperlink{ref-Choi2012}{146}{]}, Polyphen2, ou encore CADD
    vont eux chercher à prédire la pathogénicité de ce variant, c'est à
    dire la probabilité que ce variant soit délétère pour la fonction de
    la protéine. Bien que cette information soit important, elle est à
    pondérer étant donné le peu de concordance qu'il existe entre les
    prédictions de ces différents logiciel (\textbf{Figure :}
    \ref{fig:vennpred}).
  \end{enumerate}
  
  \begin{figure}
  
  {\centering \includegraphics[scale=.7]{figure/venn_Diag_patho_pred} 
  
  }
  
  \caption[Diagramme de Venn des prédictions de pathogénicités de six logiciels]{Diagramme de Venn des prédictions de pathogénicités de six logiciels d'après [@Salgado2016] : }\label{fig:vennpred}
  \end{figure}
  
  \newpage
  
  \begin{enumerate}
  \def\labelenumi{\arabic{enumi}.}
  \setcounter{enumi}{1}
  \tightlist
  \item
    \textbf{Au niveau de l'unité génétique} : DÉCRIRE UNITÉ GÉNÉTIQUE
    (gène, transcrit). L'annotation au niveau de l'unité génétique
    consiste à récupérer l'ensemble des informations disponible non plus
    sur le variant uniquement mais sur la ou les unités génétiques qu'il
    impacte. Ce ``dézoom'' permet d'ajouter des informations
    complémentaires particulièrement utiles notamment lorsque peu
    d'informations sont disponibles sur le variant lui-même. En pratique,
    la plupart des variants connus pour impliquer une pathologie sont des
    variants privés, c'est à dire spécifiques à une famille ou un individu
    limitant ainsi la quantité d'information disponible sur ce variant.
    Élargir l'annotation au niveau des unités génétiques impactés par des
    variants permet d'augmenter considérablement la quantité d'information
    disponible et permet donc d'améliorer la capacité des algorithmes à
    filtrer et / ou prioriser les variants rendant donc les analyses plus
    efficaces. On peut relever certains logiciels tel que le \emph{Protein
    ANalysis THrough Evolutionary Relationships} (PANTHER)
    {[}\protect\hyperlink{ref-Mi2017}{147}{]} qui permet par exemple de
    classer une liste de gènes en fonction de leurs fonctions
    moléculaires, des processus biologiques et des voies de signalisation
    dans lesquels ils sont impliqués. On peut également noter \emph{the
    Human Phenotype Ontology project} (HPO)
    {[}\protect\hyperlink{ref-Kohler2014}{148}{]} qui fournit une
    classification (À compléter). Plus récemment, on a pu voir émerger des
    ``scores mutationnel'' tel que RVIS {[}{\textbf{???}}{]} ou encore le
    pLI {[}\protect\hyperlink{ref-Lek2016}{141}{]}. En se basant sur les
    bases de données telle que ESP ou encore ExAC, ces scores permettent
    de classer les gènes en fonction de leur tolérance (ou intolérance)
    aux variations avec l'idée sous-jacente que ``les gènes impliqués dans
    des pathologies à transmission Mendéliennes'' devraient être moins
    tolérant aux variations que les autres.
  \end{enumerate}
  
  Comme nous l'avons vu, le développement d'outils permettant l'analyse
  des données NGS est extrêmement important puisqu'il permet aux
  biologistes de faire face à la masse de données générés par le
  séquençage haute débit l'aidant ainsi dans ses prises de décisions. Il
  est à noter que la plupart de ces données filtrées sont extrêmement
  dépendantes du jeu de gènes utilisés, les prédictions seront donc
  différentes si l'on se base sur les gènes RefSeq, Ensembl ou UCSC
  {[}\protect\hyperlink{ref-McCarthy2014}{149}{]} bien que les gènes du
  \emph{Consensus Coding Sequence project} (CCDS) soient bien représentés
  par ces trois listes {[}\protect\hyperlink{ref-Pruitt2009}{151}{]}. De
  même, pour une même liste de gêne, de nombreuses différences seront
  observées en fonction du ou des logiciels de prédiction utilisés
  {[}\protect\hyperlink{ref-McCarthy2014}{149},
  \protect\hyperlink{ref-Salgado2016}{152}{]}.
  
  \newpage
  
  \subsection{Conclusion NGS}\label{conclusion-ngs}
  
  En moins de 10 ans, les technologies NGS sont passées du séquençage de
  panels de gènes (environ 100 Mb pour le Roche GS FLX system) au
  séquençage de génomes entiers (environs 1500 GB pour l'Illumina Hiseq
  4000) et d'une utilisation exclusive à la recherche à l'analyse en
  routine dans un cadre de diagnostics cliniques. Le nombre croissant
  d'études utilisant le WGS ou le WES démontre le pouvoir de ces approches
  dans des analyses phénotypes-génotypes impliquant des pathologies à
  transmission Mendélienne. De plus, la diminution constante des coûts par
  génomes / exomes séquencés laisse supposer que ces technologies
  deviendront d'ici peut le fer de lance de la génétique clinique moderne.
  Cependant, cette quantité de données produites crées de nouvelles
  problématiques pour les généticiens qui se retrouvent désormais face au
  ``déluge de données génétiques''
  {[}\protect\hyperlink{ref-Schatz2013}{153}{]}. Le succès d'une étude
  n'étant plus lié aux capacités de séquençage mais aux compétences dans
  l'analyse et l'interprétation des données produites. Bien que de
  nombreux efforts soient faits pour palier la contrainte instaurée par
  les \emph{reads} courts dans le cadre d'analyse génomique, les solutions
  informatiques et bioinformatiques proposées jusqu'à présent restent en
  dessous des besoins créés par NGS
  {[}\protect\hyperlink{ref-McPherson2009}{154}{]}. Cette masse de données
  produite, à l'origine du succès du séquençage haut-débit dans le domaine
  de la génomique et de la post-génomique, se trouve désormais être un
  frein à la compréhension et l'interprétation des réseaux de gènes et
  leurs implications dans des pathologies. La limitation de cette
  technologie n'est donc plus le séquençage d'un, de plusieurs, ou de
  l'ensemble des gènes, mais plutôt l'analyse et l'interprétation des
  donnée générée. Le processus allant de l'extraction de l'ADN à
  l'identification d'un variant responsable d'une pathologie comprend de
  nombreuses étapes apportant avec elles leur lot d'erreurs. Bien que dans
  chacune de ces phases, de nombreux acteurs soient en concurrence et
  cherchent à atteindre une solution idéale, celle-ci n'a toujours pas été
  trouvée et la prolifération des logiciels et algorithmes d'analyses,
  bien que nécessaire, peut également parfois augmenter la confusion.
  
  Malgré les dizaines de milliers d'exomes et de génomes ayant été jusqu'à
  présent étudiés, notre compréhension des mécanismes moléculaires qui
  sous-tendent la variété génomique humaine reste limitée, et ce
  particulièrement dans le contexte de l'analyse de pathologies
  génétiques. En effet, à l'heure actuelle, plus de 3700 pathologie à
  transmission Mendélienne ont été caractérisées mais un nombre similaire
  ont toujours une cause inconnue
  {[}\protect\hyperlink{ref-Amberger2011}{155}{]}. L'élucidation de ces
  mystères passera probablement par une harmonisation des méthodes de
  production des données ainsi que par l'amélioration des techniques
  d'analyses.
  
  \chapter{Mise en place d'une stratégie pour l'analyse des données
  exomiques -- application en recherche
  clinique}\label{mise-en-place-dune-strategie-pour-lanalyse-des-donnees-exomiques-application-en-recherche-clinique}
  
  \newpage
  
  En 2011, les bases moléculaires d'environ 3700 pathologies à
  transmission Mendélienne avaient été élucidées. Cependant, pour une
  quantité équivalente de pathologies Mendéliennes (ou suspectées de
  l'être) cette cause reste un mystère
  {[}\protect\hyperlink{ref-Amberger2011}{155}{]}. Avec plusieurs
  centaines de pathologies caractérisées depuis 2010
  {[}\protect\hyperlink{ref-Ng}{156}{]}, les séquençages WGS et WES ont,
  depuis leur émergence, révolutionnés les méthodes de recherche dans le
  cadre d'étude phénotype-génotype en permettant de manière rapide et à
  moindre coup le séquençage de la quasi-totalité des gènes humains. Dès
  lors, le défis de ces analyses n'est plus le séquençage de l'ADN mais
  l'interprétation des données massives produites. En effet, l'un des plus
  grands challenges des analyses phénotype-génotype réalisées par WES
  réside dans l'analyse de l'importante quantité de variant portés par
  chaque individu s'élevant à plusieurs dizaines de milliers lorsque l'on
  compare avec le génome de référence. Même après avoir retiré les
  variants retrouvés fréquemment dans la population générale, des méthodes
  additionnelles sont nécessaires pour prédire, parmi les variants
  restant, lesquels induisent des conséquences fonctionnelles sérieuses
  afin de les prioriser {[}\protect\hyperlink{ref-Pelak2010}{157}{]}. De
  nombreux logiciels tel que Variant Effect Predictor
  {[}\protect\hyperlink{ref-McLaren2016}{142}{]}, SnpEff
  {[}\protect\hyperlink{ref-Cingolani2012}{143}{]} ou encore ANNOVAR
  {[}\protect\hyperlink{ref-Wang2010}{144}{]} permettent d'identifier
  quels sont les variants qui ont un effet tronquant sur la protéine.
  Cependant, avec en moyenne 165 variants homozygotes ayant un effet
  tronquant retrouvés dans chaque exomes
  {[}\protect\hyperlink{ref-Pelak2010}{157}{]} ces méthodes, bien
  qu'efficaces sont souvent insuffisantes.
  
  D'autres logiciels tel que Exomiser
  {[}\protect\hyperlink{ref-Robinson2014}{158}{]} vont, à partir d'une
  liste de variants \textbf{déjà} appelés effectuer les étapes
  d'annotation, de filtrage et de priorisation. Malgré l'efficacité de ces
  logiciels, aucun d'entre eux ne couvrent l'ensemble des étapes allant de
  l'alignement des \emph{reads} à la priorisation des variants. La plupart
  ayant pour point de départ une liste de variants appelés en amont. Ils
  ne contrôlent donc en aucune manière les étapes d'alignement et d'appel
  des variants. Or, comme il a été dit plus tôt, ces deux étapes
  constituent la base de l'analyse {[}{]}.
  
  Ce chapitre décrit à la fois la constitution d'un pipeline d'analyse des
  données de séquençage exomique recouvrant l'ensemble des étapes allant
  de l'allignement des séquences à la priorisation des variants ainsi que
  son utilisation dans le cadre de la recherche de mutations entrainant
  différents phénotypes d'infertilité d'une part de cas familiaux composés
  de duos ou trio et, pour finir, d'une large cohorte d'individus non
  apparentés présentant tous le même phénotype.
  
  \newpage
  
  \section{Méthode : Description du
  pipeline}\label{methode-description-du-pipeline}
  
  \subsection{\texorpdfstring{L'alignement des
  \emph{reads}}{L'alignement des reads}}\label{lalignement-des-reads}
  
  Comme expliqué plus tôt, l'étape d'alignement à pour objectif de
  repositioner l'ensemble des \emph{reads} d'un individu le long d'un
  génome de référence. Cette étape peut ainsi être comparée à la
  reconstruction d'un puzzle dans lequel chaque \emph{reads} peut-être
  assimilé à l'une des pièces tandis que le génome de référence serait ici
  le modèle (\textbf{Figure : }\ref{fig:picdnamapping}).
  
  L'ensemble de nos exomes ayant été réalisés en \emph{paired-end}, les
  deux extrémités de chaque fragment sont séquencées. Chaque \emph{end}
  d'un même \emph{read} peut donc être considérée comme un \emph{read} à
  part entière qui sont alignées \textbf{indépendamment} le long du génome
  de référence. L'information fournit par le \emph{paired-end} n'étant
  utilisé qu'à \emph{posteriori} en tant que critère qualité. Au sein de
  notre pipeline, cette étape est effectué par le logiciel MAGIC
  {[}\protect\hyperlink{ref-Su2014}{127}{]} qui dans le cadre de nos
  études, s'est basé sur la version hg19 / GHRC37 du génome de référence.
  Suite à cet alignement, plusieurs critères sont observés afin de filtrer
  les \emph{reads} présentant une faible qualité d'alignement.
  
  Ainsi, le premier de ces filtres consiste à tout d'abord filtrer
  l'ensemble des \emph{reads} dupliqués, c'est à dire les \emph{reads}
  ayant des séquences parfaitement identiques, ceux-ci étant souvent le
  résultats d'un excès d'amplification au moment des PCRs effectuées en
  amont. De la même manière, afin d'éviter toute ambiguité au momen de
  l'interprétation des résultats, l'ensemble des \emph{reads} s'étant
  alignés sur plusieurs région du génome sont aussi filtrés. Une fois cela
  fait, nous vérifions la ``compatibilité'' des deux \emph{ends} composant
  chacun des \emph{reads} restant. Un \emph{reads} est dit compatible
  lorsque les deux \emph{ends} qui le composent s'alignent face à face
  (une sur le brin sens du génome de référence et l'autre sur le brin
  anti-sens) et couvrent une zone ne faisant pas plus de 3 fois la taille
  médiane de l'insert. Les \emph{reads} dont les deux \emph{ends} se sont
  alignées mais ne remplissant pas ces conditions seront dit ``Non
  compatible'', ceux dont une seule des deux \emph{ends} s'est alignés
  seront appelés ``orphelins'' et enfin ceux pour lesquels aucune des deux
  \emph{ends} ne se sont allignées sont appellés ``non-aligné''.
  L'ensemble des \emph{reads} ``non-compatible'', ``orphelins'' et
  ``non-alignés'' sont, en raison de leur faible qualité, filtré et donc
  non considérés pour les analyses en aval. Les \emph{reads} ayant passé
  l'ensemble des critères qualité mentionnés précédemment seront, eux,
  utilisés pour effectuer l'appel des variants.
  
  \newpage
  
  \subsection{L'appel des variants}\label{lappel-des-variants}
  
  Si l'allignement des séquences peut être comparé à la reconstruction
  d'un puzzle, l'appel des variant pourrait lui être vu comme un jeu des 7
  erreurs, au cours duquel, pour chaque position couverte, les différence
  entre la séquence de l'individu séquencé et le génome de référence
  seront listés et appelé variants. Comme nous l'avons vu plus
  \protect\hyperlink{varcall}{tôt}, il est fortement conseillé d'effectuer
  l'appel des variants en tenant compte de l'aligneur choisi
  {[}\protect\hyperlink{ref-Nielsen2011}{123},
  \protect\hyperlink{ref-DePristo2011}{131},
  \protect\hyperlink{ref-Lunter2011}{132}{]}. C'est pourquoi, nous avons
  développé notre propre algorithme d'appel des variants spécialement
  conçu pour l'analyse des données de MAGIC. Ainsi, l'appel des variants
  sera directement basé sur quatre comptages (R\(_+\), R\(_-\), V\(_+\) et
  V\(_-\)) fourni directement par MAGIC pour chaque position suffisement
  couvertes :
  
  \begin{enumerate}
  \def\labelenumi{\arabic{enumi}.}
  \tightlist
  \item
    \textbf{R}\(_+\) \textbf{et R}\(_-\) : Ces deux comptages
    correspondent au nombre de \emph{reads} \emph{forward} (+) et
    \emph{reverse} (-) sur lesquels est observé l'allèle de
    \textbf{référence} (R) à une position donnée.\\
  \item
    \textbf{V}\(_+\) \textbf{et V}\(_-\) : À l'inverse de R\(_+\) et
    R\(_-\), ces comptages correspondent au nombre de \emph{reads}
    \emph{forward} et \emph{reverse} sur lesquels est observé un allèle de
    \textbf{variant} (V) à une position donnée.
  \end{enumerate}
  
  Ainsi, les sommes : \(R_+ + V_+\) et \(R_- + V_-\) indiqueront
  respectivement la couverture d'une position en ne tenant que des
  \emph{reads forward} et \emph{reverse}. En fonction de ces couverturs
  nos appels seront classés en trois catégories :
  
  \begin{enumerate}
  \def\labelenumi{\arabic{enumi}.}
  \tightlist
  \item
    \textbf{Les appels \emph{double strand} (DS) :} Qualifie les positions
    ayant une couverture \(\ge\) 10 sur \textbf{les deux} strands. Ces
    appels sont ceux sont ceux ayant la meilleure qualité.\\
  \item
    \textbf{Les appels \emph{single strand} (SS) :} Ces appels définissent
    les positions pour lesquels \textbf{un des deux} \emph{strands}
    présentent une couverture \(\le\) 10. Dans ce cas, ce \emph{strand}
    est ignoré et l'appel est effectué uniquement en utilisant le second
    \emph{strand}.\\
  \item
    \textbf{Les appels \emph{non strand} (NS) :} Les positions NS sont
    celles pour lesquelles la couverture est \(\le\) 10 sur \textbf{les
    deux} strands. Aucun appel n'est effectué à ces positions qui
    \textbf{ne sont pas conservés dans la suite des analyses}.
  \end{enumerate}
  
  Ensuite, chaque position couverte, des appels indépendants seront
  effectués pour chaque \emph{strand} de telle sorte que, pour chaqune de
  ces position si :
  
  \begin{enumerate}
  \def\labelenumi{\arabic{enumi}.}
  \tightlist
  \item
    0 à 20\% des \emph{reads} portent un variant, la position est appelée
    \textbf{homozygote référence}.\\
  \item
    40 à 75\% des \emph{reads} portent un variant, la position est appelée
    \textbf{hétérozygote}.\\
  \item
    85 à 100\% des \emph{reads} portent un variant, la position est
    appelée \textbf{homozygote variant}.\\
  \item
    20 à 40\% des \emph{reads} portent un variant, l'appel sera considéré
    comme \textbf{ambigu bas}.\\
  \item
    75 à 85\% des \emph{reads} portent un variant, l'appel sera considéré
    comme \textbf{ambigu haut}.
  \end{enumerate}
  
  Pour les positions SS, l'appel final coresspondra directement à l'appel
  effectué sur l'unique \emph{strand} suffisemment couvert. Pour les
  positions DS, la concordance des appels fournis par chaque \emph{end}
  est verifié. Ainsi, un variant sera considéré :
  
  \begin{enumerate}
  \def\labelenumi{\arabic{enumi}.}
  \tightlist
  \item
    \textbf{Homozygote référence} si les deux appels sont homozygote
    référence, ou, un des appels est homozygote référence et l'autre
    ambigu bas.
  \item
    \textbf{Hétérozygote} si les deux appels sont hétérozygotes, ou, si
    l'un des appels est hétérozygote et l'autre ambigu bas ou haut.\\
  \item
    \textbf{Homozygote variant} si les deux appels sont homozygote
    variant, ou, un des appels est homozygote variant et l'autre ambigu
    haut
  \item
    \textbf{Ambigu} si les deux appels sont ambigu bas ou si ils sont tous
    les deux ambigu haut.\\
  \item
    \textbf{Discordant} pour toutes les combinaisons restantes.
  \end{enumerate}
  
  Dans le cadre de nos analyses, les appels ambigu et discordant sont
  filtrés.
  
  \newpage
  
  \subsection{L'annotation}\label{lannotation}
  
  Chaque variant retenu sera ensuite annoté tout d'abord par le logiciel
  \emph{variant effect predictor} (VEP)
  {[}\protect\hyperlink{ref-McLaren2016}{142}{]} qui nous indiquera pour
  chaque variant la conséquence que celui-ci aura sur la séquence codante
  de l'ensemble des transcrits Ensembl qu'il chevauche (\textbf{Figure :
  }\ref{fig:figvepcsq}, \textbf{Table : }\ref{tab:tabvepcsq}). Dans le cas
  de substitution faux-sens, c'est à dire entrainant le changement d'un
  seul acide-aminé de la séquence protéique, nous utiliserons les
  prédictions fournies par SIFT et PolyPhen afin d'estimer leur
  pathogénicité. Ensuite, nous ajoutons, pour chaque gène, son expression
  tissulaire en nous basant sur les données Ensembl
  {[}\protect\hyperlink{ref-Aken2017}{159}{]} générées par le projet
  Illumina BodyMap qui recense les données RNAseq des gènes humains pour
  16 tissus différents. Suite à cela nous ajoutons, lorsque celle-ci est
  disponible, la fréquence du variant dans les bases de données ExAC
  {[}\protect\hyperlink{ref-Lek2016}{141}{]}, ESP600
  (\href{http://evs.gs.washington.edu/EVS/}{Exome Variant Server, NHLBI GO
  Exome Sequencing Project (ESP), Seattle, WA}) et 1000Genomes
  {[}\protect\hyperlink{ref-1000GenomesProjectConsortium2015}{140}{]}
  donnant ainsi une estimation de sa fréquence dans la population
  générale. De même, la particularité de ce pipeline est qu'elle conserve
  l'ensemble des variants identifiés dans les études effectuées
  précédemment permettant d'ajouter aux annotations la fréquence d'un
  variant chez les individus déjà séquencé et donc la fréquence d'un
  variant dans chaque phénotype étudié créant ainsi une base de données
  interne qui pourra servir de contrôle dans les études ultérieur.
  
  \begin{figure}
  
  {\centering \includegraphics[scale=.9]{figure/vep_csq} 
  
  }
  
  \caption[Listes des différentes conséquences prédites par VEP et leur positionnement sur le transcrit]{Listes des différentes conséquences prédites par VEP et leur positionnement sur le transcrit d'après [VEP site](http://www.ensembl.org/info/genome/variation/consequences.jpg)}\label{fig:figvepcsq}
  \end{figure}
  
  \newpage
  
  \blandscape
  
  \begin{longtable}[]{@{}lll@{}}
  \caption{\label{tab:tabvepcsq} Liste simplifiée des conséquences prédites
  par VEP avec leur description et impact associée}\tabularnewline
  \toprule
  \begin{minipage}[b]{0.18\columnwidth}\raggedright\strut
  VEP consequence\strut
  \end{minipage} & \begin{minipage}[b]{0.11\columnwidth}\raggedright\strut
  VEP impact\strut
  \end{minipage} & \begin{minipage}[b]{0.63\columnwidth}\raggedright\strut
  Description\strut
  \end{minipage}\tabularnewline
  \midrule
  \endfirsthead
  \toprule
  \begin{minipage}[b]{0.18\columnwidth}\raggedright\strut
  VEP consequence\strut
  \end{minipage} & \begin{minipage}[b]{0.11\columnwidth}\raggedright\strut
  VEP impact\strut
  \end{minipage} & \begin{minipage}[b]{0.63\columnwidth}\raggedright\strut
  Description\strut
  \end{minipage}\tabularnewline
  \midrule
  \endhead
  \begin{minipage}[t]{0.18\columnwidth}\raggedright\strut
  Splice acceptor / donor\strut
  \end{minipage} & \begin{minipage}[t]{0.11\columnwidth}\raggedright\strut
  HIGH\strut
  \end{minipage} & \begin{minipage}[t]{0.63\columnwidth}\raggedright\strut
  A splice variant that changes the 2 base region at the 3' / 5' end of an
  intron\strut
  \end{minipage}\tabularnewline
  \begin{minipage}[t]{0.18\columnwidth}\raggedright\strut
  Stop gained\strut
  \end{minipage} & \begin{minipage}[t]{0.11\columnwidth}\raggedright\strut
  HIGH\strut
  \end{minipage} & \begin{minipage}[t]{0.63\columnwidth}\raggedright\strut
  A sequence variant whereby at least one base of a codon is changed,
  resulting in a premature stop codon, leading to a shortened
  transcript\strut
  \end{minipage}\tabularnewline
  \begin{minipage}[t]{0.18\columnwidth}\raggedright\strut
  Frameshift\strut
  \end{minipage} & \begin{minipage}[t]{0.11\columnwidth}\raggedright\strut
  HIGH\strut
  \end{minipage} & \begin{minipage}[t]{0.63\columnwidth}\raggedright\strut
  A sequence variant which causes a disruption of the translational
  reading frame, because the number of nucleotides inserted or deleted is
  not a multiple of three\strut
  \end{minipage}\tabularnewline
  \begin{minipage}[t]{0.18\columnwidth}\raggedright\strut
  Stop lost\strut
  \end{minipage} & \begin{minipage}[t]{0.11\columnwidth}\raggedright\strut
  HIGH\strut
  \end{minipage} & \begin{minipage}[t]{0.63\columnwidth}\raggedright\strut
  A sequence variant where at least one base of the terminator codon
  (stop) is changed, resulting in an elongated transcript\strut
  \end{minipage}\tabularnewline
  \begin{minipage}[t]{0.18\columnwidth}\raggedright\strut
  Start lost\strut
  \end{minipage} & \begin{minipage}[t]{0.11\columnwidth}\raggedright\strut
  HIGH\strut
  \end{minipage} & \begin{minipage}[t]{0.63\columnwidth}\raggedright\strut
  A codon variant that changes at least one base of the canonical start
  codo\strut
  \end{minipage}\tabularnewline
  \begin{minipage}[t]{0.18\columnwidth}\raggedright\strut
  Inframe insertion / deletion\strut
  \end{minipage} & \begin{minipage}[t]{0.11\columnwidth}\raggedright\strut
  MODERATE\strut
  \end{minipage} & \begin{minipage}[t]{0.63\columnwidth}\raggedright\strut
  An inframe non synonymous variant that inserts / deletes bases into in
  the coding sequenc\strut
  \end{minipage}\tabularnewline
  \begin{minipage}[t]{0.18\columnwidth}\raggedright\strut
  Missense\strut
  \end{minipage} & \begin{minipage}[t]{0.11\columnwidth}\raggedright\strut
  MODERATE\strut
  \end{minipage} & \begin{minipage}[t]{0.63\columnwidth}\raggedright\strut
  A sequence variant, that changes one or more bases, resulting in a
  different amino acid sequence but where the length is preserved\strut
  \end{minipage}\tabularnewline
  \begin{minipage}[t]{0.18\columnwidth}\raggedright\strut
  Splice region\strut
  \end{minipage} & \begin{minipage}[t]{0.11\columnwidth}\raggedright\strut
  LOW\strut
  \end{minipage} & \begin{minipage}[t]{0.63\columnwidth}\raggedright\strut
  A sequence variant in which a change has occurred within the region of
  the splice site, either within 1-3 bases of the exon or 3-8 bases of the
  intron\strut
  \end{minipage}\tabularnewline
  \begin{minipage}[t]{0.18\columnwidth}\raggedright\strut
  Stop retained\strut
  \end{minipage} & \begin{minipage}[t]{0.11\columnwidth}\raggedright\strut
  LOW\strut
  \end{minipage} & \begin{minipage}[t]{0.63\columnwidth}\raggedright\strut
  A sequence variant where at least one base in the terminator codon is
  changed, but the terminator remains\strut
  \end{minipage}\tabularnewline
  \begin{minipage}[t]{0.18\columnwidth}\raggedright\strut
  Synonymous\strut
  \end{minipage} & \begin{minipage}[t]{0.11\columnwidth}\raggedright\strut
  LOW\strut
  \end{minipage} & \begin{minipage}[t]{0.63\columnwidth}\raggedright\strut
  A sequence variant where there is no resulting change to the encoded
  amino acid\strut
  \end{minipage}\tabularnewline
  \begin{minipage}[t]{0.18\columnwidth}\raggedright\strut
  5 / 3 prime UTR\strut
  \end{minipage} & \begin{minipage}[t]{0.11\columnwidth}\raggedright\strut
  MODIFIER\strut
  \end{minipage} & \begin{minipage}[t]{0.63\columnwidth}\raggedright\strut
  A UTR variant of the 5' / 3' UTR\strut
  \end{minipage}\tabularnewline
  \begin{minipage}[t]{0.18\columnwidth}\raggedright\strut
  Intron\strut
  \end{minipage} & \begin{minipage}[t]{0.11\columnwidth}\raggedright\strut
  MODIFIER\strut
  \end{minipage} & \begin{minipage}[t]{0.63\columnwidth}\raggedright\strut
  A transcript variant occurring within an intron\strut
  \end{minipage}\tabularnewline
  \begin{minipage}[t]{0.18\columnwidth}\raggedright\strut
  NMD transcript\strut
  \end{minipage} & \begin{minipage}[t]{0.11\columnwidth}\raggedright\strut
  MODIFIER\strut
  \end{minipage} & \begin{minipage}[t]{0.63\columnwidth}\raggedright\strut
  A variant in a transcript that is the target of NMD\strut
  \end{minipage}\tabularnewline
  \begin{minipage}[t]{0.18\columnwidth}\raggedright\strut
  Non coding transcript\strut
  \end{minipage} & \begin{minipage}[t]{0.11\columnwidth}\raggedright\strut
  MODIFIER\strut
  \end{minipage} & \begin{minipage}[t]{0.63\columnwidth}\raggedright\strut
  A transcript variant of a non coding RNA gene\strut
  \end{minipage}\tabularnewline
  \bottomrule
  \end{longtable}
  
  \elandscape
  \newpage
  
  \subsection{Le filtrage des variants}\label{le-filtrage-des-variants}
  
  L'étape de filtrage est extrêmement importante si l'on souhaite analyser
  de manière efficace les données provenant de WES. C'est pourquoi elle
  occupe une place importante dans notre pipeline. L'intégralité des
  paramètres de cette étape peuvent être modifiés par l'utilisateur de
  sorte à faire correspondre les critères de filtre aux besoins de
  l'étude. Afin de rendre son utilisation le plus efficace possible, nous
  avons souhaité définir des paramètres par défauts pertinent dans la
  plupart des études de séquençage exomique de sorte que à moins que le
  contraire ne soit spécifié les filtres suivant seront appliqués :
  
  \begin{enumerate}
  \def\labelenumi{\arabic{enumi}.}
  \item
    \textbf{Filtre 1 : L'union des variants :} Dans le cas ou des
    individus présentant un lien de parenté et présentant le même
    phénotype sont analysé, seuls les variants observés chez l'ensemble
    des individus sont conservés.
  \item
    \textbf{Filtre 2 : Génotype des variants :} Ce pipeline d'analyse a
    avant tout été développé pour la recherche de variant impliqué dans
    des pathologies à transmission récessives. C'est pourquoi, dans le
    cadre d'étude d'individus présentant un historique de consanguinité,
    l'ensemble des variants hétérozygotes sont filtrés. En revanche, dans
    le cas d'individus issus d'unions non consanguin nous procédons à la
    recherches de variants hétérozygotes composites, c'est à dire
    \textbf{au moins deux variants hétérozygotes différents situés sur
    chacun des deux allèles du même gène d'un patient}. Dès lors, bien que
    les variants soient différents, les deux allèles sont altérés rendant
    posiible l'apparition de phénotype récessifs. Malheureusement, dans le
    cadre des séquençages WES et WGS, il est impossible de connaitre le
    ``phasage'' de ces variants, c'est à dire que l'on ne peut déterminer
    si deux variants hétérozygotes sont situés sur le même allèles ou sur
    deux allèles différents (\textbf{Figure : }\ref{fig:compositehet}).
    Pour cela, des analyses de biologie moléculaire sont nécessaires.
  \end{enumerate}
  
  \newpage
  
  \begin{figure}
  
  {\centering \includegraphics[scale=0.35]{figure/hetero_composites} 
  
  }
  
  \caption[Représentation schématique des phasages de deux variants avec les génotypes associés]{Représentation schématique des phasages de deux variants avec les génotypes associés : Un variant est dit homozygote lorsque le **même** variant est présents sur les deux allèles d'un gène et hétérozygote lorsqu'il est présent sur **un seul** des deux allèles. On parlera d'hétérozygotes *cis* lorsque deux variants hétérozygotes différents seront positionnés sur **le même** allèle et d'hétérozygote *trans* (ou composite) lorsque ces deux variants hétérozygotes seront positionnés sur **deux allèles différents**. En WES et en WGS il est impossible de différentié les hétérozygotes *cis* des hétérozygotes *trans*}\label{fig:piccompositehet}
  \end{figure}
  
  \begin{enumerate}
  \def\labelenumi{\arabic{enumi}.}
  \setcounter{enumi}{2}
  \item
    \textbf{Filtre 3 : Les transcrits ``non pertinents'' :} Au cours de
    nos analyses nous nous sommes concentré uniquement sur les transcrits
    codant pour une protéine. Ainsi, l'ensemble des transcrits annotés
    comme étant non codant furent filtrés. De même pour les transcrits
    annotés comme étant NMD (\emph{nonsense-mediated decay}). En effet, ce
    mécanisme a pour but de contrôler la qualité des ARNm cellulaires chez
    les eucaryotes {[}\protect\hyperlink{ref-Chang2007}{160}{]} en
    éliminant les ARNm qui comportent un codon stop prématuré
    {[}\protect\hyperlink{ref-Baker2004}{161}{]}, pouvant être le résultat
    d'une erreur de transcription, d'une mutation ou encore d'une erreur
    d'épissage. Il est donc peu probable que les variants présents sur des
    transcrits annotés NMD soient responsables du phénotype. Dès lors, ces
    transcrits ont été également filtrés. Ainsi, l'ensemble des variants
    impactant \textbf{uniquement} des transcrits non codant et / ou annoté
    NMD sont filtrés.
  \item
    \textbf{Filtre 4 : Impact du variant :} Afin de ne conserver que les
    variants ayant le plus probablement délétère sur la protéine, seuls
    sont conservés ceux impactant la séquence codante d'un transcrit. De
    plus les variants synonymes ne sont pas conservés (exceptés ceux se
    trouvant proches des régions d'épissage) car ceux-ci n'ont aucun effet
    sur la séquence protéique. Pour les variants faux sens (changement
    d'un seul acide-aminé de la séquence protéique) il est plus difficile
    de se trancher, dès lors, seuls ceux étant prédit comme
    \emph{tolerated} par SIFT {[}\protect\hyperlink{ref-Kumar2009}{145}{]}
    \textbf{et} comme \emph{benign} par Polyphen
    {[}\protect\hyperlink{ref-Adzhubei2010}{162}{]} sont filtrés.
  \item
    \textbf{Filtre 5 : Fréquence des variants :} La fréquence d'un variant
    dans la population générale est un moyen rapide d'avoir une prédiction
    fiable de l'effet délétère ou non de celui-ci. En effet, il est peu
    probable qu'un variant retrouvé fréquemment dans la population
    générale soit causal d'une pathologie sévère. C'est pourquoi,
    l'ensemble des variants ayant une fréquence \(\ge\) 1\% dans l'une des
    trois bases de données que sont ExAC, ESP et 1KG sont filtrés.
  \item
    \textbf{Filtre 6 : Présence des variants dans la cohorte contrôle :}
    Au sien de notre pipeline, les données de l'ensemble des patients
    analysés dans les études anterieures sont conservés créant ainsi une
    base de donnée interne de variants. Dès lors, il devient possible
    d'utiliser chacun de ces patient comme contrôle lorsqu'ils ne sont pas
    porteur du même phénotype que celui des patients des études
    ulterieures. Ce filtre se révèle particulièrement intéréssant lorsque
    plusieurs patients porteurs de phénotypes différents ont subi le même
    protocole de séquençage ainsi l'ensemble des variants faux-positifs
    résultant d'artéfact liés au différentes étapes en amont de l'analyse
    bioinformatique pourront alors être filtré. De même ce filtre permet
    de mettre en évidence les variants propores à une population lorsque
    des patients provenant de la même région géographique et ne présentant
    toujours pas le même phénotype sont comparés.
  \end{enumerate}
  
  \newpage
  
  \section{Résultats 1 : Analyse de 3 cas
  familiaux}\label{resultats-1-analyse-de-3-cas-familiaux}
  
  Dans cette partie, se concentre sur l'analyse bioinformatique des
  résultats des séquençages exomiques de 6 individus infertiles provenant
  de 3 familles et présentant toutes un phénotype d'infertilité masculine
  différenteffectués entre 2012 et 2014 \ref{tab:tabfam} :
  
  \begin{enumerate}
  \def\labelenumi{\arabic{enumi}.}
  \tightlist
  \item
    \textbf{Famille FAM} : Cette famille est composée de 2 frères
    azoospermes. Comme nous avons pu le voir, l'azoospermie est un
    phénotype d'infertilité masculine caractérisé par l'absence de
    spermatozoïde dans l'éjaculât. Des 6 patients de cette étude, les
    frères Ghs44 et Ghs45 sont les deux seuls à ne pas avoir été séquencés
    au Génopole d'Évry.\\
  \item
    \textbf{Famille FF} : Les spermatozoïdes des 2 frères de cette famille
    sont caractérisés par leur incapacité à féconder l'ovocyte malgré leur
    morphologie et leur mibilité normales.\\
  \item
    \textbf{Famille MMAF} : Le syndrome MMAF (\emph{multiple morphological
    abnormalities of the sperm flagella}) dont souffrent les frères de
    cette famille se caractérise comme son nom l'indique, par la présence
    d'une majorité de spermatozoïdes présentant par une mosaïque
    d'anomalie morphologique du flagelle.
  \end{enumerate}
  
  \begin{longtable}[t]{lllrl}
  \caption{\label{tab:tabfam}Tableau récapitulatif des familles séquencées et de leur phénotype}\\
  \toprule
  Family & Individuals & Phenotype & Year & Place\\
  \midrule
  AZ & Ghs44, Ghs45 & Azoospermia & 2012 & Mount Sinai Institut\\
  FF & Ghs113, Ghs117 & Fertilization failure & 2014 & Genoscope (Evry)\\
  MMAF3 & Ghs62, Ghs130 & MMAF & 2014 & Genoscope (Evry)\\
  \bottomrule
  \end{longtable}
  
  \newpage 
  
  \subsection{Résultats des différents étapes de
  l'analyse}\label{resultats-des-differents-etapes-de-lanalyse}
  
  \subsubsection{Résultat de l'alignement}\label{resultat-de-lalignement}
  
  Pour rappel, l'\href{\%7B\#lalignement\%7D}{alignement} consiste à
  repositionner l'ensemble des \emph{reads} générés au cours de l'étape de
  séquençage le long d'un génome de référence.
  
  La quantité de \emph{reads} composant les exomes de chaque individu peut
  varier en fonction de plusieurs paramètres et n'est donc pas égale pour
  chaque patient bien que l'ordre de grandeur reste le même avec une
  médiane de 91978875 \emph{reads}. Seuls les deux frères AZ1 et AZ2 se
  distinguent avec près de 3 fois plus de \emph{reads} que les autres
  patients. Cette différence peut être expliquée car ces deux patients
  sont les deux seuls à voir été séquencés au Mount Sinaï Institut or leur
  protocole d'amplification précédent le séquençage contient un nombre de
  cycles de PCR supérieur à ceux appliqués au Génopole d'Évry où ont été
  séquencés les autres patients. Il faut noter que ce nombre plus
  important de \emph{reads} n'est en rien le reflet d'une meilleure
  qualité. En effet, celui-ci est causé par une grande quantité de
  \emph{reads} dupliqués qui seront pour la plupart filtrés au cours des
  analyses ultérieures (\textbf{Table :} \ref{tab:tabfam}, \textbf{Figure
  : }\ref{fig:plotfammapping} - \textbf{A}).
  
  La première étape du contrôle qualité des \emph{reads} consiste à
  filtrer les \emph{reads} ne s'étant pas alignés sur le génome. Ces
  \emph{reads} sont extrêmement minoritaires puisqu'ils ne représentent
  qu'entre 1.2 et 5.5 \% des \emph{reads} de nos individus (\textbf{Figure
  : }\ref{fig:plotfammapping} - \textbf{B}).
  
  Dans nos analyses, seuls les \emph{reads} compatibles sont conservés,
  c'est à dire environs 95.2 \% des \emph{reads} s'étant correctement
  alignés. (\textbf{Figure : }\ref{fig:plotfammapping} - \textbf{C}).
  
  La dernière étape de ce contrôle-qualité consiste à analyser le nombre
  de site auxquels se sont alignés les \emph{reads}. En effet, certaine
  zone du génome étant dupliqué, l'une des problématiques des
  \emph{short-reads} est qu'il est possible que ceux-ci s'alignent à
  plusieurs régions différentes du génome. Afin d'éviter toute ambiguïté,
  seul ceux s'étant aligné sur un site unique sont conservés pour la suite
  des analyses. Ces \emph{reads} représente entre 92.3 et 96.9 \% des
  \emph{reads} ayant passé les précédents filtres (\textbf{Figure :
  }\ref{fig:plotfammapping} - \textbf{D}).
  
  \newpage 
  
  \begin{figure}
  
  {\centering \includegraphics{thesis_files/figure-latex/plotfammapping-1} 
  
  }
  
  \caption[Processus simplifié du contrôle qualité des *reads*]{Processus simplifié du contrôle qualité des *reads* : Pour chacun des graphiques, les *reads* représentés en vert sont conservés tandis que ceux en rouge sont filtrés. **A** : Quantité de *reads* bruts générés pour chaque patient au cours de l'étape de séquençage. La médiane des *reads* est représentée en bleue. **B** : Pourcentage pour chaque individu de *reads* s'étant aligné correctement et ne s'étant pas alignés sur le génome de référence. **C** : Distribution pour chaque patient des *reads* compatibles (Comp), non compatibles (Non comp) et orphelins (Orphans). **D** : Présentation pour chaque *reads* du nombre de site auxquels ils s'alignent}\label{fig:plotfammapping}
  \end{figure}
  
  \newpage  
  
  \subsubsection{L'appel des variants}\label{lappel-des-variants-1}
  
  Dans nos données, les appels SS sont majoritaires et représentent
  environ 46 \% de nos appels (contre 38.6 \% d'appels DS). Au vus de
  l'importance de ces appels, nous avons fait le choix de les conserver
  afin de ne pas filtrer une quantité trop importante de données. Ces
  appels seront cependant considérés comme étant de faible qualité, de
  fait, leurs analyses et interprétation seront plus précautionneuses. En
  revanche, au vus de la trop grande incertitude de l'appel des variants
  NS, ceux-ci sont systématiquement filtrés éliminant ainsi entre 12.4 et
  18.7 \% des positions appelées pour chaque patient (\textbf{Figure :
  }\ref{fig:plotvarcall} - \textbf{A}).
  
  Les appels discordant et ambigus sont filtrés, soit environ 85.8 \% des
  variants DS. Il est intéressant de noter que bien que les variants
  \emph{single strand} (SS) soient conservés, on peut s'attendre à ce
  qu'environ 14.2 \% de ceux-ci soient aberrants, ceux-ci n'ayant pu subir
  le même contrôle que les SS (\textbf{Figure : }\ref{fig:plotvarcall} -
  \textbf{B}).
  
  Pour l'ensemble des variants ayant passé les filtres énoncés ci-dessus,
  c'est à dire les variants SS et les variants DS avec appels concordants,
  le génotype est déterminé en fonction du pourcentage de \emph{reads}
  portant le variant à cette position. Ainsi, pour chaque individu nous
  avons pu établir une liste de SNVs et d'indels avec leur génotype
  associé. Pour chacun de nos 6 patients les ordres de grandeur du nombre
  de variants appelés sont identique. Ainsi pour chaque patient nous avons
  appelés environ 46684 variants hétérozygotes (44126 SNVs et 2558 indels)
  et 64944 variants homozygotes (32472 SNVs et 1773 indels)
  (\textbf{Figure : }\ref{fig:plotvarcall} - \textbf{C}).
  
  \newpage
  
  \begin{figure}
  
  {\centering \includegraphics{thesis_files/figure-latex/plotvarcall-1} 
  
  }
  
  \caption[Contrôle qualité des variants appelés]{Contrôle qualité des variants appelés : Pour chacun des graphiques, les variants représentés en vert et en orange sont conservés tandis que ceux en rouge sont filtrés. **A** : Distribution du *stranding* des appels pour chaque patient. **B** : Comparaison des appels entre les deux *ends* des variants appelés DS. **C** : Distribution des SNVs et indels en fonction de leur génotype pour chaque patients (représentés par une barre}\label{fig:plotvarcall}
  \end{figure}
  
  \newpage
  
  \subsubsection{L'annotation des
  variants}\label{lannotation-des-variants}
  
  Après avoir annoté nos variants, nous avons pu constater que pour chaque
  patient 24851 gènes sont en moyenne affecté par au moins un variant
  homozygote pour en moyenne 122630 transcrits (soit environ 5 transcrits
  par gènes). Il faut noter que parmi ces gènes se trouvent à la fois des
  gènes codant pour des protéine \textbf{et} d'autres non codant
  (\textbf{Figure : }\ref{fig:plotannotation} - \textbf{A}).
  
  Chaque variant affectera l'ensemble des transcrits qu'il chevauche,
  ainsi un même variant pourra impacter plusieurs transcrits. Ces impacts
  sont ensuite classés par VEP en quatre catégories qui sont, de la plus
  délétère à la moins délétère : \emph{HIGH}, \emph{MODERATE}, \emph{LOW},
  \emph{MODIFIER} (\textbf{Table :}\ref{tab:tabvepcsq}).
  
  Comme attendu, les variants ayant un impact tronquant se retrouvent être
  les moins fréquent chez chacun de nos patients. Ceci est d'autant plus
  flagrant pour l'impact \emph{HIGH} qui regroupe, entre autres, les
  variants créant un codon stop ou encore ceux causant un décalage du
  cadre de lecture (\textbf{Table :}\ref{tab:tabvepcsq}), se retrouvent,
  par rapport aux autres impacts, en quantité extrêmement faible
  puisqu'ils ne représentent en moyenne que 0.16 \% des variants.
  Cependant, bien que ce pourcentage soit faible, cela représente tout de
  même une moyenne de 504 variants \emph{HIGH} hétérozygotes par patients
  et 332 variants \emph{HIGH} homozygotes par patient) (\textbf{Figure :
  }\ref{fig:plotannotation} - \textbf{B}).
  
  \newpage
  
  \begin{figure}
  
  {\centering \includegraphics{thesis_files/figure-latex/plotannotation-1} 
  
  }
  
  \caption[Annotation des variants]{Annotation des variants : **A** : Quantification du nombre de gènes (en bleu) / transcrits (en rose) impactés par au moins un variant pour chaque patient chacun représentés par une barre. **B** : Distribution des impacts HIGH MODERATE LOW et MODIFIER en fonction des patients et du génotype du variant}\label{fig:plotannotation}
  \end{figure}
  
  \subsubsection{Le filtrage des
  variants}\label{le-filtrage-des-variants-1}
  
  Les étapes précédentes nous ont permis de mettre en évidence pour chaque
  patient une liste de variants passant l'ensemble de nos critères
  qualités. Ces variants ont dès lors pu être annotés nous permettant
  notamment d'avoir connaissance de leurs impacts sur les différents
  transcrits qu'ils chevauchent ou encore leur fréquence dans la
  population générale. Désormais, afin de ne conserver que les variants
  ayant la plus forte probabilité d'être responsable du phénotype de ces
  patients, nous avons appliqué successivement les six filtres
  précédemment décrits.
  
  \begin{enumerate}
  \def\labelenumi{\arabic{enumi}.}
  \item
    \textbf{Filtre 1 : L'union des variants :} Dans cette étude nous
    analysons les données génétiques de 3 chacune composée de 2 frères.
    Nous avons donc émmi l'hypothèse que le phénotype de chacun de frères
    d'une même famille était dûe à une cause génétique commune. C'est
    pourquoi, seul les variants observés chez l'ensemble des membres d'une
    même famille furent conservés. Ainsi se filtre a permis de filtrer
    entre \ldots{} et \ldots{} variants pour chacun des patients.
  \item
    \textbf{Filtre 2 : Génotype des variants :} Ici, nous avons émis
    l'hypothèse d'une transmission récessive du phénotype. Ainsi, seuls
    les variants homozygotes ont été conservés. filtrant en moyenne
    \ldots{} variant par individu soit une moyenne de \ldots{} \% de leurs
    variants (\textbf{Figure : }\ref{fig:plotvarcall},
    \ref{fig:plotcomparefilter}).
  \item
    \textbf{Filtre 3 : Impact du variant :} Ce filtre consistant à se
    baser à la fois sur les prédiction VEP mais aussi, dans le cas de
    variants faux-sens, sur les prédiction SIFT et PolyPhen permet de ne
    conserver que les variant ayant les effet les plus délétères. Ce
    filtre est, de prime abord le plus efficace puisqu'il permet de
    filtrer environs \ldots{} \% (médiane) des variants de chaque
    individu.
  \item
    \textbf{Filtre 4 : Les transcrits ``non pertinents'' :} Cette étape de
    filtre permet de filtrer systématiquement entre 13712 et 17407
    transcrits différents par patients. Cependant, un même variant pouvant
    impacter à la fois des transcrits ``non pertinents'' \textbf{et} des
    transcrits ``pertinents'', seuls ceux impactant \textbf{uniquement}
    des transcrits ``non pertinents'' sont filtrés, soit une moyenne de
    1870 variants par individus (\textbf{Figure :
    }\ref{fig:plotfilternonpertinanttr}).
  \end{enumerate}
  
  \begin{figure}
  
  {\centering \includegraphics{thesis_files/figure-latex/plotfilternonpertinanttr-1} 
  
  }
  
  \caption[Filtrage des transcrits jugés "non pertinents" et des variants les chevauchant]{Filtrage des transcrits jugés "non pertinents" et des variants les chevauchant : Pour chaque patient nous avons filtrer les transcrits jugés "non pertinents" pour l'analyse, c'est à dire ceux ne codant pas pour une protéine et ceux annoté NMD. Dès lors, l'intégralité des variants chevauchant uniquement des transcrits non pertinents ont pu systématiquement être filtrés (boites rouges). Les autres furent conservés (boites vertes)}\label{fig:plotfilternonpertinanttr}
  \end{figure}
  
  \begin{enumerate}
  \def\labelenumi{\arabic{enumi}.}
  \setcounter{enumi}{4}
  \item
    \textbf{Fréquence des variants :} Filtrer systématiquement les
    variants retrouvés avec une fréquence \(\ge\) 0.01 dans l'une des
    trois bases de données que sont ExAC, 1KG et ESP6500 permet de filtrer
    entre \ldots{} et \ldots{} variants par patients.
  \item
    \textbf{Présence des variants dans la cohorte contrôle :} Au cours de
    nos différentes études, nous avons été amenés à séquencé un total de
    \texttt{n\_tot\_runs} individus présentant un des
    \texttt{n\_pheno\_tot} phénotypes que nous avons étudiés
    (\textbf{Table : }\ref{tab:TODO}). Ces phénotypes étant très
    différent, on peut émettre l'hypothèse que leurs causes génétiques
    soient également différentes. De même, les variants recherchés étant
    rares, il est peu probable qu'un individu porte les variants de deux
    phénotypes différents. Ainsi, pour chacune des 3 familles, nous avons
    pu constituer une cohorte contrôle composée dans l'ensemble des
    patients précédemment analysés et ne présentant pas le même phénotype
    que celui étudié dans la famille (\textbf{Figure :}
    \ref{fig:plotsamplectrl}). Dès lors, nous avons pu filtrer l'ensemble
    des variants retrouvés à la fois chez nos patients et observés à
    l'état homozygote dans la cohorte contrôle. Cette cohorte contrôle
    présente ainsi le même rôle que les bases de données publiques. Sont
    intérêt principale par rapport à celles-ci est que les individus qui
    la composent ont pour la plupart la même origine ethnico-géographique
    que nos patients. De plus ceux-ci ont été séquencés en même temps dans
    les mêmes centres permettant ainsi d'identifier les artefacts dus aux
    protocoles de séquençage.
  \end{enumerate}
  
  \begin{figure}
  
  {\centering \includegraphics{thesis_files/figure-latex/plotsamplectrl-1} 
  
  }
  
  \caption[Nombre d'individus composant la cohorte contrôle de chaque famille]{Nombre d'individus composant la cohorte contrôle de chaque famille : Ici, chaque barre représente une famille et sa hauteur est déterminée par le nombre d'individus composant la cohorte contrôle à laquelle elle a été confronté. Chaque individu de la cohorte contrôle a été séquencés en WES par notre équipe. Afin d'être considéré comme "contrôle" et intégrer cette cohorte, un individu doit être sain ou présenter un phénotype d'infertilité différent de la famille étudiée. Par exemple, un individus MMAF pourra servir de contrôle aux familles AZ et FF mais pas aux familles MMAF1-4}\label{fig:plotsamplectrl}
  \end{figure}
  
  \newpage
  
  \newpage
  
  Comme on pouvait s'y attendre, ces six filtres ont un pouvoir
  discriminant extrêmement différent. En effet, tandis que le filtre
  ``Transcript relevance'' (filtre n°4) éliminer en moyenne 3.9 \% des
  variants de chaque individu, le filtre ``Variant impact'' (filtre n° 3)
  élimine jusqu'à 90.1 \% de ces mêmes variants. Cette différence n'est
  pas surprenante. En effet, comme nous l'avions vu plus tôt, les variants
  de la catégorie VEP \emph{MODIFIER} qui regroupe entre autres les
  variants chevauchant les séquences UTRs et introniques (\textbf{Table :}
  \ref{tab:tabvepcsq}) représentent en moyenne 87\% des variants de nos
  patients. Ceux-ci étant tous filtrés, on s'attendait donc à une valeur
  aussi élevée. On peut également constater l'importance de la cohorte
  contrôle qui, je le rappelle, permet de filtrer l'ensemble des variants
  homozygotes observés en son sein, puisque ce filtre permet retirer entre
  76.5 et 88.4\% des variants de chaque individus (\textbf{Figure :}
  \ref{fig:plotcomparefilter} - \textbf{A}).
  
  Cependant, regarder uniquement le pourcentage de variants filtrés par
  chaque filtre révèle une information partielle. En effet, dans ce cas de
  figure, on observe la quantité de variant éliminé par chaque filtre
  indépendamment les uns des autres. Ainsi, un même variant peut donc être
  filtré par plusieurs filtres. Dès lors, il faut également analyser la
  quantité de variants filtrés \textbf{spécifiquement} par chaque filtre.
  Ainsi, on peut constater que le classement des filtres en fonctions de
  leur stringence reste quasiment identique. Il est tout de même
  intéressant de noter que désormais le filtre ``Variant impact'' apparait
  moins efficace que les filtres ``Ctrl'' et ``Genotype'' en filtrant
  spécifiquement une moyenne de 253 variants par individu contre 423 pour
  le filtre génotype et 882 pour le filtre ``Ctrl''. Ainsi, ce dernier
  devient celui filtrant spécifiquement le plus de variants avec entre 364
  et 1060 variants spécifiquement filtrés par patients confirmant ainsi
  l'importance de ce filtre dans nos analyses. Aussi, les filtres
  ``Transcript relevance'', ``Union'' et ``Frequency'' apparaissent
  désormais comme étant anecdotiques en comparaison aux trois autres
  filtres puisqu'ils filtrent au maximum 43 variants spécifiques
  (\textbf{Figure :} \ref{fig:plotcomparefilter} - \textbf{B}).
  
  \newpage
  
  \begin{figure}
  
  {\centering \includegraphics{thesis_files/figure-latex/plotcomparefilter-1} 
  
  }
  
  \caption[Comparaison de l'efficacité de chacun des six filtres utilisés]{Comparaison de l'efficacité de chacun des six filtres utilisés : **A** : Comparaison du pourcentage de variants filtrés par chacun des six filtres indépendamment les uns des autres pour chaque patient (représenté par les points. Dès lors, un même variant peut-être filtré par plusieurs filtres. **B** : Comparaison du nombre de variant filtrés spécifiquement par chacun des filtres. Ici, un variant ne peut-être filtré que par un seul filtre}\label{fig:plotcomparefilter}
  \end{figure}
  
  Après avoir appliqué l'ensemble de ces filtres, seuls quelques variants
  subsistent nous permettant d'obtenir une liste de gènes restreinte pour
  chaque famille et ainsi de tirer des conclusions quant au variant
  responsable du phénotype de chacune d'entre elles. Ces travaux ont ainsi
  pu mener à l'écriture de trois articles dont je suis co-auteur.
  
  \newpage
  
  \newpage  
  
  \begin{enumerate}
  \def\labelenumi{\arabic{enumi}.}
  \setcounter{enumi}{1}
  \tightlist
  \item
    \textbf{Famille FF} : Pour cette famille, seul le gène
    \emph{PLC}\(\zeta 1\) a passé l'ensemble des filtres. Nos
    connaissances sur la fonction de se gène et notamment son rôle dans
    l'activation ovocytaire {[}\protect\hyperlink{ref-Amdani2013}{81}{]}
    ainsi que sa forte expression testiculaire ont fait de ce gène le
    candidat idéal pour expliquer le phénotype d'échec de fécondation de
    ces deux frères (\textbf{Figure : }\ref{fig:plotexpplcz1}).
  \end{enumerate}
  
  \begin{figure}
  
  {\centering \includegraphics{thesis_files/figure-latex/plotexpplcz1-1} 
  
  }
  
  \caption[Expression tissulaire du gène *PLCZ1*]{Expression tissulaire du gène *PLCZ1* : D'après les données du Illumina BodyMap}\label{fig:plotexpplcz1}
  \end{figure}
  
  \begin{enumerate}
  \def\labelenumi{\arabic{enumi}.}
  \setcounter{enumi}{3}
  \tightlist
  \item
    \textbf{Famille MMAF3} : À l'issue des filtres,
    \texttt{n\_gene\_mmaf3} gènes ressortaient chez ces deux frères :
    \emph{MYH11} et \emph{DNAH1}. Or, notre équipe ayant déjà établit le
    lien entre des mutations du gène \emph{DNAH1} et le syndrome MMAF
    {[}\protect\hyperlink{ref-BenKhelifa2014}{77}{]} ce gène s'est révélé
    être un candidat idéal pour expliquer le phénotype de ces 2 frères. De
    plus, l'implication de \emph{MYH11} dans le phénotype de dissection
    aortique {[}\protect\hyperlink{ref-Imai2015}{163}{]} l'ont écarté des
    candidats pour le phénotype MMAF.
  \end{enumerate}
  
  \newpage
  
  \subsection{Article n° 3}\label{article-n-3}
  
  \textbf{SPINK2 deficiency causes infertility by inducing sperm defects
  in heterozygotes and azoospermia in homozygotes}
  
  Kherraf ZE\textsuperscript{*}, Christou-Kent M\textsuperscript{*},
  \textbf{Karaouzène T}, Amiri-Yekta A, Martinez G, Vargas AS, Lambert E,
  Borel C, Dorphin B, Aknin-Seifer I, Mitchell MJ, Metzler-Guillemain C,
  Escoffier J, Nef S, Grepillat M, Thierry-Mieg N, Satre V, Bailly M,
  Boitrelle F, Pernet-Gallay K, Hennebicq S, Fauré J, Bottari SP, Coutton
  C, Ray PF, Arnoult C
  
  \textsuperscript{*} Co-premiers auteurs
  
  EMBO Molecular Medicine, Mai 2017
  
  \newpage
  
  \subsubsection{Contexte et objectifs}\label{contexte-et-objectifs}
  
  L'oligospermie, comme l'azoospermie sont des phénotypes d'infertilités
  masculines liées à la quantité de spermatozoïdes présent dans
  l'éjaculât. Les différentes études publiées ces dernières années
  montrent que les microdélétions du chromosome Y sont retrouvées chez
  10\% des hommes avec une azoospermie non-obstructives et chez 5\% des
  patients avec une oligozoospermie sévères
  {[}\protect\hyperlink{ref-Hotaling2014}{49}{]}. Ces taux bien qu'élevé
  ne représente qu'une infime partie des cas d'azoospermie et
  d'oligospermie suggérant l'implication de nombreux autres gènes dans ce
  phénotype.
  
  Entre 2005 et 2014 deux frères issus d'un union consanguin ont demandé
  des conseils médicaux auprès de différentes cliniques d'infertilité
  après deux ans de tentatives infructueuses de consevoir un enfant. Ces
  deux frères étant marriés à des femmes non-apparentées la piste de
  l'implication d'une cause féminine fut exclu et les recherches
  concentrées sur l'analyse des deux frères. Après analyse de leur
  éjaculât, (et de l'épydidime) \ldots{} tout deux présentèrent de sévères
  défauts de production de spermatozoïdes. Au vu de la similarité du
  phénotype et du lien de parenté les liant, l'hypothèse d'une cause
  génétique commune fut émmise. L'analyse de leur karyotype et du locus
  AZF du chromosome Y ne révélant aucune anomalie, la procédure d'un
  séquençage WES fut décidé.
  
  Dans ce contexte, l'objectif de mon travail sur l'analyse phénotype de
  ces deux frères a été d'effectuer l'ensemble des analyses des données
  WES obtenues après leur séquençage afin de mettre en évidence une
  mutation homozygote commune aux deux frères pouvant expliquer leur
  phénotype. Dans un second temps, j'ai pu mettre en place le protocole de
  génotypage des souris au locus du gène \emph{Spink2} permettant
  d'identifier les souris sauvages \emph{Spink2}\textsuperscript{+/+} des
  souris KO \emph{Spink2}\textsuperscript{-/-}. Pour finir, afin d'estimer
  l'importance des variants du gène \emph{SPINK2} comme cause
  d'infertilité masculine chez l'humain, j'ai également contribué au
  séquençage Sanger de la séquence codante de \emph{SPINK2} d'une partie
  des 611 patients séquencés dans cette étude.
  
  \newpage
  
  \includepdf[pages=-]{bib/SPINK2_2017.pdf}
  
  \newpage
  
  \subsubsection{Principaux résultats}\label{principaux-resultats}
  
  Après avoir analyser les données de séquençage des deux frères au sein
  de notre pipeline décrite précédement, seul 2 variants passèrent
  l'ensemble des filtres. Ces deux variants impactant respectivement les
  gènes \emph{GUF1} et \emph{SPINK2}. Parmis ces deux gènes, seul
  \emph{SPINK2} présentaient une forte expression testiculaire dans les
  données Ensembl (\textbf{Figure : }\ref{fig:plotexpfamaz}) que nous
  avons pu confirmer par RT-PCR dans cette étude. De plus, des mutations
  du gène \emph{Spink2} chez la souris avait déjà été identifiée comme
  induisant des défauts de la spermatogenèse
  {[}\protect\hyperlink{ref-Lee2011}{164}{]}. Ces arguments ont ainsi fait
  de \emph{SPINK2} le candidat évident pour expliquer le phénotype de ces
  deux frères. Après avoir confirmé en séquençage Sanger la mutation de ce
  gène à l'état homozygote pour les deux frères et hétérozygotes pour les
  parents, nous avons, afin de continuer nos investigations, développés un
  modèle murin KO \emph{Spink2}\textsuperscript{-/-} confirmant une
  azoospermie complète pour les sours mâle spermiogenesis causé par un
  arret de la spermatogénèse au stage des spermatides rondes. De plus,
  malgré une fertilité normale, nous avons pu noter un taux élevé
  d'anomalies morphologiques du spermatozoïde ainsi qu'une motilité
  spermatique réduite chez les souris mâles hétérozygotes
  \emph{Spink2}\textsuperscript{+/-}. Les femelles, elles ne présentaient
  aucun phénotype apparent. L'étude de la localisation de la protéine
  Spink2 chez la souris et SPINK2 chez l'humain a révélés que ces deux
  protéines localisaient dans la vésicule acrosomale depuis le début de la
  biogénèse de l'acrosome jusqu'au spermatozoïde mature.
  
  Suite à cela, afin d'évaluer l'importance des variants du gène
  \emph{SPINK2} dans l'infertilité humaine, nous avons effectués le
  séquençage Sanger de 611 patient parmi lesquels 210 étaient azoospermes,
  393 oligozoospermes et 8 dont la cause n'étaient pas spécifiée. Parmi
  cet ensemble de patient, seul 1 (le patient p105) s'est révélé porter un
  variant non répertorié dans ExAC sur le gène \emph{SPINK2}. Ce patient
  présentant un phénotype d'oligozoospermie porte à l'état hétérozygote un
  variant altérant le codon start du gène \emph{SPINK2}. Ces résultats
  laissent donce supposer que chez l'homme, la présence de mutations
  homozygotes sur le gène \emph{SPINK2} induit un phénotype d'azoospermie
  tandis que les mutation hétérozygotes entrainent, elles, un phénotype
  d'oligozoospermie. Cette forte séléction négative pouvant expliquer la
  rareté des mutations observées sur ce gènes.
  
  \begin{longtable}[t]{llllll}
  \caption{\label{tab:tabrecapaz}Liste des variants ayant passé l'ensemble des filtres pour les deux fères de la famille AZ}\\
  \toprule
  \multicolumn{1}{c}{ } & \multicolumn{2}{c}{Variant impact} & \multicolumn{3}{c}{Variant frequency} \\
  \cmidrule(l{2pt}r{2pt}){2-3} \cmidrule(l{2pt}r{2pt}){4-6}
  Gene & HGVSc, HGVSp & Consequence & ExAC & ESP & 1KG\\
  \midrule
  GUF1 & c.443A>T ; p.Ser148Ile & missense & 0.00207 & 0.0028 & 9e-04\\
  SPINK2 & c.56-3C>G ; . & splice region & . & . & .\\
  \bottomrule
  \end{longtable}
  
  \newpage
  
  \begin{figure}
  
  {\centering \includegraphics{thesis_files/figure-latex/plotexpfamaz-1} 
  
  }
  
  \caption[Expression tissulaire des gènes *SPINK2* et *GUF1*]{Expression tissulaire des gènes *SPINK2* et *GUF1* : Données provenant du projet de transcriptome Illumina bodyMap}\label{fig:plotexpfamaz}
  \end{figure}
  
  \newpage
  
  \subsection{Article n° 4}\label{article-n-4}
  
  \textbf{Homozygous mutation of PLCZ1 leads to defective human oocyte
  activation and infertility that is not rescued by the WW-binding protein
  PAWP}
  
  Jessica Escoffier J\textsuperscript{*}, Lee HC\textsuperscript{*},
  Yassine S\textsuperscript{*}, Zouari R, Martinez G, \textbf{Karaouzène
  T}, Coutton C, Kherraf ZE, Halouani L, Triki C, Nef S, Thierry-Mieg N,
  Savinov SN, Fissore R, Ray PF, Arnoult C
  
  \textsuperscript{*} Co-premiers auteurs
  
  Human Molecular Genetics, Décembre 2015
  
  \newpage
  
  \subsubsection{Contexte et objectifs}\label{contexte-et-objectifs-1}
  
  L'activation ovocitaire regroupe une série de processus intervenant au
  cour de la fécondation d'un ovocyte par un spermatozoïde. en 1990,
  plusieurs études démontrèrent que chez les mamifères ces processus
  reposent principalement sur le relargage par le spermatozoïde de
  ``facteurs spermatiques'' qui déclenchent un signal de calcium,
  constitué d'oscillations Ca\textsuperscript{2+} {[}ref{]}. Plus tard, la
  protéine PLC\(\zeta\) fu identifiée comme la molécule responsable de ces
  oscilations calciques. Cependant, en raison de l'incapacité àproduire
  des modèles animaux \emph{PLC}\(\zeta\) KO capable de produire des
  spermatozoïdes mature a empeché d'attribuer l'exclusivité de ce rôle à
  \emph{PLC}\(\zeta\) laissant ouverte la possibilité de la nécessité
  d'autres facteurs spermatiques. C'est ainsi qu'en 2014 fut proposé la
  protéine PAWP comme facteur spermatique alternatif ou complémentaire de
  PLC\(\zeta\) {[}\protect\hyperlink{ref-Aarabi2014}{165},
  \protect\hyperlink{ref-Aarabi2014a}{166}{]}.
  
  Les travaux ci-dessous décrivent les analyses effectuées sur deux frères
  issus d'un union consanguin ayant tout deux été dans l'incapacité de
  concevoir un enfant par voies naturelles et pour qui, malgré des
  paramètres spermatiques normaux, l'ensemble des procédures de
  reproduction assités effectuées se sont soldés par un échec d'activation
  ovocitaire.
  
  Comme dans l'étude précédente, en raison de l'historique de
  cansanguinité de la famille des deux frères ainsi que le fait que leur
  femmes respectives soient non apparentées nous a permi d'exclure
  l'hypothèse d'une cause féminine et nous a conduit à rechercher un
  variant homozygote commun aux deux frères. Nous avons ainsi effectué un
  séquençage WES de ces deux frères. Comme précédemment, dans cette étude,
  j'ai été en charge de l'ensemble des analyses des données issus du
  séquençage des deux frères.
  
  \newpage
  
  \includepdf[pages=-]{bib/PLCZ1_2016}
  
  \newpage
  
  \subsubsection{Principaux résultats}\label{principaux-resultats-1}
  
  Suite à l'analyse bioinformatique de ces deux frères, un seul variant
  subsistait après l'application de l'ensemble des filtres. Celui-ci était
  recancé uniquement dans la base de donnée ExAC avec une fréquence de
  8.24e-06 et entrait un faux-sens prédit comme \emph{deleterious} par
  SIFT et \emph{possibly damaging} par PolyPhen sur la séquence du gène
  \emph{PLC}\(\zeta 1\). La forte expression testiculaire de ce gène
  (\textbf{Figure : }\ref{fig:plotexpfamff}) couplée à l'implication déjà
  connu de celui-ci dans l'activationn ovocitaire, ont fait de ce variant
  le candidat évident pour expliquer le phénotype de ces deux frères. De
  plus, aucun variant n'a été retrouvé sur la séquence du gène
  \emph{WBP2NL} codant pour la protéine PAWP bien que l'intégralité de la
  séquence codante de \emph{WBP2NL} ait une couverture \(\ge\) 40x (les
  zones moins couvertes du début de l'exon 1 et de la fin de l'exon 6
  correspondant aux régions UTR) (\textbf{Figure :
  }\ref{fig:plotcovplcz}). Ces résultats suggérant une parfaite
  fonctionnalité de la protéine PAWP ont pu être confirmée par
  \emph{Western blot}, de même, la bonne localisation de la protéine PAWP
  a pu être observée chez les deux patients par Immunofluorescence.
  
  Cette étude présente le premier cas de \emph{knock-down} de PLC\(\zeta\)
  n'entrainant pas d'effets sur la spermatogénèse démontrant ainsi le rôle
  primordial de cette protéine dans l'activation ovocitaire. De plus, la
  parfaite localisation et fonctionnalité de la protéine PAWP la disculpe
  de toute implication dans le phénotype de nos patients.
  
  \begin{longtable}[t]{llllll}
  \caption{\label{tab:tabrecapff}Liste des variants ayant passé l'ensemble des filtres pour les deux fères de la famille FF}\\
  \toprule
  \multicolumn{1}{c}{ } & \multicolumn{4}{c}{Variant impact} & \multicolumn{1}{c}{Variant frequency} \\
  \cmidrule(l{2pt}r{2pt}){2-5} \cmidrule(l{2pt}r{2pt}){6-6}
  Gene & HGVSc, HGVSp & Consequence & SIFT & PolyPhen & ExAC\\
  \midrule
  PLCZ1 & c.1465G>T ; p.Ile489Phe & missense & deleterious & possib damaging & 8.24e-06\\
  \bottomrule
  \end{longtable}
  
  \newpage
  
  \begin{figure}
  
  {\centering \includegraphics{thesis_files/figure-latex/plotexpfamff-1} 
  
  }
  
  \caption[Expression tissulaire du gène PLCZ1*]{Expression tissulaire du gène PLCZ1* : Données provenant du projet de transcriptome Illumina bodyMap}\label{fig:plotexpfamff}
  \end{figure}
  
  \begin{figure}
  
  {\centering \includegraphics[scale=.4]{figure/pawp_coverage} 
  
  }
  
  \caption[Couverture des 6 éxons de *WBP2NL* pour les deux frères de la famille FF]{Couverture des 6 éxons de *WBP2NL* pour les deux frères de la famille FF}\label{fig:plotcovplcz}
  \end{figure}
  
  \newpage
  
  \subsection{Article n° 5}\label{article-n-5}
  
  \subsubsection{Whole-exome sequencing of familial cases of multiple
  morphological abnormalities of the sperm flagella (MMAF) reveals new
  DNAH1
  mutations}\label{whole-exome-sequencing-of-familial-cases-of-multiple-morphological-abnormalities-of-the-sperm-flagella-mmaf-reveals-new-dnah1-mutations}
  
  Amiri-Yekta A\textsuperscript{*}, Coutton C\textsuperscript{*}, Kherraf
  ZE, \textbf{Karaouzène T}, Le Tanno P, Sanati MH, Sabbaghian M, Almadani
  N, Sadighi Gilani MA, Seyedeh Hanieh Hosseini, Bahrami S, Daneshipour A,
  Bini M, Arnoult C, Colombo R, Gourabi H, Ray PF
  
  \textsuperscript{*} Co-premiers auteurs
  
  Human Reproduction, Octobre 2016
  
  \newpage
  
  \subsubsection{Contexte et objectifs}\label{contexte-et-objectifs-2}
  
  Dans une étude précédente non détaillée dans ce manuscrit , notre équipe
  a pu identifier le gène \emph{DNAH1} comme le premier gène codant pour
  une dynéine axonèmale responsable uniquement d'infertilité masculine.
  Dans cette première étude, \ldots{} de nos \ldots{} patients (non
  apparentés), soit environs \ldots{} d'entre eux, étaient porteur d'une
  mutation homozygote sur le gène \emph{DNAH1} responsable de leur
  phénotype MMAF. Ces résultats ont ainsi démontrés l'importance de
  l'implication de ce gène dans ce phénotype.
  
  Dans cette nouvelle étude, nous nous concentrons sur l'analyses des
  données génétiques de 5 familles iraniène et d'une famille italienne
  pour un total de 12 individus. Parmis eux 10 sont nés d'un union
  consanguain. Pour des raisons techniques et materiels, seuls deux de ces
  individus ont, pour l'instant, été analysés sur notre pipeline décrite
  précédement (famille MMAF). Pour les autres les données familles
  séquencés en haut débit, les variants utilisés furent ceux fournit par
  les centres de séquençage auxquels nous avons ensuite appliqué les même
  critères de filtre.
  
  Dans ce contexte j'ai effectué les l'ensemble des analyses des deux
  patients patients de la familles MMAF décrites ci-dessus, mais également
  l'ensemble des analyses de filtrage appliqué aux données des 8 autres
  patients séquencés en WES.
  
  \newpage
  
  \includepdf[pages=-]{bib/Fam_DNAH1_2016.pdf}
  
  \newpage
  
  \subsubsection{Principaux résultats}\label{principaux-resultats-2}
  
  Dans cette étude nous avons séquencé un total de 12 patients provenant
  de 6 familles différentes. Tout ces patients souffrent d'un syndrome
  MMAF induisant une infertilité. Parmi ceux-ci 10 ont été séquençé en
  WES, cependant, seulement \ldots{} d'entre eux ont été analysé au sein
  de notre pipeline d'analyse. Après l'aplication de l'ensemble des
  filtres seul \ldots{} variants subsistaient impactant respectivement les
  gènes \ldots{} .
  
  \begin{longtable}[t]{llllll}
  \caption{\label{tab:tabrecapmmaf}Liste des variants ayant passé l'ensemble des filtres pour les deux fères P5 et P6 de la famille MMAF3}\\
  \toprule
  \multicolumn{1}{c}{ } & \multicolumn{2}{c}{Variant impact} & \multicolumn{3}{c}{Variant frequency} \\
  \cmidrule(l{2pt}r{2pt}){2-3} \cmidrule(l{2pt}r{2pt}){4-6}
  Gene & HGVSc, HGVSp & Consequence & ExAC & ESP & 1KG\\
  \midrule
  MYH11 & c.4625G>A ; p.Arg1542Gln & missense & 0.00234 & 0.0016 & 5e-04\\
  DNAH1 & c.8626-1G > A ; . & splice acceptor & . & . & .\\
  \bottomrule
  \end{longtable}
  
  \newpage
  
  \begin{figure}
  
  {\centering \includegraphics{thesis_files/figure-latex/plotexpfammmaf-1} 
  
  }
  
  \caption[Expression tissulaire des gènes DNAH1 et MYH11]{Expression tissulaire des gènes DNAH1 et MYH11 : Données provenant du projet de transcriptome Illumina bodyMap}\label{fig:plotexpfammmaf}
  \end{figure}
  
  \newpage
  
  \section{Résultats 2 : Étude d'une cohorte de femmes
  infertiles}\label{resultats-2-etude-dune-cohorte-de-femmes-infertiles}
  
  \subsection{Article n° 6}\label{article-n-6}
  
  \textbf{PATL2 Gene Mutation Causes Oocyte Meiotic Deficiency and Female
  Infertility}
  
  Christou-Kent M, Amiri-Yekta A, Kherraf ZE, \textbf{Karaouzène T},
  Escoffier J, Guttin A, Martinez G, Le Blévec E, Lambert E, Fourati Ben
  Mustapha S, Cedrin-Durnerin I, Halouani L, Marrakchi O, Makni M, Latrous
  H, Kharouf M, Bottari S, Thierry-Mieg N, Coutton C, Zouari R, Issartel
  JP, Ray PF, Arnoult C
  
  New England Journal of Medicine, 07 Juillet 2017 (soummis)
  
  \newpage
  
  \subsubsection{Contexte et objectifs}\label{contexte-et-objectifs-3}
  
  Entre 2013 et 2014 notre équipe a pris en charge le diagnostique de 23
  femmes nord africaine présentant toutes une déficience méiotique
  ovocytaire (DMO) caractérisé par un blocquage de la méïose au stade M1
  et unduisant une infertilité. À l'heure actuelle, seul le gène
  \emph{TUBB8} retrouvé muté à l'état hétérozygote chez des patientes
  chinoise avait pu être lié à ce phénotype. Cette étude a donc pour
  objectif de caractériser la cause génétique responsable du phénotype DMO
  de ces 23 femmes. Parmi cells-ci, 15 ont été analysées par séquençage
  haut-débit. Dans ce contexte, j'ai, au cours de ma thèse, été en charge
  de l'ensemble des analyses bio-informatiques de ces 15 femmes.
  
  \newpage
  
  \includepdf[pages=-]{bib/PATL2_2017.pdf}
  
  \newpage
  
  \subsubsection{Principaux résultats}\label{principaux-resultats-3}
  
  L'application de notre pipeline d'analyse sur les données de ces femmes
  nous a permis d'obtenir une liste de 316 variants impactant 299 gènes
  différents. Parmi ces variants, aucin n'impactait le gène \emph{TUBB8}.
  Afin de restreindre à nouveau la liste de gènes, nous nous sommes
  concentré sur ceux retrouvés mutés à l'état homozygote chez au moins 2
  femmes. Seul 3 gènes ont passés ce nouveau critère : \emph{FAM58A},
  \emph{MGAM} et \emph{PATL2}. Aucune investigation n'a pour l'instant été
  effectuée sur les 296 autres. En raison de la fréquence élevée du
  variant retrouvé sur \emph{FAM58A} et de l'impact peu délétère du
  variant chevauchant \emph{MGAM}, ces deux gènes ont été considéré comme
  de mauvais candidats.
  
  Ainsi, nous nous sommes dans un premier temps concentré sur la
  caractérisation de \emph{PATL2} dont l'orthologue \emph{xpat1a} chez le
  xenope a été décrit comme étant exprimé au cours du dévelopement de
  l'ovocyte {[}\protect\hyperlink{ref-Marnef2010}{167},
  \protect\hyperlink{ref-Nakamura2010}{168}{]} faisant de ce gène un
  excellent candidat. 5 de nos patients, soit 33.3\% d'entre elles,
  révélèrent porter une même mutation : c.478G\textgreater{}T, p.Arg160Ter
  induisant un codon stop prématuré dans la séquence codante du transcrit
  canonique \emph{PATL2}: ENST00000434130. Au vu de ces résultats un
  séquençage Sanger de la séquence codante de ce gène fut réalisé pour ces
  5 femmes afin de confirmer cette mutation, de même que sur 8 femmes
  supplémentaires souffrant du même phénotypes. Parmi ces dernières, une
  s'est révélée porter la même mutation à l'état homozygote.
  
  Dans un second temps, l'étude du modèle murin KO
  \emph{Patl2}\textsuperscript{-/-} nous a permi de mettre en évidence une
  subfertilité importante chez les souris femelles tandis qu'aucun
  phénotype n'était observable chez les mâles.
  
  Pour finir, \emph{xpat1a}, l'orthologue de \emph{PATL2} chez le xénope,
  ayant été décrit comme réprimant la traduction de l'ARNm dans l'ococyte
  nous avons cherché à savoir si les souris femelles KO
  \emph{Patl2}\textsuperscript{-/-} présentait des dérégulation de leur
  transcriptome ovocytaire. Pour cela, nous avons procédés à une étude
  comparative des transcriptome ovocytaires au stades GC et MII murins sur
  puces Affymetrix mesurant les valeurs d'expression d'environ 66,000
  transcrits différents. Ainsi, nous avons pu mettre en évidence 134
  transcrits différentiellement exprimés au stade GV parmi lesquels 95
  étaient sous-exprimés tandis que 39 étaient sur-exprimés. Au stade MII,
  ces dérégulation se révélèrent être plus impréssionnantes puisque 124
  étaient sous-exprimés et 122 sur-exprimés démontrant ainsi un forte
  implication de \emph{Patl2} dans la transcription ovocytaire des gènes
  murins.
  
  \newpage  
  
  \section{Résultats 3 : Étude d'une large cohorte de patients
  MMAF}\label{resultats-3-etude-dune-large-cohorte-de-patients-mmaf}
  
  \subsection{Article n° 7}\label{article-n-7}
  
  \textbf{Whole exome cohort study and analysis of mouse and Trypanosoma
  models demonstrate the importance of WDR proteins in flagellogenesis and
  male fertility}
  
  Coutton C, Vargas A, Amiri-Yekta A, Kherraf ZE, Fourati Ben Mustapha S,
  Le Tanno P, Wambergue-Legrand C, \textbf{Karaouzène T}, Martinez G,
  Daneshipour A, Hanieh Hosseini S, Mitchell V, Halouani L, Marrakchi O,
  Makni M, Latrous H, Kharouf M, Deleuze JF, Boland A, Hennebicq S, Satre
  V, Jouk PS, Bottari SP, Thierry-Mieg N, Conne B, Dacheux-Deschamps D,
  Schmitt A, Stouvenel L, Lorès P, El Khouri E, Fauré J, Wolf JP,
  Escoffier J, Gourabi H, Robinson DR, Nef S, Dulioust E, Zouari R,
  Bonhivers M, Touré A, Arnoult C, Ray PF
  
  EMBO Molecular Medicine, Mai 2017
  
  \newpage
  
  \subsubsection{Contexte et objectifs}\label{contexte-et-objectifs-4}
  
  Après avoir mis en évidence l'implication du gène \emph{DNAH1} dans le
  phénotype MMAF notre équipe s'est en partie spécialisé dans la
  caractérisation ce syndrome. Ainsi, entre 2012 et 2016, notre équipe a
  effectué le séquençage de 90 individus présentant tous ce phénotype afin
  d'en établir la cause génétique. Ces séquençages ont été effectué dans 5
  centres différents que sont Genoscope, Integragen, MountSinai, Novogene
  et Strasbourg. La plupart de ces séquençages ont été effectués sur des
  Illumina HiSeq2000 exceptés les \ldots{} plus récent qui ont été
  effectués sur un Illumina HiSeq4000 (\textbf{Table :
  }\ref{tab:tabrunbigmmaf}).
  
  \begin{longtable}[t]{lrlr}
  \caption{\label{tab:tabrunbigmmaf}Liste des différents individus présentant un phénotype MMAF séquencés en WES}\\
  \toprule
  Place & Year & Platform & Nb of individuals\\
  \midrule
  MountSinai & 2012 & Illumina Hiseq2000 & 2\\
  Strasbourg & 2012 & Illumina Hiseq2000 & 13\\
  Genoscope & 2013 & Illumina Hiseq2000? & 13\\
  Genoscope & 2014 & Illumina Hiseq2000 & 28\\
  Genoscope & 2015 & Illumina Hiseq2000 & 6\\
  \addlinespace
  Integragen & 2016 & Illumina HiSeq4000 & 18\\
  Novogene & 2016 & Illumina HiSeq4000 & 10\\
  \bottomrule
  \end{longtable}
  
  \includepdf[pages=-]{bib/CFAP_2017.pdf}
  
  \newpage
  
  \subsubsection{Principaux résultats}\label{principaux-resultats-4}
  
  L'application de de notre pipeline d'analyse sur les données de ces 90
  patients nous a permi d'obtenir une liste de 3955 variant distincts
  ayant passés l'ensemble de nos filtres (3178 SNPs et 777 indels),
  ceux-ci impactant un total de 2987 gènes différents.
  
  Le gène \emph{DNAH1} étant le candidat évident, nous avons chercher en
  priorité l'ensemble des variants retrouvés sur ce gène. Ainsi, nous
  avons obtenus une liste de 20 patients portant tous soit au moins
  variant homozygote sur le gène \emph{DNAH1} soit deux variants
  hétérozygotes sur ce même gène (\textbf{Table : }\ref{tab:tabdnah1}).
  
  \textbf{décrire les variants DNAH1}
  
  Au vu du nombre important de gènes restant et afin d'étudier en priorité
  ceux pouvant expliquer le phénotype d'un maximum de patients nous avons,
  comme précédemment, limité nos recherches aux gènes sur lesquels
  \textbf{au moins 3 patients portaient un variant tronquant} réduisant à
  nouveaux cette liste à 11 gènes différents : \emph{CFAP43},
  \emph{CFAP44}, \emph{CRIPAK}, \emph{FAM58A}, \emph{NOS2P1},
  \emph{PTCHD3}, \emph{RBMX2}, \emph{RETNLB}, \emph{TBP}, \emph{TRAV26-1}
  et \emph{TTC29} (\textbf{Figure : }\ref{fig:plottwohomohigh}).
  
  Cela nous a ainsi permi de mettre en évidence les gènes \emph{CFAP43} et
  \emph{CFAP44} sur lesquels des variants homozygotes tronquants ont été
  retrouvés chez respectivement 8 et 6 patiens (\textbf{Tables :
  }\ref{tab:tabcfap43} et \ref{tab:tabcfap44}). Ces deux gènes CFAP (pour
  \emph{Cilia and Flagella Associated Protein}) avaient déjà été
  répertorié dans les bases de données publiques comme ayant une forte
  expression testiculaire, et comme étant probablement impliqués dans la
  structure et / ou fonction du flagelle spermatique
  {[}\protect\hyperlink{ref-Ivliev2012}{169}{]}. De plus, ces deux gènes
  codent tous deux pour des protéines appartenant à la famille des WDR et
  comportent tous deux neuf répétitions WD
  {[}\protect\hyperlink{ref-Smith2008}{170}{]}. Ainsi, en tenant compte du
  nombre important de patients portantdes variants sur un de ces deux
  gènes et le fait qu'ils codent tous deux pour des protéines appartenant
  à la même famille, nous avons décidé de nous concentrer dans un premiers
  temps à la caractérisation de ces deux seuls gènes, ceux-ci étant les
  meilleurs candidats pour expliquer le phénotype d'infertilité de 17 de
  nos patients.
  
  Ainsi, un total de 9 de nos patients révélèrent porter des variants
  homozygotes sur le gène \emph{CFAP43}. En méttant de cotés les 8 portant
  un variant ayant un effet tronquant évident, les variants portés par les
  1 autres patients révélèrent eux aussi avoir un effet probablement
  délétère. Le premier d'entre eux est un variants intronique non listé
  dans ExAC et prédit par Human Splicing Finder
  (\url{http://www.umd.be/HSF3}) comme altérant probablement le site
  consensus d'épissage de l'éxon 16 de \emph{CFAP43}. Pour le second
  \ldots{}
  
  \textbf{détail des variants CFAP 44 (pareil que pour \emph{CFAP43})}
  
  L'analyse au microscope éléctronique à transmission des celulles
  spermatiques d'un patient portant un variant sur \emph{CFAP43} et d'un
  autre portant un variant sur \emph{CFAP44} révéla des défauts au niveau
  de l'axonème ainsi qu'une gaine fibreuse désorganisée pour chacun des
  deux patients.
  
  Ensuite, afin de compenser l'absence d'anticorps anti-CFAP43 et
  anti-CFAP44 fiables chez l'humain comme chez la souris nous avons décidé
  de caractériser leur orthologues chez \emph{Trypanosoma brucei}
  (\emph{T. brucei}), un protozoaire flagellé utilisé comme organisme
  modèle dans l'études des flagelles chez qui, les protéines
  \emph{Tb}CFAP43 et \emph{Tb}CFAP44 respectivement orthologues de CFAP43
  et CFAP44 avaient déjà été identifées comme des protéines du flagelles
  {[}\protect\hyperlink{ref-Broadhead2006}{171},
  \protect\hyperlink{ref-Subota2014}{172}{]}. Ensuite, l'utilisation d'ARN
  interférence nous a permi de produire des organismes \emph{knock-down}
  pour ces deux gènes \emph{TbCFAP43}\textsuperscript{RAi} et
  \emph{TbCFAP44}\textsuperscript{RNAi} nous permettant ainsi d'évaluer la
  fonction de ces deux gènes au sein du flagelle du trypanosome. Cela nous
  a permi d'observerun arret de la prolifération cellulaire au bout de 24h
  ainsi que de nombreux defauts au niveau des flagelles pour l'ensemble
  des lignées cellulaires \emph{TbCFAP43}\textsuperscript{RAi} et
  \emph{TbCFAP44}\textsuperscript{RNAi}.
  
  Pour finir, l'impact de l'absence des protéines CFAP43 et 44 sur la
  spermatogénèse murine a été déterminé grâce à la génération de modèle KO
  utilisant la technologie CRISPR-Cas9 qui nous a permi d'obtenir des
  phénotype reproductible pour nos deux modèles de souris KO. Ces modèles
  nous ont permis d'observer que les mâles
  \emph{Cfap43}\textsuperscript{-/-} et \emph{Cfap44}\textsuperscript{-/-}
  présentaient tous deux de nombreuses anomalies au niveaux des flagelles
  tandis que les femelles \emph{Cfap43}\textsuperscript{-/-} et
  \emph{Cfap44}\textsuperscript{-/-} étaient elles parfaitement fertiles.
  
  Pour conclure, cette étude portant sur la caractérisation du phénotype
  MMAF de \ldots{} individus non apparentés. L'utilisation de notre
  pipeline pour l'analyse des données NGS nous a permis à la fois de
  confirmer l'importance du gène \emph{DNAH1} dans la structure du
  flagelle et son implication dans ce phénotype, mais aussi d'identifier
  deux nouveaux gènes, \emph{CFAP43} et \emph{CFAP44} respectivement
  responsables du phénotype de 11 et 17 de nos patients soit 12.2 et
  12.2\% de notre cohorte.
  
  \newpage
  
  \begin{figure}
  
  {\centering \includegraphics{thesis_files/figure-latex/plottwohomohigh-1} 
  
  }
  
  \caption[Listes des gènes sur lesquels un variants tronquant à l'état homozygote a été retrouvé chez au moins deux de nos patients]{Listes des gènes sur lesquels un variants tronquant à l'état homozygote a été retrouvé chez au moins deux de nos patients : La couleur des barres indiquent le type de variants portés par chaque patient. Vert : variant homozygote tronquant, Orange variant homozygote non tronquant et rouge au moins deux variants hétérozygotes. Les gènes sont classés dans l'ordre décroissant en fonction du nombre de variants homozygotes tronquant retrouvés parmi nos patients.}\label{fig:plottwohomohigh}
  \end{figure}
  
  \newpage
  
  \section{Conclusion}\label{conclusion}
  
  \appendix
  
  \chapter{: Article DNAH1}\label{article-dnah1}
  
  Article DNAH1 2014
  
  \newpage
  
  \chapter{\texorpdfstring{Table des variants retrouvés sur le gène
  \emph{PATL2}}{Table des variants retrouvés sur le gène PATL2}}\label{table-des-variants-retrouves-sur-le-gene-patl2}
  
  \begin{longtable}[t]{llll}
  \caption{\label{tab:tabpatl2}Table des variants retrouvés sur le gène *PATL2*}\\
  \toprule
  \multicolumn{2}{c}{ } & \multicolumn{2}{c}{Variant impact} \\
  \cmidrule(l{2pt}r{2pt}){3-4}
  Patient & Geno & HGVSc, HGVSp & Consequence\\
  \midrule
  Ghs103 & Homo & c.478G>T ; p.Arg160Ter & stop gained\\
  Ghs104 & Homo & c.478G>T ; p.Arg160Ter & stop gained\\
  Ghs114 & Homo & c.478G>T ; p.Arg160Ter & stop gained\\
  Ghs5 & Homo & c.478G>T ; p.Arg160Ter & stop gained\\
  Ghs98 & Homo & c.478G>T ; p.Arg160Ter & stop gained\\
  \bottomrule
  \end{longtable}
  
  \newpage
  
  \chapter{\texorpdfstring{Table des variants retrouvés sur le gène
  \emph{DNAH1}}{Table des variants retrouvés sur le gène DNAH1}}\label{table-des-variants-retrouves-sur-le-gene-dnah1}
  
  \begin{landscape}
  \begin{longtable}[t]{lllll}
  \caption{\label{tab:tabdnah1}Table des variants retrouvés sur le gène *DNAH1*}\\
  \toprule
  \multicolumn{2}{c}{ } & \multicolumn{2}{c}{Variant impact} & \multicolumn{1}{c}{Variant frequency} \\
  \cmidrule(l{2pt}r{2pt}){3-4} \cmidrule(l{2pt}r{2pt}){5-5}
  Patient & Geno & HGVSc, HGVSp & Consequence & ExAC ; ESP ; 1KG\\
  \midrule
  Ghs122 & Homo & c.7533delC ; p.Gln2511HisfsTer27 & frameshift & . ; . ; .\\
  Ghs178 & Homo & c.4734\_4742delGATGGGTAG ; p.Pro1582\_Gly1584del & inframe deletion & . ; . ; .\\
  Ghs90 & Homo & c.2122G>C ; p.Ile708Leu & missense & . ; . ; .\\
  Ghs90 & Homo & c.2123A>C ; p.Ile708Thr & missense & . ; . ; .\\
  Ghs90 & Homo & c.2125A>C ; p.Phe709Leu & missense & . ; . ; .\\
  \addlinespace
  Ghs90 & Homo & c.4504C>A ; p.Val1502Met & missense & 0.00733 ; . ; .\\
  Ghs95 & Homo & c.9278C>G ; p.Ala3093Gly & missense & . ; . ; .\\
  Ghs126 & Hete & . ; p.Ile2739Val & missense & 0.00219 ; . ; .\\
  Ghs126 & Hete & . ; p.Ile2739Val & missense & 0.00219 ; 0.008 ; 0.0023\\
  Ghs127 & Hete & . ; p.Ser1763Leu & missense & 0.0058 ; . ; .\\
  \addlinespace
  Ghs127 & Hete & c.9121C>T ; p.Arg3041Cys & missense & 0.00527 ; . ; .\\
  Ghs129 & Hete & . ; p.Arg3169Gly & missense & 8.26e-06 ; . ; .\\
  Ghs129 & Hete & c.7153G>A ; p.Trp2385Arg & missense & . ; . ; .\\
  Ghs155 & Hete & c.3877T>A ; p.Asp1293Asn & missense & 0.000149 ; . ; .\\
  Ghs155 & Hete & c.3877T>A ; p.Asp1293Asn & missense & 0.000149 ; 1e-04 ; 0.0019\\
  \addlinespace
  Ghs167 & Hete & . ; p.Ile2739Val & missense & 0.00219 ; . ; .\\
  Ghs167 & Hete & . ; p.Ile2739Val & missense & 0.00219 ; 0.008 ; 0.0023\\
  Ghs172 & Hete & . ; p.Ser1763Leu & missense & 0.0058 ; . ; .\\
  Ghs172 & Hete & c.9121C>T ; p.Arg3041Cys & missense & 0.00527 ; . ; .\\
  Ghs175 & Hete & . ; p.Asp3440Asn & missense & 9.91e-05 ; . ; .\\
  \addlinespace
  Ghs175 & Hete & . ; p.Asp3440Asn & missense & 9.91e-05 ; 2e-04 ; .\\
  Ghs177 & Hete & c.1351A>G ; p.Lys451Glu & missense & 0.00103 ; . ; .\\
  Ghs177 & Hete & c.4504C>A ; p.Val1502Met & missense & 0.00733 ; . ; .\\
  Ghs26 & Hete & . ; p.Ile2739Val & missense & 0.00219 ; . ; .\\
  Ghs26 & Hete & . ; p.Ile2739Val & missense & 0.00219 ; 0.008 ; 0.0023\\
  \addlinespace
  Ghs27 & Hete & . ; p.Ser1763Leu & missense & 0.0058 ; . ; .\\
  Ghs27 & Hete & c.9121C>T ; p.Arg3041Cys & missense & 0.00527 ; . ; .\\
  Ghs28 & Hete & c.1172A>G ; p.Tyr391Cys & missense & 0.00233 ; 0.0027 ; 0.0019\\
  Ghs28 & Hete & c.752C>G ; p.Glu251Gly & missense & 1.65e-05 ; 1e-04 ; .\\
  Ghs36 & Hete & c.3931G>T ; p.Gln1311Ter & stop gained & 8.29e-06 ; . ; .\\
  \addlinespace
  Ghs36 & Hete & c.845A>G ; p.Leu282Trp & missense & . ; . ; .\\
  Ghs39 & Hete & . ; p.Asp3440Asn & missense & 9.91e-05 ; . ; .\\
  Ghs39 & Hete & . ; p.Asp3440Asn & missense & 9.91e-05 ; 2e-04 ; .\\
  Ghs42 & Hete & . ; . & splice region & . ; . ; .\\
  Ghs42 & Hete & . ; . & splice region & . ; . ; .\\
  \addlinespace
  Ghs42 & Hete & c.6916G>A ; p.Ala2306Thr & missense & . ; . ; .\\
  Ghs42 & Hete & c.6916G>A ; p.Ala2306Thr & missense & . ; . ; .\\
  Ghs87 & Hete & c.3431C>A ; p.Ser1144Asn & missense & 0.00024 ; . ; .\\
  Ghs87 & Hete & c.8885G>C ; p.Lys2962Thr & missense & 0.000457 ; 7e-04 ; 5e-04\\
  Ghs88 & Hete & c.2209C>A ; p.Val737Met & missense & 0.000115 ; 1e-04 ; .\\
  \addlinespace
  Ghs88 & Hete & c.3877T>A ; p.Asp1293Asn & missense & 0.000149 ; . ; .\\
  Ghs88 & Hete & c.3877T>A ; p.Asp1293Asn & missense & 0.000149 ; 1e-04 ; 0.0019\\
  \bottomrule
  \end{longtable}
  \end{landscape}
  
  \newpage
  
  \chapter{\texorpdfstring{Table des variants retrouvés sur le gène
  \emph{CFAP43}}{Table des variants retrouvés sur le gène CFAP43}}\label{table-des-variants-retrouves-sur-le-gene-cfap43}
  
  \begin{landscape}
  \begin{longtable}[t]{lllll}
  \caption{\label{tab:tabcfap43}Table des variants retrouvés sur le gène *CFAP43*}\\
  \toprule
  \multicolumn{2}{c}{ } & \multicolumn{2}{c}{Variant impact} & \multicolumn{1}{c}{Variant frequency} \\
  \cmidrule(l{2pt}r{2pt}){3-4} \cmidrule(l{2pt}r{2pt}){5-5}
  Patient & Geno & HGVSc, HGVSp & Consequence & ExAC ; ESP ; 1KG\\
  \midrule
  Ghs102 & Homo & c.3882delA ; p.Glu1294AspfsTer47 & frameshift & . ; . ; .\\
  Ghs105 & Homo & c.3541-2A>C ; . & splice acceptor & . ; . ; .\\
  Ghs105 & Homo & c.3541-2A>C ; . & splice acceptor & . ; . ; .\\
  Ghs126 & Homo & c.3352C>T ; p.Arg1118Ter & stop gained & 3.29e-05 ; . ; .\\
  Ghs160 & Homo & c.3541-2A>C ; . & splice acceptor & . ; . ; .\\
  \addlinespace
  Ghs160 & Homo & c.3541-2A>C ; . & splice acceptor & . ; . ; .\\
  Ghs162 & Homo & c.1240\_1241delGC ; p.Val414LeufsTer46 & frameshift & . ; . ; .\\
  Ghs164 & Homo & c.1240\_1241delGC ; p.Val414LeufsTer46 & frameshift & . ; . ; .\\
  Ghs17 & Homo & c.2658G>A ; p.Trp886Ter & stop gained & 9.88e-05 ; . ; .\\
  Ghs17 & Homo & c.2658G>A ; p.Trp886Ter & stop gained & 9.88e-05 ; 2e-04 ; .\\
  \addlinespace
  Ghs25 & Homo & c.2141+5T>A ; . & splice region & . ; . ; .\\
  Ghs41 & Homo & c.2680C>T ; p.Arg894Ter & stop gained & 8.24e-06 ; . ; .\\
  Ghs132 & Hete & c.1040G>C ; p.Val347Ala & missense & 7.41e-05 ; 2e-04 ; .\\
  Ghs132 & Hete & c.1300\_1301insT ; p.Leu435SerfsTer26 & frameshift & . ; . ; .\\
  Ghs154 & Hete & c.589C>A ; p.Val197Met & missense & 0.00385 ; . ; .\\
  Ghs154 & Hete & c.589C>A ; p.Val197Met & missense & 0.00385 ; 0.0042 ; 0.0037\\
  \bottomrule
  \end{longtable}
  \end{landscape}
  
  \newpage
  
  \chapter{\texorpdfstring{Table des variants retrouvés sur le gène
  \emph{CFAP44}}{Table des variants retrouvés sur le gène CFAP44}}\label{table-des-variants-retrouves-sur-le-gene-cfap44}
  
  \begin{landscape}
  \begin{longtable}[t]{lllll}
  \caption{\label{tab:tabcfap44}Table des variants retrouvés sur le gène *CFAP44*}\\
  \toprule
  \multicolumn{2}{c}{ } & \multicolumn{2}{c}{Variant impact} & \multicolumn{1}{c}{Variant frequency} \\
  \cmidrule(l{2pt}r{2pt}){3-4} \cmidrule(l{2pt}r{2pt}){5-5}
  Patient & Geno & HGVSc, HGVSp & Consequence & ExAC ; ESP ; 1KG\\
  \midrule
  Ghs155 & Homo & . ; p.Glu463Ter & stop gained & . ; . ; .\\
  Ghs168 & Homo & c.2818\_2819insG ; p.Glu940GlyfsTer19 & frameshift & 8.24e-06 ; . ; .\\
  Ghs177 & Homo & . ; p.Glu463Ter & stop gained & . ; . ; .\\
  Ghs181 & Homo & c.4767delC ; p.Ile1589MetfsTer6 & frameshift & . ; . ; .\\
  Ghs22 & Homo & . ; p.Arg1059Ter & stop gained & . ; . ; .\\
  \addlinespace
  Ghs34 & Homo & . ; . & splice donor & . ; . ; .\\
  Ghs89 & Homo & c.1457A>T ; p.Ala486Val & missense & 0.000692 ; . ; .\\
  Ghs89 & Homo & c.1457A>T ; p.Ala486Val & missense & 0.000692 ; 0.0012 ; 0.0014\\
  Ghs106 & Hete & . ; p.Arg1651Trp & missense & 0.00146 ; . ; .\\
  Ghs106 & Hete & . ; p.Arg1651Trp & missense & 0.00146 ; 0.0015 ; 0.0014\\
  \addlinespace
  Ghs125 & Hete & c.1457A>T ; p.Ala486Val & missense & 0.000692 ; . ; .\\
  Ghs125 & Hete & c.1457A>T ; p.Ala486Val & missense & 0.000692 ; 0.0012 ; 0.0014\\
  Ghs126 & Hete & . ; p.Arg1495Trp & missense & 0.00459 ; . ; .\\
  Ghs126 & Hete & . ; p.Arg1495Trp & missense & 0.00459 ; 0.0096 ; 0.007\\
  Ghs160 & Hete & . ; p.Arg1495Trp & missense & 0.00459 ; . ; .\\
  \addlinespace
  Ghs160 & Hete & . ; p.Arg1495Trp & missense & 0.00459 ; 0.0096 ; 0.007\\
  Ghs183 & Hete & . ; p.Arg1495Trp & missense & 0.00459 ; . ; .\\
  Ghs183 & Hete & . ; p.Arg1495Trp & missense & 0.00459 ; 0.0096 ; 0.007\\
  Ghs21 & Hete & . ; p.Arg1651Trp & missense & 0.00146 ; . ; .\\
  Ghs21 & Hete & . ; p.Arg1651Trp & missense & 0.00146 ; 0.0015 ; 0.0014\\
  \addlinespace
  Ghs41 & Hete & c.1457A>T ; p.Ala486Val & missense & 0.000692 ; . ; .\\
  Ghs41 & Hete & c.1457A>T ; p.Ala486Val & missense & 0.000692 ; 0.0012 ; 0.0014\\
  Ghs43 & Hete & . ; p.Arg1495Trp & missense & 0.00459 ; . ; .\\
  Ghs43 & Hete & . ; p.Arg1495Trp & missense & 0.00459 ; 0.0096 ; 0.007\\
  Ghs46 & Hete & . ; p.Arg1651Trp & missense & 0.00146 ; . ; .\\
  \addlinespace
  Ghs46 & Hete & . ; p.Arg1651Trp & missense & 0.00146 ; 0.0015 ; 0.0014\\
  Ghs47 & Hete & . ; p.Arg1651Trp & missense & 0.00146 ; . ; .\\
  Ghs47 & Hete & . ; p.Arg1651Trp & missense & 0.00146 ; 0.0015 ; 0.0014\\
  \bottomrule
  \end{longtable}
  \end{landscape}
  
  \backmatter
  
  \chapter*{References}\label{references}
  \addcontentsline{toc}{chapter}{References}
  
  \noindent
  
  \setlength{\parindent}{-0.20in} \setlength{\leftskip}{0.20in}
  \setlength{\parskip}{8pt}
  
  \hypertarget{refs}{}
  \hypertarget{ref-Gnessi1997}{}
  1. L. Gnessi, A. Fabbri, and G. Spera: ``Gonadal peptides as mediators
  of development and functional control of the testis: An integrated
  system with hormones and local environment.'' \emph{Endocrine Reviews}.
  vol. 18, no. 4, pp. 541--609, 1997.
  
  \hypertarget{ref-Sharpe1994}{}
  2. R.M. Sharpe, C. McKinnell, T. McLaren, M. Millar, T.P. West, S.
  Maguire, J. Gaughan, V. Syed, B. J?gou, J.B. Kerr, and P.T.K. Saunders:
  ``Interactions Between Androgens, Sertoli Cells and Germ Cells in the
  Control of Spermatogenesis.'' Molecular and cellular endocrinology of
  the testis. pp. 115--142. \emph{Springer Berlin Heidelberg}, Berlin,
  Heidelberg (1994).
  
  \hypertarget{ref-KIERSZENBAUM1994}{}
  3. A.L. KIERSZENBAUM: ``Mammalian Spermatogenesis
  \textless{}i\textgreater{}in Vivo\textless{}/i\textgreater{} and
  \textless{}i\textgreater{}in Vitro\textless{}/i\textgreater{} : A
  Partnership of Spermatogenic and Somatic Cell Lineages*.''
  \emph{Endocrine Reviews}. vol. 15, no. 1, pp. 116--134, 1994.
  
  \hypertarget{ref-Johnson1980}{}
  4. L. JOHNSON, C.S. PETTY, and W.B. NEAVES: ``A Comparative Study of
  Daily Sperm Production and Testicular Composition in Humans and Rats.''
  \emph{Biol Reprod}. vol. 22, no. 5, pp. 1233--1243, 1980.
  
  \hypertarget{ref-Clermont1963}{}
  5. Y. Clermont: ``The cycle of the seminiferous epithelium in man.''
  \emph{American Journal of Anatomy}. vol. 112, no. 1, pp. 35--51, 1963.
  
  \hypertarget{ref-Clermont1966}{}
  6. Y. Clermont: ``Renewal of spermatogonia in man.'' \emph{American
  Journal of Anatomy}. vol. 118, no. 2, pp. 509--524, 1966.
  
  \hypertarget{ref-Goossens2013}{}
  7. E. Goossens and H. Tournaye: ``Adult stem cells in the human
  testis.'' \emph{Seminars in Reproductive Medicine}. vol. 31, no. 1, pp.
  39--48, 2013.
  
  \hypertarget{ref-Handyside2012}{}
  8. A.H. Handyside: ``Molecular origin of female meiotic aneuploidies.''
  \emph{Biochimica et Biophysica Acta (BBA) - Molecular Basis of Disease}.
  vol. 1822, no. 12, pp. 1913--1920, 2012.
  
  \hypertarget{ref-YvesClermontRichardOko1993}{}
  9. L.H. Yves Clermont, Richard Oko: ``Cell and molecular biology of the
  testis.'' \emph{Oxford University Press}, 1993.
  
  \hypertarget{ref-Escalier1991}{}
  10. D. Escalier, J.M. Gallo, M. Albert, G. Meduri, D. Bermudez, G.
  David, and J. Schrevel: ``Human acrosome biogenesis: immunodetection of
  proacrosin in primary spermatocytes and of its partitioning pattern
  during meiosis.'' \emph{Development (Cambridge, England)}. vol. 113, no.
  3, pp. 779--788, 1991.
  
  \hypertarget{ref-Hamilton1987}{}
  11. G.M.H. Hamilton, D. W., Waites: ``Cellular and Molecular Events in
  Spermiogenesis.'' \emph{Cambridge University Press}, 1990.
  
  \hypertarget{ref-Papic}{}
  12. Z. Papic, G. Katona, and Z. Skrabalo: ``The cytologic identification
  and quantification of testicular cell subtypes. Reproducibility and
  relation to histologic findings in the diagnosis of male infertility.''
  \emph{Acta cytologica}. vol. 32, no. 5, pp. 697--706, 1988.
  
  \hypertarget{ref-Schenck}{}
  13. U. Schenck and W.B. Schill: ``Cytology of the human seminiferous
  epithelium.'' \emph{Acta cytologica}. vol. 32, no. 5, pp. 689--96,
  
  \hypertarget{ref-Adelman1989}{}
  14. M.M. Adelman and E.M. Cahill: ``Atlas of sperm morphology.''
  \emph{ASCP Press}, 1989.
  
  \hypertarget{ref-WorldHealthOrganization1992}{}
  15. World Health Organization: ``WHO laboratory manual for the
  examination of human semen and sperm-cervical mucus interaction.''
  \emph{Cambridge University Press}, 1992.
  
  \hypertarget{ref-Ogura1994}{}
  16. a Ogura, J. Matsuda, and R. Yanagimachi: ``Birth of normal young
  after electrofusion of mouse oocytes with round spermatids.''
  \emph{Proceedings of the National Academy of Sciences of the United
  States of America}. vol. 91, no. 16, pp. 7460--7462, 1994.
  
  \hypertarget{ref-Kimura1995}{}
  17. A. Ogura, J. Matsuda, T. Asano, O. Suzuki, and R. Yanagimachi:
  ``Mouse oocytes injected with cryopreserved round spermatids can develop
  into normal offspring.'' \emph{Journal of Assisted Reproduction and
  Genetics}. vol. 13, no. 5, pp. 431--434, 1996.
  
  \hypertarget{ref-Sasagawa}{}
  18. I. Sasagawa and R. Yanagimachi: ``Spermatids from mice after
  cryptorchid and reversal operations can initiate normal embryo
  development.'' \emph{Journal of andrology}. vol. 18, no. 2, pp.
  203--209, 1997.
  
  \hypertarget{ref-Tanaka2015}{}
  19. A. Tanaka, M. Nagayoshi, Y. Takemoto, I. Tanaka, H. Kusunoki, S.
  Watanabe, K. Kuroda, S. Takeda, M. Ito, and R. Yanagimachi: ``Fourteen
  babies born after round spermatid injection into human oocytes.''
  \emph{Proceedings of the National Academy of Sciences}. vol. 112, no.
  March 2014, pp. 201517466, 2015.
  
  \hypertarget{ref-Asimakopoulos2003}{}
  20. B. Asimakopoulos: ``Is There a Place for Round and Elongated
  Spermatids Injection in.'' vol. 1, no. 1, pp. 1--6, 2003.
  
  \hypertarget{ref-Moreno2006}{}
  21. R.D. Moreno, J. Palomino, and G. Schatten: ``Assembly of spermatid
  acrosome depends on microtubule organization during mammalian
  spermiogenesis.'' \emph{Developmental Biology}. vol. 293, no. 1, pp.
  218--227, 2006.
  
  \hypertarget{ref-Hermo2010}{}
  22. L. Hermo, R.M. Pelletier, D.G. Cyr, and C.E. Smith: ``Surfing the
  wave, cycle, life history, and genes/proteins expressed by testicular
  germ cells. Part 3: Developmental changes in spermatid flagellum and
  cytoplasmic droplet and interaction of sperm with the zona pellucida and
  egg plasma membrane.'' \emph{Microscopy Research and Technique}. vol.
  73, no. 4, pp. 320--363, 2010.
  
  \hypertarget{ref-Kierszenbaum2004}{}
  23. A.L. Kierszenbaum and L.L. Tres: ``The
  acrosome-acroplaxome-manchette complex and the shaping of the spermatid
  head.'' \emph{Archives of histology and cytology}. vol. 67, no. 4, pp.
  271--84, 2004.
  
  \hypertarget{ref-Cho2001}{}
  24. C. Cho, W.D. Willis, E.H. Goulding, H. Jung-Ha, Y.C. Choi, N.B.
  Hecht, and E.M. Eddy: ``Haploinsufficiency of protamine-1 or -2 causes
  infertility in mice.'' \emph{Nature genetics}. vol. 28, no. 1, pp.
  82--6, 2001.
  
  \hypertarget{ref-Kierszenbaum1978}{}
  25. A.L. Kierszenbaum and L.L. Tres: ``RNA transcription and chromatin
  structure during meiotic and postmeiotic stages of spermatogenesis.''
  \emph{Federation proceedings}. vol. 37, no. 11, pp. 2512--6, 1978.
  
  \hypertarget{ref-Ward1994}{}
  26. W.S. Ward: ``The structure of the sleeping genome: implications of
  sperm DNA organization for somatic cells.'' \emph{Journal of cellular
  biochemistry}. vol. 55, no. 1, pp. 77--82, 1994.
  
  \hypertarget{ref-Inaba2003}{}
  27. K. Inaba: ``Molecular Architecture of the Sperm Flagella: Molecules
  for Motility and Signaling.'' \emph{Zoological Science}. vol. 20, no. 9,
  pp. 1043--1056, 2003.
  
  \hypertarget{ref-Eddy2007}{}
  28. E.M. Eddy: ``The scaffold role of the fibrous sheath.''
  \emph{Society of Reproduction and Fertility supplement}. vol. 65, pp.
  45--62, 2007.
  
  \hypertarget{ref-Boivin2007a}{}
  29. J. Boivin, L. Bunting, J.A. Collins, and K.G. Nygren:
  ``International estimates of infertility prevalence and
  treatment-seeking: potential need and demand for infertility medical
  care.'' \emph{Human Reproduction}. vol. 22, no. 6, pp. 1506--1512, 2007.
  
  \hypertarget{ref-Grudzinskas1995}{}
  30. J.G.(.G. Grudzinskas and J. Yovich: ``Gametes : the spermatozoon.''
  \emph{Cambridge University Press}, 1995.
  
  \hypertarget{ref-Michael1937}{}
  31. M. Michael and K. Joel: ``Zellformen in normalen und pathologischen
  Ejakulaten und ihre klinische Bedeutung.'' \emph{Schweiz. Med. Wsch}.
  1937.
  
  \hypertarget{ref-Tomlinson1993a}{}
  32. M. Tomlinson, C. Barrati, A. Bolton, E. Lenton, H. Roberts, and I.
  Cooke: ``Round cells and sperm fertilizing capacity: The presence of
  immature germ cells but not seminal leukocytes are associated with
  reduced success of in vitro fertilization.'' \emph{International Journal
  of Gynecology \& Obstetrics}. vol. 42, no. 2, pp. 223--224, 1993.
  
  \hypertarget{ref-MacLeod1970}{}
  33. J. MacLeod: ``The Significance of Deviations in Human Sperm
  Morphology.'' Presented at the (1970).
  
  \hypertarget{ref-Tomlinson1993}{}
  34. M.J. Tomlinson, C.L.R. Barratt, and I.D. Cooke: ``Prospective study
  of leukocytes and leukocyte subpopulations in semen suggests they are
  not a cause of male infertility**Supported by the Infertility Research
  Trust, and the University of Sheffield, Sheffield, United Kingdom
  (M.J.T.).'' \emph{Fertility and Sterility}. vol. 60, no. 6, pp.
  1069--1075, 1993.
  
  \hypertarget{ref-Kurilo}{}
  35. L.F. Kurilo, I.A. Liubashevskaia, V.P. Dubinskaia, and T.N. Gaeva:
  ``{[}Karyological analysis of the count of immature germ cells in the
  ejaculate{]}.'' \emph{Urologiia i nefrologiia}. no. 2, pp. 45--7, 1993.
  
  \hypertarget{ref-SPERLING1971}{}
  36. K. SPERLING and R. KADEN: ``Meiotic Studies of the Ejaculated
  Seminal Fluid of Humans with Normal Sperm Count and Oligospermia.''
  \emph{Nature}. vol. 232, no. 5311, pp. 481--481, 1971.
  
  \hypertarget{ref-Girgis}{}
  37. S.M. Girgis, A.N. Etriby, A.A. Ibrahim, and S.A. Kahil: ``Testicular
  biopsy in azoospermia. A review of the last ten years' experiences of
  over 800 cases.'' \emph{Fertility and sterility}. vol. 20, no. 3, pp.
  467--77, 1969.
  
  \hypertarget{ref-Cooper2010}{}
  38. T.G. Cooper, E. Noonan, S. von Eckardstein, J. Auger, H.W.G. Baker,
  H.M. Behre, T.B. Haugen, T. Kruger, C. Wang, M.T. Mbizvo, and K.M.
  Vogelsong: ``World Health Organization reference values for human semen
  characteristics.'' \emph{Human Reproduction Update}. vol. 16, no. 3, pp.
  231--245, 2010.
  
  \hypertarget{ref-Colgan1980}{}
  39. T.J. Colgan, Y.C. Bedard, H.T. Strawbridge, M.B. Buckspan, and P.G.
  Klotz: ``Reappraisal of the Value of Testicular Biopsy in the
  Investigation of Infertility.'' \emph{Fertility and Sterility}. vol. 33,
  no. 1, pp. 56--60, 1980.
  
  \hypertarget{ref-Levin1979}{}
  40. H.S. Levin: ``Testicular biopsy in the study of male infertility.''
  \emph{Human Pathology}. vol. 10, no. 5, pp. 569--584, 1979.
  
  \hypertarget{ref-Soderstrom1980}{}
  41. K.O. Soderström and J. Suominen: ``Histopathology and ultrastructure
  of meiotic arrest in human spermatogenesis.'' \emph{Archives of
  pathology \& laboratory medicine}. vol. 104, no. 9, pp. 476--82, 1980.
  
  \hypertarget{ref-WONG1973}{}
  42. T.-W. WONG, F.H.I. STRAUS, and N.E. WARNER: ``TESTICULAR BIOPSY IN
  THE STUDY OF MALE INFERTILITY: II. POST... : Obstetrical \&
  Gynecological Survey.'' \emph{Obstetrical \& Gynecological Survey}. vol.
  28, no. 9, pp. 660--661, 1973.
  
  \hypertarget{ref-Palermo1992}{}
  43. G. Palermo, H. Joris, P. Devroey, and A.C. Van Steirteghem:
  ``Pregnancies after intracytoplasmic injection of single spermatozoon
  into an oocyte.'' \emph{Lancet (London, England)}. vol. 340, no. 8810,
  pp. 17--8, 1992.
  
  \hypertarget{ref-Lindholmer1974}{}
  44. C. Lindholmer: ``The importance of seminal plasma for human sperm
  motility.'' \emph{Biology of reproduction}. vol. 10, no. 5, pp. 533--42,
  1974.
  
  \hypertarget{ref-Bjorndahl2010}{}
  45. L. Björndahl: ``The usefulness and significance of assessing rapidly
  progressive spermatozoa.'' \emph{Asian journal of andrology}. vol. 12,
  no. 1, pp. 33--5, 2010.
  
  \hypertarget{ref-Aitken1985}{}
  46. R.J. Aitken, M. Sutton, P. Warner, and D.W. Richardson:
  ``Relationship between the movement characteristics of human spermatozoa
  and their ability to penetrate cervical mucus and zona-free hamster
  oocytes.'' \emph{Journal of reproduction and fertility}. vol. 73, no. 2,
  pp. 441--9, 1985.
  
  \hypertarget{ref-Tuttelmann2011}{}
  47. F. Tüttelmann, M. Simoni, S. Kliesch, S. Ledig, B. Dworniczak, P.
  Wieacker, and A. Röpke: ``Copy number variants in patients with severe
  oligozoospermia and Sertoli-cell-only syndrome.'' \emph{PloS one}. vol.
  6, no. 4, pp. e19426, 2011.
  
  \hypertarget{ref-Skaletsky2003}{}
  48. H. Skaletsky, T. Kuroda-Kawaguchi, P.J. Minx, H.S. Cordum, L.
  Hillier, L.G. Brown, S. Repping, T. Pyntikova, J. Ali, T. Bieri, A.
  Chinwalla, A. Delehaunty, K. Delehaunty, H. Du, G. Fewell, L. Fulton, R.
  Fulton, T. Graves, S.-F. Hou, P. Latrielle, S. Leonard, E. Mardis, R.
  Maupin, J. McPherson, T. Miner, W. Nash, C. Nguyen, P. Ozersky, K.
  Pepin, S. Rock, T. Rohlfing, K. Scott, B. Schultz, C. Strong, A.
  Tin-Wollam, S.-P. Yang, R.H. Waterston, R.K. Wilson, S. Rozen, and D.C.
  Page: ``The male-specific region of the human Y chromosome is a mosaic
  of discrete sequence classes.'' \emph{Nature}. vol. 423, no. 6942, pp.
  825--837, 2003.
  
  \hypertarget{ref-Hotaling2014}{}
  49. J. Hotaling and D.T. Carrell: ``Clinical genetic testing for male
  factor infertility: current applications and future directions.''
  \emph{Andrology}. vol. 2, no. 3, pp. 339--350, 2014.
  
  \hypertarget{ref-Ravel2006}{}
  50. C. Ravel, I. Berthaut, J.L. Bresson, J.P. Siffroi, and Genetics
  Commission of the French Federation of CECOS: ``Prevalence of
  chromosomal abnormalities in phenotypically normal and fertile adult
  males: large-scale survey of over 10 000 sperm donor karyotypes.''
  \emph{Human Reproduction}. vol. 21, no. 6, pp. 1484--1489, 2006.
  
  \hypertarget{ref-Bojesen2011}{}
  51. A. Bojesen and C.H. Gravholt: ``Morbidity and mortality in
  Klinefelter syndrome (47,XXY).'' \emph{Acta Paediatrica}. vol. 100, no.
  6, pp. 807--813, 2011.
  
  \hypertarget{ref-Gekas2001}{}
  52. J. Gekas, F. Thepot, C. Turleau, J.P. Siffroi, J.P. Dadoune, S.
  Briault, M. Rio, G. Bourouillou, F. Carré-Pigeon, R. Wasels, B.
  Benzacken, and Association des Cytogeneticiens de Langue Francaise:
  ``Chromosomal factors of infertility in candidate couples for ICSI: an
  equal risk of constitutional aberrations in women and men.'' \emph{Human
  reproduction (Oxford, England)}. vol. 16, no. 1, pp. 82--90, 2001.
  
  \hypertarget{ref-Elliott1997}{}
  53. D.J. Elliott and H.J. Cooke: ``The molecular genetics of male
  infertility.'' \emph{BioEssays}. vol. 19, no. 9, pp. 801--809, 1997.
  
  \hypertarget{ref-OFlynnOBrien2010}{}
  54. K.L. O'Flynn O'Brien, A.C. Varghese, and A. Agarwal: ``The genetic
  causes of male factor infertility: A review.'' \emph{Fertility and
  Sterility}. vol. 93, no. 1, pp. 1--12, 2010.
  
  \hypertarget{ref-Krausz2000}{}
  55. C. Krausz and G. Forti: ``Clinical aspects of male infertility.''
  \emph{Results and problems in cell differentiation}. vol. 28, pp. 1--21,
  2000.
  
  \hypertarget{ref-Vorona2007}{}
  56. E. Vorona, M. Zitzmann, J. Gromoll, A.N. Schüring, and E. Nieschlag:
  ``Clinical, Endocrinological, and Epigenetic Features of the 46,XX Male
  Syndrome, Compared with 47,XXY Klinefelter Patients.'' \emph{The Journal
  of Clinical Endocrinology \& Metabolism}. vol. 92, no. 9, pp.
  3458--3465, 2007.
  
  \hypertarget{ref-Yu2012}{}
  57. J. Yu, Z. Chen, Y. Ni, and Z. Li: ``CFTR mutations in men with
  congenital bilateral absence of the vas deferens (CBAVD): a systemic
  review and meta-analysis.'' \emph{Human Reproduction}. vol. 27, no. 1,
  pp. 25--35, 2012.
  
  \hypertarget{ref-Minase2017}{}
  58. G. Minase, T. Miyamoto, Y. Miyagawa, M. Iijima, H. Ueda, Y. Saijo,
  M. Namiki, and K. Sengoku: ``Single-nucleotide polymorphisms in the
  human \textless{}i\textgreater{}RAD21L\textless{}/i\textgreater{} gene
  may be a genetic risk factor for Japanese patients with azoospermia
  caused by meiotic arrest and Sertoli cell-only syndrome.'' \emph{Human
  Fertility}. pp. 1--4, 2017.
  
  \hypertarget{ref-Yatsenko2015}{}
  59. A.N. Yatsenko, A.P. Georgiadis, A. Röpke, A.J. Berman, T. Jaffe, M.
  Olszewska, B. Westernströer, J. Sanfilippo, M. Kurpisz, A. Rajkovic,
  S.A. Yatsenko, S. Kliesch, S. Schlatt, and F. Tüttelmann: ``X-linked
  TEX11 mutations, meiotic arrest, and azoospermia in infertile men.''
  \emph{The New England journal of medicine}. vol. 372, no. 22, pp.
  2097--107, 2015.
  
  \hypertarget{ref-Yang2015}{}
  60. F. Yang, S. Silber, N.A. Leu, R.D. Oates, J.D. Marszalek, H.
  Skaletsky, L.G. Brown, S. Rozen, D.C. Page, and P.J. Wang: ``TEX11 is
  mutated in infertile men with azoospermia and regulates genome-wide
  recombination rates in mouse.'' \emph{EMBO molecular medicine}. vol. 7,
  no. 9, pp. 1198--210, 2015.
  
  \hypertarget{ref-Maor-Sagie2015}{}
  61. E. Maor-Sagie, Y. Cinnamon, B. Yaacov, A. Shaag, H. Goldsmidt, S.
  Zenvirt, N. Laufer, C. Richler, and A. Frumkin: ``Deleterious mutation
  in SYCE1 is associated with non-obstructive azoospermia.'' \emph{Journal
  of assisted reproduction and genetics}. vol. 32, no. 6, pp. 887--91,
  2015.
  
  \hypertarget{ref-Nistal}{}
  62. M. Nistal, R. Paniagua, and A. Herruzo: ``Multi-tailed spermatozoa
  in a case with asthenospermia and teratospermia.'' \emph{Virchows Archiv
  B}. vol. 26, no. 1, pp. 111--118, 1978.
  
  \hypertarget{ref-Dieterich2007}{}
  63. K. Dieterich, R. Soto Rifo, A.K. Faure, S. Hennebicq, B. Ben Amar,
  M. Zahi, J. Perrin, D. Martinez, B. Sèle, P.-S. Jouk, T. Ohlmann, S.
  Rousseaux, J. Lunardi, and P.F. Ray: ``Homozygous mutation of AURKC
  yields large-headed polyploid spermatozoa and causes male infertility.''
  \emph{Nature genetics}. vol. 39, no. 5, pp. 661--5, 2007.
  
  \hypertarget{ref-BenKhelifa2011}{}
  64. M. Ben Khelifa, R. Zouari, R. Harbuz, L. Halouani, C. Arnoult, J.
  Lunardi, and P.F. Ray: ``A new AURKC mutation causing macrozoospermia:
  implications for human spermatogenesis and clinical diagnosis.''
  \emph{Molecular Human Reproduction}. vol. 17, no. 12, pp. 762--768,
  2011.
  
  \hypertarget{ref-Dieterich2009}{}
  65. K. Dieterich, R. Zouari, R. Harbuz, F. Vialard, D. Martinez, H.
  Bellayou, N. Prisant, A. Zoghmar, M.R. Guichaoua, I. Koscinski, M.
  Kharouf, M. Noruzinia, S. Nadifi, A. Sefiani, J. Lornage, M. Zahi, S.
  Viville, B. Sele, P.-S. Jouk, M.-C. Jacob, D. Escalier, Y. Nikas, S.
  Hennebicq, J. Lunardi, and P.F. Ray: ``The Aurora Kinase C c.144delC
  mutation causes meiosis I arrest in men and is frequent in the North
  African population.'' \emph{Human Molecular Genetics}. vol. 18, no. 7,
  pp. 1301--1309, 2009.
  
  \hypertarget{ref-Dam2006}{}
  66. A. Dam, I. Feenstra, J. Westphal, L. Ramos, R. van Golde, and J.
  Kremer: ``Globozoospermia revisited.'' \emph{Human Reproduction Update}.
  vol. 13, no. 1, pp. 63--75, 2006.
  
  \hypertarget{ref-Sen2009}{}
  67. C.G.S. Sen, A.F. Holstein, and C. Schirren: ``über die Morphogenese
  rundköpfiger Spermatozoen des Menschen.'' \emph{Andrologia}. vol. 3, no.
  3, pp. 117--125, 1971.
  
  \hypertarget{ref-Holstein1973}{}
  68. A.F. Holstein, C. Schirren, and C.G. Schirren: ``Human spermatids
  and spermatozoa lacking acrosomes.'' \emph{Journal of reproduction and
  fertility}. vol. 35, no. 3, pp. 489--91, 1973.
  
  \hypertarget{ref-Dam2007a}{}
  69. A.H. Dam, I. Koscinski, J.A. Kremer, C. Moutou, A.-S. Jaeger, A.R.
  Oudakker, H. Tournaye, N. Charlet, C. Lagier-Tourenne, H. van Bokhoven,
  and S. Viville: ``Homozygous Mutation in SPATA16 Is Associated with Male
  Infertility in Human Globozoospermia.'' \emph{The American Journal of
  Human Genetics}. vol. 81, no. 4, pp. 813--820, 2007.
  
  \hypertarget{ref-Lu2006}{}
  70. L. Lu, M. Lin, M. Xu, Z.-M. Zhou, and J.-H. Sha: ``Gene functional
  research using polyethylenimine-mediated in vivo gene transfection into
  mouse spermatogenic cells.'' \emph{Asian Journal of Andrology}. vol. 8,
  no. 1, pp. 53--59, 2006.
  
  \hypertarget{ref-Chemes2010}{}
  71. H.E. Chemes and V.Y. Rawe: ``The making of abnormal spermatozoa:
  cellular and molecular mechanisms underlying pathological
  spermiogenesis.'' \emph{Cell and Tissue Research}. vol. 341, no. 3, pp.
  349--357, 2010.
  
  \hypertarget{ref-Panidis2001}{}
  72. D. Panidis, D. Rousso, A. Kourtis, C. Gianoulis, K. Papathanasiou,
  and J. Kalachanis: ``Headless spermatozoa in semen specimens from
  fertile and subfertile men.'' \emph{The Journal of reproductive
  medicine}. vol. 46, no. 11, pp. 947--50, 2001.
  
  \hypertarget{ref-Chemes1987}{}
  73. H.E. Chemes, C. Carizza, F. Scarinci, S. Brugo, N. Neuspiller, and
  L. Schwarsztein: ``Lack of a head in human spermatozoa from sterile
  patients: a syndrome associated with impaired fertilization.''
  \emph{Fertility and sterility}. vol. 47, no. 2, pp. 310--6, 1987.
  
  \hypertarget{ref-Zhu2016}{}
  74. F. Zhu, F. Wang, X. Yang, J. Zhang, H. Wu, Z. Zhang, Z. Zhang, X.
  He, P. Zhou, Z. Wei, J. Gecz, and Y. Cao: ``Biallelic SUN5 Mutations
  Cause Autosomal-Recessive Acephalic Spermatozoa Syndrome.'' \emph{The
  American Journal of Human Genetics}. vol. 99, no. 4, pp. 942--949, 2016.
  
  \hypertarget{ref-Yassine2015}{}
  75. S. Yassine, J. Escoffier, R. Abi Nahed, R.A. Nahed, V. Pierre, T.
  Karaouzene, P.F. Ray, and C. Arnoult: ``Dynamics of Sun5 localization
  during spermatogenesis in wild type and Dpy19l2 knock-out mice indicates
  that Sun5 is not involved in acrosome attachment to the nuclear
  envelope.'' \emph{PloS one}. vol. 10, no. 3, pp. e0118698, 2015.
  
  \hypertarget{ref-Coutton2015}{}
  76. C. Coutton, J. Escoffier, G. Martinez, C. Arnoult, and P.F. Ray:
  ``Teratozoospermia: spotlight on the main genetic actors in the human.''
  \emph{Human Reproduction Update}. vol. 21, no. 4, pp. 455--485, 2015.
  
  \hypertarget{ref-BenKhelifa2014}{}
  77. M. Ben Khelifa, C. Coutton, R. Zouari, T. Karaouzène, J. Rendu, M.
  Bidart, S. Yassine, V. Pierre, J. Delaroche, S. Hennebicq, D. Grunwald,
  D. Escalier, K. Pernet-Gallay, P.S. Jouk, N. Thierry-Mieg, A. Touré, C.
  Arnoult, and P.F. Ray: ``Mutations in DNAH1, which encodes an inner arm
  heavy chain dynein, lead to male infertility from multiple morphological
  abnormalities of the sperm flagella.'' \emph{American Journal of Human
  Genetics}. vol. 94, no. 1, pp. 95--104, 2014.
  
  \hypertarget{ref-Wang2017}{}
  78. X. Wang, H. Jin, F. Han, Y. Cui, J. Chen, C. Yang, P. Zhu, W. Wang,
  G. Jiao, W. Wang, C. Hao, and Z. Gao: ``Homozygous
  \textless{}i\textgreater{}DNAH1\textless{}/i\textgreater{} frameshift
  mutation causes multiple morphological anomalies of the sperm flagella
  in Chinese.'' \emph{Clinical Genetics}. vol. 91, no. 2, pp. 313--321,
  2017.
  
  \hypertarget{ref-Amiri-Yekta2016}{}
  79. A. Amiri-Yekta, C. Coutton, Z.-E. Kherraf, T. Karaouzène, P. Le
  Tanno, M.H. Sanati, M. Sabbaghian, N. Almadani, M.A. Sadighi Gilani,
  S.H. Hosseini, S. Bahrami, A. Daneshipour, M. Bini, C. Arnoult, R.
  Colombo, H. Gourabi, and P.F. Ray: ``Whole-exome sequencing of familial
  cases of multiple morphological abnormalities of the sperm flagella
  (MMAF) reveals new
  \textless{}i\textgreater{}DNAH1\textless{}/i\textgreater{} mutations.''
  \emph{Human Reproduction}. vol. 31, no. 12, pp. 2872--2880, 2016.
  
  \hypertarget{ref-Nomikos2013}{}
  80. M. Nomikos, J. Kashir, K. Swann, and F.A. Lai: ``Sperm PLC\(\zeta\):
  From structure to Ca
  \textless{}sup\textgreater{}2+\textless{}/sup\textgreater{}
  oscillations, egg activation and therapeutic potential.'' \emph{FEBS
  Letters}. vol. 587, no. 22, pp. 3609--3616, 2013.
  
  \hypertarget{ref-Amdani2013}{}
  81. S.N. Amdani, C. Jones, and K. Coward: ``Phospholipase C zeta
  (PLC\(\zeta\)): Oocyte activation and clinical links to male factor
  infertility.'' \emph{Advances in Biological Regulation}. vol. 53, no. 3,
  pp. 292--308, 2013.
  
  \hypertarget{ref-Heytens2009}{}
  82. E. Heytens, J. Parrington, K. Coward, C. Young, S. Lambrecht, S.-Y.
  Yoon, R.A. Fissore, R. Hamer, C.M. Deane, M. Ruas, P. Grasa, R.
  Soleimani, C.A. Cuvelier, J. Gerris, M. Dhont, D. Deforce, L. Leybaert,
  and P. De Sutter: ``Reduced amounts and abnormal forms of phospholipase
  C zeta (PLCzeta) in spermatozoa from infertile men.'' \emph{Human
  reproduction (Oxford, England)}. vol. 24, no. 10, pp. 2417--28, 2009.
  
  \hypertarget{ref-Escoffier2016}{}
  83. J. Escoffier, H.C. Lee, S. Yassine, R. Zouari, G. Martinez, T.
  Karaouzène, C. Coutton, Z.-E. Kherraf, L. Halouani, C. Triki, S. Nef, N.
  Thierry-Mieg, S.N. Savinov, R. Fissore, P.F. Ray, and C. Arnoult:
  ``Homozygous mutation of PLCZ1 leads to defective human oocyte
  activation and infertility that is not rescued by the WW-binding protein
  PAWP.'' \emph{Human molecular genetics}. vol. 25, no. 5, pp. 878--91,
  2016.
  
  \hypertarget{ref-DeBoer2015}{}
  84. P. de Boer, M. de Vries, and L. Ramos: ``A mutation study of sperm
  head shape and motility in the mouse: lessons for the clinic.''
  \emph{Andrology}. vol. 3, no. 2, pp. 174--202, 2015.
  
  \hypertarget{ref-ElInati2012}{}
  85. E. ElInati, P. Kuentz, C. Redin, S. Jaber, F. Vanden Meerschaut, J.
  Makarian, I. Koscinski, M.H. Nasr-Esfahani, A. Demirol, T. Gurgan, N.
  Louanjli, N. Iqbal, M. Bisharah, F.C. Pigeon, H. Gourabi, D. De Briel,
  F. Brugnon, S.A. Gitlin, J.-M. Grillo, K. Ghaedi, M.R. Deemeh, S.
  Tanhaei, P. Modarres, B. Heindryckx, M. Benkhalifa, D. Nikiforaki, S.C.
  Oehninger, P. De Sutter, J. Muller, and S. Viville: ``Globozoospermia is
  mainly due to DPY19L2 deletion via non-allelic homologous recombination
  involving two recombination hotspots.'' \emph{Human Molecular Genetics}.
  vol. 21, no. 16, pp. 3695--3702, 2012.
  
  \hypertarget{ref-Miyamoto2003}{}
  86. T. Miyamoto, S. Hasuike, L. Yogev, M.R. Maduro, M. Ishikawa, H.
  Westphal, and D.J. Lamb: ``Azoospermia in patients heterozygous for a
  mutation in SYCP3.'' \emph{The Lancet}. vol. 362, no. 9397, pp.
  1714--1719, 2003.
  
  \hypertarget{ref-Yatsenko2006}{}
  87. A.N. Yatsenko, A. Roy, R. Chen, L. Ma, L.J. Murthy, W. Yan, D.J.
  Lamb, and M.M. Matzuk: ``Non-invasive genetic diagnosis of male
  infertility using spermatozoal RNA: KLHL10mutations in oligozoospermic
  patients impair homodimerization.'' \emph{Human Molecular Genetics}.
  vol. 15, no. 23, pp. 3411--3419, 2006.
  
  \hypertarget{ref-Bashamboo2010}{}
  88. A. Bashamboo, B. Ferraz-de-Souza, D. Lourenço, L. Lin, N.J. Sebire,
  D. Montjean, J. Bignon-Topalovic, J. Mandelbaum, J.-P. Siffroi, S.
  Christin-Maitre, U. Radhakrishna, H. Rouba, C. Ravel, J. Seeler, J.C.
  Achermann, and K. McElreavey: ``Human male infertility associated with
  mutations in NR5A1 encoding steroidogenic factor 1.'' \emph{American
  journal of human genetics}. vol. 87, no. 4, pp. 505--12, 2010.
  
  \hypertarget{ref-Alon1999}{}
  89. U. Alon, N. Barkai, D.A. Notterman, K. Gish, S. Ybarra, D. Mack, and
  A.J. Levine: ``Broad patterns of gene expression revealed by clustering
  analysis of tumor and normal colon tissues probed by oligonucleotide
  arrays.'' \emph{Proceedings of the National Academy of Sciences of the
  United States of America}. vol. 96, no. 12, pp. 6745--50, 1999.
  
  \hypertarget{ref-Wang2000}{}
  90. T. Wang, D. Hopkins, C. Schmidt, S. Silva, R. Houghton, H. Takita,
  E. Repasky, and S.G. Reed: ``Identification of genes differentially
  over-expressed in lung squamous cell carcinoma using combination of cDNA
  subtraction and microarray analysis.'' \emph{Oncogene}. vol. 19, no. 12,
  pp. 1519--1528, 2000.
  
  \hypertarget{ref-Singh2002}{}
  91. D. Singh, P.G. Febbo, K. Ross, D.G. Jackson, J. Manola, C. Ladd, P.
  Tamayo, A.A. Renshaw, A.V. D'Amico, J.P. Richie, E.S. Lander, M. Loda,
  P.W. Kantoff, T.R. Golub, and W.R. Sellers: ``Gene expression correlates
  of clinical prostate cancer behavior.'' \emph{Cancer cell}. vol. 1, no.
  2, pp. 203--9, 2002.
  
  \hypertarget{ref-VantVeer2002}{}
  92. L.J. van 't Veer, H. Dai, M.J. van de Vijver, Y.D. He, A.A.M. Hart,
  M. Mao, H.L. Peterse, K. van der Kooy, M.J. Marton, A.T. Witteveen, G.J.
  Schreiber, R.M. Kerkhoven, C. Roberts, P.S. Linsley, R. Bernards, and
  S.H. Friend: ``Gene expression profiling predicts clinical outcome of
  breast cancer.'' \emph{Nature}. vol. 415, no. 6871, pp. 530--536, 2002.
  
  \hypertarget{ref-Brachat2002}{}
  93. A. Brachat, B. Pierrat, A. Xynos, K. Brecht, M. Simonen, A.
  Brüngger, and J. Heim: ``A microarray-based, integrated approach to
  identify novel regulators of cancer drug response and apoptosis.''
  \emph{Oncogene}. vol. 21, no. 54, pp. 8361--8371, 2002.
  
  \hypertarget{ref-Cutler2001}{}
  94. D.J. Cutler, M.E. Zwick, M.M. Carrasquillo, C.T. Yohn, K.P. Tobin,
  C. Kashuk, D.J. Mathews, N.A. Shah, E.E. Eichler, J.A. Warrington, and
  A. Chakravarti: ``High-throughput variation detection and genotyping
  using microarrays.'' \emph{Genome research}. vol. 11, no. 11, pp.
  1913--25, 2001.
  
  \hypertarget{ref-Wang1998}{}
  95. D.G. Wang, J.B. Fan, C.J. Siao, A. Berno, P. Young, R. Sapolsky, G.
  Ghandour, N. Perkins, E. Winchester, J. Spencer, L. Kruglyak, L. Stein,
  L. Hsie, T. Topaloglou, E. Hubbell, E. Robinson, M. Mittmann, M.S.
  Morris, N. Shen, D. Kilburn, J. Rioux, C. Nusbaum, S. Rozen, T.J.
  Hudson, R. Lipshutz, M. Chee, and E.S. Lander: ``Large-scale
  identification, mapping, and genotyping of single-nucleotide
  polymorphisms in the human genome.'' \emph{Science (New York, N.Y.)}.
  vol. 280, no. 5366, pp. 1077--82, 1998.
  
  \hypertarget{ref-Brown1999}{}
  96. P.O. Brown, J.R. Pollack, C.M. Perou, A.A. Alizadeh, M.B. Eisen, A.
  Pergamenschikov, C.F. Williams, S.S. Jeffrey, and D. Botstein:
  ``Genome-wide analysis of DNA copy-number changes using cDNA
  microarrays.'' \emph{Nature Genetics}. vol. 23, no. 1, pp. 41--46, 1999.
  
  \hypertarget{ref-Collins2003}{}
  97. F.S. Collins, M. Morgan, and A. Patrinos: ``The Human Genome
  Project: Lessons from Large-Scale Biology.'' \emph{Science}. vol. 300,
  no. 5617, pp. 286--290, 2003.
  
  \hypertarget{ref-Metzker2010}{}
  98. M.L. Metzker: ``Sequencing technologies - the next generation.''
  \emph{Nature reviews. Genetics}. vol. 11, no. 1, pp. 31--46, 2010.
  
  \hypertarget{ref-Sims2014}{}
  99. D. Sims, I. Sudbery, N.E. Ilott, A. Heger, and C.P. Ponting:
  ``Sequencing depth and coverage: key considerations in genomic
  analyses.'' \emph{Nature reviews. Genetics}. vol. 15, no. 2, pp.
  121--32, 2014.
  
  \hypertarget{ref-Ng2010}{}
  100. S.B. Ng, E.H. Turner, P.D. Robertson, S.D. Flygare, W. Abigail, C.
  Lee, T. Shaffer, M. Wong, A. Bhattacharjee, E. Evan, M. Bamshad, D. a
  Nickerson, and J. Shendure: ``Targeted Capture and Massicely Parallel
  Sequencing of twelve human exomes.'' \emph{Nature}. vol. 461, no. 7261,
  pp. 272--276, 2010.
  
  \hypertarget{ref-Lelieveld2015}{}
  101. S.H. Lelieveld, M. Spielmann, S. Mundlos, J. a Veltman, and C.
  Gilissen: ``Comparison of Exome and Genome Sequencing Technologies for
  the Complete Capture of Protein-Coding Regions.'' \emph{Human mutation}.
  vol. 36, no. 8, pp. 815--22, 2015.
  
  \hypertarget{ref-Meienberg2016}{}
  102. J. Meienberg, R. Bruggmann, K. Oexle, and G. Matyas: ``Clinical
  sequencing: is WGS the better WES?'' \emph{Human Genetics}. vol. 135,
  no. 3, pp. 359--362, 2016.
  
  \hypertarget{ref-Goodwin2016}{}
  103. S. Goodwin, J.D. McPherson, and W.R. McCombie: ``Coming of age: ten
  years of next-generation sequencing technologies.'' \emph{Nat Rev
  Genet}. vol. 17, no. 6, pp. 333--351, 2016.
  
  \hypertarget{ref-Guo2008}{}
  104. J. Guo, N. Xu, Z. Li, S. Zhang, J. Wu, D.H. Kim, M. Sano Marma, Q.
  Meng, H. Cao, X. Li, S. Shi, L. Yu, S. Kalachikov, J.J. Russo, N.J.
  Turro, and J. Ju: ``Four-color DNA sequencing with 3'-O-modified
  nucleotide reversible terminators and chemically cleavable fluorescent
  dideoxynucleotides.'' \emph{Proceedings of the National Academy of
  Sciences of the United States of America}. vol. 105, no. 27, pp.
  9145--9150, 2008.
  
  \hypertarget{ref-Tomkinson2006}{}
  105. A.E. Tomkinson, S. Vijayakumar, J.M. Pascal, and T. Ellenberger:
  ``DNA Ligases:~ Structure, Reaction Mechanism, and Function.''
  \emph{Chemical Reviews}. vol. 106, no. 2, pp. 687--699, 2006.
  
  \hypertarget{ref-Wold2007}{}
  106. B. Wold and R.M. Myers: ``Sequence census methods for functional
  genomics.'' \emph{Nature Methods}. vol. 5, no. 1, pp. 19--21, 2007.
  
  \hypertarget{ref-Yang2009}{}
  107. M.Q. Yang, B.D. Athey, H.R. Arabnia, A.H. Sung, Q. Liu, J.Y. Yang,
  J. Mao, and Y. Deng: ``High-throughput next-generation sequencing
  technologies foster new cutting-edge computing techniques in
  bioinformatics.'' \emph{BMC genomics}. vol. 10 Suppl 1, pp. I1, 2009.
  
  \hypertarget{ref-Qin2010}{}
  108. J. Qin, R. Li, J. Raes, M. Arumugam, S. Burgdorf, C. Manichanh, T.
  Nielsen, N. Pons, T. Yamada, D.R. Mende, J. Li, J. Xu, S. Li, D. Li, J.
  Cao, B. Wang, H. Liang, H. Zheng, Y. Xie, J. Tap, P. Lepage, M.
  Bertalan, J.-m. Batto, T. Hansen, D.L. Paslier, A. Linneberg, H.B.
  Nielsen, E. Pelletier, P. Renault, Y. Zhou, Y. Li, X. Zhang, S. Li, N.
  Qin, and H. Yang: ``A human gut microbial gene catalog established by
  metagenomic sequencing.'' \emph{Nature}. vol. 464, no. 7285, pp. 59--65,
  2010.
  
  \hypertarget{ref-VanTassell2008}{}
  109. C.P. Van Tassell, T.P.L. Smith, L.K. Matukumalli, J.F. Taylor, R.D.
  Schnabel, C.T. Lawley, C.D. Haudenschild, S.S. Moore, W.C. Warren, and
  T.S. Sonstegard: ``SNP discovery and allele frequency estimation by deep
  sequencing of reduced representation libraries.'' \emph{Nature Methods}.
  vol. 5, no. 3, pp. 247--252, 2008.
  
  \hypertarget{ref-Alkan2010}{}
  110. C. Alkan, J.M. Kidd, T. Marques-bonet, G. Aksay, F. Hormozdiari,
  J.O. Kitzman, C. Baker, M. Malig, S.C. Sahinalp, R.A. Gibbs, and E.E.
  Eichler: ``Personalized Copy-Number and Segmental Duplication Maps using
  Next-Generation Sequencing.'' \emph{Nature Genetics}. vol. 41, no. 10,
  pp. 1061--1067, 2010.
  
  \hypertarget{ref-Medvedev2009}{}
  111. P. Medvedev, M. Stanciu, and M. Brudno: ``Computational methods for
  discovering structural variation with next-generation sequencing.''
  \emph{Nature Methods}. vol. 6, no. 11s, pp. S13--S20, 2009.
  
  \hypertarget{ref-Taylor2007}{}
  112. K.H. Taylor, R.S. Kramer, J.W. Davis, J. Guo, D.J. Duff, D. Xu,
  C.W. Caldwell, and H. Shi: ``Ultradeep Bisulfite Sequencing Analysis of
  DNA Methylation Patterns in Multiple Gene Promoters by 454 Sequencing.''
  \emph{Cancer Research}. vol. 67, no. 18, pp. 8511--8518, 2007.
  
  \hypertarget{ref-Sultan2008}{}
  113. M. Sultan, M.H. Schulz, H. Richard, A. Magen, A. Klingenhoff, M.
  Scherf, M. Seifert, T. Borodina, A. Soldatov, D. Parkhomchuk, D.
  Schmidt, S. O'Keeffe, S. Haas, M. Vingron, H. Lehrach, and M.-L. Yaspo:
  ``A Global View of Gene Activity and Alternative Splicing by Deep
  Sequencing of the Human Transcriptome.'' \emph{Science}. vol. 321, no.
  5891, pp. 956--960, 2008.
  
  \hypertarget{ref-Guffanti2009}{}
  114. A. Guffanti, M. Iacono, P. Pelucchi, N. Kim, G. Soldà, L.J. Croft,
  R.J. Taft, E. Rizzi, M. Askarian-Amiri, R.J. Bonnal, M. Callari, F.
  Mignone, G. Pesole, G. Bertalot, L. Bernardi, A. Albertini, C. Lee, J.S.
  Mattick, I. Zucchi, and G. De Bellis: ``A transcriptional sketch of a
  primary human breast cancer by 454 deep sequencing.'' \emph{BMC
  Genomics}. vol. 10, no. 1, pp. 163, 2009.
  
  \hypertarget{ref-Auffray2009}{}
  115. C. Auffray, Z. Chen, and L. Hood: ``Systems medicine: the future of
  medical genomics and healthcare.'' \emph{Genome medicine}. vol. 1, no.
  1, pp. 2, 2009.
  
  \hypertarget{ref-Horner2009}{}
  116. D.S. Horner, G. Pavesi, T. Castrignano', P.D.O. de Meo, S. Liuni,
  M. Sammeth, E. Picardi, and G. Pesole: ``Bioinformatics approaches for
  genomics and post genomics applications of next-generation sequencing.''
  \emph{Briefings in Bioinformatics}. vol. 11, no. 2, pp. 181--197, 2009.
  
  \hypertarget{ref-Mardis2008}{}
  117. E.R. Mardis: ``The impact of next-generation sequencing technology
  on genetics.'' \emph{Trends in Genetics}. vol. 24, no. 3, pp. 133--141,
  2008.
  
  \hypertarget{ref-Bentley2006}{}
  118. D.R. Bentley: ``Whole-genome re-sequencing.'' \emph{Current Opinion
  in Genetics and Development}. vol. 16, no. 6, pp. 545--552, 2006.
  
  \hypertarget{ref-Li2008}{}
  119. H. Li, J. Ruan, R. Durbin, H. Li, J. Ruan, and R. Durbin: ``Mapping
  short DNA sequencing reads and calling variants using mapping quality
  scores Mapping short DNA sequencing reads and calling variants using
  mapping quality scores.'' pp. 1851--1858, 2008.
  
  \hypertarget{ref-Korbel2009}{}
  120. J.O. Korbel, A.E. Urban, J.P. Affourtit, B. Godwin, F. Grubert,
  J.F. Simons, P.M. Kim, D. Palejev, J. Nicholas, L. Du, B.E. Taillon, Z.
  Chen, A. Tanzer, a C. Eugenia, J. Chi, F. Yang, N.P. Carter, M.E.
  Hurles, S.M. Weissman, T.T. Harkins, M.B. Gerstein, M. Egholm, and M.
  Snyder: ``Paired-End Mapping Reveals Extensive Structural Variation in
  the Human Genome.'' \emph{October}. vol. 318, no. 5849, pp. 420--426,
  2009.
  
  \hypertarget{ref-Cock2009}{}
  121. P.J.A. Cock, C.J. Fields, N. Goto, M.L. Heuer, and P.M. Rice: ``The
  Sanger FASTQ file format for sequences with quality scores, and the
  Solexa/Illumina FASTQ variants.'' \emph{Nucleic Acids Research}. vol.
  38, no. 6, pp. 1767--1771, 2009.
  
  \hypertarget{ref-Flicek2009}{}
  122. P. Flicek and E. Birney: ``Sense from sequence reads: methods for
  alignment and assembly.'' \emph{Nature methods}. vol. 6, no. 11 Suppl,
  pp. S6--S12, 2009.
  
  \hypertarget{ref-Nielsen2011}{}
  123. R. Nielsen, J.S. Paul, A. Albrechtsen, and Y.S. Song: ``Genotype
  and SNP calling from next-generation sequencing data.'' \emph{Nature
  reviews. Genetics}. vol. 12, no. 6, pp. 443--51, 2011.
  
  \hypertarget{ref-Langmead2012}{}
  124. B. Langmead and S.L. Salzberg: ``Fast gapped-read alignment with
  Bowtie 2.'' \emph{Nature Methods}. vol. 9, no. 4, pp. 357--359, 2012.
  
  \hypertarget{ref-Treangen2013}{}
  125. T.J. Treangen and S.L. Salzberg: ``Repetitive DNA and
  next-generation sequencing: computational challenges and solutions.''
  \emph{Nat Rev Genet.} vol. 13, no. 1, pp. 36--46, 2013.
  
  \hypertarget{ref-Langmead2009}{}
  126. B. Langmead, C. Trapnell, M. Pop, and S. Salzberg: ``Ultrafast and
  memory-efficient alignment of short DNA sequences to the human genome.''
  \emph{Genome biology}. vol. 10, no. 3, pp. R25, 2009.
  
  \hypertarget{ref-Su2014}{}
  127. Z. Su, P.P. Łabaj, S.S. Li, J. Thierry-Mieg, D. Thierry-Mieg, W.
  Shi, C. Wang, G.P. Schroth, R. a Setterquist, J.F. Thompson, W.D. Jones,
  W. Xiao, W. Xu, R.V. Jensen, R. Kelly, J. Xu, A. Conesa, C. Furlanello,
  H.H. Gao, H. Hong, N. Jafari, S. Letovsky, Y. Liao, F. Lu, E.J. Oakeley,
  Z. Peng, C.A. Praul, J. Santoyo-Lopez, A. Scherer, T. Shi, G.K. Smyth,
  F. Staedtler, P. Sykacek, X.-X. Tan, E.A. Thompson, J. Vandesompele,
  M.D. Wang, J.J.J. Wang, R.D. Wolfinger, J. Zavadil, S.S. Auerbach, W.
  Bao, H. Binder, T. Blomquist, M.H. Brilliant, P.R. Bushel, W. Cai, J.G.
  Catalano, C.-W. Chang, T. Chen, G. Chen, R. Chen, M. Chierici, T.-M.
  Chu, D.-A. Clevert, Y. Deng, A. Derti, V. Devanarayan, Z. Dong, J.
  Dopazo, T. Du, H. Fang, Y. Fang, M. Fasold, A. Fernandez, M. Fischer, P.
  Furió-Tari, J.C. Fuscoe, F. Caimet, S. Gaj, J. Gandara, H.H. Gao, W. Ge,
  Y. Gondo, B. Gong, M. Gong, Z. Gong, B. Green, C. Guo, L.-W.L. Guo,
  L.-W.L. Guo, J. Hadfield, J. Hellemans, S. Hochreiter, M. Jia, M. Jian,
  C.D. Johnson, S. Kay, J. Kleinjans, S. Lababidi, S. Levy, Q.-Z. Li, L.
  Li, P. Li, Y. Li, H. Li, J. Li, S.S. Li, S.M. Lin, F.J. López, X. Lu, H.
  Luo, X. Ma, J. Meehan, D.B. Megherbi, N. Mei, B. Mu, B. Ning, A. Pandey,
  J. Pérez-Florido, R.G. Perkins, R. Peters, J.H. Phan, M. Pirooznia, F.
  Qian, T. Qing, L. Rainbow, P. Rocca-Serra, L. Sambourg, S.-A. Sansone,
  S. Schwartz, R. Shah, J. Shen, T.M. Smith, O. Stegle, N. Stralis-Pavese,
  E. Stupka, Y. Suzuki, L.T. Szkotnicki, M. Tinning, B. Tu, J. van Delft,
  A. Vela-Boza, E. Venturini, S.J. Walker, L. Wan, W. Wang, J.J.J. Wang,
  J.J.J. Wang, E.D. Wieben, J.C. Willey, P.-Y. Wu, J. Xuan, Y. Yang, Z.
  Ye, Y. Yin, Y. Yu, Y.-C. Yuan, J. Zhang, K.K. Zhang, W.W. Zhang, W.W.
  Zhang, Y. Zhang, C. Zhao, Y. Zheng, Y. Zhou, P. Zumbo, W. Tong, D.P.
  Kreil, C.E. Mason, and L. Shi: ``A comprehensive assessment of RNA-seq
  accuracy, reproducibility and information content by the Sequencing
  Quality Control Consortium.'' \emph{Nature Biotechnology}. vol. 32, no.
  9, pp. 903--14, 2014.
  
  \hypertarget{ref-Ruffalo2011}{}
  128. M. Ruffalo, T. Laframboise, and M. Koyutürk: ``Comparative analysis
  of algorithms for next-generation sequencing read alignment.''
  \emph{Bioinformatics}. vol. 27, no. 20, pp. 2790--2796, 2011.
  
  \hypertarget{ref-Thankaswamy-Kosalai2017}{}
  129. S. Thankaswamy-Kosalai, P. Sen, and I. Nookaew: ``Evaluation and
  assessment of read-mapping by multiple next-generation sequencing
  aligners based on genome-wide characteristics.'' \emph{Genomics}. 2017.
  
  \hypertarget{ref-Bao2011}{}
  130. S. Bao, R. Jiang, W. Kwan, B. Wang, X. Ma, and Y.-Q. Song:
  ``Evaluation of next-generation sequencing software in mapping and
  assembly.'' \emph{Journal of Human Genetics}. vol. 56, no. May, pp.
  406--414, 2011.
  
  \hypertarget{ref-DePristo2011}{}
  131. M.A. DePristo, E. Banks, R. Poplin, K.V. Garimella, J.R. Maguire,
  C. Hartl, A.A. Philippakis, G. del Angel, M.A. Rivas, M. Hanna, A.
  McKenna, T.J. Fennell, A.M. Kernytsky, A.Y. Sivachenko, K. Cibulskis,
  S.B. Gabriel, D. Altshuler, M.J. Daly, S. Keenan, M. Komorowska, E.
  Kulesha, I. Longden, T. Maurel, W. McLaren, M. Muffato, R. Nag, B.
  Overduin, M. Pignatelli, B. Pritchard, and E. Pritchard: ``A framework
  for variation discovery and genotyping using next-generation DNA
  sequencing data.'' \emph{Nature Genetics}. vol. 43, no. 5, pp. 491--498,
  2011.
  
  \hypertarget{ref-Lunter2011}{}
  132. G. Lunter and M. Goodson: ``Stampy: A statistical algorithm for
  sensitive and fast mapping of Illumina sequence reads.'' \emph{Genome
  Research}. vol. 21, no. 6, pp. 936--939, 2011.
  
  \hypertarget{ref-Li2009}{}
  133. H. Li, B. Handsaker, A. Wysoker, T. Fennell, J. Ruan, N. Homer, G.
  Marth, G. Abecasis, and R. Durbin: ``The Sequence Alignment/Map format
  and SAMtools.'' \emph{Bioinformatics}. vol. 25, no. 16, pp. 2078--2079,
  2009.
  
  \hypertarget{ref-McKenna2010}{}
  134. A. McKenna, M. Hanna, E. Banks, A. Sivachenko, K. Cibulskis, A.
  Kernytsky, K. Garimella, D. Altshuler, S. Gabriel, M. Daly, and M.A.
  DePristo: ``The Genome Analysis Toolkit: a MapReduce framework for
  analyzing next-generation DNA sequencing data.'' \emph{Genome research}.
  vol. 20, no. 9, pp. 1297--303, 2010.
  
  \hypertarget{ref-Hwang2015}{}
  135. S. Hwang, E. Kim, I. Lee, and E.M. Marcotte: ``Systematic
  comparison of variant calling pipelines using gold standard personal
  exome variants.'' \emph{Scientific Reports}. vol. 5, no. December, pp.
  17875, 2015.
  
  \hypertarget{ref-Baes2014}{}
  136. C.F. Baes, M.A. Dolezal, J.E. Koltes, B. Bapst, E. Fritz-Waters, S.
  Jansen, C. Flury, H. Signer-Hasler, C. Stricker, R. Fernando, R. Fries,
  J. Moll, D.J. Garrick, J.M. Reecy, and B. Gredler: ``Evaluation of
  variant identification methods for whole genome sequencing data in dairy
  cattle.'' \emph{BMC genomics}. vol. 15, no. 1, pp. 948, 2014.
  
  \hypertarget{ref-ORawe2013}{}
  137. J. O'Rawe, T. Jiang, G. Sun, Y. Wu, W. Wang, J. Hu, P. Bodily, L.
  Tian, H. Hakonarson, W.E. Johnson, Z. Wei, K. Wang, and G.J. Lyon: ``Low
  concordance of multiple variant-calling pipelines: practical
  implications for exome and genome sequencing.'' \emph{Genome Medicine}.
  vol. 5, no. 3, pp. 28, 2013.
  
  \hypertarget{ref-Rosenfeld2012}{}
  138. J.A. Rosenfeld, C.E. Mason, T.M. Smith, C. Wallin, and M. Diekhans:
  ``Limitations of the Human Reference Genome for Personalized Genomics.''
  \emph{PLoS ONE}. vol. 7, no. 7, pp. e40294, 2012.
  
  \hypertarget{ref-Gonzaga-Jauregui2012}{}
  139. C. Gonzaga-Jauregui, J.R. Lupski, and R.A. Gibbs: ``Human genome
  sequencing in health and disease.'' \emph{Annual review of medicine}.
  vol. 63, pp. 35--61, 2012.
  
  \hypertarget{ref-1000GenomesProjectConsortium2015}{}
  140. T.1.G.P. 1000 Genomes Project Consortium, A. Auton, L.D. Brooks,
  R.M. Durbin, E.P. Garrison, H.M. Kang, J.O. Korbel, J.L. Marchini, S.
  McCarthy, G.A. McVean, and G.R. Abecasis: ``A global reference for human
  genetic variation.'' \emph{Nature}. vol. 526, no. 7571, pp. 68--74,
  2015.
  
  \hypertarget{ref-Lek2016}{}
  141. M. Lek, K.J. Karczewski, E.V. Minikel, K.E. Samocha, E. Banks, T.
  Fennell, A.H. O'Donnell-Luria, J.S. Ware, A.J. Hill, B.B. Cummings, T.
  Tukiainen, D.P. Birnbaum, J.A. Kosmicki, L.E. Duncan, K. Estrada, F.
  Zhao, J. Zou, E. Pierce-Hoffman, J. Berghout, D.N. Cooper, N. Deflaux,
  M. DePristo, R. Do, J. Flannick, M. Fromer, L. Gauthier, J. Goldstein,
  N. Gupta, D. Howrigan, A. Kiezun, M.I. Kurki, A.L. Moonshine, P.
  Natarajan, L. Orozco, G.M. Peloso, R. Poplin, M.A. Rivas, V.
  Ruano-Rubio, S.A. Rose, D.M. Ruderfer, K. Shakir, P.D. Stenson, C.
  Stevens, B.P. Thomas, G. Tiao, M.T. Tusie-Luna, B. Weisburd, H.-H. Won,
  D. Yu, D.M. Altshuler, D. Ardissino, M. Boehnke, J. Danesh, S. Donnelly,
  R. Elosua, J.C. Florez, S.B. Gabriel, G. Getz, S.J. Glatt, C.M. Hultman,
  S. Kathiresan, M. Laakso, S. McCarroll, M.I. McCarthy, D. McGovern, R.
  McPherson, B.M. Neale, A. Palotie, S.M. Purcell, D. Saleheen, J.M.
  Scharf, P. Sklar, P.F. Sullivan, J. Tuomilehto, M.T. Tsuang, H.C.
  Watkins, J.G. Wilson, M.J. Daly, D.G. MacArthur, and D.G. Exome
  Aggregation Consortium: ``Analysis of protein-coding genetic variation
  in 60,706 humans.'' \emph{Nature}. vol. 536, no. 7616, pp. 285--91,
  2016.
  
  \hypertarget{ref-McLaren2016}{}
  142. W. McLaren, L. Gil, S.E. Hunt, H.S. Riat, G.R.S. Ritchie, A.
  Thormann, P. Flicek, and F. Cunningham: ``The Ensembl Variant Effect
  Predictor.'' \emph{Genome biology}. vol. 17, no. 1, pp. 122, 2016.
  
  \hypertarget{ref-Cingolani2012}{}
  143. P. Cingolani, A. Platts, L.L. Wang, M. Coon, T. Nguyen, L. Wang,
  S.J. Land, X. Lu, and D.M. Ruden: ``A program for annotating and
  predicting the effects of single nucleotide polymorphisms, SnpEff.''
  \emph{Fly}. vol. 6, no. 2, pp. 80--92, 2012.
  
  \hypertarget{ref-Wang2010}{}
  144. K. Wang, M. Li, and H. Hakonarson: ``ANNOVAR: functional annotation
  of genetic variants from high-throughput sequencing data.''
  \emph{Nucleic Acids Research}. vol. 38, no. 16, pp. e164--e164, 2010.
  
  \hypertarget{ref-Kumar2009}{}
  145. P. Kumar, S. Henikoff, and P.C. Ng: ``Predicting the effects of
  coding non-synonymous variants on protein function using the SIFT
  algorithm.'' \emph{Nature protocols}. vol. 4, no. 7, pp. 1073--1081,
  2009.
  
  \hypertarget{ref-Choi2012}{}
  146. Y. Choi, G.E. Sims, S. Murphy, J.R. Miller, and A.P. Chan:
  ``Predicting the Functional Effect of Amino Acid Substitutions and
  Indels.'' \emph{PLoS ONE}. vol. 7, no. 10, 2012.
  
  \hypertarget{ref-Mi2017}{}
  147. H. Mi, X. Huang, A. Muruganujan, H. Tang, C. Mills, D. Kang, and
  P.D. Thomas: ``PANTHER version 11: expanded annotation data from Gene
  Ontology and Reactome pathways, and data analysis tool enhancements.''
  \emph{Nucleic Acids Research}. vol. 45, no. D1, pp. D183--D189, 2017.
  
  \hypertarget{ref-Kohler2014}{}
  148. S. Köhler, S.C. Doelken, C.J. Mungall, S. Bauer, H.V. Firth, I.
  Bailleul-Forestier, G.C.M. Black, D.L. Brown, M. Brudno, J. Campbell,
  D.R. FitzPatrick, J.T. Eppig, A.P. Jackson, K. Freson, M. Girdea, I.
  Helbig, J.A. Hurst, J. Jähn, L.G. Jackson, A.M. Kelly, D.H. Ledbetter,
  S. Mansour, C.L. Martin, C. Moss, A. Mumford, W.H. Ouwehand, S.-M. Park,
  E.R. Riggs, R.H. Scott, S. Sisodiya, S. Van Vooren, R.J. Wapner, A.O.M.
  Wilkie, C.F. Wright, A.T. Vulto-van Silfhout, N. de Leeuw, B.B.A. de
  Vries, N.L. Washingthon, C.L. Smith, M. Westerfield, P. Schofield, B.J.
  Ruef, G.V. Gkoutos, M. Haendel, D. Smedley, S.E. Lewis, and P.N.
  Robinson: ``The Human Phenotype Ontology project: linking molecular
  biology and disease through phenotype data.'' \emph{Nucleic acids
  research}. vol. 42, no. Database issue, pp. D966--74, 2014.
  
  \hypertarget{ref-McCarthy2014}{}
  149. D.J. McCarthy, P. Humburg, A. Kanapin, M. a Rivas, K. Gaulton,
  J.-B. Cazier, and P. Donnelly: ``Choice of transcripts and software has
  a large effect on variant annotation.'' \emph{Genome medicine}. vol. 6,
  no. 3, pp. 26, 2014.
  
  \hypertarget{ref-Zhao2015}{}
  150. S. Zhao and B. Zhang: ``A comprehensive evaluation of ensembl,
  RefSeq, and UCSC annotations in the context of RNA-seq read mapping and
  gene quantification.'' \emph{BMC genomics}. vol. 16, no. 1, pp. 97,
  2015.
  
  \hypertarget{ref-Pruitt2009}{}
  151. K.D. Pruitt, J. Harrow, R.A. Harte, C. Wallin, M. Diekhans, D.R.
  Maglott, S. Searle, C.M. Farrell, J.E. Loveland, B.J. Ruef, E. Hart,
  M.M. Suner, M.J. Landrum, B. Aken, S. Ayling, R. Baertsch, J.
  Fernandez-Banet, J.L. Cherry, V. Curwen, M. DiCuccio, M. Kellis, J. Lee,
  M.F. Lin, M. Schuster, A. Shkeda, C. Amid, G. Brown, O. Dukhanina, A.
  Frankish, J. Hart, B.L. Maidak, J. Mudge, M.R. Murphy, T. Murphy, J.
  Rajan, B. Rajput, L.D. Riddick, C. Snow, C. Steward, D. Webb, J.A.
  Weber, L. Wilming, W. Wu, E. Birney, D. Haussler, T. Hubbard, J. Ostell,
  R. Durbin, and D. Lipman: ``The consensus coding sequence (CCDS)
  project: Identifying a common protein-coding gene set for the human and
  mouse genomes.'' \emph{Genome Research}. vol. 19, no. 7, pp. 1316--1323,
  2009.
  
  \hypertarget{ref-Salgado2016}{}
  152. D. Salgado, M.I. Bellgard, J.P. Desvignes, and C. B??roud: ``How to
  Identify Pathogenic Mutations among All Those Variations: Variant
  Annotation and Filtration in the Genome Sequencing Era.'' \emph{Human
  Mutation}. vol. 37, no. 12, pp. 1272--1282, 2016.
  
  \hypertarget{ref-Schatz2013}{}
  153. M.C. Schatz and B. Langmead: ``The DNA Data Deluge: Fast, efficient
  genome sequencing machines are spewing out more data than geneticists
  can analyze.'' \emph{IEEE spectrum}. vol. 50, no. 7, pp. 26--33, 2013.
  
  \hypertarget{ref-McPherson2009}{}
  154. J.D. McPherson: ``Next-generation gap.'' \emph{Nature Methods}.
  vol. 6, no. 11s, pp. S2--S5, 2009.
  
  \hypertarget{ref-Amberger2011}{}
  155. J. Amberger, C. Bocchini, and A. Hamosh: ``A new face and new
  challenges for Online Mendelian Inheritance in Man (OMIM).'' \emph{Human
  Mutation}. vol. 32, no. 5, pp. 564--567, 2011.
  
  \hypertarget{ref-Ng}{}
  156. S.B. Ng, K.J. Buckingham, C. Lee, A.W. Bigham, H.K. Tabor, K.M.
  Dent, C.D. Huff, P.T. Shannon, E.W. Jabs, D.A. Nickerson, J. Shendure,
  and M.J. Bamshad: ``Exome sequencing identifies the cause of a Mendelian
  disorder.''
  
  \hypertarget{ref-Pelak2010}{}
  157. K. Pelak, K.V. Shianna, D. Ge, J.M. Maia, M. Zhu, J.P. Smith, E.T.
  Cirulli, J. Fellay, S.P. Dickson, C.E. Gumbs, E.L. Heinzen, A.C. Need,
  E.K. Ruzzo, A. Singh, C.R. Campbell, L.K. Hong, K.A. Lornsen, A.M.
  McKenzie, N.L.M. Sobreira, J.E. Hoover-Fong, J.D. Milner, R. Ottman,
  B.F. Haynes, J.J. Goedert, and D.B. Goldstein: ``The characterization of
  twenty sequenced human genomes.'' \emph{PLoS genetics}. vol. 6, no. 9,
  pp. e1001111, 2010.
  
  \hypertarget{ref-Robinson2014}{}
  158. P.N. Robinson, S. Köhler, A. Oellrich, S.M.G. Sanger Mouse Genetics
  Project, K. Wang, C.J. Mungall, S.E. Lewis, N. Washington, S. Bauer, D.
  Seelow, P. Krawitz, C. Gilissen, M. Haendel, and D. Smedley: ``Improved
  exome prioritization of disease genes through cross-species phenotype
  comparison.'' \emph{Genome research}. vol. 24, no. 2, pp. 340--8, 2014.
  
  \hypertarget{ref-Aken2017}{}
  159. B.L. Aken, P. Achuthan, W. Akanni, M.R. Amode, F. Bernsdorff, J.
  Bhai, K. Billis, D. Carvalho-Silva, C. Cummins, P. Clapham, L. Gil, C.G.
  Girón, L. Gordon, T. Hourlier, S.E. Hunt, S.H. Janacek, T. Juettemann,
  S. Keenan, M.R. Laird, I. Lavidas, T. Maurel, W. McLaren, B. Moore, D.N.
  Murphy, R. Nag, V. Newman, M. Nuhn, C.K. Ong, A. Parker, M. Patricio,
  H.S. Riat, D. Sheppard, H. Sparrow, K. Taylor, A. Thormann, A. Vullo, B.
  Walts, S.P. Wilder, A. Zadissa, M. Kostadima, F.J. Martin, M. Muffato,
  E. Perry, M. Ruffier, D.M. Staines, S.J. Trevanion, F. Cunningham, A.
  Yates, D.R. Zerbino, and P. Flicek: ``Ensembl 2017.'' \emph{Nucleic
  acids research}. vol. 45, no. D1, pp. D635--D642, 2017.
  
  \hypertarget{ref-Chang2007}{}
  160. Y.-F. Chang, J.S. Imam, and M.F. Wilkinson: ``The Nonsense-Mediated
  Decay RNA Surveillance Pathway.'' \emph{Annual Review of Biochemistry}.
  vol. 76, no. 1, pp. 51--74, 2007.
  
  \hypertarget{ref-Baker2004}{}
  161. K.E. Baker and R. Parker: ``Nonsense-mediated mRNA decay:
  terminating erroneous gene expression.'' \emph{Current opinion in cell
  biology}. vol. 16, no. 3, pp. 293--9, 2004.
  
  \hypertarget{ref-Adzhubei2010}{}
  162. I.A. Adzhubei, S. Schmidt, L. Peshkin, V.E. Ramensky, A.
  Gerasimova, P. Bork, A.S. Kondrashov, and S.R. Sunyaev: ``A method and
  server for predicting damaging missense mutations.'' \emph{Nature
  methods}. vol. 7, no. 4, pp. 248--9, 2010.
  
  \hypertarget{ref-Imai2015}{}
  163. Y. Imai, H. Morita, N. Takeda, F. Miya, H. Hyodo, D. Fujita, T.
  Tajima, T. Tsunoda, R. Nagai, M. Kubo, and I. Komuro: ``A deletion
  mutation in myosin heavy chain 11 causing familial thoracic aortic
  dissection in two Japanese pedigrees.'' \emph{International Journal of
  Cardiology}. vol. 195, pp. 290--292, 2015.
  
  \hypertarget{ref-Lee2011}{}
  164. B. Lee, I. Park, S. Jin, H. Choi, J.T. Kwon, J. Kim, J. Jeong,
  B.-N. Cho, E.M. Eddy, and C. Cho: ``Impaired spermatogenesis and
  fertility in mice carrying a mutation in the Spink2 gene expressed
  predominantly in testes.'' \emph{The Journal of biological chemistry}.
  vol. 286, no. 33, pp. 29108--17, 2011.
  
  \hypertarget{ref-Aarabi2014}{}
  165. M. Aarabi, H. Balakier, S. Bashar, S.I. Moskovtsev, P. Sutovsky,
  C.L. Librach, and R. Oko: ``Sperm content of postacrosomal WW binding
  protein is related to fertilization outcomes in patients undergoing
  assisted reproductive technology.'' \emph{Fertility and Sterility}. vol.
  102, no. 2, pp. 440--447, 2014.
  
  \hypertarget{ref-Aarabi2014a}{}
  166. M. Aarabi, H. Balakier, S. Bashar, S.I. Moskovtsev, P. Sutovsky,
  C.L. Librach, and R. Oko: ``Sperm-derived WW domain-binding protein,
  PAWP, elicits calcium oscillations and oocyte activation in humans and
  mice.'' \emph{FASEB journal : official publication of the Federation of
  American Societies for Experimental Biology}. vol. 28, no. 10, pp.
  4434--40, 2014.
  
  \hypertarget{ref-Marnef2010}{}
  167. A. Marnef, M. Maldonado, A. Bugaut, S. Balasubramanian, M. Kress,
  D. Weil, and N. Standart: ``Distinct functions of maternal and somatic
  Pat1 protein paralogs.'' \emph{RNA}. vol. 16, no. 11, pp. 2094--2107,
  2010.
  
  \hypertarget{ref-Nakamura2010}{}
  168. Y. Nakamura, K.J. Tanaka, M. Miyauchi, L. Huang, M. Tsujimoto, and
  K. Matsumoto: ``Translational repression by the oocyte-specific protein
  P100 in Xenopus.'' \emph{Developmental biology}. vol. 344, no. 1, pp.
  272--83, 2010.
  
  \hypertarget{ref-Ivliev2012}{}
  169. A.E. Ivliev, P.A.C. 't Hoen, W.M.C. van Roon-Mom, D.J.M. Peters,
  and M.G. Sergeeva: ``Exploring the Transcriptome of Ciliated Cells Using
  In Silico Dissection of Human Tissues.'' \emph{PLoS ONE}. vol. 7, no. 4,
  pp. e35618, 2012.
  
  \hypertarget{ref-Smith2008}{}
  170. T.F. Smith: ``Diversity of WD-Repeat proteins.'' The coronin family
  of proteins. pp. 20--30. \emph{Springer New York}, New York, NY (2008).
  
  \hypertarget{ref-Broadhead2006}{}
  171. R. Broadhead, H.R. Dawe, H. Farr, S. Griffiths, S.R. Hart, N.
  Portman, M.K. Shaw, M.L. Ginger, S.J. Gaskell, P.G. McKean, and K. Gull:
  ``Flagellar motility is required for the viability of the bloodstream
  trypanosome.'' \emph{Nature}. vol. 440, no. 7081, pp. 224--227, 2006.
  
  \hypertarget{ref-Subota2014}{}
  172. I. Subota, D. Julkowska, L. Vincensini, N. Reeg, J. Buisson, T.
  Blisnick, D. Huet, S. Perrot, J. Santi-Rocca, M. Duchateau, V. Hourdel,
  J.-C. Rousselle, N. Cayet, A. Namane, J. Chamot-Rooke, and P. Bastin:
  ``Proteomic Analysis of Intact Flagella of Procyclic
  \textless{}i\textgreater{}Trypanosoma brucei\textless{}/i\textgreater{}
  Cells Identifies Novel Flagellar Proteins with Unique Sub-localization
  and Dynamics.'' \emph{Molecular \& Cellular Proteomics}. vol. 13, no. 7,
  pp. 1769--1786, 2014.


  % Index?

\end{document}

