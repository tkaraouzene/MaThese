% This is the Reed College LaTeX thesis template. Most of the work
% for the document class was done by Sam Noble (SN), as well as this
% template. Later comments etc. by Ben Salzberg (BTS). Additional
% restructuring and APA support by Jess Youngberg (JY).
% Your comments and suggestions are more than welcome; please email
% them to cus@reed.edu
%
% See http://web.reed.edu/cis/help/latex.html for help. There are a
% great bunch of help pages there, with notes on
% getting started, bibtex, etc. Go there and read it if you're not
% already familiar with LaTeX.
%
% Any line that starts with a percent symbol is a comment.
% They won't show up in the document, and are useful for notes
% to yourself and explaining commands.
% Commenting also removes a line from the document;
% very handy for troubleshooting problems. -BTS

% As far as I know, this follows the requirements laid out in
% the 2002-2003 Senior Handbook. Ask a librarian to check the
% document before binding. -SN

%%
%% Preamble
%%
% \documentclass{<something>} must begin each LaTeX document
\documentclass[12pt,twoside]{ugathesis}
% Packages are extensions to the basic LaTeX functions. Whatever you
% want to typeset, there is probably a package out there for it.
% Chemistry (chemtex), screenplays, you name it.
% Check out CTAN to see: http://www.ctan.org/
%%
\usepackage{graphicx,latexsym}
\usepackage[french]{babel}
\usepackage{amsmath}
\usepackage{amssymb,amsthm}
\usepackage[dvipsnames]{xcolor} % tk: for more color
\usepackage{xcolor}
\usepackage{eso-pic}
\usepackage{longtable,booktabs,setspace}
\usepackage{chemarr} %% Useful for one reaction arrow, useless if you're not a chem major
\usepackage[hyphens]{url}
\usepackage{pdfpages}
\usepackage{tikz}
\usetikzlibrary{calc}
% Added by CII
\usepackage{hyperref}
\usepackage{lmodern}
\usepackage{float}
\floatplacement{figure}{H}
% End of CII addition
\usepackage{rotating}
\usepackage{upgreek} % tk : pour pouvoir utiliser le symbole µ droit (pas en itallic)
\usepackage{pdfpages} % tk : pour pouvoir insérer des fichiers pdf dans le corp de texte
\usepackage{lscape} % tk : pour pouvoir insérer des images au format paysage
\newcommand{\blandscape}{\begin{landscape}}
\newcommand{\elandscape}{\end{landscape}}
\usepackage[utf8]{inputenc}

% Next line commented out by CII
%%% \usepackage{natbib}
% Comment out the natbib line above and uncomment the following two lines to use the new
% biblatex-chicago style, for Chicago A. Also make some changes at the end where the
% bibliography is included.
%\usepackage{biblatex-chicago}
%\bibliography{thesis}


% Added by CII (Thanks, Hadley!)
% Use ref for internal links
\renewcommand{\hyperref}[2][???]{\autoref{#1}}
\def\chapterautorefname{Chapter}
\def\sectionautorefname{Section}
\def\subsectionautorefname{Subsection}
% End of CII addition

% Added by CII
\usepackage{caption}
\captionsetup{width=5in}
% End of CII addition

% \usepackage{times} % other fonts are available like times, bookman, charter, palatino


% To pass between YAML and LaTeX the dollar signs are added by CII
\title{THÈSE}
\author{Thomas Karaouzene}
\lab{Génétique, Epigénétique et Thérapies de l'Infertilité (GETI) et
Techniques de l'Ingénierie Médicale et de la Complexité - Informatique,
Mathématiques et Applications de Grenoble (TIMC-IMAG)}
\date{07 novembre 2017}
\division{Mathematics and Natural Sciences}
\advisor{Pierre Ray}
%If you have two advisors for some reason, you can use the following
% Uncommented out by CII
\altadvisor{Nicolas Thierry-Mieg}
% End of CII addition
%\institution{}
%\degree{}

%%% Remember to use the correct department!
\department{Ingénierie de la Santé, de la Cognition et Environnement (EDISCE)}
% if you're writing a thesis in an interdisciplinary major,
% uncomment the line below and change the text as appropriate.
% check the Senior Handbook if unsure.
%\thedivisionof{The Established Interdisciplinary Committee for}
% if you want the approval page to say "Approved for the Committee",
% uncomment the next line
%\approvedforthe{Committee}

% Added by CII
%%% Copied from knitr
%% maxwidth is the original width if it's less than linewidth
%% otherwise use linewidth (to make sure the graphics do not exceed the margin)
\makeatletter
\def\maxwidth{ %
  \ifdim\Gin@nat@width>\linewidth
    \linewidth
  \else
    \Gin@nat@width
  \fi
}
\makeatother

\renewcommand{\contentsname}{Table of Contents}
% End of CII addition

\setlength{\parskip}{0pt}

% Added by CII
  %\setlength{\parskip}{\baselineskip}
  \usepackage[parfill]{parskip}

\providecommand{\tightlist}{%
  \setlength{\itemsep}{0pt}\setlength{\parskip}{0pt}}

\Acknowledgements{

}

\Dedication{

}

\Preface{

}

\Abstract{

}

	\usepackage{tikz}
% End of CII addition
%%
%% End Preamble
%%
%

\begin{document}

% Everything below added by CII
  \maketitle

\frontmatter % this stuff will be roman-numbered
\pagestyle{empty} % this removes page numbers from the frontmatter



  \hypersetup{linkcolor=black}
  \setcounter{tocdepth}{3}
  \tableofcontents

  \listoftables

  \listoffigures



\mainmatter % here the regular arabic numbering starts
\pagestyle{fancyplain} % turns page numbering back on

\chapter{thesisdown::thesis\_epub:
default}\label{thesisdownthesis_epub-default}

\chapter*{Résumé}\label{resume}
\addcontentsline{toc}{chapter}{Résumé}

\newpage

Ces dix dernières années, l'investigation des maladies génétiques a été
bouleversée par l'émergence des techniques de séquençage haut-débit.
Celles-ci permettent désormais de ne plus séquencer les gènes un par un,
mais d'avoir accès à l'intégralité de la séquence génomique ou
transcriptomique d'un individu. La difficulté devient alors d'identifier
les variants causaux parmi une multitude d'artefacts techniques et de
variants bénins, pour ensuite comprendre la physiopathologie des gènes
identifiés.

L'application du séquençage haut débit est particulièrement prometteuse
dans le champ de la génétique de l'infertilité masculine car il s'agit
d'une pathologie dont l'étiologie est souvent génétique, qui est
génétiquement très hétérogène et pour laquelle peu de gènes ont été
identifiés. Mon travail de thèse est donc centré sur la l'infertilité et
comporte deux parties majeures : l'analyse des données issues du
séquençage haut débit d'homme infertiles et de modèles animaux et la
caractérisation moléculaire d'un phénotype spécifique d'infertilité, la
globozoospermie.

Le nombre de variants identifiés dans le cadre d'un séquençage exomique
pouvant s'élever à plusieurs dizaines de milliers, l'utilisation d'un
outil informatique performant est indispensable. Pour arriver à une
liste de variants suffisamment restreinte pour pouvoir être interprétée,
plusieurs traitements sont nécessaires. Ainsi, j'ai développé un
pipeline d'analyse de données issues de séquençage haut-débit effectuant
de manière successive l'intégralité des étapes de l'analyse
bio-informatique, c'est-à-dire l'alignement des \emph{reads} sur un
génome de référence, l'appel des génotypes, l'annotation des variants
obtenus ainsi que le filtrage de ceux considérés comme non pertinents
dans le contexte de l'analyse. L'ensemble de ces étapes étant
interdépendantes, les réaliser au sein du même pipeline permet de mieux
les calibrer pour ainsi réduire le nombre d'erreurs générées. Ce
pipeline a été utilisé dans cinq études au sein du laboratoire, et a
permis l'identification de variants impactant des gènes candidats
prometteurs pouvant expliquer le phénotype d'infertilité des patients.
L'ensemble des variants retenus ont ensuite pu être validés
expérimentalement.

J'ai également pris part aux investigations génétiques et moléculaires
permettant la caractérisation du gène \emph{DPY19L2}, identifié au
laboratoire et dont la délétion homozygote entraine une globozoospermie,
caractérisée par la présence dans l'éjaculât de spermatozoïdes à tête
ronde dépourvus d'acrosome. Pour cela, j'ai contribué à caractériser les
mécanismes responsables de cette délétion récurrente, puis, en utilisant
le modèle murin \emph{Dpy19l2 knock out} (KO) mimant le phénotype
humain, j'ai réalisé une étude comparative des transcriptomes
testiculaires de souris sauvages et de souris KO
\emph{Dpy19l2\textsuperscript{-/-}}. Cette étude a ainsi permis de
mettre en évidence la dérégulation de 76 gènes chez la souris KO. Parmi
ceux-ci, 23 sont impliqués dans la liaison d'acides nucléiques et de
protéines, pouvant ainsi expliquer les défauts d'ancrage de l'acrosome
au noyau chez les spermatozoïdes globozoocéphales.

Mon travail a donc permis de mieux comprendre la globozoospermie et de
développer un pipeline d'analyse bioinformatique qui a déjà permis
l'identification de plus de 15 gènes de la gamétogénèse humaine
impliqués dans différents phénotypes d'infertilité.

\newpage

\chapter*{Abstract}\label{abstract}
\addcontentsline{toc}{chapter}{Abstract}

\newpage

Même chose en anglais

\chapter{Mise en place d'une stratégie pour l'analyse des données
exomiques -- application en recherche
clinique}\label{mise-en-place-dune-strategie-pour-lanalyse-des-donnees-exomiques-application-en-recherche-clinique}

\section{Méthode : Description du
pipeline}\label{methode-description-du-pipeline}

\subsection{\texorpdfstring{L'alignement des
\emph{reads}}{L'alignement des reads}}\label{lalignement-des-reads}

\subsection{L'appel des variants}\label{lappel-des-variants}

\subsection{L'annotation}\label{lannotation}

\subsection{Le filtrage des variants}\label{le-filtrage-des-variants}

\section{Résultats 1 : Analyse de 3 phénotypes par des cas
familiaux}\label{resultats-1-analyse-de-3-phenotypes-par-des-cas-familiaux}

\subsection{Résultats des différentes étapes de
l'analyse}\label{resultats-des-differentes-etapes-de-lanalyse}

\subsubsection{Résultat de l'alignement}\label{resultat-de-lalignement}

\subsubsection{L'appel des variants}\label{lappel-des-variants-1}

\subsubsection{L'annotation des
variants}\label{lannotation-des-variants}

\subsubsection{Le filtrage des
variants}\label{le-filtrage-des-variants-1}

\subsection{Article n° 3}\label{article-n-3}

\subsubsection{Contexte et objectifs}\label{contexte-et-objectifs}

\subsubsection{Principaux résultats}\label{principaux-resultats}

\subsection{Article n° 4}\label{article-n-4}

\subsubsection{Contexte et objectifs}\label{contexte-et-objectifs-1}

\subsubsection{Principaux résultats}\label{principaux-resultats-1}

\subsection{Article n° 5}\label{article-n-5}

\subsubsection{Whole-exome sequencing of familial cases of multiple
morphological abnormalities of the sperm flagella (MMAF) reveals new
DNAH1
mutations}\label{whole-exome-sequencing-of-familial-cases-of-multiple-morphological-abnormalities-of-the-sperm-flagella-mmaf-reveals-new-dnah1-mutations}

\subsubsection{Contexte et objectifs}\label{contexte-et-objectifs-2}

\subsubsection{Principaux résultats}\label{principaux-resultats-2}

\section{Résultats 2 : Étude d'une cohorte de femmes
infertiles}\label{resultats-2-etude-dune-cohorte-de-femmes-infertiles}

\subsection{Article n° 6}\label{article-n-6}

\subsubsection{Contexte et objectifs}\label{contexte-et-objectifs-3}

\subsubsection{Principaux résultats}\label{principaux-resultats-3}

\section{Résultats 3 : Étude d'une large cohorte de patients
MMAF}\label{resultats-3-etude-dune-large-cohorte-de-patients-mmaf}

\subsection{Article n° 7}\label{article-n-7}

\subsubsection{Contexte et objectifs}\label{contexte-et-objectifs-4}

\subsubsection{Principaux résultats}\label{principaux-resultats-4}

\chapter{Mutations in DNAH1, which Encodes an Inner Arm Heavy Chain
Dynein, Lead to Male Infertility from Multiple Morphological
Abnormalities of the Sperm Flagella}\label{dnah12014}


% Index?

\end{document}

