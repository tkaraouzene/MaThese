% This is the Reed College LaTeX thesis template. Most of the work
% for the document class was done by Sam Noble (SN), as well as this
% template. Later comments etc. by Ben Salzberg (BTS). Additional
% restructuring and APA support by Jess Youngberg (JY).
% Your comments and suggestions are more than welcome; please email
% them to cus@reed.edu
%
% See http://web.reed.edu/cis/help/latex.html for help. There are a
% great bunch of help pages there, with notes on
% getting started, bibtex, etc. Go there and read it if you're not
% already familiar with LaTeX.
%
% Any line that starts with a percent symbol is a comment.
% They won't show up in the document, and are useful for notes
% to yourself and explaining commands.
% Commenting also removes a line from the document;
% very handy for troubleshooting problems. -BTS

% As far as I know, this follows the requirements laid out in
% the 2002-2003 Senior Handbook. Ask a librarian to check the
% document before binding. -SN

%%
%% Preamble
%%
% \documentclass{<something>} must begin each LaTeX document
\documentclass[12pt,twoside]{reedthesis}
% Packages are extensions to the basic LaTeX functions. Whatever you
% want to typeset, there is probably a package out there for it.
% Chemistry (chemtex), screenplays, you name it.
% Check out CTAN to see: http://www.ctan.org/
%%
\usepackage{graphicx,latexsym}
\usepackage[french]{babel} 
\usepackage{amsmath}
\usepackage{amssymb,amsthm}
\usepackage[dvipsnames]{xcolor} % tk: for more color
\usepackage{xcolor}
\usepackage{eso-pic}
\usepackage{longtable,booktabs,setspace}
\usepackage{chemarr} %% Useful for one reaction arrow, useless if you're not a chem major
\usepackage[hyphens]{url}
\usepackage{tikz}
\usetikzlibrary{calc}
\newcommand\HRule{\rule{\textwidth}{1pt}}
% Added by CII
\usepackage{hyperref}
\usepackage{lmodern}
\usepackage{float}
\floatplacement{figure}{H}
% End of CII addition
\usepackage{rotating}
\usepackage{upgreek} % tk : pour pouvoir utiliser le symbole µ droit (pas en itallic)
\usepackage{lscape}
\newcommand{\blandscape}{\begin{landscape}}
\newcommand{\elandscape}{\end{landscape}}

% Next line commented out by CII
%%% \usepackage{natbib}
% Comment out the natbib line above and uncomment the following two lines to use the new
% biblatex-chicago style, for Chicago A. Also make some changes at the end where the
% bibliography is included.
%\usepackage{biblatex-chicago}
%\bibliography{thesis}


% Added by CII (Thanks, Hadley!)
% Use ref for internal links
\renewcommand{\hyperref}[2][???]{\autoref{#1}}
\def\chapterautorefname{Chapter}
\def\sectionautorefname{Section}
\def\subsectionautorefname{Subsection}
% End of CII addition

% Added by CII
\usepackage{caption}
\captionsetup{width=5in}
% End of CII addition

% \usepackage{times} % other fonts are available like times, bookman, charter, palatino


% To pass between YAML and LaTeX the dollar signs are added by CII
\title{THÈSE}
\author{Thomas Karaouzene}
\labo{}
% The month and year that you submit your FINAL draft TO THE LIBRARY (May or December)
\date{31 octobre 2017}
\division{}
\advisor{Pierre Ray}
%If you have two advisors for some reason, you can use the following
% Uncommented out by CII
\altadvisor{Nicolas Thierry-Mieg}
% End of CII addition

%%% Remember to use the correct department!
\department{Ingénierie de la Santé, de la Cognition et Environnement (EDISCE)}
% if you're writing a thesis in an interdisciplinary major,
% uncomment the line below and change the text as appropriate.
% check the Senior Handbook if unsure.
%\thedivisionof{The Established Interdisciplinary Committee for}
% if you want the approval page to say "Approved for the Committee",
% uncomment the next line
%\approvedforthe{Committee}

% Added by CII
%%% Copied from knitr
%% maxwidth is the original width if it's less than linewidth
%% otherwise use linewidth (to make sure the graphics do not exceed the margin)
\makeatletter
\def\maxwidth{ %
  \ifdim\Gin@nat@width>\linewidth
    \linewidth
  \else
    \Gin@nat@width
  \fi
}
\makeatother

\renewcommand{\contentsname}{Table of Contents}
% End of CII addition

\setlength{\parskip}{0pt}

% Added by CII
  %\setlength{\parskip}{\baselineskip}
  \usepackage[parfill]{parskip}

\providecommand{\tightlist}{%
  \setlength{\itemsep}{0pt}\setlength{\parskip}{0pt}}

\Acknowledgements{

}

\Dedication{

}

\Preface{
This is an example of a thesis setup to use the reed thesis document
class (for LaTeX) and the R bookdown package, in general.
}

\Abstract{

}

	\usepackage{tikz}
% End of CII addition
%%
%% End Preamble
%%
%

\usepackage{amsthm}
\newtheorem{theorem}{Theorem}[section]
\newtheorem{lemma}{Lemma}[section]
\theoremstyle{definition}
\newtheorem{definition}{Definition}[section]
\newtheorem{corollary}{Corollary}[section]
\newtheorem{proposition}{Proposition}[section]
\theoremstyle{definition}
\newtheorem{example}{Example}[section]
\theoremstyle{remark}
\newtheorem*{remark}{Remark}
\begin{document}

% Everything below added by CII
      \maketitle
  
  \frontmatter % this stuff will be roman-numbered
  \pagestyle{empty} % this removes page numbers from the frontmatter

  
      \begin{preface}
      This is an example of a thesis setup to use the reed thesis document
      class (for LaTeX) and the R bookdown package, in general.
    \end{preface}
  
      \hypersetup{linkcolor=black}
    \setcounter{tocdepth}{3}
    \tableofcontents
  
      \listoftables
  
      \listoffigures
  
  
  
  \mainmatter % here the regular arabic numbering starts
  \pagestyle{fancyplain} % turns page numbering back on

  \chapter{Delete line 6 if you only have one
  advisor}\label{delete-line-6-if-you-only-have-one-advisor}
  
  \chapter*{Remerciements}\label{remerciements}
  \addcontentsline{toc}{chapter}{Remerciements}
  
  \chapter*{Résumé}\label{resume}
  \addcontentsline{toc}{chapter}{Résumé}
  
  \chapter{Introduction}\label{introInf}
  
  \chapter{Investigation génétique et physiologique de la
  globozoospermie}\label{globo}
  
  \chapter{Mise en place d'une stratégie pour l'analyse des données
  exomiques -- application en recherche
  clinique}\label{mise-en-place-dune-strategie-pour-lanalyse-des-donnees-exomiques-application-en-recherche-clinique}
  
  \section{Intro}\label{intro}
  
  Comme vu précédemment, l'émergence du séquençage haut débit, avec
  notamment le WGS et le WES, a révolutionné les méthodes de recherche
  dans le cadre d'étude phénotype-génotype en permettant de manière rapide
  et à moindre coup le séquençage de la quasi totalité des gènes humains.
  Les causes de plusieurs centaines de pathologies ont pu être identifiées
  grâce à ces technique depuis leur premier succès pubilié en 2010 (Ng et
  al., n.d.). Dès lors, l'analyse des données issues du séquençage est
  devenu la clef dans la réussite de ces études.
  
  Il existe de nombreux logiciels qui à partir des variants appelés
  effectuent les étapes d'annotation et de filtrage. C'est par exemple le
  cas d'Exomiser {[}TODO: insert ref and Exomiser describtion{]} ou encore
  de {[}TODO: insert at least one other soft{]}. La plupart de ces
  logiciels fonctionnent très bien, cependant tous prennent pour point de
  départ des variants appelés en amont. Ils ne contrôlent donc en aucune
  manière les étapes d'alignement et d'appel des variants. Or, comme il a
  été dit plus tôt, ces deux étapes constituent la bases de l'analyse
  {[}TODO insert ref{]} et les résultats
  
  Dans ce chapitre, je détaillerai les résultats de 4 articles dont je
  suis coauteur :
  
  \begin{enumerate}
  \def\labelenumi{\arabic{enumi}.}
  \tightlist
  \item
    \protect\hyperlink{famdnah1}{\textbf{Whole-exome sequencing of
    familial cases of multiple morphological abnormalities of the sperm
    flagella (MMAF) reveals new DNAH1 mutations}} : {[}todo{]}
  \item
    \protect\hyperlink{plcz}{\textbf{Homozygous mutation of PLCZ1 leads to
    defective human oocyte activation and infertility that is not rescued
    by the WW-binding protein PAWP}} : Dans cet article j'ai, comme
    précédemment, effectué l'integralité des analyses bioinformatiques des
    données d'exomes effectués sur deux frères infertiles présentant des
    échecs de fécondation.\\
  \item
    \protect\hyperlink{spink2}{\textbf{SPINK2 deficiency causes
    infertility by inducing sperm defects in heterozygotes and azoospermia
    in homozygotes}} : Dans cet article j'ai effectuer non seulement
    l'intégralité des analyses bioinformatiques des données d'exomes de
    deux frères infertiles présentant un phénotype d'azoospermie mais
    aussi séquencer en Sanger les séquences codantes du gène \emph{SPINK2}
    pour une parie des 611 individus analyser ainsi que contribué à
    l'extraction de l'ARN testiculaire des souris pour l'analyse
    fonctionelle du gène \emph{Spink2} sur le modèle murin.\\
  \item
    \protect\hyperlink{cohortemmah}{****} : {[}todo{]}
  \end{enumerate}
  
  \section{Résultats}\label{resultats}
  
  \subsection{Description de la
  pipeline}\label{description-de-la-pipeline}
  
  Notre pipeline d'analyse effectue l'ensemble des étapes allant de
  l'alignement des données jusqu'au filtrage des variants
  
  \begin{enumerate}
  \def\labelenumi{\arabic{enumi}.}
  \tightlist
  \item
    \textbf{L'alignement} : L'alignement des \emph{reads} le long du
    génome de référence est effectué par le logiciel MAGIC (Su et al.,
    \protect\hyperlink{ref-Su2014}{2014}). Celui-ci l'intégralité pour
    l'ensemble des analyses en aval l'ensemble des \emph{reads} dupliqués
    et / ou s'alignant à plusieurs zone du génome. Au cours de cette
    étape, MAGIC va produire également quatre comptages pour chaque
    position couverte du génome : R+, V+, R- et V- :
  
    \begin{enumerate}
    \def\labelenumii{\alph{enumii}.}
    \tightlist
    \item
      \textbf{R+ et R-} : Ces deux comptages correspondent au nombres de
      \emph{reads} \emph{forward} (+) et \emph{reverse} (-) sur lesquels
      est observé l'allere de \textbf{référence} (R) à une position
      donnée.\\
    \item
      \textbf{V+ et V-} : À l'inverse de R+ et R-, ces comptages
      correspondent au nombres de \emph{reads} \emph{forward} et
      \emph{reverse} sur lesquels est observé un allele de
      \textbf{variant} (V) à une position donnée.\\
    \end{enumerate}
  \item
    \textbf{L'appel des variants} : Comme nous l'avons vu plus
    \protect\hyperlink{varcall}{tôt}, il est fortement conseillé
    d'effectuer l'appel des variants en tenant compte de l'aligneur choisi
    (Nielsen, Paul, Albrechtsen, \& Song,
    \protect\hyperlink{ref-Nielsen2011}{2011}, M. A. DePristo et al.
    (\protect\hyperlink{ref-DePristo2011}{2011}), Lunter \& Goodson
    (\protect\hyperlink{ref-Lunter2011}{2011})). C'est pourquoi, nous
    avons conçu notre propre algorithme d'appel des variants spécialement
    conçu pour l'analyse des données de MAGIC. Ainsi, l'appel des variants
    sera directement basé sur les quatre comptages vu précédement. Tout
    d'abord, les positions ayant une couverture \textless{} 10 sur l'un
    des deux \emph{strands} sera considérée comme de faible qualité,
    celles aynant une couverture \textless{} 10 sur les deux
    \emph{strands} seront exclus. Ensuite pour chaque variant, des appels
    indépendant seront effectués pour chaque \emph{strand}. L'appel final
    sera une synthèse de ces deux appels où seul les cas où ces deux
    appels sont concordants seront considérés comme de bone qualité.\\
  \item
    \textbf{L'annotation} : Chaque variant retenu sera ensuite annoté tout
    d'abord par le logiciel \emph{variant effect predictor} (VEP) (W.
    McLaren et al., \protect\hyperlink{ref-McLaren2016}{2016}) qui nous
    indiquera pour chaque variant la conséquence que celui-ci aura sur la
    séquence codante de l'ensemble des transcrits Ensembl qu'il chevauche
    (\textbf{Figure : }\ref{fig:figvepcsq}) (\textbf{Table :
    }\ref{tab:tabvepcsq}). Suite à cela nous ajoutons, lorsque celle-ci
    est disponible, la fréquence du variant dans les bases de données ExAC
    (Lek et al., \protect\hyperlink{ref-Lek2016}{2016}), ESP600 {[}TODO{]}
    et 1000Genomes {[}TODO{]} donnant ainsi une estimation de sa fréquence
    dans la population générale. De même, la particularité de cette
    pipeline est qu'elle conserve l'ensemble des variants identifiés dans
    les études effectués précédement permettant d'ajouter aux annotations
    la fréquences d'un variant chez les individus déjà séquencé et donc la
    fréquence d'un variant dans chaque phénotype étudié créant ainsi une
    base de données interne qui pourra servir de contrôle dans les études
    ulterieur.
  \end{enumerate}
  
  \begin{figure}
  
  {\centering \includegraphics[scale=.9]{figure/vep_csq} 
  
  }
  
  \caption[Listes des différentes conséquences prédites par VEP et leurs positionement sur le transcrit]{Listes des différentes conséquences prédites par VEP et leurs positionement sur le transcrit d'après [VEP site](http://www.ensembl.org/info/genome/variation/consequences.jpg)}\label{fig:figvepcsq}
  \end{figure}
  
  \begin{enumerate}
  \def\labelenumi{\arabic{enumi}.}
  \setcounter{enumi}{3}
  \tightlist
  \item
    \textbf{Le filtrage des variants} : L'étape de filtrage est
    extremement importante si l'on souhaite analyser de manière efficace
    les données provenant de WES. C'est pourquoi elle occupe une place
    importante dans notre pipeline. L'intégralité des paramètres de cette
    étape peuvent être modifier par l'utilisateur de sorte à faire
    correspondre les critères de filtre aux bsoins de l'étude. Afin de
    rendre son utilisation le plus efficace possibe, nous avons souhaité
    définir des paramètres par défauts pertinent dans la plupart des étude
    de séquençage exomique de sorte que à moins que le contraire ne soit
    spécifié, seul les variants impactant les transcrits codant pour une
    protéine sont conservés. De même les variants synonymes ou affectant
    les séquences UTRs sont filtrés ainsi que les variants ayant une
    fréquence \(\ge\) 1\% dans les bases dans l'une des bases données
    (ExAC, ESP6500 ou 1KH). Aussi, pour un phénotype donné, l'ensemble des
    variants observés chez les individus étudiés présentant un phénotype
    différent sont de même enlevés de la liste finale.
  \end{enumerate}
  
  \subsection{Utilisation de la pipeline dans des cas familiaux
  :}\label{utilisation-de-la-pipeline-dans-des-cas-familiaux}
  
  \subsubsection{Description des familles}\label{description-des-familles}
  
  Dans cette partie, je me concentre sur l'analyse bioinformatique des
  résultats des séquençages exomiques effectués entre 2012 et 2014 de 13
  individus infertiles provenant de 6 familles différentes. Parmi
  celles-ci, 3 phénotypes différents ont été observés :
  
  \begin{enumerate}
  \def\labelenumi{\arabic{enumi}.}
  \tightlist
  \item
    \textbf{\protect\hyperlink{infquant}{L'Azoospermie} :} Comme nous
    avons pu le voir, l'azoospermie est un phénotype d'infertilité
    masculine caractérisé par l'absence de spermatozoïde dans
    l'éjaculat.\\
  \item
    \textbf{Echec de fécondation :} Ce phénotype d'infertilité se
    caractérise par l'incapacité des spermatozoïdes à féconder
    l'ovocyte.\\
  \item
    \textbf{MMAF} : Le syndrome MMAF (\emph{multiple morphological
    abnormalities of the sperm flagella}) caractérise comme son nom
    l'indique les patients présentant une majorité de spermatozoïdes
    atteins par une mosaïque d'anomalie morphologique du flagelle.
  \end{enumerate}
  
  Un récapitulatif des familles et de leur phénotype est disponible dans
  la table \ref{tab:recapfam}.
  
  \begin{longtable}[t]{lrlrll}
  \caption{\label{tab:recapfam}Tableau recapitulatif des familles séquencées et de leur phénotype}\\
  \toprule
  Familly & Individuals & Phenotype & Year & Plateform & Place\\
  \midrule
  Az & 2 & Azoospermia & 2012 & Illumina HiSeq2000 & Mount Sinai Institut\\
  FF & 2 & Fertilization failure & 2014 & Illumina HiSeq2000 & Genoscope (Evry)\\
  MMAF1 & 2 & MMAF & 2014 & Illumina HiSeq2000 & Genoscope (Evry)\\
  MMAF2 & 2 & MMAF & 2014 & Illumina HiSeq2000 & Genoscope (Evry)\\
  MMAF3 & 2 & MMAF & 2014 & Illumina HiSeq2000 & Genoscope (Evry)\\
  MMAF4 & 3 & MMAF & 2014 & Illumina HiSeq2000 & Genoscope (Evry)\\
  \bottomrule
  \end{longtable}
  
  \subsubsection{Resultats des exomes}\label{resultats-des-exomes}
  
  \paragraph{Résultat de l'alignement
  les}\label{resultat-de-lalignement-les}
  
  Pour rappel, l'\href{\%7B\#lalignement\%7D}{alignement} consiste à
  repositionnerl'ensemble des \emph{reads} générés au cours de l'étape de
  séquençage le long d'un génome de référence. La quantité de \emph{reads}
  peut varier en fonction de plusieurs paramètres et n'est donc pas éguale
  pour chaque patient bien que l'ordre de grandeur reste le même exceptés
  pour les deux frères AZ1 et AZ2 pour lesquels on a près de 3 fois plus
  de \emph{reads} que pour les autres patients (\textbf{Figure :
  }\ref{fig:readsselection} - \textbf{A}). Ceci peut être expliqué car ces
  deux patients sont les deux seuls à voir été séquencé au Mount Sinaï
  Institut or leur protocol d'amplification contient un nombre de cycles
  de PCR superieur à ceux appliqué au Génopole d'Evry où ont été séquencé
  les autres patients (\textbf{Table :} \ref{tab:recapfam}), L'ensemble de
  nos exomes ayant été réalisés en \emph{paired-end}, les deux extrémités
  de chaques fragments sont séquencés chaque \emph{end} d'un même
  \emph{read} peut donc être considéré comme un \emph{read} à part
  entière. Celle-ci sont ensuite alignées \textbf{indépendemment} le long
  du génome de référence, l'information fournit par le \emph{paired-end}
  n'est utilisé qu'à \emph{posteriri} en tant que critère qualité. Ainsi,
  après avoir filtré les \emph{reads} ne s'étant pas aligné sur le génome
  et le \emph{reads} orphelines (une seule des deux \emph{ends} s'est
  alignée sur le génome), la ``compatibilité entre les deux \emph{ends}
  d'un même \emph{reads} est analysée. Un \emph{read} est dit''compatible"
  lorsque les deux \emph{ends} qui le composent s'alignent face à face
  (une sur le \emph{strand} + et l'autre sur le \emph{strand} -) et
  couvrent une zone ne faisant pas plus de 3 fois la taille médiane de
  l'insert. Ici encore, seul les \emph{reads} ayant des \emph{ends}
  ``compatibles'' seront conservés. Pour l'ensemble de nos patients, les
  \emph{reads} compatibles sont environs 10 fois plus important que la
  somme des \emph{reads} non compatible, orphelins ou non mappés
  (\textbf{Figure : }\ref{fig:readsselection} - \textbf{B}). Suite à cela,
  le nombre de site auxquels se sont alignés les \emph{reads} est analysé.
  En effet, certaine zone du génome étant dupliqué, l'une des
  problématique des \emph{short-reads} est qu'il est possible que ceux-ci
  s'alignent à plusieurs endroit du génome. Afin d'éviter toute ambiguité,
  seul ceux s'étant aligné sur un site unique sont conservés pour la suite
  des analyse. ces \emph{reads} représente entre \ldots{} et \ldots{} \%
  des \emph{reads} ayant passé les précédents filtres (\textbf{Figure :
  }\ref{fig:readsselection} - \textbf{C}). Les \emph{reads} ayant passé
  l'ensemble des critères qualité seront ensuite utilisé pour effectué
  l'appel des variants.
  
  \newpage
  
  \begin{figure}
  
  {\centering \includegraphics{thesis_files/figure-latex/readsselection-1} 
  
  }
  
  \caption[Processus simplifié du contrôle qualité des *reads*]{Processus simplifié du contrôle qualité des *reads* : **A** : Quantité de *reads* bruts générés pour chaque patients au cours de l'étape de séquençage. On constate que ce nombre reste pour chaque patient dans le même ordre de grandeur sauf pour les frères AZ1 et AZ2  qui contienent presque 3 fois plus de *reads* que la mediane représentée en bleue.. **B** : Distribution pour chaque patient des *reads* compatibles, incompatibles, orphelins et non mappés. Comme attendu, les reads compatible sont les plus important. Ils sont les seuls à être utilisé dans le reste de l'analyse.. **C** : Présentation pour chaque *reads* du nombre de site auxquels ils s'alignent. Seuls les reads s'alignant sur un site unique sont conservés}\label{fig:readsselection}
  \end{figure}
  
  \newpage
  
  \paragraph{Résultat de l'appel des
  variants}\label{resultat-de-lappel-des-variants}
  
  Comme dit précédement, l'appel des variants fait suite à l'alignement et
  consiste à comparer la séquence d'un individus avec celle d'un génome de
  référence afin d'en relever les différences. La particularité de notre
  algorithme d'appel est d'effectuer pour chaque position deux appels
  indépendents. Le premier sera effectué en utilisant uniquement les
  \emph{reads forward} et le second le \emph{reads reverse}. Les positions
  ayant une couverture \(\le\) 10 sur \textbf{les deux} \emph{strands}
  seront filtrés (NS). Les autres seront conservés bien que ceux ayant une
  couverture \(\le\) 10 sur \textbf{un des deux} \emph{strands} (SS)
  seront considéré comme de faible qualité et seront leurs interprétations
  sera plus précotionneuse. Ainsi, chez nos \texttt{r} {[}TODO nb of
  patients{]} entre \ldots{} et \ldots{} variants sont filtrés car leur
  couverture est \textless{} 10 sur les deux \emph{ends} du \emph{reads}
  (\textbf{Figure : }\ref{fig:plotvarcall} - \textbf{A}).
  
  Pour les position ayant une couverture \(\ge\) 10 sur \textbf{les deux}
  \emph{strands} (DS) les résultats des deux appels sont comparés et seul
  les appel concordant seront conservés, c'est à dire environs \ldots{}\%
  des variants DS. Les appels ambigus et discordants seront filtrés et non
  considérés dans les analyses en aval (\textbf{Figure :
  }\ref{fig:plotvarcall} - \textbf{B}).
  
  Dès lors il est intéréssant de noter que bien que les variants
  \emph{single strand} (SS) soient conservés, on peut s'attendre à ce
  qu'également \ldots{} \% de ces variants soient abérents, ceux-ci
  n'ayant put subir le même contrôle que les SS. Pour l'ensemble des
  variants ayant passé les premiers filtres, c'est à dire les variants SS
  et les variants DS avec appels concordants, le génotype est déterminé en
  fonction du pourcentage de \emph{reads} portant le variant à cette
  position. Par exemple, si à une position donnée, 0\% des \emph{reads}
  portent un variant, l'individu sera appelé ``Homozygote référence'', si
  50\% des \emph{reads} sont portent un variant, l'appel sera
  ``hétérozygote'' et si 100\% des \emph{reads} portent un variant,
  l'appel sera ``Homozygote variant''. Ainsi, pour chaque individu nous
  avons pu établire une liste de variants avec leur génotype associé
  (\textbf{Figure : }\ref{fig:plotvarcall} - \textbf{C}).
  
  \newpage
  
  \begin{figure}
  
  {\centering \includegraphics{thesis_files/figure-latex/plotvarcall-1} 
  
  }
  
  \caption[Contrôle qualité des variants appelés]{Contrôle qualité des variants appelés : Pour l'ensemble des figures les variants verts sont conservés, les gris sont filtrés. **A** : Distribution du *stranding* des appels pour chaque patients. Environs 16 pourcents des vairants ont une couverture insufisante pour l'appel *forward* et l'appel *revers* et sont donc filtrés, les autres sont conservés. **B** : Comparaison des appels entre les deux *ends* des variants appelés DS. En fonction des individus, 80 à 90 pourcent des appels sont concordants. Les autres appels sont filtrés des analyses ulterieures. **C** : Distribution des SNVs et indels en fonction de leur génotype pour chaque patients (représentés par une barre).}\label{fig:plotvarcall}
  \end{figure}
  
  \newpage
  
  \paragraph{Résultats de l'annotation}\label{resultats-de-lannotation}
  
  Afin de connaitre l'effet qu'auront chacun des variants appelés sur les
  différents transcrits qu'ils chevauchent nous utilisons le logiciel VEP.
  Grâce à celà, nous pouvons constater que pour chaque patient \ldots{}
  gènes sont en moyenne affecté par au moins un variant tandis que
  \ldots{} sont impactés (soit environs \texttt{r} transcrits par gènes)
  (\textbf{Figure : }\ref{fig:plotvarannotation} - \textbf{A}). Chaque
  variant affectera l'ensemble des transcrits qu'il chevauche, ainsi un
  même variant pourra impacter plusieurs transcrits. Ces impacts sont
  ensuite classés par VEP en quatre catégories qui sont, de la plus
  délétère à la moins délétère : HIGH, MODERATE, LOW, MODIFIER. Comme
  attendu, les variants ayant un impact tronquant se retrouvent être les
  moins fréquent chez chacun de nos patients. Ceci est d'autant plus
  flagrant pour l'impact HIGH qui regroupe, entre autre, les variants
  créant un codon stop ou encore ceux causant un décalage du cadre de
  lécture, se retrouvent en quantité extrêment faible puisqu'ils ne
  représentent en moyenne que \ldots{} \% des variants (soit environs
  \ldots{} par patient (en nb ici)) (\textbf{Figure :
  }\ref{fig:plotvarannotation} - \textbf{B}).
  
  \begin{figure}
  
  {\centering \includegraphics{thesis_files/figure-latex/plotvarannotation-1} 
  
  }
  
  \caption[TODO]{TODO : TODOooooooo ooooooooo oooooooooooooooooo oooooooooooooooo oooooooooooo ooooooooooo oooo oooooooo oooooooooooooooooooooooooooo oooooooooo ooooo ooooooooooooo oooooooooo oooooooooo ooooooo.}\label{fig:plotvarannotation}
  \end{figure}
  
  \begin{figure}
  
  {\centering \includegraphics{thesis_files/figure-latex/plotvarfreq-1} 
  
  }
  
  \caption[TODO]{TODO : TODOooooooo ooooooooo oooooooooooooooooo oooooooooooooooo oooooooooooo ooooooooooo oooo oooooooo oooooooooooooooooooooooooooo oooooooooo ooooo ooooooooooooo oooooooooo oooooooooo ooooooo.}\label{fig:plotvarfreq}
  \end{figure}
  
  \newpage
  
  \paragraph{Résultats du filtrage}\label{resultats-du-filtrage}
  
  \begin{verbatim}
  ##             used  (Mb) gc trigger   (Mb)  max used   (Mb)
  ## Ncells    754897  40.4    2564361  137.0   3205452  171.2
  ## Vcells 115550903 881.6  363498973 2773.3 454172341 3465.1
  \end{verbatim}
  
  Les étapes précédentes nous ont permis de mettre en évidence pour chaque
  patient une liste de variants passant l'ensemble de nos critères
  qualités. Ces variants ont dès lors put être annotés nous permettant
  entre autre d'avoir connaissance de leurs l'impacts sur les différents
  transcrits qu'ils chevauchent ou encore leur fréquence dans la
  population générale. Desormai, afin de ne conserver que les variants
  ayant la plus forte probabilité d'être responsable du phénotype de ces
  patients, nous avons appliqué succesivement six filtres basés à la fois
  sur les différentes annotations que nous avons ajouté mais aussi sur nos
  connaissance du mode de transmission du phénotype :
  
  \begin{enumerate}
  \def\labelenumi{\arabic{enumi}.}
  \tightlist
  \item
    \textbf{Filtre 1 : L'union des variants :} Dans ces différentes
    études, nous avons à chaque fois séquencé des duos ou des trios
    d'individus provenant de même frateries et étant caractérisés par le
    même phénotype. Ainsi nous avons pu formuler l'hypothèse d'une cause
    génétique commune entre les différents patients d'une même famille et
    donc filtrer l'ensemble des variants qui ne sont pas partagés par
    l'ensemble des membre de la fraterie {[}TODO: discussion de
    l'efficacité du filtre{]}.\\
  \item
    \textbf{Filtre 2 : Genotype des variants :} Dans ces études, nous
    avons emmis l'hypothèse d'une transmission recessive du phénotype.
    Ainsi, seul les variants homozygotes ont été conservés. Ce filtre est
    le plus efficace du pipeline en permettant de filtrer entre \ldots{}
    et \ldots{} variants par individus (\textbf{Figure :
    }\ref{fig:resvarcall}, \ref{fig:comparefilter}).\\
  \item
    \textbf{Filtre 3 : Impact du variant :} Afin de ne conserver que les
    variants ayant un effet potentiellement tronquant sur la protéine,
    nous avons filtré les variants intonique et ceux tombant dans les
    sequences UTRs. De même les variants synonymes ne sont pas conservés
    (exeptés ceux se trouvant proches des régions d'épissage) car ceux-ci
    n'ont aucun effet sur séquences protéique. Pour les variants faux sens
    (changement d'un seul aa de la séquence protéique) il est plus
    difficile de se décider {[}TODO insert citation{]} nous avons donc
    utilisé les logiciels SIFT et Polyphen et filtré l'ensemble des
    fauxsens prédit comme \emph{tolerated} par SIFT et \emph{benign} par
    Polyphen.\\
  \item
    \textbf{Filtre 4 : Les transcrits non pertinents :} Au cours de nos
    analyses nous nous sommes concentré uniquement sur les transcrits
    codant pour une protéine. Ainsi, l'ensemble des transcrits annotés
    comme étant non codant furent filtrés. De même Le mécanisme NMD
    (\emph{nonsense-mediated decay}) a pour but de controler la qualité
    des ARNm cellulaires chez les eucaryotes (Y.-F. Chang, Imam, \&
    Wilkinson, \protect\hyperlink{ref-Chang2007}{2007}) en éliminant les
    ARNm qui comportent un codon stop prématuré (Baker \& Parker,
    \protect\hyperlink{ref-Baker2004}{2004}), pouvant être le résultat
    d'une erreur de transcription, d'une mutation ou encore d'une erreur
    d'épissage. Il est donc peu probable que les variants présents sur
    transcrits annotés NMD soient responsables du phénotype. Dès lors, ces
    transcripts furent eux aussi filtrés. Ainsi, nous avons pu retirer de
    nos listes de variants l'ensemble des mutations impactant
    \textbf{uniquement} des transcrits non codant et / ou annoté NMD.
    Cette étape de filtre permet à elle seule de systematiquement filtrer
    entre 36576 et 44581 transcrits différents par patients, soit une
    moyenne de NaN variants par individus (\textbf{Figure :
    }\ref{fig:plotfilternonpertinanttr}).
  \end{enumerate}
  
  \begin{figure}
  
  {\centering \includegraphics{thesis_files/figure-latex/plotfilternonpertinanttr-1} 
  
  }
  
  \caption[Filtrage des transcrits jugés "non pertinents" et des variants les chevauchant]{Filtrage des transcrits jugés "non pertinents" et des variants les chevauchant : Pour chaque patients nous avons filtrer les transcrits jugés "non pertinents" pour l'analyse, c'est à dire ceux ne codant pas pour une protéine et ceux annoté NMD. Dès lors, l'intégralité des variants chevauchant uniquement des transcrits non pertinents ont put systématiquement être filtrés (boites rouges). les autres furent conservés (boites vertes)}\label{fig:plotfilternonpertinanttr}
  \end{figure}
  
  \begin{enumerate}
  \def\labelenumi{\arabic{enumi}.}
  \setcounter{enumi}{4}
  \tightlist
  \item
    \textbf{Frequence des variants :} La fréquence d'un variant dans la
    population générale est un moyen rapide d'avoir un avis sur l'effet
    délétère de celui-ci. En efft, il est peu probable qu'un retrouvé
    fréquement dans la population générale soit causal d'une pathologie
    sévère. Ainsi nous avons filtré pour l'ensemble de nos patients
    l'ensemble des variants ayant une fréquence \(\ge\) 0.01 dans l'une
    des trois bases de données que sont ExAC, ESP et 1KG.\\
  \item
    \textbf{Présence des variants dans la cohorte contrôle :} Au cours de
    nos différentes études, nous avons été ammené à séquencé 134.
    L'ensemble de ces individus peuvent être soit sains soit présenter
    l'un des 6 phénotypes étudié au cours de nos différentes études
    (\textbf{Table : }\ref{tab:TODO}). Ces phénotypes étant très
    différent, il n'est pas abérant d'emmetre l'hypothèse qu'ils que leurs
    causes génétiques soient diffrentes. De même, les variants recherché
    étant rares, il est peu probable qu'un individu porte les variants de
    deux phénotypes différents. Ainsi, pour chacune des 6 familles, nous
    avons pu constituer une cohorte contrôle composée dans l'ensemble des
    patients précédemment analysés et ne présentant pas le même phénotype
    que celui étudié dans la famille (\textbf{Figure :}
    \ref{fig:plotsamplectrl}). Dès lors, nous avons put filtrer l'ensemble
    des variants retrouvés à la fois chez nos patients et observés à
    l'état homozygote dans la cohorte contrôle.
  \end{enumerate}
  
  \begin{figure}
  
  {\centering \includegraphics{thesis_files/figure-latex/plotsamplectrl-1} 
  
  }
  
  \caption[Nombre d'individus composant la cohorte contrôle de chaque famille]{Nombre d'individus composant la cohorte contrôle de chaque famille : Ici, chaque barre représente une famille et sa hauteur est déterminée par le nombre d'individus composant la cohorte contrôle à laquelle elle a été confronté. Chaque individus de la cohorte contrôle a été séquencés en WES par notre équipe. Afin d'être considéré comme "contrôle" et intégrer cette cohorte, un individus doit être sain ou présenter un phénotype d'infertilité différent de la famille étudiée. Par exemple, un individus MMAF pourra servir de contrôle aux familles AZ et FF mais pas aux familles MMAF1-4}\label{fig:plotsamplectrl}
  \end{figure}
  
  \newpage
  
  {[}TODO : COMPARAISON DE L'EFFICACITE DES FILTRES{]}
  
  Après avoir effectuer l'ensemble de ces filtres, seuls quelques variants
  subsistent nous permettant d'obtenir unle liste de gènes restrainte pour
  chaque famille (\textbf{Table : }\ref{tab:tablegene}). Ainsi, la cause
  génétique expliquant le phénotype d'une famille a pu être mis en
  évidence dans \ldots{} familles sur \ldots{} {[}TODO{]} (\textbf{Figure
  : }\ref{fig:plotremaininggenes}). Il est a noté que l'ensemble des
  familles pour lesquelles la cause génétique a été identifiée présente un
  historique consanguin {[}figure arbre{]} ce qui n'était pas le cas pour
  les \ldots{} autres. Cette consanguinité observée dans une partie des
  famille nous a permi de justifier l'exclusion des variants
  hétérozygotes. En revanche pour les \ldots{} autres fa milles, rien ne
  justifiait un tel filtre. Ainsi, pour celles-ci il est probable que les
  variants responsables se soient vu exclus par ce filtre. C'est pourquoi,
  notre équipe se concentre actuellement sur les variants hétérozygotes de
  ces familles.
  
  \begin{figure}
  
  {\centering \includegraphics{thesis_files/figure-latex/plotremaininggenes-1} 
  
  }
  
  \caption[Nombre de gènes passant l'ensemble des filtres par famille]{Nombre de gènes passant l'ensemble des filtres par famille  :  Chaque barre représente une des familles analysées. La hauteure de cette barre correspond au nombre de gènes ayant passé l'ensemble des filtres pour chaque famille. Les barres bleues caractérisent les familles pour lesquelles le gène responsable de la pathologie a été identifié parmi la liste de gène (dans ce cas le symbole du gène est écrit au dessus de la barre). Les barres rouges indique qu'aucun des gènes ayant passé les filtres pour ne semble expliquer le phénotype (dans ce cas il est écrit "???" au dessus de la barre)}\label{fig:plotremaininggenes}
  \end{figure}
  
  \chapter{MutaScript}\label{mutascript}
  
  \chapter*{Conclusion}\label{conclusion}
  \addcontentsline{toc}{chapter}{Conclusion}
  
  \chapter{The First Appendix}\label{the-first-appendix}
  
  \chapter*{References}\label{references}
  \addcontentsline{toc}{chapter}{References}
  
  \hypertarget{refs}{}
  \hypertarget{ref-Baker2004}{}
  Baker, K. E., \& Parker, R. (2004). Nonsense-mediated mRNA decay:
  terminating erroneous gene expression. \emph{Current Opinion in Cell
  Biology}, \emph{16}(3), 293--9.
  \url{http://doi.org/10.1016/j.ceb.2004.03.003}
  
  \hypertarget{ref-Chang2007}{}
  Chang, Y.-F., Imam, J. S., \& Wilkinson, M. F. (2007). The
  Nonsense-Mediated Decay RNA Surveillance Pathway. \emph{Annual Review of
  Biochemistry}, \emph{76}(1), 51--74.
  \url{http://doi.org/10.1146/annurev.biochem.76.050106.093909}
  
  \hypertarget{ref-DePristo2011}{}
  DePristo, M. A., Banks, E., Poplin, R., Garimella, K. V., Maguire, J.
  R., Hartl, C., \ldots{} Pritchard, E. (2011). A framework for variation
  discovery and genotyping using next-generation DNA sequencing data.
  \emph{Nature Genetics}, \emph{43}(5), 491--498.
  \url{http://doi.org/10.1038/ng.806}
  
  \hypertarget{ref-Lek2016}{}
  Lek, M., Karczewski, K. J., Minikel, E. V., Samocha, K. E., Banks, E.,
  Fennell, T., \ldots{} Exome Aggregation Consortium, D. G. (2016).
  Analysis of protein-coding genetic variation in 60,706 humans.
  \emph{Nature}, \emph{536}(7616), 285--91.
  \url{http://doi.org/10.1038/nature19057}
  
  \hypertarget{ref-Lunter2011}{}
  Lunter, G., \& Goodson, M. (2011). Stampy: A statistical algorithm for
  sensitive and fast mapping of Illumina sequence reads. \emph{Genome
  Research}, \emph{21}(6), 936--939.
  \url{http://doi.org/10.1101/gr.111120.110}
  
  \hypertarget{ref-McLaren2016}{}
  McLaren, W., Gil, L., Hunt, S. E., Riat, H. S., Ritchie, G. R. S.,
  Thormann, A., \ldots{} Cunningham, F. (2016). The Ensembl Variant Effect
  Predictor. \emph{Genome Biology}, \emph{17}(1), 122.
  \url{http://doi.org/10.1186/s13059-016-0974-4}
  
  \hypertarget{ref-Ng}{}
  Ng, S. B., Buckingham, K. J., Lee, C., Bigham, A. W., Tabor, H. K.,
  Dent, K. M., \ldots{} Bamshad, M. J. (n.d.). Exome sequencing identifies
  the cause of a Mendelian disorder. \url{http://doi.org/10.1038/ng.499}
  
  \hypertarget{ref-Nielsen2011}{}
  Nielsen, R., Paul, J. S., Albrechtsen, A., \& Song, Y. S. (2011).
  Genotype and SNP calling from next-generation sequencing data.
  \emph{Nature Reviews. Genetics}, \emph{12}(6), 443--51.
  \url{http://doi.org/10.1038/nrg2986}
  
  \hypertarget{ref-Su2014}{}
  Su, Z., Łabaj, P. P., Li, S. S., Thierry-Mieg, J., Thierry-Mieg, D.,
  Shi, W., \ldots{} Shi, L. (2014). A comprehensive assessment of RNA-seq
  accuracy, reproducibility and information content by the Sequencing
  Quality Control Consortium. \emph{Nature Biotechnology}, \emph{32}(9),
  903--14. \url{http://doi.org/10.1038/nbt.2957}


  % Index?

\end{document}

