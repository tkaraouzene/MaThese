% This is the Reed College LaTeX thesis template. Most of the work
% for the document class was done by Sam Noble (SN), as well as this
% template. Later comments etc. by Ben Salzberg (BTS). Additional
% restructuring and APA support by Jess Youngberg (JY).
% Your comments and suggestions are more than welcome; please email
% them to cus@reed.edu
%
% See http://web.reed.edu/cis/help/latex.html for help. There are a
% great bunch of help pages there, with notes on
% getting started, bibtex, etc. Go there and read it if you're not
% already familiar with LaTeX.
%
% Any line that starts with a percent symbol is a comment.
% They won't show up in the document, and are useful for notes
% to yourself and explaining commands.
% Commenting also removes a line from the document;
% very handy for troubleshooting problems. -BTS

% As far as I know, this follows the requirements laid out in
% the 2002-2003 Senior Handbook. Ask a librarian to check the
% document before binding. -SN

%%
%% Preamble
%%
% \documentclass{<something>} must begin each LaTeX document
\documentclass[12pt,twoside]{ugathesis}
% Packages are extensions to the basic LaTeX functions. Whatever you
% want to typeset, there is probably a package out there for it.
% Chemistry (chemtex), screenplays, you name it.
% Check out CTAN to see: http://www.ctan.org/
%%
\usepackage{graphicx,latexsym}
\usepackage[french]{babel}
\usepackage{amsmath}
\usepackage{amssymb,amsthm}
\usepackage[dvipsnames]{xcolor} % tk: for more color
\usepackage{xcolor}
\usepackage{eso-pic}
\usepackage{longtable,booktabs,setspace}
\usepackage{chemarr} %% Useful for one reaction arrow, useless if you're not a chem major
\usepackage[hyphens]{url}
\usepackage{pdfpages}
\usepackage{tikz}
\usetikzlibrary{calc}
% Added by CII
\usepackage{hyperref}
\usepackage{lmodern}
\usepackage{float}
\floatplacement{figure}{H}
% End of CII addition
\usepackage{rotating}
\usepackage{upgreek} % tk : pour pouvoir utiliser le symbole µ droit (pas en itallic)
\usepackage{pdfpages} % tk : pour pouvoir insérer des fichiers pdf dans le corp de texte
\usepackage{lscape} % tk : pour pouvoir insérer des images au format paysage
\newcommand{\blandscape}{\begin{landscape}}
\newcommand{\elandscape}{\end{landscape}}
\usepackage[utf8]{inputenc}

% Next line commented out by CII
%%% \usepackage{natbib}
% Comment out the natbib line above and uncomment the following two lines to use the new
% biblatex-chicago style, for Chicago A. Also make some changes at the end where the
% bibliography is included.
%\usepackage{biblatex-chicago}
%\bibliography{thesis}


% Added by CII (Thanks, Hadley!)
% Use ref for internal links
\renewcommand{\hyperref}[2][???]{\autoref{#1}}
\def\chapterautorefname{Chapter}
\def\sectionautorefname{Section}
\def\subsectionautorefname{Subsection}
% End of CII addition

% Added by CII
\usepackage{caption}
\captionsetup{width=5in}
% End of CII addition

% \usepackage{times} % other fonts are available like times, bookman, charter, palatino


% To pass between YAML and LaTeX the dollar signs are added by CII
\title{THÈSE}
\author{Thomas Karaouzene}
\lab{Génétique, Epigénétique et Thérapies de l'Infertilité (GETI) et
Techniques de l'Ingénierie Médicale et de la Complexité - Informatique,
Mathématiques et Applications de Grenoble (TIMC-IMAG)}
\date{07 novembre 2017}
\division{Mathematics and Natural Sciences}
\advisor{Pierre Ray}
%If you have two advisors for some reason, you can use the following
% Uncommented out by CII
\altadvisor{Nicolas Thierry-Mieg}
% End of CII addition
%\institution{}
%\degree{}

%%% Remember to use the correct department!
\department{Ingénierie de la Santé, de la Cognition et Environnement (EDISCE)}
% if you're writing a thesis in an interdisciplinary major,
% uncomment the line below and change the text as appropriate.
% check the Senior Handbook if unsure.
%\thedivisionof{The Established Interdisciplinary Committee for}
% if you want the approval page to say "Approved for the Committee",
% uncomment the next line
%\approvedforthe{Committee}

% Added by CII
%%% Copied from knitr
%% maxwidth is the original width if it's less than linewidth
%% otherwise use linewidth (to make sure the graphics do not exceed the margin)
\makeatletter
\def\maxwidth{ %
  \ifdim\Gin@nat@width>\linewidth
    \linewidth
  \else
    \Gin@nat@width
  \fi
}
\makeatother

\renewcommand{\contentsname}{Table of Contents}
% End of CII addition

\setlength{\parskip}{0pt}

% Added by CII
  %\setlength{\parskip}{\baselineskip}
  \usepackage[parfill]{parskip}

\providecommand{\tightlist}{%
  \setlength{\itemsep}{0pt}\setlength{\parskip}{0pt}}

\Acknowledgements{

}

\Dedication{

}

\Preface{

}

\Abstract{

}

	\usepackage{tikz}
% End of CII addition
%%
%% End Preamble
%%
%

\begin{document}

% Everything below added by CII
  \maketitle

\frontmatter % this stuff will be roman-numbered
\pagestyle{empty} % this removes page numbers from the frontmatter



  \hypersetup{linkcolor=black}
  \setcounter{tocdepth}{3}
  \tableofcontents

  \listoftables

  \listoffigures



\mainmatter % here the regular arabic numbering starts
\pagestyle{fancyplain} % turns page numbering back on

\chapter{thesisdown::thesis\_epub:
default}\label{thesisdownthesis_epub-default}

\chapter*{Résumé}\label{resume}
\addcontentsline{toc}{chapter}{Résumé}

\chapter{Introduction}\label{introInf}

\section{La spermatogenèse}\label{la-spermatogenese}

\subsection{Rappels sur le testicule}\label{rappels-sur-le-testicule}

\subsection{La phase de
multiplication}\label{la-phase-de-multiplication}

\subsection{La méiose}\label{meiose}

\subsection{La spermiogenèse}\label{spermiogenese}

\section{Structure et fonction du
spermatozoïde}\label{structure-et-fonction-du-spermatozoide}

\subsection{La tête}\label{la-tete}

\subsection{Le flagelle}\label{le-flagelle}

\section{L'infertilité masculine}\label{linfertilite-masculine}

\subsection{Les différents phénotypes d'infertilité
masculine}\label{les-differents-phenotypes-dinfertilite-masculine}

\subsubsection{Anomalies liées à la quantité
spermatique}\label{infquant}

\subsubsection{Anomalies liées à la
morphologie}\label{anomalies-liees-a-la-morphologie}

\subsubsection{Anomalies liées à la
mobilité}\label{anomalies-liees-a-la-mobilite}

\subsection{La génétique de
l'infertilité}\label{la-genetique-de-linfertilite}

\subsubsection{Les causes fréquentes}\label{les-causes-frequentes}

\subsubsection{Les nouveaux gènes}\label{les-nouveaux-genes}

\section{Les techniques d'analyses
génétiques}\label{les-techniques-danalyses-genetiques}

\subsection{\texorpdfstring{Approche ``gènes
candidats''}{Approche gènes candidats}}\label{approche-genes-candidats}

\subsection{Les puces}\label{les-puces}

\subsubsection{Les puces à expression}\label{les-puces-a-expression}

\subsubsection{Les puces à SNP, plateforme
génotypage}\label{les-puces-a-snp-plateforme-genotypage}

\subsubsection{Les puces à indels}\label{les-puces-a-indels}

\subsubsection{Limitation}\label{limitation}

\subsection{Le séquençage NGS}\label{ngs}

\subsubsection{La capture des parties à séquencer, avantages et
inconvénients}\label{la-capture-des-parties-a-sequencer-avantages-et-inconvenients}

\subsubsection{L'amplification}\label{lamplification}

\subsubsection{La réaction de séquence}\label{la-reaction-de-sequence}

\section{L'analyse bioinformatique des données de
NGS}\label{lanalyse-bioinformatique-des-donnees-de-ngs}

\subsection{Les données fournies par le
NGS}\label{les-donnees-fournies-par-le-ngs}

\subsubsection{\texorpdfstring{Un \emph{read}, c'est quoi
?}{Un read, c'est quoi ?}}\label{un-read-cest-quoi}

\subsubsection{Le format FASTQ}\label{fastq}

\subsection{L'alignement}\label{lalignement}

\subsection{L'appel des variants}\label{varcall}

\subsection{L'annotation des variants}\label{lannotation-des-variants}

\subsection{Le filtrage des variants}\label{le-filtrage-des-variants}

\subsection{Conclusion NGS}\label{conclusion-ngs}

\chapter{Mise en place d'une stratégie pour l'analyse des données
exomiques -- application en recherche
clinique}\label{mise-en-place-dune-strategie-pour-lanalyse-des-donnees-exomiques-application-en-recherche-clinique}

\section{Méthode : Description du
pipeline}\label{methode-description-du-pipeline}

\subsection{\texorpdfstring{L'alignement des
\emph{reads}}{L'alignement des reads}}\label{lalignement-des-reads}

\subsection{L'appel des variants}\label{lappel-des-variants}

\subsection{L'annotation}\label{lannotation}

\subsection{Le filtrage des variants}\label{le-filtrage-des-variants-1}

\section{Résultats 1 : Analyse de 3 phénotypes par des cas
familliaux}\label{resultats-1-analyse-de-3-phenotypes-par-des-cas-familliaux}

\subsection{Résultats des différents étapes de
l'analyse}\label{resultats-des-differents-etapes-de-lanalyse}

\subsubsection{Résultat de l'alignement}\label{resultat-de-lalignement}

\subsubsection{L'appel des variants}\label{lappel-des-variants-1}

\subsubsection{L'annotation des
variants}\label{lannotation-des-variants-1}

\subsubsection{Le filtrage des
variants}\label{le-filtrage-des-variants-2}

\subsection{Article n° 3}\label{article-n-3}

\subsubsection{Contexte et objectifs}\label{contexte-et-objectifs}

\subsubsection{Principaux résultats}\label{principaux-resultats}

\subsection{Article n° 4}\label{article-n-4}

\subsubsection{Contexte et objectifs}\label{contexte-et-objectifs-1}

\subsubsection{Principaux résultats}\label{principaux-resultats-1}

\subsection{Article n° 5}\label{article-n-5}

\subsubsection{Whole-exome sequencing of familial cases of multiple
morphological abnormalities of the sperm flagella (MMAF) reveals new
DNAH1
mutations}\label{whole-exome-sequencing-of-familial-cases-of-multiple-morphological-abnormalities-of-the-sperm-flagella-mmaf-reveals-new-dnah1-mutations}

\subsubsection{Contexte et objectifs}\label{contexte-et-objectifs-2}

\subsubsection{Principaux résultats}\label{principaux-resultats-2}

\section{Résultats 2 : Étude d'une cohorte de femmes
infertiles}\label{resultats-2-etude-dune-cohorte-de-femmes-infertiles}

\subsection{Article n° 6}\label{article-n-6}

\subsubsection{Contexte et objectifs}\label{contexte-et-objectifs-3}

\subsubsection{Principaux résultats}\label{principaux-resultats-3}

\section{Résultats 3 : Étude d'une large cohorte de patients
MMAF}\label{resultats-3-etude-dune-large-cohorte-de-patients-mmaf}

\subsection{Article n° 7}\label{article-n-7}

\subsubsection{Contexte et objectifs}\label{contexte-et-objectifs-4}

\subsubsection{Principaux résultats}\label{principaux-resultats-4}

\chapter{Investigation génétique et physiologique de la
globozoospermie}\label{globo}

\section{Introduction sur la
globozoospermie}\label{introduction-sur-la-globozoospermie}

\section{\texorpdfstring{Résultats 1 : Les mécanismes mutationnels
entrainant la délétion au locus de \emph{DPY19L2} chez
l'humain}{Résultats 1 : Les mécanismes mutationnels entrainant la délétion au locus de DPY19L2 chez l'humain}}\label{mecamut}

\subsection{Article n° 1:}\label{article-n-1}

\subsubsection{Contexte et objectifs}\label{contexte-et-objectifs-5}

\subsubsection{Principaux résultats}\label{principaux-resultats-5}

\section{Résultat 2 : La transcriptomique}\label{transcriptome}

\subsection{Article n° 2:}\label{article-n-2}

\subsubsection{Contexte et objectifs}\label{contexte-et-objectifs-6}

\subsubsection{Principaux résultats :}\label{principaux-resultats-6}

\chapter*{Conclusion et discussion}\label{conclusion-et-discussion}
\addcontentsline{toc}{chapter}{Conclusion et discussion}

L'infertilité est une problématique qui concerne entre 10 et 15\% des
couples {[}\protect\hyperlink{ref-Boivin2007a}{1}{]} faisant de cette
pathologie un enjeu de santé publique. Bien que les causes de ce
phénotype puissent être multifactorielles et acquises au cours de la vie
de l'l'individu, notamment suite à des infections du système urogénitale
ou encore à des perturbations du système endocrinien, la composante
génétique est extrêmement importante. À ce jour, malgré les efforts de
nombreuses équipes, incluant la notre, seulement une poignée de gènes a
pu être reliée à ce phénotype. De plus, pour nombre d'entre eux, les
bases moléculaires reliant une mutation au phénotype d'infertilité
restent inconnues.

Dès lors, la première partie de mon travail de thèse a consisté à
contribuer à la caractérisation du phénotype de globozoospermie. Ce
phénotype, entrainant la production de 100\% de spermatozoïdes à têtes
rondes et dépourvus d'acrosomes est principalement causé, chez l'humain,
par une délétion homozygote récurrente entrainant la perte de la
totalité de la séquence du gène \emph{DPY19L2}. Ainsi, dans deux études
différentes, nous avons pu, dans un premier temps, mieux caractériser
les mécanismes moléculaires responsables de cette délétion. Ainsi, nous
avons pu mettre en évidence cinq points de cassures au niveau des LCRs
flanquant la séquence de \emph{DPY19L2} chez l'humain. Ceux-ci étant
tous concentrés dans une région d'environ 1150 pb contenant en son
centre un site de reconnaissance consensus de la protéine PRDM9 connue
pour son implication dans la recombinaison chromosomique chez l'humain
et la souris {[}\protect\hyperlink{ref-Parvanov2010}{2},
\protect\hyperlink{ref-Baudat2010}{3}{]}. Cette même étude a également
permis de démontrer que les effets de la sélection naturelle étaient
responsables du paradoxe consistant à observer plus fréquemment, dans la
population générale, l'allèle dupliqué à ce locus que l'allèle délété
tandis que \emph{de novo} l'allèle délété est produit, en théorie et en
pratique, plus fréquemment que l'allèle dupliqué. L'étude de ce
phénotype nous a par la suite poussée à étudier le modèle murin KO
\emph{Dpy19l2}\textsuperscript{-/-} présentant le même phénotype que
l'humain. Afin d'expliquer l'absence de la protéine PLCZ1 chez l'humain
globozoosperme, nous avons effectué une analyse comparative des
transcriptomes testiculaires de souris sauvages
\emph{Dpy19l2}\textsuperscript{+/+} et KO
\emph{Dpy19l2}\textsuperscript{-/-}. Bien qu'aucun dérèglement
transcriptionnel n'ait pu être observé pour le gène \emph{Plcz1} cette
étude nous a permis de mettre en évidence un total de 75 gènes
présentant des dérégulations transcriptionnelles pouvant expliquer en
partie les anomalies physiologiques et morphologiques des spermatozoïdes
des souris\emph{Dpy19l2}\textsuperscript{-/-}.

Notre équipe ayant, entre 2012 et 2016, effectué le séquençage exomique
de nombreux patients présentant tous un phénotype d'infertilité, la
seconde partie de mon travail de thèse a été de mettre au point un
pipeline permettant l'analyse des données générées lors du processus de
séquençage exomique. Celui-ci ayant pour vocation première de mettre en
évidence les variants responsables des phénotypes de ces patients.
Contrairement à la plupart des pipelines d'analyse de données WES
existant, celui-ci prend en charge l'ensemble des étapes de l'analyse
allant de l'alignement des \emph{short-reads} sur le génome de référence
jusqu'à la priorisation des variants en passant par l'appel des variants
et leur annotation. Les résultats de chacune de ces étapes pouvant être
contrôlées et personnalisées grâce à des paramètres ajustables.
L'alignement des \emph{reads} est effectué par le logiciel MAGIC tandis
que les variants et leur génotype sont appelés par un algorithme
développé dans notre laboratoire spécifiquement conçu pour analyser les
informations fournies par MAGIC et dont les paramètres sont ajustables
en fonctions de la distribution des pourcentages de \emph{reads}
variants observés dans les données analysées. Pour l'annotation nous
avons utilisé plusieurs ressources extérieures tel que le logiciel
Variant Effect Predictor qui va nous informer de l'effet d'un variant
sur l'ensemble des transcrits qu'il chevauche. De même, les bases de
données ExAC ESP6500 ou encore 1KG nous donne une indication de la
fréquence des variants dans la population générale. Une fois ces étapes
effectuées, nous avons mis en place plusieurs filtres successifs afin
d'éliminer de nos listes les variants ayant le moins de chances d'être
responsables du phénotype des différents patients. Ceux-ci s'appuient à
la fois sur les critères qualité des résultats de séquençage, le
génotype des variants, leur fréquence ou encore leur impact sur la
protéine.

L'efficacité de ce pipeline a pu être démontré grâce à son utilisation
sur des cas familiaux mais aussi sur des cohortes d'individus non
apparentés. Ainsi, nous avons pu dans un premier temps confirmer
l'importance de l'implication du gène \emph{DNAH1} dans le syndrome MMAF
en le retrouvant muté chez \ldots{} de nos patients. Ensuite, dans un
second temps, ce pipeline nous a permis de mettre en évidence un total
de 5 nouveaux gènes dans des phénotypes d'infertilité masculine et
féminine. Ainsi les gènes \emph{CFPA43}, \emph{CFAP44}, retrouvés
respectivement mutés chez \ldots{} et \ldots{} de nos patients ont pu
être liés à leur syndrome MMAF. Aussi, une même mutation impactant le
gène PATL2 a pu être reliée au phénotype de déficience méiotique
ovocytaire de cinq femmes. Pour finir, des mutations sur les gènes
\emph{SPINK2} et \emph{PLCZ1} ont, elles aussi, pu être liées aux
phénotypes d'azoospermie et d'echec de fécondation dont étaient atteint
deux fratries.

Ces résultats sont cependant à relativiser puisque pour plusieurs de nos
patients, aucun candidat n'a pu être identifié. Plusieurs raisons
peuvent expliquer cela. Tout d'abord, au cours des analyses décrites
dans ces manuscrits nous nous concentrons uniquement sur les SNPs et les
indels. Cependant de nombreux logiciels tel que ExomeDepth
{[}\protect\hyperlink{ref-Plagnol2012}{4}{]}, CoNIFER
{[}\protect\hyperlink{ref-Krumm2012}{5}{]} ou encore ExomeCNV
{[}\protect\hyperlink{ref-Sathirapongsasuti2011}{6}{]} permettent de
détecter des CNVs à partir de données WES et / ou WGS. Les stratégies de
prédictions de ces logiciels pouvant être extrêmement différents
(\textbf{Figure : }\ref{fig:pictcnvdetection}), le profil des CNVs
détectés ou non le sera aussi {[}\protect\hyperlink{ref-Zhao2013}{7},
\protect\hyperlink{ref-Guo2013}{8}{]}. Ainsi, dans des analyses non
décrites dans ce manuscrit, j'ai pu chercher à identifier des CNVs à
partir de nos données d'exome à l'aide du logiciel ExomeDepth
{[}\protect\hyperlink{ref-Plagnol2012}{4}{]}. Cette approche a été
extrêmement concluante puisqu'elle a permis d'identifier une délétion
homozygote sur le gène \emph{WDR66} chez 7 de nos patients pour lesquels
aucun candidat n'avait été alors identifié. Ces délétions ont ensuite pu
être confirmées par PCR et la caractérisation de ce gène est
actuellement en cours au sein de notre équipe. Au vu de cette réussite,
il est désormais prévu d'intégrer ce genre d'analyse de manière
automatique et systématique au sein de notre pipeline. Pour les autres
patients n'ayant eu aucun candidat identifié, il est possible que le
choix de la stratégie du séquençage exomique plutôt que du génome entier
ait masqué la cause génétique du phénotype de certains de nos patients.
En effet, dans ces analyses, nous nous sommes concentrés sur l'analyse
des variants situés dans les parties codantes \textbf{uniquement}. Ainsi
les variants situés par exemple dans les microARN n'ont pu être
observés. Or, les microARN jouent un rôle important dans la régulation
génique principalement en influant sur la stabilité d'ARNm cibles et
sont présent en grande quantité au sein des cellules germinales et leur
importance dans la spermatogenèse a déjà été démontrée chez la souris
{[}\protect\hyperlink{ref-Comazzetto2014}{9}{]} ainsi que plus récemment
chez d'autres mammifères dont l'humain
{[}\protect\hyperlink{ref-Chen2017}{10}{]} laissant penser que des
défauts altérants ces microARN pourraient entrainer des
dysfonctionnements de la spermatogenèse. Aussi, il faut noter que les
analyses WES \textbf{et} WGS ne permettent pas d'observer les défauts
épigénétiques, or, ceux-ci représentent une part croissante des causes
impliquées dans les cas d'infertilité masculine
{[}\protect\hyperlink{ref-Carrell2011}{11}--\protect\hyperlink{ref-Dada2012}{13}{]}.
Aussi, au vus du grand nombre de gène impliqué dans la spermatogénèse il
est très possible que les causes génétiques responsables d'un même
phénotype puissent être très hétérogène. Par exemple, dans le cas de
l'analyse de la cohorte de patients MMAF, 3630 variants subsistaient
après avoir appliqué l'ensemble des filtres. Ces variants impactaient
2780 gènes différents parmi lesquels 1684 étaient retrouvés mutés chez
uniquement un seul des 77 patients de la cohorte. Au vu de ce nombre
important de gènes, il est très compliqué d'effectué des analyses
poussées sur l'ensemble d'entre eux. Dès lors, il est possible que la
cause génétique responsable du phénotype d'un patient soit ``noyé''
parmi les nombreux variants restant mettant ainsi en évidence la
nécessité de créer de nouveaux filtres afin de pouvoir réduire encore
cette liste.

\newpage

\begin{figure}

{\centering \includegraphics[scale=0.55]{figure/cvn_detection_strategies} 

}

\caption[Présentation de cinq approches permettant la détection de CNVs à partir de données NGS]{\textbf{\emph{Présentation de cinq approches
permettant la détection de CNVs à partir de données NGS} d'après
{[}\protect\hyperlink{ref-Zhao2013}{7}{]}} : \textbf{A} : Cette
stratégie, permet de prédire des CNVs à partir des alignements
discordant des deux \emph{ends} d'un même \emph{read}, c'est à dire en
répertoriant les \emph{reads} pour lesquels la distance séparant les
deux \emph{ends} après l'alignement est significativement supérieure à
la taille moyenne de l'insert. \textbf{B} : La méthode \emph{split-read}
se base sur les \emph{reads} s'alignant de manière partielle sur
plusieurs régions génomiques. \textbf{C} : L'approche \emph{read depth}
compare la couverture observée sur plusieurs région génomique pour
prédire des CNVs. \textbf{D} : Cette méthode effectue un assemblage
\emph{de novo} (sans utilisé de génome de référence) ; les résultats de
l'assemblage appelés \emph{contigs} sont comparés au génome de référence
a posteriori pour détecter les CNVs. \textbf{E} : Cette méthode combine
les approches \textbf{A} et \textbf{C}.}\label{fig:pictcnvdetection}
\end{figure}


















\newpage

C'est dans ce but que notre équipe travaille actuellement au
développement du score MutaScript. Ce score a pour but de classer
l'ensemble des transcrit codant en fonction de leur charge mutationnelle
avec l'idée sous-jacente que les transcrits les plus mutés dans la
population générale ne sont probablement pas impliqués dans des
pathologies sévères à transmission Mendélienne, et \emph{a contrario}
ceux retrouvés comme n'étant pas / peu mutés le sont probablement. Pour
ce faire, le score MutaScript repose sur trois informations principales.
La première étant le jeu de transcrits fournit par Ensembl
{[}\protect\hyperlink{ref-Aken2017}{14}{]}. Afin de connaitre la charge
mutationnelle de ces transcrits, nous nous sommes basées sur les
variants mis à disposition par ExAC
{[}\protect\hyperlink{ref-Lek2016}{15}{]} qui réunit les données d'exome
de 60.706 individus non apparentés que nous avons ensuite annoté grâce
au logiciel \emph{variant effect predictor} (VEP)
{[}\protect\hyperlink{ref-McLaren2016}{16}{]} afin de prédire l'impact
de chaque variant sur l'ensemble des transcrits qu'ils chevauchent de
sorte à ce que les variants ayant un impact prédit comme étant délétère
aient une plus grosse contribution au score MutaScript que ceux ayant un
impact faible. Ces dernières années, des scores tel que le
\emph{residual variance intolerance score} (RVIS)
{[}\protect\hyperlink{ref-Petrovski2013}{17}{]} ou encore \emph{the the
Probability of loss-of-function Incoherency} (pLI)
{[}\protect\hyperlink{ref-Lek2016}{15}{]} ont vu le jour. MutaScript se
présente comme une alternative à ces derniers scores et, bien que sa
fonction soit similaire, il diffère de ceux-ci sur de nombreux points.
Tout d'abord, MutaScript donne un score à l'ensemble des transcrits
codant pour une protéine là où pLI donne un score seulement au transcrit
consensus de chaque gène et RVIS qui agrège les séquences codantes de
l'ensemble des transcrits d'un même gène créant ainsi un transcrit
``chimérique''. Ce procédé facilite l'interprétation du score mais
engendre une perte d'information puisque l'on se retrouve avec un seul
score par gène et non par transcrits. De plus, dans la conception de
leur score, RVIS et pLI ne considère que les variants dit
\emph{loss-of-function} (LoF), c'est à dire les variants impactant
l'épissage, engendrant un codon stop ou un décalage du cadre de lecture.
Cependant, ces variants ne représentent qu'une faible proportion des
variants fournit par la base de données ExAC. C'est pourquoi, MutaScript
prend en compte l'ensemble des variants, peu importe leur impact sur les
différents transcrits qu'ils chevauchent, et leur attribue un poids en
fonction de cet impact de sorte à ce que les variants considérés comme
étant les plus délétères contribuent plus au score d'un transcrits que
les autres. Aussi, l'étude des scores RVIS et pLI nous a permis de
mettre en évidence une corrélation forte entre le score qu'ils
attribuent à un gène et la taille de la séquence codante (CDS) de ce
même gène. Cette corrélation étant due à un biais causé par leur manière
de calculer leur score et non à une réalité biologique, MutaScript fut
construit de sorte à éviter cette corrélation qui peut mener à des
erreurs d'interprétations. Le développement de ce score étant en cours
de finalisation.

\begin{center}\rule{0.5\linewidth}{\linethickness}\end{center}

Pendant de nombreuses années, la Science et la Technique étaient
considérées comme des disciplines distinctes. Elles étaient pratiquées,
dans la grande majorité des cas, de manière indépendante l'une de
l'autre et surtout par des personnes différentes n'entretenant que peu
d'interactions. Bien que la distinction entre Science et Technique soit
réelle, la première pouvant être définie comme la quête de la
connaissance et de la compréhension du monde tandis que la seconde met
en œuvre un ensemble de moyen afin de modifier celui-ci d'une manière
déterminée à l'avance, l'interdépendance liant ces deux notions n'a
jamais été aussi forte qu'à notre époque tant et si bien qu'elles sont
souvent confondues. En effet, il est courant d'entendre parler de
progrès scientifique pour présenter une innovation technologique et
\emph{vice versa}. Ainsi, si la Science n'est pas la Technique, elle est
dans de nombreux cas dépendante de celle-ci. En effet, comme nous avons
pu le voir, l'étude et la connaissance du génome ont dû attendre les
progrès techniques permettant notamment le séquençage de l'ADN. La
Technique, elle, n'a pas nécessairement besoin de savoirs scientifiques
pour être conçue : des savoirs empiriquement acquis suffisent à
l'application d'une technique. Par exemple, bien qu'ils n'aient eu
aucune conscience des mécanismes scientifiques sous-jacent, les premiers
hommes ont su maitriser plusieurs techniques de production et
d'entretien du feu. De la même manière les agriculteurs n'ont pas eu
besoin d'attendre et de comprendre les travaux sur la génétique et
l'hérédité pour observer que la mise en reproduction des bêtes les plus
productives permettait de maximiser les chances que la descendance soit
elle aussi très productive. Cependant la Technique utilise de plus en
plus des connaissances scientifiques et a ainsi finit par beaucoup
dépendre d'elle en utilisant et appliquant des savoirs scientifiques.
Ainsi est née la Technologie. Les travaux décrits dans ce manuscrit
illustrent parfaitement cette relation d'interdépendance entre la
Science et le Technique / Technologie. En effet, la connaissance du
génome a été permise par l'émergence des différentes technologies de
séquençage qui s'appuient elles aussi sur de nombreuses connaissances
scientifiques. On peut dès lors s'attendre à ce que Science et
Techniques / Technologie continuent d'évoluer de manière concomitante en
s'entre alimentant. Dès lors, on peut prédire que les prochains progrès
technologiques seront à l'origine de découvertes scientifiques qui
serviront elles même à la fois de socle mais aussi de guide aux
évolutions technologiques futures.

\chapter{Mutations in DNAH1, which Encodes an Inner Arm Heavy Chain
Dynein, Lead to Male Infertility from Multiple Morphological
Abnormalities of the Sperm Flagella}\label{dnah12014}

\chapter*{References}\label{references}
\addcontentsline{toc}{chapter}{References}

\hypertarget{refs}{}
\hypertarget{ref-Boivin2007a}{}
1. J. Boivin, L. Bunting, J.A. Collins, and K.G. Nygren: ``International
estimates of infertility prevalence and treatment-seeking: potential
need and demand for infertility medical care.'' \emph{Human
Reproduction}. vol. 22, no. 6, pp. 1506--1512, 2007.

\hypertarget{ref-Parvanov2010}{}
2. E.D. Parvanov, P.M. Petkov, and K. Paigen: ``Prdm9 controls
activation of mammalian recombination hotspots.'' \emph{Science (New
York, N.Y.)}. vol. 327, no. 5967, pp. 835, 2010.

\hypertarget{ref-Baudat2010}{}
3. F. Baudat, J. Buard, C. Grey, A. Fledel-Alon, C. Ober, M. Przeworski,
G. Coop, and B. de Massy: ``PRDM9 is a major determinant of meiotic
recombination hotspots in humans and mice.'' \emph{Science (New York,
N.Y.)}. vol. 327, no. 5967, pp. 836--40, 2010.

\hypertarget{ref-Plagnol2012}{}
4. V. Plagnol, J. Curtis, M. Epstein, K.Y. Mok, E. Stebbings, S.
Grigoriadou, N.W. Wood, S. Hambleton, S.O. Burns, A.J. Thrasher, D.
Kumararatne, R. Doffinger, and S. Nejentsev: ``A robust model for read
count data in exome sequencing experiments and implications for copy
number variant calling.'' \emph{Bioinformatics (Oxford, England)}. vol.
28, no. 21, pp. 2747--54, 2012.

\hypertarget{ref-Krumm2012}{}
5. N. Krumm, P.H. Sudmant, A. Ko, B.J. O'Roak, M. Malig, B.P. Coe,
N.E.S. NHLBI Exome Sequencing Project, A.R. Quinlan, D.A. Nickerson, and
E.E. Eichler: ``Copy number variation detection and genotyping from
exome sequence data.'' \emph{Genome research}. vol. 22, no. 8, pp.
1525--32, 2012.

\hypertarget{ref-Sathirapongsasuti2011}{}
6. J.F. Sathirapongsasuti, H. Lee, B.A.J. Horst, G. Brunner, A.J.
Cochran, S. Binder, J. Quackenbush, and S.F. Nelson: ``Exome
sequencing-based copy-number variation and loss of heterozygosity
detection: ExomeCNV.'' \emph{Bioinformatics (Oxford, England)}. vol. 27,
no. 19, pp. 2648--54, 2011.

\hypertarget{ref-Zhao2013}{}
7. M. Zhao, Q. Wang, Q. Wang, P. Jia, and Z. Zhao: ``Computational tools
for copy number variation (CNV) detection using next-generation
sequencing data: features and perspectives.'' \emph{BMC Bioinformatics}.
vol. 14, pp. S1, 2013.

\hypertarget{ref-Guo2013}{}
8. Y. Guo, Q. Sheng, D.C. Samuels, B. Lehmann, J.A. Bauer, J. Pietenpol,
and Y. Shyr: ``Comparative study of exome copy number variation
estimation tools using array comparative genomic hybridization as
control.'' \emph{BioMed research international}. vol. 2013, pp. 915636,
2013.

\hypertarget{ref-Comazzetto2014}{}
9. S. Comazzetto, M. Di Giacomo, K.D. Rasmussen, C. Much, C. Azzi, E.
Perlas, M. Morgan, and D. O'Carroll: ``Oligoasthenoteratozoospermia and
Infertility in Mice Deficient for miR-34b/c and miR-449 Loci.''
\emph{PLoS Genetics}. vol. 10, no. 10, pp. e1004597, 2014.

\hypertarget{ref-Chen2017}{}
10. X. Chen, X. Li, J. Guo, P. Zhang, and W. Zeng: ``The roles of
microRNAs in regulation of mammalian spermatogenesis.'' \emph{Journal of
animal science and biotechnology}. vol. 8, pp. 35, 2017.

\hypertarget{ref-Carrell2011}{}
11. D.T. Carrell and K.I. Aston: ``The search for SNPs, CNVs, and
epigenetic variants associated with the complex disease of male
infertility.'' \emph{Systems Biology in Reproductive Medicine}. vol. 57,
no. 1-2, pp. 17--26, 2011.

\hypertarget{ref-Dada2011}{}
12. R. Dada, M. Shamsi, and K. Kumar: ``Genetic and epigenetic factors:
Role in male infertility.'' \emph{Indian Journal of Urology}. vol. 27,
no. 1, pp. 110, 2011.

\hypertarget{ref-Dada2012}{}
13. R. Dada, M. Kumar, R. Jesudasan, J.L. Fernández, J. Gosálvez, and A.
Agarwal: ``Epigenetics and its role in male infertility.'' \emph{Journal
of Assisted Reproduction and Genetics}. vol. 29, no. 3, pp. 213--223,
2012.

\hypertarget{ref-Aken2017}{}
14. B.L. Aken, P. Achuthan, W. Akanni, M.R. Amode, F. Bernsdorff, J.
Bhai, K. Billis, D. Carvalho-Silva, C. Cummins, P. Clapham, L. Gil, C.G.
Girón, L. Gordon, T. Hourlier, S.E. Hunt, S.H. Janacek, T. Juettemann,
S. Keenan, M.R. Laird, I. Lavidas, T. Maurel, W. McLaren, B. Moore, D.N.
Murphy, R. Nag, V. Newman, M. Nuhn, C.K. Ong, A. Parker, M. Patricio,
H.S. Riat, D. Sheppard, H. Sparrow, K. Taylor, A. Thormann, A. Vullo, B.
Walts, S.P. Wilder, A. Zadissa, M. Kostadima, F.J. Martin, M. Muffato,
E. Perry, M. Ruffier, D.M. Staines, S.J. Trevanion, F. Cunningham, A.
Yates, D.R. Zerbino, and P. Flicek: ``Ensembl 2017.'' \emph{Nucleic
acids research}. vol. 45, no. D1, pp. D635--D642, 2017.

\hypertarget{ref-Lek2016}{}
15. M. Lek, K.J. Karczewski, E.V. Minikel, K.E. Samocha, E. Banks, T.
Fennell, A.H. O'Donnell-Luria, J.S. Ware, A.J. Hill, B.B. Cummings, T.
Tukiainen, D.P. Birnbaum, J.A. Kosmicki, L.E. Duncan, K. Estrada, F.
Zhao, J. Zou, E. Pierce-Hoffman, J. Berghout, D.N. Cooper, N. Deflaux,
M. DePristo, R. Do, J. Flannick, M. Fromer, L. Gauthier, J. Goldstein,
N. Gupta, D. Howrigan, A. Kiezun, M.I. Kurki, A.L. Moonshine, P.
Natarajan, L. Orozco, G.M. Peloso, R. Poplin, M.A. Rivas, V.
Ruano-Rubio, S.A. Rose, D.M. Ruderfer, K. Shakir, P.D. Stenson, C.
Stevens, B.P. Thomas, G. Tiao, M.T. Tusie-Luna, B. Weisburd, H.-H. Won,
D. Yu, D.M. Altshuler, D. Ardissino, M. Boehnke, J. Danesh, S. Donnelly,
R. Elosua, J.C. Florez, S.B. Gabriel, G. Getz, S.J. Glatt, C.M. Hultman,
S. Kathiresan, M. Laakso, S. McCarroll, M.I. McCarthy, D. McGovern, R.
McPherson, B.M. Neale, A. Palotie, S.M. Purcell, D. Saleheen, J.M.
Scharf, P. Sklar, P.F. Sullivan, J. Tuomilehto, M.T. Tsuang, H.C.
Watkins, J.G. Wilson, M.J. Daly, D.G. MacArthur, and D.G. Exome
Aggregation Consortium: ``Analysis of protein-coding genetic variation
in 60,706 humans.'' \emph{Nature}. vol. 536, no. 7616, pp. 285--91,
2016.

\hypertarget{ref-McLaren2016}{}
16. W. McLaren, L. Gil, S.E. Hunt, H.S. Riat, G.R.S. Ritchie, A.
Thormann, P. Flicek, and F. Cunningham: ``The Ensembl Variant Effect
Predictor.'' \emph{Genome biology}. vol. 17, no. 1, pp. 122, 2016.

\hypertarget{ref-Petrovski2013}{}
17. S. Petrovski, Q. Wang, E.L. Heinzen, A.S. Allen, D.B. Goldstein, E.
Davydov, D. Goode, M. Sirota, G. Cooper, A. Sidow, I. Adzhubei, S.
Schmidt, L. Peshkin, V. Ramensky, A. Gerasimova, W. Lee, P. Yue, Z.
Zhang, N. Sim, P. Kumar, J. Hu, S. Henikoff, G. Schneider, S. Hicks, D.
Wheeler, S. Plon, M. Kimmel, G. Cooper, J. Shendure, B. Neale, Y. Kou,
L. Liu, A. Ma'ayan, K. Samocha, B. O'Roak, L. Vives, S. Girirajan, E.
Karakoc, N. Krumm, S. Sanders, M. Murtha, A. Gupta, J. Murdoch, M.
Raubeson, J. de Ligt, M. Willemsen, B. van Bon, T. Kleefstra, H. Yntema,
A. Rauch, D. Wieczorek, E. Graf, T. Wieland, S. Endele, I. Iossifov, M.
Ronemus, D. Levy, Z. Wang, I. Hakker, J. Tennessen, A. Bigham, T.
O'Connor, W. Fu, E. Kenny, K. Pruitt, J. Harrow, R. Harte, C. Wallin, M.
Diekhans, A. McKenna, M. Hanna, E. Banks, A. Sivachenko, K. Cibulskis,
E. Heinzen, K. Swoboda, Y. Hitomi, F. Gurrieri, S. Nicole, J. Eppig, J.
Blake, C. Bult, J. Kadin, J. Richardson, B. Georgi, B. Voight, M. Bucan,
N. Goldman, Z. Yang, W. Li, C. Wu, C. Luo, M. Nei, T. Gojobori, C.
Zhang, J. Wang, M. Long, C. Fan, K. Goh, M. Cusick, D. Valle, B. Childs,
M. Vidal, E. DeLong, D. DeLong, D. Clarke-Pearson, X. Robin, N. Turck,
A. Hainard, N. Tiberti, and F. Lisacek: ``Genic Intolerance to
Functional Variation and the Interpretation of Personal Genomes.''
\emph{PLoS Genetics}. vol. 9, no. 8, pp. e1003709, 2013.


% Index?

\end{document}

