% This is the Reed College LaTeX thesis template. Most of the work
% for the document class was done by Sam Noble (SN), as well as this
% template. Later comments etc. by Ben Salzberg (BTS). Additional
% restructuring and APA support by Jess Youngberg (JY).
% Your comments and suggestions are more than welcome; please email
% them to cus@reed.edu
%
% See http://web.reed.edu/cis/help/latex.html for help. There are a
% great bunch of help pages there, with notes on
% getting started, bibtex, etc. Go there and read it if you're not
% already familiar with LaTeX.
%
% Any line that starts with a percent symbol is a comment.
% They won't show up in the document, and are useful for notes
% to yourself and explaining commands.
% Commenting also removes a line from the document;
% very handy for troubleshooting problems. -BTS

% As far as I know, this follows the requirements laid out in
% the 2002-2003 Senior Handbook. Ask a librarian to check the
% document before binding. -SN

%%
%% Preamble
%%
% \documentclass{<something>} must begin each LaTeX document
\documentclass[12pt,twoside]{reedthesis}
% Packages are extensions to the basic LaTeX functions. Whatever you
% want to typeset, there is probably a package out there for it.
% Chemistry (chemtex), screenplays, you name it.
% Check out CTAN to see: http://www.ctan.org/
%%
\usepackage{graphicx,latexsym}
\usepackage[french]{babel} 
\usepackage{amsmath}
\usepackage{amssymb,amsthm}
\usepackage[dvipsnames]{xcolor} % tk: for more color
\usepackage{xcolor}
\usepackage{eso-pic}
\usepackage{longtable,booktabs,setspace}
\usepackage{chemarr} %% Useful for one reaction arrow, useless if you're not a chem major
\usepackage[hyphens]{url}
\usepackage{tikz}
\usetikzlibrary{calc}
\newcommand\HRule{\rule{\textwidth}{1pt}}
% Added by CII
\usepackage{hyperref}
\usepackage{lmodern}
\usepackage{float}
\floatplacement{figure}{H}
% End of CII addition
\usepackage{rotating}
\usepackage{upgreek} % tk : pour pouvoir utiliser le symbole µ droit (pas en itallic)
\usepackage{pdfpages}
\usepackage{lscape}
\newcommand{\blandscape}{\begin{landscape}}
\newcommand{\elandscape}{\end{landscape}}
\usepackage[utf8]{inputenc}




% Next line commented out by CII
%%% \usepackage{natbib}
% Comment out the natbib line above and uncomment the following two lines to use the new
% biblatex-chicago style, for Chicago A. Also make some changes at the end where the
% bibliography is included.
%\usepackage{biblatex-chicago}
%\bibliography{thesis}


% Added by CII (Thanks, Hadley!)
% Use ref for internal links
\renewcommand{\hyperref}[2][???]{\autoref{#1}}
\def\chapterautorefname{Chapter}
\def\sectionautorefname{Section}
\def\subsectionautorefname{Subsection}
% End of CII addition

% Added by CII
\usepackage{caption}
\captionsetup{width=5in}
% End of CII addition

% \usepackage{times} % other fonts are available like times, bookman, charter, palatino


% To pass between YAML and LaTeX the dollar signs are added by CII
\title{THÈSE}
\author{Thomas Karaouzene}
\labo{}
% The month and year that you submit your FINAL draft TO THE LIBRARY (May or December)
\date{31 octobre 2017}
\division{}
\advisor{Pierre Ray}
%If you have two advisors for some reason, you can use the following
% Uncommented out by CII
\altadvisor{Nicolas Thierry-Mieg}
% End of CII addition

%%% Remember to use the correct department!
\department{Ingénierie de la Santé, de la Cognition et Environnement (EDISCE)}
% if you're writing a thesis in an interdisciplinary major,
% uncomment the line below and change the text as appropriate.
% check the Senior Handbook if unsure.
%\thedivisionof{The Established Interdisciplinary Committee for}
% if you want the approval page to say "Approved for the Committee",
% uncomment the next line
%\approvedforthe{Committee}

% Added by CII
%%% Copied from knitr
%% maxwidth is the original width if it's less than linewidth
%% otherwise use linewidth (to make sure the graphics do not exceed the margin)
\makeatletter
\def\maxwidth{ %
  \ifdim\Gin@nat@width>\linewidth
    \linewidth
  \else
    \Gin@nat@width
  \fi
}
\makeatother

\renewcommand{\contentsname}{Table of Contents}
% End of CII addition

\setlength{\parskip}{0pt}

% Added by CII
  %\setlength{\parskip}{\baselineskip}
  \usepackage[parfill]{parskip}

\providecommand{\tightlist}{%
  \setlength{\itemsep}{0pt}\setlength{\parskip}{0pt}}

\Acknowledgements{

}

\Dedication{

}

\Preface{
This is an example of a thesis setup to use the reed thesis document
class (for LaTeX) and the R bookdown package, in general.
}

\Abstract{

}

	\usepackage{tikz}
% End of CII addition
%%
%% End Preamble
%%
%

\usepackage{amsthm}
\newtheorem{theorem}{Theorem}[section]
\newtheorem{lemma}{Lemma}[section]
\theoremstyle{definition}
\newtheorem{definition}{Definition}[section]
\newtheorem{corollary}{Corollary}[section]
\newtheorem{proposition}{Proposition}[section]
\theoremstyle{definition}
\newtheorem{example}{Example}[section]
\theoremstyle{remark}
\newtheorem*{remark}{Remark}
\begin{document}

% Everything below added by CII
      \maketitle
  
  \frontmatter % this stuff will be roman-numbered
  \pagestyle{empty} % this removes page numbers from the frontmatter

  
      \begin{preface}
      This is an example of a thesis setup to use the reed thesis document
      class (for LaTeX) and the R bookdown package, in general.
    \end{preface}
  
      \hypersetup{linkcolor=black}
    \setcounter{tocdepth}{3}
    \tableofcontents
  
      \listoftables
  
      \listoffigures
  
  
  
  \mainmatter % here the regular arabic numbering starts
  \pagestyle{fancyplain} % turns page numbering back on

  \chapter{Delete line 6 if you only have one
  advisor}\label{delete-line-6-if-you-only-have-one-advisor}
  
  \chapter{Mise en place d'une stratégie pour l'analyse des données
  exomiques -- application en recherche
  clinique}\label{mise-en-place-dune-strategie-pour-lanalyse-des-donnees-exomiques-application-en-recherche-clinique}
  
  \newpage
  
  En 2011, les bases moléculaires d'environ 3700 pathologies à
  transmission Mendélienne avaient été élucidées. Cependant, pour une
  quantité équivalente de pathologies Mendéliennes (ou suspectées de
  l'être) cette cause reste un mystère (Amberger, Bocchini, \& Hamosh,
  \protect\hyperlink{ref-Amberger2011}{2011}). Avec plusieurs centaines de
  pathologies caractérisées depuis 2010 (S. B. Ng et al., n.d.), les
  séquençages WGS et WES ont, depuis leur émergence, révolutionnés les
  méthodes de recherche dans le cadre d'étude phénotype-génotype en
  permettant de manière rapide et à moindre coup le séquençage de la
  quasi-totalité des gènes humains. Dès lors, le défis de ces analyses
  n'est plus le séquençage de l'ADN mais l'interprétation des données
  massives produites. En effet, l'un des plus grands challenges des
  analyses phénotype-génotype réalisées par WES réside dans l'analyse de
  l'importante quantité de variant portés par chaque individu s'élevant à
  plusieurs dizaines de milliers lorsque l'on compare avec le génome de
  référence. Même après avoir retiré les variants retrouvés fréquemment
  dans la population générale, des méthodes additionnelles sont
  nécessaires pour prédire, parmi les variants restant, lesquels induisent
  des conséquences fonctionnelles sérieuses afin de les prioriser (Pelak
  et al., \protect\hyperlink{ref-Pelak2010}{2010}). De nombreux logiciels
  tel que Variant Effect Predictor (W. McLaren et al.,
  \protect\hyperlink{ref-McLaren2016}{2016}), SnpEff (Cingolani et al.,
  \protect\hyperlink{ref-Cingolani2012}{2012}) ou encore ANNOVAR (K. Wang,
  Li, \& Hakonarson, \protect\hyperlink{ref-Wang2010}{2010}) permettent
  d'identifier quels sont les variants qui ont un effet tronquant sur la
  protéine. Cependant, avec en moyenne 165 variants homozygotes ayant un
  effet tronquant retrouvés dans chaque exomes (Pelak et al.,
  \protect\hyperlink{ref-Pelak2010}{2010}) ces méthodes, bien qu'efficaces
  sont souvent insuffisantes.
  
  D'autres logiciels tel que Exomiser (Robinson et al.,
  \protect\hyperlink{ref-Robinson2014}{2014}) vont, à partir d'une liste
  de variants \textbf{déjà} appelés effectuer les étapes d'annotation, de
  filtrage et de priorisation. Malgré l'efficacité de ces logiciels, aucun
  d'entre eux ne couvrent l'ensemble des étapes allant de l'alignement des
  \emph{reads} à la priorisation des variants. La plupart ayant pour point
  de départ une liste de variants appelés en amont. Ils ne contrôlent donc
  en aucune manière les étapes d'alignement et d'appel des variants. Or,
  comme il a été dit plus tôt, ces deux étapes constituent la base de
  l'analyse {[}{]}.
  
  Ce chapitre décrit à la fois la constitution d'un pipeline d'analyse des
  données de séquençage exomique recouvrant l'ensemble des étapes allant
  de l'allignement des séquences à la priorisation des variants ainsi que
  son utilisation dans le cadre de la recherche de mutations entrainant
  différents phénotypes d'infertilité d'une part de cas familiaux composés
  de duos ou trio et, pour finir, d'une large cohorte d'individus non
  apparentés présentant tous le même phénotype.
  
  \newpage
  
  \section{Méthode : Description du
  pipeline}\label{methode-description-du-pipeline}
  
  \subsection{\texorpdfstring{L'alignement des
  \emph{reads}}{L'alignement des reads}}\label{lalignement-des-reads}
  
  Comme expliqué plus tôt, l'étape d'alignement à pour objectif de
  repositioner l'ensemble des \emph{reads} d'un individu le long d'un
  génome de référence. Cette étape peut ainsi être comparée à la
  reconstruction d'un puzzle dans lequel chaque \emph{reads} peut-être
  assimilé à l'une des pièces tandis que le génome de référence serait ici
  le modèle (\textbf{Figure : }\ref{fig:picdnamapping}).
  
  L'ensemble de nos exomes ayant été réalisés en \emph{paired-end}, les
  deux extrémités de chaque fragment sont séquencées. Chaque \emph{end}
  d'un même \emph{read} peut donc être considérée comme un \emph{read} à
  part entière qui sont alignées \textbf{indépendamment} le long du génome
  de référence. L'information fournit par le \emph{paired-end} n'étant
  utilisé qu'à \emph{posteriori} en tant que critère qualité. Au sein de
  notre pipeline, cette étape est effectué par le logiciel MAGIC (Su et
  al., \protect\hyperlink{ref-Su2014}{2014}) qui dans le cadre de nos
  études, s'est basé sur la version hg19 / GHRC37 du génome de référence.
  Suite à cet alignement, plusieurs critères sont observés afin de filtrer
  les \emph{reads} présentant une faible qualité d'alignement.
  
  Ainsi, le premier de ces filtres consiste à tout d'abord filtrer
  l'ensemble des \emph{reads} dupliqués, c'est à dire les \emph{reads}
  ayant des séquences parfaitement identiques, ceux-ci étant souvent le
  résultats d'un excès d'amplification au moment des PCRs effectuées en
  amont. De la même manière, afin d'éviter toute ambiguité au momen de
  l'interprétation des résultats, l'ensemble des \emph{reads} s'étant
  alignés sur plusieurs région du génome sont aussi filtrés. Une fois cela
  fait, nous vérifions la ``compatibilité'' des deux \emph{ends} composant
  chacun des \emph{reads} restant. Un \emph{reads} est dit compatible
  lorsque les deux \emph{ends} qui le composent s'alignent face à face
  (une sur le brin sens du génome de référence et l'autre sur le brin
  anti-sens) et couvrent une zone ne faisant pas plus de 3 fois la taille
  médiane de l'insert. Les \emph{reads} dont les deux \emph{ends} se sont
  alignées mais ne remplissant pas ces conditions seront dit ``Non
  compatible'', ceux dont une seule des deux \emph{ends} s'est alignés
  seront appelés ``orphelins'' et enfin ceux pour lesquels aucune des deux
  \emph{ends} ne se sont allignées sont appellés ``non-aligné''.
  L'ensemble des \emph{reads} ``non-compatible'', ``orphelins'' et
  ``non-alignés'' sont, en raison de leur faible qualité, filtré et donc
  non considérés pour les analyses en aval. Les \emph{reads} ayant passé
  l'ensemble des critères qualité mentionnés précédemment seront, eux,
  utilisés pour effectuer l'appel des variants.
  
  \newpage
  
  \begin{figure}
  
  {\centering \includegraphics[scale=0.35]{figure/dna_mapping} 
  
  }
  
  \caption{Représentation schématique de l'alignement de *reads paired-end* : **A** : Représentation du génome de référence ainsi que de *reads paired end* avant l'étape d'alignement. Les *reads paired-end* sont composés d'une extrémité *forward* (en vert) complémentaire du brin sens du génome de référence et d'une extrémité réverse (en jaune), complémentaire du brin anti-sens du génome de référence. Chacune de ces extrémités est séparées par un insert de taille connue mais de séquence inconnu. **B** : Après l'étape d'alignement, chaque *read* est repositioné sur la région du génome avec laquelle il présente la plus grande homologie de séquence. Le nombre de *reads* différents recouvrant une même position du génome de référence est appellée couverture}\label{fig:picdnamapping}
  \end{figure}
  
  \newpage
  
  \subsection{L'appel des variants}\label{lappel-des-variants}
  
  Si l'allignement des séquences peut être comparé à la reconstruction
  d'un puzzle, l'appel des variant pourrait lui être vu comme un jeu des 7
  erreurs, au cours duquel, pour chaque position couverte, les différence
  entre la séquence de l'individu séquencé et le génome de référence
  seront listés et appelé variants. Comme nous l'avons vu plus
  \protect\hyperlink{varcall}{tôt}, il est fortement conseillé d'effectuer
  l'appel des variants en tenant compte de l'aligneur choisi (Nielsen,
  Paul, Albrechtsen, \& Song, \protect\hyperlink{ref-Nielsen2011}{2011},
  M. A. DePristo et al. (\protect\hyperlink{ref-DePristo2011}{2011}),
  Lunter \& Goodson (\protect\hyperlink{ref-Lunter2011}{2011})). C'est
  pourquoi, nous avons développé notre propre algorithme d'appel des
  variants spécialement conçu pour l'analyse des données de MAGIC. Ainsi,
  l'appel des variants sera directement basé sur quatre comptages
  (R\(_+\), R\(_-\), V\(_+\) et V\(_-\)) fourni directement par MAGIC pour
  chaque position suffisement couvertes :
  
  \begin{enumerate}
  \def\labelenumi{\arabic{enumi}.}
  \tightlist
  \item
    \textbf{R}\(_+\) \textbf{et R}\(_-\) : Ces deux comptages
    correspondent au nombre de \emph{reads} \emph{forward} (+) et
    \emph{reverse} (-) sur lesquels est observé l'allèle de
    \textbf{référence} (R) à une position donnée.\\
  \item
    \textbf{V}\(_+\) \textbf{et V}\(_-\) : À l'inverse de R\(_+\) et
    R\(_-\), ces comptages correspondent au nombre de \emph{reads}
    \emph{forward} et \emph{reverse} sur lesquels est observé un allèle de
    \textbf{variant} (V) à une position donnée.
  \end{enumerate}
  
  Ainsi, les sommes : \(R_+ + V_+\) et \(R_- + V_-\) indiqueront
  respectivement la couverture d'une position en ne tenant que des
  \emph{reads forward} et \emph{reverse}. En fonction de ces couverturs
  nos appels seront classés en trois catégories :
  
  \begin{enumerate}
  \def\labelenumi{\arabic{enumi}.}
  \tightlist
  \item
    \textbf{Les appels \emph{double strand} (DS) :} Qualifie les positions
    ayant une couverture \(\ge\) 10 sur \textbf{les deux} strands. Ces
    appels sont ceux sont ceux ayant la meilleure qualité.\\
  \item
    \textbf{Les appels \emph{single strand} (SS) :} Ces appels définissent
    les positions pour lesquels \textbf{un des deux} \emph{strands}
    présentent une couverture \(\le\) 10. Dans ce cas, ce \emph{strand}
    est ignoré et l'appel est effectué uniquement en utilisant le second
    \emph{strand}.\\
  \item
    \textbf{Les appels \emph{non strand} (NS) :} Les positions NS sont
    celles pour lesquelles la couverture est \(\le\) 10 sur \textbf{les
    deux} strands. Aucun appel n'est effectué à ces positions qui
    \textbf{ne sont pas conservés dans la suite des analyses}.
  \end{enumerate}
  
  Ensuite, chaque position couverte, des appels indépendants seront
  effectués pour chaque \emph{strand} de telle sorte que, pour chaqune de
  ces position si :
  
  \begin{enumerate}
  \def\labelenumi{\arabic{enumi}.}
  \tightlist
  \item
    0 à 20\% des \emph{reads} portent un variant, la position est appelée
    \textbf{homozygote référence}.\\
  \item
    40 à 75\% des \emph{reads} portent un variant, la position est appelée
    \textbf{hétérozygote}.\\
  \item
    85 à 100\% des \emph{reads} portent un variant, la position est
    appelée \textbf{homozygote variant}.\\
  \item
    20 à 40\% des \emph{reads} portent un variant, l'appel sera considéré
    comme \textbf{ambigu bas}.\\
  \item
    75 à 85\% des \emph{reads} portent un variant, l'appel sera considéré
    comme \textbf{ambigu haut}.
  \end{enumerate}
  
  Pour les positions SS, l'appel final sera directement \ldots{}. Pour les
  positions DS, la concordance des appels fournis par chaque \emph{end}
  est verifié. Ainsi, un variant sera considéré :
  
  \begin{enumerate}
  \def\labelenumi{\arabic{enumi}.}
  \tightlist
  \item
    \textbf{Homozygote référence} si les deux appels sont homozygote
    référence, ou, un des appels est homozygote référence et l'autre
    ambigu bas.
  \item
    \textbf{Hétérozygote} si les deux appels sont hétérozygotes, ou, si
    l'un des appels est hétérozygote et l'autre ambigu bas ou haut.\\
  \item
    \textbf{Homozygote variant} si les deux appels sont homozygote
    variant, ou, un des appels est homozygote variant et l'autre ambigu
    haut
  \item
    \textbf{Ambigu} si les deux appels sont ambigu bas ou si ils sont tous
    les deux ambigu haut.\\
  \item
    \textbf{Discordant} pour toutes les combinaisons restantes.
  \end{enumerate}
  
  Dans le cadre de nos analyses, les appels ambigu et discordant sont
  filtrés.
  
  \newpage
  
  \subsection{L'annotation}\label{lannotation}
  
  Chaque variant retenu sera ensuite annoté tout d'abord par le logiciel
  \emph{variant effect predictor} (VEP) (W. McLaren et al.,
  \protect\hyperlink{ref-McLaren2016}{2016}) qui nous indiquera pour
  chaque variant la conséquence que celui-ci aura sur la séquence codante
  de l'ensemble des transcrits Ensembl qu'il chevauche (\textbf{Figure :
  }\ref{fig:figvepcsq}) (\textbf{Table : }\ref{tab:tabvepcsq}). Dans le
  cas de substitution faux-sens, c'est à dire entrainant le changement
  d'un seul acide-aminé de la séquence protéique, nous utiliserons les
  prédictions fournies par SIFT et PolyPhen afin d'estimer leur
  pathogénicité. Ensuite, nous ajoutons, pour chaque gène, son expression
  tissulaire en nous basant sur les données Ensembl (Aken et al.,
  \protect\hyperlink{ref-Aken2017}{2017}) générées par le projet Illumina
  BodyMap qui recense les données RNAseq des gènes humains pour 16 tissus
  différents. Suite à cela nous ajoutons, lorsque celle-ci est disponible,
  la fréquence du variant dans les bases de données ExAC (Lek et al.,
  \protect\hyperlink{ref-Lek2016}{2016}), ESP600
  (\href{http://evs.gs.washington.edu/EVS/}{Exome Variant Server, NHLBI GO
  Exome Sequencing Project (ESP), Seattle, WA}) et 1000Genomes (1000
  Genomes Project Consortium et al.,
  \protect\hyperlink{ref-1000GenomesProjectConsortium2015}{2015}) donnant
  ainsi une estimation de sa fréquence dans la population générale. De
  même, la particularité de ce pipeline est qu'elle conserve l'ensemble
  des variants identifiés dans les études effectuées précédemment
  permettant d'ajouter aux annotations la fréquence d'un variant chez les
  individus déjà séquencé et donc la fréquence d'un variant dans chaque
  phénotype étudié créant ainsi une base de données interne qui pourra
  servir de contrôle dans les études ultérieur.
  
  \begin{figure}
  
  {\centering \includegraphics[scale=.9]{figure/vep_csq} 
  
  }
  
  \caption[Listes des différentes conséquences prédites par VEP et leur positionnement sur le transcrit]{Listes des différentes conséquences prédites par VEP et leur positionnement sur le transcrit d'après [VEP site](http://www.ensembl.org/info/genome/variation/consequences.jpg)}\label{fig:figvepcsq}
  \end{figure}
  
  \newpage
  
  \blandscape
  
  \begin{longtable}[]{@{}lll@{}}
  \caption{\label{tab:tabvepcsq} Liste simplifiée des conséquences prédites
  par VEP avec leur description et impact associée}\tabularnewline
  \toprule
  \begin{minipage}[b]{0.18\columnwidth}\raggedright\strut
  VEP consequence\strut
  \end{minipage} & \begin{minipage}[b]{0.11\columnwidth}\raggedright\strut
  VEP impact\strut
  \end{minipage} & \begin{minipage}[b]{0.63\columnwidth}\raggedright\strut
  Description\strut
  \end{minipage}\tabularnewline
  \midrule
  \endfirsthead
  \toprule
  \begin{minipage}[b]{0.18\columnwidth}\raggedright\strut
  VEP consequence\strut
  \end{minipage} & \begin{minipage}[b]{0.11\columnwidth}\raggedright\strut
  VEP impact\strut
  \end{minipage} & \begin{minipage}[b]{0.63\columnwidth}\raggedright\strut
  Description\strut
  \end{minipage}\tabularnewline
  \midrule
  \endhead
  \begin{minipage}[t]{0.18\columnwidth}\raggedright\strut
  Splice acceptor / donor\strut
  \end{minipage} & \begin{minipage}[t]{0.11\columnwidth}\raggedright\strut
  HIGH\strut
  \end{minipage} & \begin{minipage}[t]{0.63\columnwidth}\raggedright\strut
  A splice variant that changes the 2 base region at the 3' / 5' end of an
  intron\strut
  \end{minipage}\tabularnewline
  \begin{minipage}[t]{0.18\columnwidth}\raggedright\strut
  Stop gained\strut
  \end{minipage} & \begin{minipage}[t]{0.11\columnwidth}\raggedright\strut
  HIGH\strut
  \end{minipage} & \begin{minipage}[t]{0.63\columnwidth}\raggedright\strut
  A sequence variant whereby at least one base of a codon is changed,
  resulting in a premature stop codon, leading to a shortened
  transcript\strut
  \end{minipage}\tabularnewline
  \begin{minipage}[t]{0.18\columnwidth}\raggedright\strut
  Frameshift\strut
  \end{minipage} & \begin{minipage}[t]{0.11\columnwidth}\raggedright\strut
  HIGH\strut
  \end{minipage} & \begin{minipage}[t]{0.63\columnwidth}\raggedright\strut
  A sequence variant which causes a disruption of the translational
  reading frame, because the number of nucleotides inserted or deleted is
  not a multiple of three\strut
  \end{minipage}\tabularnewline
  \begin{minipage}[t]{0.18\columnwidth}\raggedright\strut
  Stop lost\strut
  \end{minipage} & \begin{minipage}[t]{0.11\columnwidth}\raggedright\strut
  HIGH\strut
  \end{minipage} & \begin{minipage}[t]{0.63\columnwidth}\raggedright\strut
  A sequence variant where at least one base of the terminator codon
  (stop) is changed, resulting in an elongated transcript\strut
  \end{minipage}\tabularnewline
  \begin{minipage}[t]{0.18\columnwidth}\raggedright\strut
  Start lost\strut
  \end{minipage} & \begin{minipage}[t]{0.11\columnwidth}\raggedright\strut
  HIGH\strut
  \end{minipage} & \begin{minipage}[t]{0.63\columnwidth}\raggedright\strut
  A codon variant that changes at least one base of the canonical start
  codo\strut
  \end{minipage}\tabularnewline
  \begin{minipage}[t]{0.18\columnwidth}\raggedright\strut
  Inframe insertion / deletion\strut
  \end{minipage} & \begin{minipage}[t]{0.11\columnwidth}\raggedright\strut
  MODERATE\strut
  \end{minipage} & \begin{minipage}[t]{0.63\columnwidth}\raggedright\strut
  An inframe non synonymous variant that inserts / deletes bases into in
  the coding sequenc\strut
  \end{minipage}\tabularnewline
  \begin{minipage}[t]{0.18\columnwidth}\raggedright\strut
  Missense\strut
  \end{minipage} & \begin{minipage}[t]{0.11\columnwidth}\raggedright\strut
  MODERATE\strut
  \end{minipage} & \begin{minipage}[t]{0.63\columnwidth}\raggedright\strut
  A sequence variant, that changes one or more bases, resulting in a
  different amino acid sequence but where the length is preserved\strut
  \end{minipage}\tabularnewline
  \begin{minipage}[t]{0.18\columnwidth}\raggedright\strut
  Splice region\strut
  \end{minipage} & \begin{minipage}[t]{0.11\columnwidth}\raggedright\strut
  LOW\strut
  \end{minipage} & \begin{minipage}[t]{0.63\columnwidth}\raggedright\strut
  A sequence variant in which a change has occurred within the region of
  the splice site, either within 1-3 bases of the exon or 3-8 bases of the
  intron\strut
  \end{minipage}\tabularnewline
  \begin{minipage}[t]{0.18\columnwidth}\raggedright\strut
  Stop retained\strut
  \end{minipage} & \begin{minipage}[t]{0.11\columnwidth}\raggedright\strut
  LOW\strut
  \end{minipage} & \begin{minipage}[t]{0.63\columnwidth}\raggedright\strut
  A sequence variant where at least one base in the terminator codon is
  changed, but the terminator remains\strut
  \end{minipage}\tabularnewline
  \begin{minipage}[t]{0.18\columnwidth}\raggedright\strut
  Synonymous\strut
  \end{minipage} & \begin{minipage}[t]{0.11\columnwidth}\raggedright\strut
  LOW\strut
  \end{minipage} & \begin{minipage}[t]{0.63\columnwidth}\raggedright\strut
  A sequence variant where there is no resulting change to the encoded
  amino acid\strut
  \end{minipage}\tabularnewline
  \begin{minipage}[t]{0.18\columnwidth}\raggedright\strut
  5 / 3 prime UTR\strut
  \end{minipage} & \begin{minipage}[t]{0.11\columnwidth}\raggedright\strut
  MODIFIER\strut
  \end{minipage} & \begin{minipage}[t]{0.63\columnwidth}\raggedright\strut
  A UTR variant of the 5' / 3' UTR\strut
  \end{minipage}\tabularnewline
  \begin{minipage}[t]{0.18\columnwidth}\raggedright\strut
  Intron\strut
  \end{minipage} & \begin{minipage}[t]{0.11\columnwidth}\raggedright\strut
  MODIFIER\strut
  \end{minipage} & \begin{minipage}[t]{0.63\columnwidth}\raggedright\strut
  A transcript variant occurring within an intron\strut
  \end{minipage}\tabularnewline
  \begin{minipage}[t]{0.18\columnwidth}\raggedright\strut
  NMD transcript\strut
  \end{minipage} & \begin{minipage}[t]{0.11\columnwidth}\raggedright\strut
  MODIFIER\strut
  \end{minipage} & \begin{minipage}[t]{0.63\columnwidth}\raggedright\strut
  A variant in a transcript that is the target of NMD\strut
  \end{minipage}\tabularnewline
  \begin{minipage}[t]{0.18\columnwidth}\raggedright\strut
  Non coding transcript\strut
  \end{minipage} & \begin{minipage}[t]{0.11\columnwidth}\raggedright\strut
  MODIFIER\strut
  \end{minipage} & \begin{minipage}[t]{0.63\columnwidth}\raggedright\strut
  A transcript variant of a non coding RNA gene\strut
  \end{minipage}\tabularnewline
  \bottomrule
  \end{longtable}
  
  \elandscape
  \newpage
  
  \subsection{Le filtrage des variants}\label{le-filtrage-des-variants}
  
  L'étape de filtrage est extrêmement importante si l'on souhaite analyser
  de manière efficace les données provenant de WES. C'est pourquoi elle
  occupe une place importante dans notre pipeline. L'intégralité des
  paramètres de cette étape peuvent être modifiés par l'utilisateur de
  sorte à faire correspondre les critères de filtre aux besoins de
  l'étude. Afin de rendre son utilisation le plus efficace possible, nous
  avons souhaité définir des paramètres par défauts pertinent dans la
  plupart des études de séquençage exomique de sorte que à moins que le
  contraire ne soit spécifié les filtres suivant seront appliqués :
  
  \begin{enumerate}
  \def\labelenumi{\arabic{enumi}.}
  \item
    \textbf{Filtre 1 : L'union des variants :} Dans le cas ou des
    individus présentant un lien de parenté et présentant le même
    phénotype sont analysé, seuls les variants observés chez l'ensemble
    des individus sont conservés.
  \item
    \textbf{Filtre 2 : Génotype des variants :} Ce pipeline d'analyse a
    avant tout été développé pour la recherche de variant impliqué dans
    des pathologies à transmission récessives. Ainsi, à moins que le
    contraire ne soit spécifié, l'ensemble des variants hétérozygotes sont
    systématiquement filtrés. .
  \item
    \textbf{Filtre 4 : Les transcrits ``non pertinents'' :} Au cours de
    nos analyses nous nous sommes concentré uniquement sur les transcrits
    codant pour une protéine. Ainsi, l'ensemble des transcrits annotés
    comme étant non codant furent filtrés. De même pour les transcrits
    annotés comme étant NMD (\emph{nonsense-mediated decay}). En effet, ce
    mécanisme a pour but de contrôler la qualité des ARNm cellulaires chez
    les eucaryotes (Y.-F. Chang, Imam, \& Wilkinson,
    \protect\hyperlink{ref-Chang2007}{2007}) en éliminant les ARNm qui
    comportent un codon stop prématuré (Baker \& Parker,
    \protect\hyperlink{ref-Baker2004}{2004}), pouvant être le résultat
    d'une erreur de transcription, d'une mutation ou encore d'une erreur
    d'épissage. Il est donc peu probable que les variants présents sur des
    transcrits annotés NMD soient responsables du phénotype. Dès lors, ces
    transcrits ont été également filtrés. Ainsi, l'ensemble des variants
    impactant \textbf{uniquement} des transcrits non codant et / ou annoté
    NMD sont filtrés.
  \item
    \textbf{Filtre 3 : Impact du variant :} Afin de ne conserver que les
    variants ayant le plus probablement délétère sur la protéine, seuls
    sont conservés ceux impactant la séquence codante d'un transcrit. De
    plus les variants synonymes ne sont pas conservés (exceptés ceux se
    trouvant proches des régions d'épissage) car ceux-ci n'ont aucun effet
    sur la séquence protéique. Pour les variants faux sens (changement
    d'un seul acide-aminé de la séquence protéique) il est plus difficile
    de se trancher, dès lors, seuls ceux étant prédit comme
    \emph{tolerated} par SIFT (Kumar, Henikoff, \& Ng,
    \protect\hyperlink{ref-Kumar2009}{2009}) \textbf{et} comme
    \emph{benign} par Polyphen (Adzhubei et al.,
    \protect\hyperlink{ref-Adzhubei2010}{2010}) sont filtrés.
  \item
    \textbf{Fréquence des variants :} La fréquence d'un variant dans la
    population générale est un moyen rapide d'avoir une prédiction fiable
    de l'effet délétère ou non de celui-ci. En effet, il est peu probable
    qu'un variant retrouvé fréquemment dans la population générale soit
    causal d'une pathologie sévère. C'est pourquoi, l'ensemble des
    variants ayant une fréquence \(\ge\) 1\% dans l'une des trois bases de
    données que sont ExAC, ESP et 1KG sont filtrés.
  \item
    \textbf{Présence des variants dans la cohorte contrôle :} Au sien de
    notre pipeline, les données de l'ensemble des patients analysés dans
    les études anterieures sont conservés créant ainsi une base de donnée
    interne de variants. Dès lors, il devient possible d'utiliser chacun
    de ces patient comme contrôle lorsqu'ils ne sont pas porteur du même
    phénotype que celui des patients des études ulterieures. Ce filtre se
    révèle particulièrement intéréssant lorsque plusieurs patients
    porteurs de phénotypes différents ont subi le même protocole de
    séquençage ainsi l'ensemble des variants faux-positifs résultant
    d'artéfact liés au différentes étapes en amont de l'analyse
    bioinformatique pourront alors être filtré. De même ce filtre permet
    de mettre en évidence les variants propores à une population lorsque
    des patients provenant de la même région géographique et ne présentant
    toujours pas le même phénotype sont comparés.
  \end{enumerate}
  
  \newpage
  
  \subsection{La priorisation des gènes}\label{la-priorisation-des-genes}
  
  Dans le cas d'analyses de larges cohortes de patients, il arrive souvent
  que de nombreux variants aient passé l'ensemble des filtres ceux-ci
  pouvant affecter plusieurs centaines de gènes. Il es dès lors nécéssaire
  d'appliquer une méthode de priorisation permettant de classer les gènes
  en fonction de leur probabilité à être responsables du phénotype étudié.
  Pour celà, nous avons mis en place \ldots{}\ldots{}.
  
  \newpage
  
  \section{Résultats 1 : Analyse de 3 cas
  familiaux}\label{resultats-1-analyse-de-3-cas-familiaux}
  
  Dans cette partie, se concentre sur l'analyse bioinformatique des
  résultats des séquençages exomiques de 6 individus infertiles provenant
  de 3 familles et présentant toutes un phénotype d'infertilité masculine
  différenteffectués entre 2012 et 2014 \ref{tab:tabfam} :
  
  \begin{enumerate}
  \def\labelenumi{\arabic{enumi}.}
  \tightlist
  \item
    \textbf{Famille FAM} : Cette famille est composée de 2 frères
    azoospermes. Comme nous avons pu le voir, l'azoospermie est un
    phénotype d'infertilité masculine caractérisé par l'absence de
    spermatozoïde dans l'éjaculât. Des 6 patients de cette étude, les
    frères Ghs44 et Ghs45 sont les deux seuls à ne pas avoir été séquencés
    au Génopole d'Évry.\\
  \item
    \textbf{Famille FF} : Les spermatozoïdes des 2 frères de cette famille
    sont caractérisés par leur incapacité à féconder l'ovocyte malgré leur
    morphologie et leur mibilité normales.\\
  \item
    \textbf{Famille MMAF} : Le syndrome MMAF (\emph{multiple morphological
    abnormalities of the sperm flagella}) dont souffrent les frères de
    cette famille se caractérise comme son nom l'indique, par la présence
    d'une majorité de spermatozoïdes présentant par une mosaïque
    d'anomalie morphologique du flagelle.
  \end{enumerate}
  
  \begin{longtable}[t]{lllrl}
  \caption{\label{tab:tabfam}Tableau récapitulatif des familles séquencées et de leur phénotype}\\
  \toprule
  Family & Individuals & Phenotype & Year & Place\\
  \midrule
  AZ & Ghs44, Ghs45 & Azoospermia & 2012 & Mount Sinai Institut\\
  FF & Ghs113, Ghs117 & Fertilization failure & 2014 & Genoscope (Evry)\\
  MMAF & Ghs62, Ghs130 & MMAF & 2014 & Genoscope (Evry)\\
  \bottomrule
  \end{longtable}
  
  \newpage 
  
  \subsection{Résultats des différents étapes de
  l'analyse}\label{resultats-des-differents-etapes-de-lanalyse}
  
  \subsubsection{Résultat de l'alignement}\label{resultat-de-lalignement}
  
  Pour rappel, l'\href{\%7B\#lalignement\%7D}{alignement} consiste à
  repositionner l'ensemble des \emph{reads} générés au cours de l'étape de
  séquençage le long d'un génome de référence.
  
  La quantité de \emph{reads} composant les exomes de chaque individu peut
  varier en fonction de plusieurs paramètres et n'est donc pas égale pour
  chaque patient bien que l'ordre de grandeur reste le même avec une
  médiane de 91978875 \emph{reads}. Seuls les deux frères AZ1 et AZ2 se
  distinguent avec près de 3 fois plus de \emph{reads} que les autres
  patients. Cette différence peut être expliquée car ces deux patients
  sont les deux seuls à voir été séquencés au Mount Sinaï Institut or leur
  protocole d'amplification précédent le séquençage contient un nombre de
  cycles de PCR supérieur à ceux appliqués au Génopole d'Évry où ont été
  séquencés les autres patients. Il faut noter que ce nombre plus
  important de \emph{reads} n'est en rien le reflet d'une meilleure
  qualité. En effet, celui-ci est causé par une grande quantité de
  \emph{reads} dupliqués qui seront pour la plupart filtrés au cours des
  analyses ultérieures (\textbf{Table :} \ref{tab:tabfam}, \textbf{Figure
  : }\ref{fig:plotfammapping} - \textbf{A}).
  
  La première étape du contrôle qualité des \emph{reads} consiste à
  filtrer les \emph{reads} ne s'étant pas alignés sur le génome. Ces
  \emph{reads} sont extrêmement minoritaires puisqu'ils ne représentent
  qu'entre 1.2 et 5.5 \% des \emph{reads} de nos individus (\textbf{Figure
  : }\ref{fig:plotfammapping} - \textbf{B}).
  
  Dans nos analyses, seuls les \emph{reads} compatibles sont conservés,
  c'est à dire environs 95.2 \% des \emph{reads} s'étant correctement
  alignés. (\textbf{Figure : }\ref{fig:plotfammapping} - \textbf{C}).
  
  La dernière étape de ce contrôle-qualité consiste à analyser le nombre
  de site auxquels se sont alignés les \emph{reads}. En effet, certaine
  zone du génome étant dupliqué, l'une des problématiques des
  \emph{short-reads} est qu'il est possible que ceux-ci s'alignent à
  plusieurs régions différentes du génome. Afin d'éviter toute ambiguïté,
  seul ceux s'étant aligné sur un site unique sont conservés pour la suite
  des analyses. Ces \emph{reads} représente entre 92.3 et 96.9 \% des
  \emph{reads} ayant passé les précédents filtres (\textbf{Figure :
  }\ref{fig:plotfammapping} - \textbf{D}).
  
  \newpage 
  
  \begin{figure}
  
  {\centering \includegraphics{thesis_files/figure-latex/plotfammapping-1} 
  
  }
  
  \caption[Processus simplifié du contrôle qualité des *reads*]{Processus simplifié du contrôle qualité des *reads* : Pour chacun des graphiques, les *reads* représentés en vert sont conservés tandis que ceux en rouge sont filtrés. **A** : Quantité de *reads* bruts générés pour chaque patient au cours de l'étape de séquençage. La médiane des *reads* est représentée en bleue. **B** : Pourcentage pour chaque individu de *reads* s'étant aligné correctement et ne s'étant pas alignés sur le génome de référence. **C** : Distribution pour chaque patient des *reads* compatibles (Comp), non compatibles (Non comp) et orphelins (Orphans). **D** : Présentation pour chaque *reads* du nombre de site auxquels ils s'alignent}\label{fig:plotfammapping}
  \end{figure}
  
  \newpage  
  
  \subsubsection{L'appel des variants}\label{lappel-des-variants-1}
  
  Dans nos données, les appels SS sont majoritaires et représentent
  environ 46 \% de nos appels (contre 38.6 \% d'appels DS). Au vus de
  l'importance de ces appels, nous avons fait le choix de les conserver
  afin de ne pas filtrer une quantité trop importante de données. Ces
  appels seront cependant considérés comme étant de faible qualité, de
  fait, leurs analyses et interprétation seront plus précautionneuses. En
  revanche, au vus de la trop grande incertitude de l'appel des variants
  NS, ceux-ci sont systématiquement filtrés éliminant ainsi entre 12.4 et
  18.7 \% des positions appelées pour chaque patient (\textbf{Figure :
  }\ref{fig:plotvarcall} - \textbf{A}).
  
  Les appels discordant et ambigus sont filtrés, soit environ 85.8 \% des
  variants DS. Il est intéressant de noter que bien que les variants
  \emph{single strand} (SS) soient conservés, on peut s'attendre à ce
  qu'environ 14.2 \% de ceux-ci soient aberrants, ceux-ci n'ayant pu subir
  le même contrôle que les SS (\textbf{Figure : }\ref{fig:plotvarcall} -
  \textbf{B}).
  
  Pour l'ensemble des variants ayant passé les filtres énoncés ci-dessus,
  c'est à dire les variants SS et les variants DS avec appels concordants,
  le génotype est déterminé en fonction du pourcentage de \emph{reads}
  portant le variant à cette position. Ainsi, pour chaque individu nous
  avons pu établir une liste de SNVs et d'indels avec leur génotype
  associé. Pour chacun de nos 6 patients les ordres de grandeur du nombre
  de variants appelés sont identique. Ainsi pour chaque patient nous avons
  appelés environ 46684 variants hétérozygotes (44126 SNVs et 2558 indels)
  et 64944 variants homozygotes (32472 SNVs et 1773 indels)
  (\textbf{Figure : }\ref{fig:plotvarcall} - \textbf{C}).
  
  \newpage
  
  \begin{figure}
  
  {\centering \includegraphics{thesis_files/figure-latex/plotvarcall-1} 
  
  }
  
  \caption[Contrôle qualité des variants appelés]{Contrôle qualité des variants appelés : Pour chacun des graphiques, les variants représentés en vert et en orange sont conservés tandis que ceux en rouge sont filtrés. **A** : Distribution du *stranding* des appels pour chaque patient. **B** : Comparaison des appels entre les deux *ends* des variants appelés DS. **C** : Distribution des SNVs et indels en fonction de leur génotype pour chaque patients (représentés par une barre}\label{fig:plotvarcall}
  \end{figure}
  
  \newpage
  
  \subsubsection{L'annotation des
  variants}\label{lannotation-des-variants}
  
  Après avoir annoté nos variants, nous avons pu constater que pour chaque
  patient 24851 gènes sont en moyenne affecté par au moins un variant
  homozygote pour en moyenne 122630 transcrits (soit environ 5 transcrits
  par gènes). Il faut noter que parmi ces gènes se trouvent à la fois des
  gènes codant pour des protéine \textbf{et} d'autres non codant
  (\textbf{Figure : }\ref{fig:plotannotation} - \textbf{A}).
  
  Chaque variant affectera l'ensemble des transcrits qu'il chevauche,
  ainsi un même variant pourra impacter plusieurs transcrits. Ces impacts
  sont ensuite classés par VEP en quatre catégories qui sont, de la plus
  délétère à la moins délétère : \emph{HIGH}, \emph{MODERATE}, \emph{LOW},
  \emph{MODIFIER} (\textbf{Table :}\ref{tab:tabvepcsq}).
  
  Comme attendu, les variants ayant un impact tronquant se retrouvent être
  les moins fréquent chez chacun de nos patients. Ceci est d'autant plus
  flagrant pour l'impact \emph{HIGH} qui regroupe, entre autres, les
  variants créant un codon stop ou encore ceux causant un décalage du
  cadre de lecture (\textbf{Table :}\ref{tab:tabvepcsq}), se retrouvent,
  par rapport aux autres impacts, en quantité extrêmement faible
  puisqu'ils ne représentent en moyenne que 0.16 \% des variants.
  Cependant, bien que ce pourcentage soit faible, cela représente tout de
  même une moyenne de 504 variants \emph{HIGH} hétérozygotes par patients
  et 332 variants \emph{HIGH} homozygotes par patient) (\textbf{Figure :
  }\ref{fig:plotannotation} - \textbf{B}).
  
  \newpage
  
  \begin{figure}
  
  {\centering \includegraphics{thesis_files/figure-latex/plotannotation-1} 
  
  }
  
  \caption[Annotation des variants]{Annotation des variants : **A** : Quantification du nombre de gènes (en bleu) / transcrits (en rose) impactés par au moins un variant pour chaque patient chacun représentés par une barre. **B** : Distribution des impacts HIGH MODERATE LOW et MODIFIER en fonction des patients et du génotype du variant}\label{fig:plotannotation}
  \end{figure}
  
  \subsubsection{Le filtrage des
  variants}\label{le-filtrage-des-variants-1}
  
  Les étapes précédentes nous ont permis de mettre en évidence pour chaque
  patient une liste de variants passant l'ensemble de nos critères
  qualités. Ces variants ont dès lors pu être annotés nous permettant
  notamment d'avoir connaissance de leurs impacts sur les différents
  transcrits qu'ils chevauchent ou encore leur fréquence dans la
  population générale. Désormais, afin de ne conserver que les variants
  ayant la plus forte probabilité d'être responsable du phénotype de ces
  patients, nous avons appliqué successivement les six filtres
  précédemment décrits.
  
  \begin{enumerate}
  \def\labelenumi{\arabic{enumi}.}
  \item
    \textbf{Filtre 1 : L'union des variants :} Dans cette étude nous
    analysons les données génétiques de 3 chacune composée de 2 frères.
    Nous avons donc émmi l'hypothèse que le phénotype de chacun de frères
    d'une même famille était dûe à une cause génétique commune. C'est
    pourquoi, seul les variants observés chez l'ensemble des membres d'une
    même famille furent conservés. Ainsi se filtre a permis de filtrer
    entre \ldots{} et \ldots{} variants pour chacun des patients.
  \item
    \textbf{Filtre 2 : Génotype des variants :} Ici, nous avons émis
    l'hypothèse d'une transmission récessive du phénotype. Ainsi, seuls
    les variants homozygotes ont été conservés. filtrant en moyenne
    \ldots{} variant par individu soit une moyenne de \ldots{} \% de leurs
    variants (\textbf{Figure : }\ref{fig:plotvarcall},
    \ref{fig:plotcomparefilter}).
  \item
    \textbf{Filtre 3 : Impact du variant :} Ce filtre consistant à se
    baser à la fois sur les prédiction VEP mais aussi, dans le cas de
    variants faux-sens, sur les prédiction SIFT et PolyPhen permet de ne
    conserver que les variant ayant les effet les plus délétères. Ce
    filtre est, de prime abord le plus efficace puisqu'il permet de
    filtrer environs \ldots{} \% (médiane) des variants de chaque
    individu.
  \item
    \textbf{Filtre 4 : Les transcrits ``non pertinents'' :} Cette étape de
    filtre permet de filtrer systématiquement entre 13712 et 17407
    transcrits différents par patients. Cependant, un même variant pouvant
    impacter à la fois des transcrits ``non pertinents'' \textbf{et} des
    transcrits ``pertinents'', seuls ceux impactant \textbf{uniquement}
    des transcrits ``non pertinents'' sont filtrés, soit une moyenne de
    1870 variants par individus (\textbf{Figure :
    }\ref{fig:plotfilternonpertinanttr}).
  \end{enumerate}
  
  \begin{figure}
  
  {\centering \includegraphics{thesis_files/figure-latex/plotfilternonpertinanttr-1} 
  
  }
  
  \caption[Filtrage des transcrits jugés "non pertinents" et des variants les chevauchant]{Filtrage des transcrits jugés "non pertinents" et des variants les chevauchant : Pour chaque patient nous avons filtrer les transcrits jugés "non pertinents" pour l'analyse, c'est à dire ceux ne codant pas pour une protéine et ceux annoté NMD. Dès lors, l'intégralité des variants chevauchant uniquement des transcrits non pertinents ont pu systématiquement être filtrés (boites rouges). Les autres furent conservés (boites vertes)}\label{fig:plotfilternonpertinanttr}
  \end{figure}
  
  \begin{enumerate}
  \def\labelenumi{\arabic{enumi}.}
  \setcounter{enumi}{4}
  \item
    \textbf{Fréquence des variants :} Filtrer systématiquement les
    variants retrouvés avec une fréquence \(\ge\) 0.01 dans l'une des
    trois bases de données que sont ExAC, 1KG et ESP6500 permet de filtrer
    entre \ldots{} et \ldots{} variants par patients.
  \item
    \textbf{Présence des variants dans la cohorte contrôle :} Au cours de
    nos différentes études, nous avons été amenés à séquencé un total de
    \texttt{n\_tot\_runs} individus présentant un des
    \texttt{n\_pheno\_tot} phénotypes que nous avons étudiés
    (\textbf{Table : }\ref{tab:TODO}). Ces phénotypes étant très
    différent, on peut émettre l'hypothèse que leurs causes génétiques
    soient également différentes. De même, les variants recherchés étant
    rares, il est peu probable qu'un individu porte les variants de deux
    phénotypes différents. Ainsi, pour chacune des 3 familles, nous avons
    pu constituer une cohorte contrôle composée dans l'ensemble des
    patients précédemment analysés et ne présentant pas le même phénotype
    que celui étudié dans la famille (\textbf{Figure :}
    \ref{fig:plotsamplectrl}). Dès lors, nous avons pu filtrer l'ensemble
    des variants retrouvés à la fois chez nos patients et observés à
    l'état homozygote dans la cohorte contrôle. Cette cohorte contrôle
    présente ainsi le même rôle que les bases de données publiques. Sont
    intérêt principale par rapport à celles-ci est que les individus qui
    la composent ont pour la plupart la même origine ethnico-géographique
    que nos patients. De plus ceux-ci ont été séquencés en même temps dans
    les mêmes centres permettant ainsi d'identifier les artefacts dus aux
    protocoles de séquençage.
  \end{enumerate}
  
  \begin{figure}
  
  {\centering \includegraphics{thesis_files/figure-latex/plotsamplectrl-1} 
  
  }
  
  \caption[Nombre d'individus composant la cohorte contrôle de chaque famille]{Nombre d'individus composant la cohorte contrôle de chaque famille : Ici, chaque barre représente une famille et sa hauteur est déterminée par le nombre d'individus composant la cohorte contrôle à laquelle elle a été confronté. Chaque individu de la cohorte contrôle a été séquencés en WES par notre équipe. Afin d'être considéré comme "contrôle" et intégrer cette cohorte, un individu doit être sain ou présenter un phénotype d'infertilité différent de la famille étudiée. Par exemple, un individus MMAF pourra servir de contrôle aux familles AZ et FF mais pas aux familles MMAF1-4}\label{fig:plotsamplectrl}
  \end{figure}
  
  \newpage
  
  \newpage
  
  Comme on pouvait s'y attendre, ces six filtres ont un pouvoir
  discriminant extrêmement différent. En effet, tandis que le filtre
  ``Transcript relevance'' (filtre n°4) éliminer en moyenne 3.9 \% des
  variants de chaque individu, le filtre ``Variant impact'' (filtre n° 3)
  élimine jusqu'à 90.1 \% de ces mêmes variants. Cette différence n'est
  pas surprenante. En effet, comme nous l'avions vu plus tôt, les variants
  de la catégorie VEP \emph{MODIFIER} qui regroupe entre autres les
  variants chevauchant les séquences UTRs et introniques (\textbf{Table :}
  \ref{tab:tabvepcsq}) représentent en moyenne 87\% des variants de nos
  patients. Ceux-ci étant tous filtrés, on s'attendait donc à une valeur
  aussi élevée. On peut également constater l'importance de la cohorte
  contrôle qui, je le rappelle, permet de filtrer l'ensemble des variants
  homozygotes observés en son sein, puisque ce filtre permet retirer entre
  76.5 et 88.4\% des variants de chaque individus (\textbf{Figure :}
  \ref{fig:plotcomparefilter} - \textbf{A}).
  
  Cependant, regarder uniquement le pourcentage de variants filtrés par
  chaque filtre révèle une information partielle. En effet, dans ce cas de
  figure, on observe la quantité de variant éliminé par chaque filtre
  indépendamment les uns des autres. Ainsi, un même variant peut donc être
  filtré par plusieurs filtres. Dès lors, il faut également analyser la
  quantité de variants filtrés \textbf{spécifiquement} par chaque filtre.
  Ainsi, on peut constater que le classement des filtres en fonctions de
  leur stringence reste quasiment identique. Il est tout de même
  intéressant de noter que désormais le filtre ``Variant impact'' apparait
  moins efficace que les filtres ``Ctrl'' et ``Genotype'' en filtrant
  spécifiquement une moyenne de 253 variants par individu contre 423 pour
  le filtre génotype et 882 pour le filtre ``Ctrl''. Ainsi, ce dernier
  devient celui filtrant spécifiquement le plus de variants avec entre 364
  et 1060 variants spécifiquement filtrés par patients confirmant ainsi
  l'importance de ce filtre dans nos analyses. Aussi, les filtres
  ``Transcript relevance'', ``Union'' et ``Frequency'' apparaissent
  désormais comme étant anecdotiques en comparaison aux trois autres
  filtres puisqu'ils filtrent au maximum 43 variants spécifiques
  (\textbf{Figure :} \ref{fig:plotcomparefilter} - \textbf{B}).
  
  \newpage
  
  \begin{figure}
  
  {\centering \includegraphics{thesis_files/figure-latex/plotcomparefilter-1} 
  
  }
  
  \caption[Comparaison de l'efficacité de chacun des six filtres utilisés]{Comparaison de l'efficacité de chacun des six filtres utilisés : **A** : Comparaison du pourcentage de variants filtrés par chacun des six filtres indépendamment les uns des autres pour chaque patient (représenté par les points. Dès lors, un même variant peut-être filtré par plusieurs filtres. **B** : Comparaison du nombre de variant filtrés spécifiquement par chacun des filtres. Ici, un variant ne peut-être filtré que par un seul filtre}\label{fig:plotcomparefilter}
  \end{figure}
  
  Après avoir appliqué l'ensemble de ces filtres, seuls quelques variants
  subsistent nous permettant d'obtenir une liste de gènes restreinte pour
  chaque famille et ainsi de tirer des conclusions quant au variant
  responsable du phénotype de chacune d'entre elles. Ces travaux ont ainsi
  pu mener à l'écriture de trois articles dont je suis co-auteur.
  
  \newpage
  
  \newpage  
  
  \begin{enumerate}
  \def\labelenumi{\arabic{enumi}.}
  \setcounter{enumi}{1}
  \tightlist
  \item
    \textbf{Famille FF} : Pour cette famille, seul le gène
    \emph{PLC}\(\zeta 1\) a passé l'ensemble des filtres. Nos
    connaissances sur la fonction de se gène et notamment son rôle dans
    l'activation ovocytaire (Amdani, Jones, \& Coward,
    \protect\hyperlink{ref-Amdani2013}{2013}) ainsi que sa forte
    expression testiculaire ont fait de ce gène le candidat idéal pour
    expliquer le phénotype d'échec de fécondation de ces deux frères
    (\textbf{Figure : }\ref{fig:plotexpplcz1}).
  \end{enumerate}
  
  \begin{figure}
  
  {\centering \includegraphics{thesis_files/figure-latex/plotexpplcz1-1} 
  
  }
  
  \caption[Expression tissulaire du gène *PLCZ1*]{Expression tissulaire du gène *PLCZ1* : D'après les données du Illumina BodyMap}\label{fig:plotexpplcz1}
  \end{figure}
  
  \begin{enumerate}
  \def\labelenumi{\arabic{enumi}.}
  \setcounter{enumi}{3}
  \tightlist
  \item
    \textbf{Famille MMAF} : À l'issue des filtres, \texttt{n\_gene\_mmaf}
    gènes ressortaient chez ces deux frères : \emph{MYH11} et
    \emph{DNAH1}. Or, notre équipe ayant déjà établit le lien entre des
    mutations du gène \emph{DNAH1} et le syndrome MMAF (Ben Khelifa et
    al., \protect\hyperlink{ref-BenKhelifa2014}{2014}) ce gène s'est
    révélé être un candidat idéal pour expliquer le phénotype de ces 2
    frères. De plus, l'implication de \emph{MYH11} dans le phénotype de
    dissection aortique (Imai et al.,
    \protect\hyperlink{ref-Imai2015}{2015}) l'ont écarté des candidats
    pour le phénotype MMAF.
  \end{enumerate}
  
  \newpage
  
  \subsection{Article n° 3}\label{article-n-3}
  
  \textbf{SPINK2 deficiency causes infertility by inducing sperm defects
  in heterozygotes and azoospermia in homozygotes}
  
  Kherraf ZE\textsuperscript{*}, Christou-Kent M\textsuperscript{*},
  \textbf{Karaouzène T}, Amiri-Yekta A, Martinez G, Vargas AS, Lambert E,
  Borel C, Dorphin B, Aknin-Seifer I, Mitchell MJ, Metzler-Guillemain C,
  Escoffier J, Nef S, Grepillat M, Thierry-Mieg N, Satre V, Bailly M,
  Boitrelle F, Pernet-Gallay K, Hennebicq S, Fauré J, Bottari SP, Coutton
  C, Ray PF, Arnoult C
  
  \textsuperscript{*} Co-premiers auteurs
  
  EMBO Molecular Medicine, Mai 2017
  
  \newpage
  
  \subsubsection{Contexte et objectifs}\label{contexte-et-objectifs}
  
  L'oligospermie, comme l'azoospermie sont des phénotypes d'infertilités
  masculines liées à la quantité de spermatozoïdes présent dans
  l'éjaculât. Les différentes études publiées ces dernières années
  montrent que les microdélétions du chromosome Y sont retrouvées chez
  10\% des hommes avec une azoospermie non-obstructives et chez 5\% des
  patients avec une oligozoospermie sévères (Hotaling \& Carrell,
  \protect\hyperlink{ref-Hotaling2014}{2014}). Ces taux bien qu'élevé ne
  représente qu'une infime partie des cas d'azoospermie et d'oligospermie
  suggérant l'implication de nombreux autres gènes dans ce phénotype.
  
  Entre 2005 et 2014 deux frères issus d'un union consanguin ont demandé
  des conseils médicaux auprès de différentes cliniques d'infertilité
  après deux ans de tentatives infructueuses de consevoir un enfant. Ces
  deux frères étant marriés à des femmes non-apparentées la piste de
  l'implication d'une cause féminine fut exclu et les recherches
  concentrées sur l'analyse des deux frères. Après analyse de leur
  éjaculât, (et de l'épydidime) \ldots{} tout deux présentèrent de sévères
  défauts de production de spermatozoïdes. Au vu de la similarité du
  phénotype et du lien de parenté les liant, l'hypothèse d'une cause
  génétique commune fut émmise. L'analyse de leur karyotype et du locus
  AZF du chromosome Y ne révélant aucune anomalie, la procédure d'un
  séquençage WES fut décidé.
  
  Dans ce contexte, l'objectif de mon travail sur l'analyse phénotype de
  ces deux frères a été d'effectuer l'ensemble des analyses des données
  WES obtenues après leur séquençage afin de mettre en évidence une
  mutation homozygote commune aux deux frères pouvant expliquer leur
  phénotype. Dans un second temps, j'ai pu mettre en place le protocole de
  génotypage des souris au locus du gène \emph{Spink2} permettant
  d'identifier les souris sauvages \emph{Spink2}\textsuperscript{+/+} des
  souris KO \emph{Spink2}\textsuperscript{-/-}. Pour finir, afin d'estimer
  l'importance des variants du gène \emph{SPINK2} comme cause
  d'infertilité masculine chez l'humain, j'ai également contribué au
  séquençage Sanger de la séquence codante de \emph{SPINK2} d'une partie
  des 611 patients séquencés dans cette étude.
  
  \newpage
  
  \includepdf[pages=-]{bib/SPINK2_2017.pdf}
  
  \newpage
  
  \subsubsection{Principaux résultats}\label{principaux-resultats}
  
  Après avoir analyser les données de séquençage des deux frères au sein
  de notre pipeline décrite précédement, seul 2 variants passèrent
  l'ensemble des filtres. Ces deux variants impactant respectivement les
  gènes \emph{GUF1} et \emph{SPINK2}. Parmis ces deux gènes, seul
  \emph{SPINK2} présentaient une forte expression testiculaire dans les
  données Ensembl (\textbf{Figure : }\ref{fig:plotexpfamaz}) que nous
  avons pu confirmer par RT-PCR dans cette étude. De plus, des mutations
  du gène \emph{Spink2} chez la souris avait déjà été identifiée comme
  induisant des défauts de la spermatogenèse (B. Lee et al.,
  \protect\hyperlink{ref-Lee2011}{2011}). Ces arguments ont ainsi fait de
  \emph{SPINK2} le candidat évident pour expliquer le phénotype de ces
  deux frères. Après avoir confirmé en séquençage Sanger la mutation de ce
  gène à l'état homozygote pour les deux frères et hétérozygotes pour les
  parents, nous avons, afin de continuer nos investigations, développés un
  modèle murin KO \emph{Spink2}\textsuperscript{-/-} confirmant une
  azoospermie complète pour les sours mâle spermiogenesis causé par un
  arret de la spermatogénèse au stage des spermatides rondes. De plus,
  malgré une fertilité normale, nous avons pu noter un taux élevé
  d'anomalies morphologiques du spermatozoïde ainsi qu'une motilité
  spermatique réduite chez les souris mâles hétérozygotes
  \emph{Spink2}\textsuperscript{+/-}. Les femelles, elles ne présentaient
  aucun phénotype apparent. L'étude de la localisation de la protéine
  Spink2 chez la souris et SPINK2 chez l'humain a révélés que ces deux
  protéines localisaient dans la vésicule acrosomale depuis le début de la
  biogénèse de l'acrosome jusqu'au spermatozoïde mature.
  
  Suite à cela, afin d'évaluer l'importance des variants du gène
  \emph{SPINK2} dans l'infertilité humaine, nous avons effectués le
  séquençage Sanger de 611 patient parmi lesquels 210 étaient azoospermes,
  393 oligozoospermes et 8 dont la cause n'étaient pas spécifiée. Parmi
  cet ensemble de patient, seul 1 (le patient p105) s'est révélé porter un
  variant non répertorié dans ExAC sur le gène \emph{SPINK2}. Ce patient
  présentant un phénotype d'oligozoospermie porte à l'état hétérozygote un
  variant altérant le codon start du gène \emph{SPINK2}. Ces résultats
  laissent donce supposer que chez l'homme, la présence de mutations
  homozygotes sur le gène \emph{SPINK2} induit un phénotype d'azoospermie
  tandis que les mutation hétérozygotes entrainent, elles, un phénotype
  d'oligozoospermie. Cette forte séléction négative pouvant expliquer la
  rareté des mutations observées sur ce gènes.
  
  \begin{longtable}[t]{llllll}
  \caption{\label{tab:tabrecapaz}Liste des variants ayant passé l'ensemble des filtres pour les deux fères de la famille AZ}\\
  \toprule
  \multicolumn{1}{c}{ } & \multicolumn{2}{c}{Variant impact} & \multicolumn{3}{c}{Variant frequency} \\
  \cmidrule(l{2pt}r{2pt}){2-3} \cmidrule(l{2pt}r{2pt}){4-6}
  Gene & HGVSc, HGVSp & Consequence & ExAC & ESP & 1KG\\
  \midrule
  GUF1 & c.443A>T ; p.Ser148Ile & missense & 0.00207 & 0.0028 & 9e-04\\
  SPINK2 & c.56-3C>G ; . & splice region & . & . & .\\
  \bottomrule
  \end{longtable}
  
  \newpage
  
  \begin{figure}
  
  {\centering \includegraphics{thesis_files/figure-latex/plotexpfamaz-1} 
  
  }
  
  \caption[Expression tissulaire des gènes *SPINK2* et *GUF1*]{Expression tissulaire des gènes *SPINK2* et *GUF1* : Données provenant du projet de transcriptome Illumina bodyMap}\label{fig:plotexpfamaz}
  \end{figure}
  
  \newpage
  
  \subsection{Article n° 4}\label{article-n-4}
  
  \textbf{Homozygous mutation of PLCZ1 leads to defective human oocyte
  activation and infertility that is not rescued by the WW-binding protein
  PAWP}
  
  Jessica Escoffier J\textsuperscript{*}, Lee HC\textsuperscript{*},
  Yassine S\textsuperscript{*}, Zouari R, Martinez G, \textbf{Karaouzène
  T}, Coutton C, Kherraf ZE, Halouani L, Triki C, Nef S, Thierry-Mieg N,
  Savinov SN, Fissore R, Ray PF, Arnoult C
  
  \textsuperscript{*} Co-premiers auteurs
  
  Human Molecular Genetics, Décembre 2015
  
  \newpage
  
  \subsubsection{Contexte et objectifs}\label{contexte-et-objectifs-1}
  
  L'activation ovocitaire regroupe une série de processus intervenant au
  cour de la fécondation d'un ovocyte par un spermatozoïde. en 1990,
  plusieurs études démontrèrent que chez les mamifères ces processus
  reposent principalement sur le relargage par le spermatozoïde de
  ``facteurs spermatiques'' qui déclenchent un signal de calcium,
  constitué d'oscillations Ca\textsuperscript{2+} {[}ref{]}. Plus tard, la
  protéine PLC\(\zeta\) fu identifiée comme la molécule responsable de ces
  oscilations calciques. Cependant, en raison de l'incapacité àproduire
  des modèles animaux \emph{PLC}\(\zeta\) KO capable de produire des
  spermatozoïdes mature a empeché d'attribuer l'exclusivité de ce rôle à
  \emph{PLC}\(\zeta\) laissant ouverte la possibilité de la nécessité
  d'autres facteurs spermatiques. C'est ainsi qu'en 2014 fut proposé la
  protéine PAWP comme facteur spermatique alternatif ou complémentaire de
  PLC\(\zeta\) (Aarabi et al.,
  \protect\hyperlink{ref-Aarabi2014}{2014}\protect\hyperlink{ref-Aarabi2014}{a},
  Aarabi et al.
  (\protect\hyperlink{ref-Aarabi2014a}{2014}\protect\hyperlink{ref-Aarabi2014a}{b})).
  
  Les travaux ci-dessous décrivent les analyses effectuées sur deux frères
  issus d'un union consanguin ayant tout deux été dans l'incapacité de
  concevoir un enfant par voies naturelles et pour qui, malgré des
  paramètres spermatiques normaux, l'ensemble des procédures de
  reproduction assités effectuées se sont soldés par un échec d'activation
  ovocitaire.
  
  Comme dans l'étude précédente, en raison de l'historique de
  cansanguinité de la famille des deux frères ainsi que le fait que leur
  femmes respectives soient non apparentées nous a permi d'exclure
  l'hypothèse d'une cause féminine et nous a conduit à rechercher un
  variant homozygote commun aux deux frères. Nous avons ainsi effectué un
  séquençage WES de ces deux frères. Comme précédemment, dans cette étude,
  j'ai été en charge de l'ensemble des analyses des données issus du
  séquençage des deux frères.
  
  \newpage
  
  \includepdf[pages=-]{bib/PLCZ1_2016}
  
  \newpage
  
  \subsubsection{Principaux résultats}\label{principaux-resultats-1}
  
  Suite à l'analyse bioinformatique de ces deux frères, un seul variant
  subsistait après l'application de l'ensemble des filtres. Celui-ci était
  recancé uniquement dans la base de donnée ExAC avec une fréquence de
  8.24e-06 et entrait un faux-sens prédit comme \emph{deleterious} par
  SIFT et \emph{possibly damaging} par PolyPhen sur la séquence du gène
  \emph{PLC}\(\zeta 1\). La forte expression testiculaire de ce gène
  (\textbf{Figure : }\ref{fig:plotexpfamff}) couplée à l'implication déjà
  connu de celui-ci dans l'activationn ovocitaire, ont fait de ce variant
  le candidat évident pour expliquer le phénotype de ces deux frères. De
  plus, aucun variant n'a été retrouvé sur la séquence du gène
  \emph{WBP2NL} codant pour la protéine PAWP bien que l'intégralité de la
  séquence codante de \emph{WBP2NL} ait une couverture \(\ge\) 40x (les
  zones moins couvertes du début de l'exon 1 et de la fin de l'exon 6
  correspondant aux régions UTR) (\textbf{Figure :
  }\ref{fig:plotcovplcz}). Ces résultats suggérant une parfaite
  fonctionnalité de la protéine PAWP ont pu être confirmée par
  \emph{Western blot}, de même, la bonne localisation de la protéine PAWP
  a pu être observée chez les deux patients par Immunofluorescence.
  
  Cette étude présente le premier cas de \emph{knock-down} de PLC\(\zeta\)
  n'entrainant pas d'effets sur la spermatogénèse démontrant ainsi le rôle
  primordial de cette protéine dans l'activation ovocitaire. De plus, la
  parfaite localisation et fonctionnalité de la protéine PAWP la disculpe
  de toute implication dans le phénotype de nos patients.
  
  \begin{longtable}[t]{llllllll}
  \caption{\label{tab:tabrecapff}Liste des variants ayant passé l'ensemble des filtres pour les deux fères de la famille FF}\\
  \toprule
  \multicolumn{1}{c}{ } & \multicolumn{4}{c}{Variant impact} & \multicolumn{1}{c}{Variant frequency} \\
  \cmidrule(l{2pt}r{2pt}){2-5} \cmidrule(l{2pt}r{2pt}){6-6}
  Gene & HGVSc, HGVSp & Consequence & SIFT & PolyPhen & ExAC & ESP & 1KG\\
  \midrule
  PLCZ1 & c.1465G>T ; p.Ile489Phe & missense & deleterious & possib damaging & 8.24e-06 & . & .\\
  \bottomrule
  \end{longtable}
  
  \newpage
  
  \begin{figure}
  
  {\centering \includegraphics{thesis_files/figure-latex/plotexpfamff-1} 
  
  }
  
  \caption[Expression tissulaire du gène PLCZ1*]{Expression tissulaire du gène PLCZ1* : Données provenant du projet de transcriptome Illumina bodyMap}\label{fig:plotexpfamff}
  \end{figure}
  
  \begin{figure}
  
  {\centering \includegraphics[scale=.4]{figure/pawp_coverage} 
  
  }
  
  \caption[Couverture des 6 éxons de *WBP2NL* pour les deux frères de la famille FF]{Couverture des 6 éxons de *WBP2NL* pour les deux frères de la famille FF}\label{fig:plotcovplcz}
  \end{figure}
  
  \newpage
  
  \subsection{Article n° 5}\label{article-n-5}
  
  \subsubsection{Whole-exome sequencing of familial cases of multiple
  morphological abnormalities of the sperm flagella (MMAF) reveals new
  DNAH1
  mutations}\label{whole-exome-sequencing-of-familial-cases-of-multiple-morphological-abnormalities-of-the-sperm-flagella-mmaf-reveals-new-dnah1-mutations}
  
  Amiri-Yekta A\textsuperscript{*}, Coutton C\textsuperscript{*}, Kherraf
  ZE, \textbf{Karaouzène T}, Le Tanno P, Sanati MH, Sabbaghian M, Almadani
  N, Sadighi Gilani MA, Seyedeh Hanieh Hosseini, Bahrami S, Daneshipour A,
  Bini M, Arnoult C, Colombo R, Gourabi H, Ray PF
  
  \textsuperscript{*} Co-premiers auteurs
  
  Human Reproduction, Octobre 2016
  
  \newpage
  
  \subsubsection{Contexte et objectifs}\label{contexte-et-objectifs-2}
  
  \newpage
  
  \includepdf[pages=-]{bib/Fam_DNAH1_2016.pdf}
  
  \newpage
  
  \subsubsection{Principaux résultats}\label{principaux-resultats-2}
  
  \begin{longtable}[t]{llllllll}
  \caption{\label{tab:tabrecapmmaf}Liste des variants ayant passé l'ensemble des filtres pour les deux fères de la famille MMAF}\\
  \toprule
  \multicolumn{1}{c}{ } & \multicolumn{4}{c}{Variant impact} & \multicolumn{1}{c}{Variant frequency} \\
  \cmidrule(l{2pt}r{2pt}){2-5} \cmidrule(l{2pt}r{2pt}){6-6}
  Gene & HGVSc, HGVSp & Consequence & SIFT & PolyPhen & ExAC & ESP & 1KG\\
  \midrule
  MYH11 & c.4625G>A ; p.Arg1542Gln & missense & . & proba damaging & 0.00234 & 0.0016 & 5e-04\\
  DNAH1 & . ; . & splice acceptor & . & . & . & . & .\\
  \bottomrule
  \end{longtable}
  
  \newpage
  
  \begin{figure}
  
  {\centering \includegraphics{thesis_files/figure-latex/plotexpfammmaf-1} 
  
  }
  
  \caption[Expression tissulaire des gènes DNAH1 et MYH11]{Expression tissulaire des gènes DNAH1 et MYH11 : Données provenant du projet de transcriptome Illumina bodyMap}\label{fig:plotexpfammmaf}
  \end{figure}
  
  \newpage
  
  \section{Résultats 2 : Étude d'une cohorte de femmes
  infertiles}\label{resultats-2-etude-dune-cohorte-de-femmes-infertiles}
  
  \subsection{Article n° 6}\label{article-n-6}
  
  \textbf{PATL2 Gene Mutation Causes Oocyte Meiotic Deficiency and Female
  Infertility}
  
  Christou-Kent M, Amiri-Yekta A, Kherraf ZE, \textbf{Karaouzène T},
  Escoffier J, Guttin A, Martinez G, Le Blévec E, Lambert E, Fourati Ben
  Mustapha S, Cedrin-Durnerin I, Halouani L, Marrakchi O, Makni M, Latrous
  H, Kharouf M, Bottari S, Thierry-Mieg N, Coutton C, Zouari R, Issartel
  JP, Ray PF, Arnoult C
  
  New England Journal of Medicine, 07 Juillet 2017 (soummis)
  
  \newpage
  
  \subsubsection{Contexte et objectifs}\label{contexte-et-objectifs-3}
  
  \newpage
  
  \includepdf[pages=-]{bib/PATL2_2017.pdf}
  
  \newpage
  
  \subsubsection{Principaux résultats}\label{principaux-resultats-3}
  
  \subsubsection{Discussion et
  Perspectives}\label{discussion-et-perspectives}
  
  \newpage  
  
  \section{Résultats 3 : Étude d'une large cohorte de patients
  MMAF}\label{resultats-3-etude-dune-large-cohorte-de-patients-mmaf}
  
  \subsection{Article n° 7}\label{article-n-7}
  
  \textbf{Whole exome cohort study and analysis of mouse and Trypanosoma
  models demonstrate the importance of WDR proteins in flagellogenesis and
  male fertility}
  
  Coutton C, Vargas A, Amiri-Yekta A, Kherraf ZE, Fourati Ben Mustapha S,
  Le Tanno P, Wambergue-Legrand C, \textbf{Karaouzène T}, Martinez G,
  Daneshipour A, Hanieh Hosseini S, Mitchell V, Halouani L, Marrakchi O,
  Makni M, Latrous H, Kharouf M, Deleuze JF, Boland A, Hennebicq S, Satre
  V, Jouk PS, Bottari SP, Thierry-Mieg N, Conne B, Dacheux-Deschamps D,
  Schmitt A, Stouvenel L, Lorès P, El Khouri E, Fauré J, Wolf JP,
  Escoffier J, Gourabi H, Robinson DR, Nef S, Dulioust E, Zouari R,
  Bonhivers M, Touré A, Arnoult C, Ray PF
  
  EMBO Molecular Medicine, Mai 2017
  
  \newpage
  
  \subsubsection{Contexte et objectifs}\label{contexte-et-objectifs-4}
  
  \newpage
  
  \includepdf[pages=-]{bib/CFAP_2017.pdf}
  
  \newpage
  
  \section{Conclusion}\label{conclusion}
  
  \chapter*{References}\label{references}
  \addcontentsline{toc}{chapter}{References}
  
  \hypertarget{refs}{}
  \hypertarget{ref-1000GenomesProjectConsortium2015}{}
  1000 Genomes Project Consortium, T. 1. G. P., Auton, A., Brooks, L. D.,
  Durbin, R. M., Garrison, E. P., Kang, H. M., \ldots{} Abecasis, G. R.
  (2015). A global reference for human genetic variation. \emph{Nature},
  \emph{526}(7571), 68--74. \url{http://doi.org/10.1038/nature15393}
  
  \hypertarget{ref-Aarabi2014}{}
  Aarabi, M., Balakier, H., Bashar, S., Moskovtsev, S. I., Sutovsky, P.,
  Librach, C. L., \& Oko, R. (2014a). Sperm content of postacrosomal WW
  binding protein is related to fertilization outcomes in patients
  undergoing assisted reproductive technology. \emph{Fertility and
  Sterility}, \emph{102}(2), 440--447.
  \url{http://doi.org/10.1016/j.fertnstert.2014.05.003}
  
  \hypertarget{ref-Aarabi2014a}{}
  Aarabi, M., Balakier, H., Bashar, S., Moskovtsev, S. I., Sutovsky, P.,
  Librach, C. L., \& Oko, R. (2014b). Sperm-derived WW domain-binding
  protein, PAWP, elicits calcium oscillations and oocyte activation in
  humans and mice. \emph{FASEB Journal : Official Publication of the
  Federation of American Societies for Experimental Biology},
  \emph{28}(10), 4434--40. \url{http://doi.org/10.1096/fj.14-256495}
  
  \hypertarget{ref-Adzhubei2010}{}
  Adzhubei, I. A., Schmidt, S., Peshkin, L., Ramensky, V. E., Gerasimova,
  A., Bork, P., \ldots{} Sunyaev, S. R. (2010). A method and server for
  predicting damaging missense mutations. \emph{Nature Methods},
  \emph{7}(4), 248--9. \url{http://doi.org/10.1038/nmeth0410-248}
  
  \hypertarget{ref-Aken2017}{}
  Aken, B. L., Achuthan, P., Akanni, W., Amode, M. R., Bernsdorff, F.,
  Bhai, J., \ldots{} Flicek, P. (2017). Ensembl 2017. \emph{Nucleic Acids
  Research}, \emph{45}(D1), D635--D642.
  \url{http://doi.org/10.1093/nar/gkw1104}
  
  \hypertarget{ref-Amberger2011}{}
  Amberger, J., Bocchini, C., \& Hamosh, A. (2011). A new face and new
  challenges for Online Mendelian Inheritance in Man (OMIM). \emph{Human
  Mutation}, \emph{32}(5), 564--567.
  \url{http://doi.org/10.1002/humu.21466}
  
  \hypertarget{ref-Amdani2013}{}
  Amdani, S. N., Jones, C., \& Coward, K. (2013). Phospholipase C zeta
  (PLC\(\zeta\)): Oocyte activation and clinical links to male factor
  infertility. \emph{Advances in Biological Regulation}, \emph{53}(3),
  292--308. \url{http://doi.org/10.1016/j.jbior.2013.07.005}
  
  \hypertarget{ref-Baker2004}{}
  Baker, K. E., \& Parker, R. (2004). Nonsense-mediated mRNA decay:
  terminating erroneous gene expression. \emph{Current Opinion in Cell
  Biology}, \emph{16}(3), 293--9.
  \url{http://doi.org/10.1016/j.ceb.2004.03.003}
  
  \hypertarget{ref-BenKhelifa2014}{}
  Ben Khelifa, M., Coutton, C., Zouari, R., Karaouzène, T., Rendu, J.,
  Bidart, M., \ldots{} Ray, P. F. (2014). Mutations in DNAH1, which
  encodes an inner arm heavy chain dynein, lead to male infertility from
  multiple morphological abnormalities of the sperm flagella.
  \emph{American Journal of Human Genetics}, \emph{94}(1), 95--104.
  \url{http://doi.org/10.1016/j.ajhg.2013.11.017}
  
  \hypertarget{ref-Chang2007}{}
  Chang, Y.-F., Imam, J. S., \& Wilkinson, M. F. (2007). The
  Nonsense-Mediated Decay RNA Surveillance Pathway. \emph{Annual Review of
  Biochemistry}, \emph{76}(1), 51--74.
  \url{http://doi.org/10.1146/annurev.biochem.76.050106.093909}
  
  \hypertarget{ref-Cingolani2012}{}
  Cingolani, P., Platts, A., Wang, L. L., Coon, M., Nguyen, T., Wang, L.,
  \ldots{} Ruden, D. M. (2012). A program for annotating and predicting
  the effects of single nucleotide polymorphisms, SnpEff. \emph{Fly},
  \emph{6}(2), 80--92. \url{http://doi.org/10.4161/fly.19695}
  
  \hypertarget{ref-DePristo2011}{}
  DePristo, M. A., Banks, E., Poplin, R., Garimella, K. V., Maguire, J.
  R., Hartl, C., \ldots{} Pritchard, E. (2011). A framework for variation
  discovery and genotyping using next-generation DNA sequencing data.
  \emph{Nature Genetics}, \emph{43}(5), 491--498.
  \url{http://doi.org/10.1038/ng.806}
  
  \hypertarget{ref-Hotaling2014}{}
  Hotaling, J., \& Carrell, D. T. (2014). Clinical genetic testing for
  male factor infertility: current applications and future directions.
  \emph{Andrology}, \emph{2}(3), 339--350.
  \url{http://doi.org/10.1111/j.2047-2927.2014.00200.x}
  
  \hypertarget{ref-Imai2015}{}
  Imai, Y., Morita, H., Takeda, N., Miya, F., Hyodo, H., Fujita, D.,
  \ldots{} Komuro, I. (2015). A deletion mutation in myosin heavy chain 11
  causing familial thoracic aortic dissection in two Japanese pedigrees.
  \emph{International Journal of Cardiology}, \emph{195}, 290--292.
  \url{http://doi.org/10.1016/j.ijcard.2015.05.178}
  
  \hypertarget{ref-Kumar2009}{}
  Kumar, P., Henikoff, S., \& Ng, P. C. (2009). Predicting the effects of
  coding non-synonymous variants on protein function using the SIFT
  algorithm. \emph{Nature Protocols}, \emph{4}(7), 1073--1081.
  \url{http://doi.org/10.1038/nprot.2009.86}
  
  \hypertarget{ref-Lee2011}{}
  Lee, B., Park, I., Jin, S., Choi, H., Kwon, J. T., Kim, J., \ldots{}
  Cho, C. (2011). Impaired spermatogenesis and fertility in mice carrying
  a mutation in the Spink2 gene expressed predominantly in testes.
  \emph{The Journal of Biological Chemistry}, \emph{286}(33), 29108--17.
  \url{http://doi.org/10.1074/jbc.M111.244905}
  
  \hypertarget{ref-Lek2016}{}
  Lek, M., Karczewski, K. J., Minikel, E. V., Samocha, K. E., Banks, E.,
  Fennell, T., \ldots{} Exome Aggregation Consortium, D. G. (2016).
  Analysis of protein-coding genetic variation in 60,706 humans.
  \emph{Nature}, \emph{536}(7616), 285--91.
  \url{http://doi.org/10.1038/nature19057}
  
  \hypertarget{ref-Lunter2011}{}
  Lunter, G., \& Goodson, M. (2011). Stampy: A statistical algorithm for
  sensitive and fast mapping of Illumina sequence reads. \emph{Genome
  Research}, \emph{21}(6), 936--939.
  \url{http://doi.org/10.1101/gr.111120.110}
  
  \hypertarget{ref-McLaren2016}{}
  McLaren, W., Gil, L., Hunt, S. E., Riat, H. S., Ritchie, G. R. S.,
  Thormann, A., \ldots{} Cunningham, F. (2016). The Ensembl Variant Effect
  Predictor. \emph{Genome Biology}, \emph{17}(1), 122.
  \url{http://doi.org/10.1186/s13059-016-0974-4}
  
  \hypertarget{ref-Ng}{}
  Ng, S. B., Buckingham, K. J., Lee, C., Bigham, A. W., Tabor, H. K.,
  Dent, K. M., \ldots{} Bamshad, M. J. (n.d.). Exome sequencing identifies
  the cause of a Mendelian disorder. \url{http://doi.org/10.1038/ng.499}
  
  \hypertarget{ref-Nielsen2011}{}
  Nielsen, R., Paul, J. S., Albrechtsen, A., \& Song, Y. S. (2011).
  Genotype and SNP calling from next-generation sequencing data.
  \emph{Nature Reviews. Genetics}, \emph{12}(6), 443--51.
  \url{http://doi.org/10.1038/nrg2986}
  
  \hypertarget{ref-Pelak2010}{}
  Pelak, K., Shianna, K. V., Ge, D., Maia, J. M., Zhu, M., Smith, J. P.,
  \ldots{} Goldstein, D. B. (2010). The characterization of twenty
  sequenced human genomes. \emph{PLoS Genetics}, \emph{6}(9), e1001111.
  \url{http://doi.org/10.1371/journal.pgen.1001111}
  
  \hypertarget{ref-Robinson2014}{}
  Robinson, P. N., Köhler, S., Oellrich, A., Sanger Mouse Genetics
  Project, S. M. G., Wang, K., Mungall, C. J., \ldots{} Smedley, D.
  (2014). Improved exome prioritization of disease genes through
  cross-species phenotype comparison. \emph{Genome Research},
  \emph{24}(2), 340--8. \url{http://doi.org/10.1101/gr.160325.113}
  
  \hypertarget{ref-Su2014}{}
  Su, Z., Łabaj, P. P., Li, S. S., Thierry-Mieg, J., Thierry-Mieg, D.,
  Shi, W., \ldots{} Shi, L. (2014). A comprehensive assessment of RNA-seq
  accuracy, reproducibility and information content by the Sequencing
  Quality Control Consortium. \emph{Nature Biotechnology}, \emph{32}(9),
  903--14. \url{http://doi.org/10.1038/nbt.2957}
  
  \hypertarget{ref-Wang2010}{}
  Wang, K., Li, M., \& Hakonarson, H. (2010). ANNOVAR: functional
  annotation of genetic variants from high-throughput sequencing data.
  \emph{Nucleic Acids Research}, \emph{38}(16), e164--e164.
  \url{http://doi.org/10.1093/nar/gkq603}


  % Index?

\end{document}

